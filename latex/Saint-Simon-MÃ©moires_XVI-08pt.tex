\PassOptionsToPackage{unicode=true}{hyperref} % options for packages loaded elsewhere
\PassOptionsToPackage{hyphens}{url}
%
\documentclass[oneside,8pt,french,]{extbook} % cjns1989 - 27112019 - added the oneside option: so that the text jumps left & right when reading on a tablet/ereader
\usepackage{lmodern}
\usepackage{amssymb,amsmath}
\usepackage{ifxetex,ifluatex}
\usepackage{fixltx2e} % provides \textsubscript
\ifnum 0\ifxetex 1\fi\ifluatex 1\fi=0 % if pdftex
  \usepackage[T1]{fontenc}
  \usepackage[utf8]{inputenc}
  \usepackage{textcomp} % provides euro and other symbols
\else % if luatex or xelatex
  \usepackage{unicode-math}
  \defaultfontfeatures{Ligatures=TeX,Scale=MatchLowercase}
%   \setmainfont[]{EBGaramond-Regular}
    \setmainfont[Numbers={OldStyle,Proportional}]{EBGaramond-Regular}      % cjns1989 - 20191129 - old style numbers 
\fi
% use upquote if available, for straight quotes in verbatim environments
\IfFileExists{upquote.sty}{\usepackage{upquote}}{}
% use microtype if available
\IfFileExists{microtype.sty}{%
\usepackage[]{microtype}
\UseMicrotypeSet[protrusion]{basicmath} % disable protrusion for tt fonts
}{}
\usepackage{hyperref}
\hypersetup{
            pdftitle={SAINT-SIMON},
            pdfauthor={Mémoires XVI},
            pdfborder={0 0 0},
            breaklinks=true}
\urlstyle{same}  % don't use monospace font for urls
\usepackage[papersize={4.80 in, 6.40  in},left=.5 in,right=.5 in]{geometry}
\setlength{\emergencystretch}{3em}  % prevent overfull lines
\providecommand{\tightlist}{%
  \setlength{\itemsep}{0pt}\setlength{\parskip}{0pt}}
\setcounter{secnumdepth}{0}

% set default figure placement to htbp
\makeatletter
\def\fps@figure{htbp}
\makeatother

\usepackage{ragged2e}
\usepackage{epigraph}
\renewcommand{\textflush}{flushepinormal}

\usepackage{indentfirst}
\usepackage{relsize}

\usepackage{fancyhdr}
\pagestyle{fancy}
\fancyhf{}
\fancyhead[R]{\thepage}
\renewcommand{\headrulewidth}{0pt}
\usepackage{quoting}
\usepackage{ragged2e}

\newlength\mylen
\settowidth\mylen{...................}

\usepackage{stackengine}
\usepackage{graphicx}
\def\asterism{\par\vspace{1em}{\centering\scalebox{.9}{%
  \stackon[-0.6pt]{\bfseries*~*}{\bfseries*}}\par}\vspace{.8em}\par}

\usepackage{titlesec}
\titleformat{\chapter}[display]
  {\normalfont\bfseries\filcenter}{}{0pt}{\Large}
\titleformat{\section}[display]
  {\normalfont\bfseries\filcenter}{}{0pt}{\Large}
\titleformat{\subsection}[display]
  {\normalfont\bfseries\filcenter}{}{0pt}{\Large}

\setcounter{secnumdepth}{1}
\ifnum 0\ifxetex 1\fi\ifluatex 1\fi=0 % if pdftex
  \usepackage[shorthands=off,main=french]{babel}
\else
  % load polyglossia as late as possible as it *could* call bidi if RTL lang (e.g. Hebrew or Arabic)
%   \usepackage{polyglossia}
%   \setmainlanguage[]{french}
%   \usepackage[french]{babel} % cjns1989 - 1.43 version of polyglossia on this system does not allow disabling the autospacing feature
\fi

\title{SAINT-SIMON}
\author{Mémoires XVI}
\date{}

\begin{document}
\maketitle

\hypertarget{chapitre-premier.}{%
\chapter{CHAPITRE PREMIER.}\label{chapitre-premier.}}

1718

~

{\textsc{L'empereur accepte le projet de paix.}} {\textsc{- Les Anglais
haïssent, se plaignent, demandent le rappel de Châteauneuf de
Hollande.}} {\textsc{- Leur impudence à l'égard du régent.}} {\textsc{-
Guidés par Dubois, ils pressent et menacent l'Espagne.}} {\textsc{-
L'empereur ménage enfin les Hollandais.}} {\textsc{- Erreur de
Monteléon.}} {\textsc{- Difficulté et conduite de la négociation du roi
de Sicile à Vienne.}} {\textsc{- Énormité contre M. le duc d'Orléans des
agents du roi de Sicile à Vienne, qui échouent en tout.}} {\textsc{-
Sage conduite et avis de Monteléon.}} {\textsc{- La Hollande, pressée
d'accéder au traité, recule.}} {\textsc{- Beretti, par ordre d'Albéroni,
qui voudrait jeter le Prétendant en Angleterre, tâche à lier l'Espagne
avec la Suède et le czar prêts à faire leur paix ensemble.}} {\textsc{-
Sages réflexions de Cellamare.}} {\textsc{- Son adresse à donner de bons
avis pacifiques en Espagne.}} {\textsc{- Dangereuses propositions pour
la France du roi de Sicile à l'empereur.}} {\textsc{- Provane les traite
d'impostures\,; proteste contre l'abandon de la Sicile, et menace la
France dans Paris.}} {\textsc{- Nouvelles scélératesses du nonce
Bentivoglio.}} {\textsc{- Fortes démarches du pape pour obliger le roi
d'Espagne de cesser ses préparatifs de guerre contre l'empereur.}}
{\textsc{- Autres griefs du pape contre le roi d'Espagne.}} {\textsc{-
Menaces de l'Espagne au pape.}} {\textsc{- Souplesses et lettres de Sa
Sainteté en Espagne.}} {\textsc{- Fortes démarches de l'Espagne sur les
bulles de Séville.}} {\textsc{- Manège d'Aldovrandi.}}

~

Enfin les incertitudes de la cour de Vienne cessèrent, et on apprit par
un courrier qu'en reçut Penterrieder à Londres que l'empereur acceptait
un projet que toute l'Europe regardait comme très avantageux à la maison
d'Autriche. Toutefois il s'était fait prier longtemps pour y consentir,
et ce n'était qu'avec des peines infinies, au moins en apparence, qu'il
s'était désisté de prétendre pour lui-même la succession du grand-duc de
Toscane. Ceux qui négociaient de la part du roi d'Angleterre furent si
contents d'avoir obtenu ce point, dont ils firent un mérite particulier
à Schaub, qu'ils préparaient déjà le régent à se relâcher sur des
conditions moins importantes qu'on pourrait lui demander\,; et pour
l'obtenir comme un effet de reconnaissance légitime, ils assuraient que
Schaub avait parfaitement bien plaidé la cause de Son Altesse Royale. La
nouvelle de l'acceptation de l'empereur causa beaucoup de joie à la cour
d'Angleterre, même aux négociants, parce qu'ils se flattèrent que le roi
d'Espagne ne pourrait se dispenser d'accepter, par conséquent qu'il n'y
aurait point de guerre, et que le commerce deviendrait plus florissant
que jamais. Au contraire les torys et généralement tous les mécontents
du gouvernement s'élevèrent contre le projet dans le fond, parce que
c'était l'ouvrage des ministres, mais en apparence à cause de la
disposition de la Sicile, en faveur de l'empereur et de celle de la
Sardaigne donnée en échange.

La cour d'Angleterre, après cette nouvelle, résolut de ménager la
communication qu'elle devait faire du projet à la Hollande, et de ne lui
en apprendre le véritable état que par degrés\,; mais elle se plaignit
que Châteauneuf, ambassadeur de France à la Haye, avait dérangé ces
mesures. Elle l'accusait depuis longtemps de mauvaises intentions et
d'agir suivant les principes de l'ancien gouvernement de France, crime
capital à l'égard des Anglais. Ainsi les ministres d'Angleterre
pressèrent le régent de rappeler au plus tôt cet ambassadeur, d'envoyer
Morville le relever, nommé depuis quelque temps pour lui succéder, et de
le faire aller directement à la Haye sans le faire passer à Londres, où
on avait dit qu'il irait pour se mettre au fait des affaires en y
recevant les instructions de l'abbé Dubois. Mais les ministres
d'Angleterre jugèrent qu'il suffisait qu'il se laissât conduire par
Widword, envoyé d'Angleterre en Hollande, et par Cadogan, que cette cour
avait résolu d'y faire passer immédiatement après avoir reçu
l'acceptation de l'empereur. Ils assuraient donc tous que tout irait le
mieux du monde, pourvu que le régent sût bien prendre son parti, et
qu'on fût en état de montrer de la vigueur aux. Espagnols, car il n'y
avait pas le moindre lieu, disaient-ils, de douter de la sincérité de la
cour de Vienne. Sur ce fondement le roi d'Angleterre envoya par un
courrier, de nouveaux ordres à son ministre à Madrid de presser plus que
jamais le roi d'Espagne de souscrire au traité, et pour le persuader le
colonel Stanhope eut ordre de lui déclarer que le départ de l'escadre
anglaise ne pouvait plus être différé, et que dans trois semaines au
plus tard elle serait en état de mettre à la voile.

Prié, commandant général des Pays-Bas pour le prince Eugène, gouverneur
général, reçut des ordres très exprès de terminer au plus tôt les
difficultés qui avaient jusqu'alors empêché l'exécution du traité de la
Barrière. Prié avait déjà reçu plusieurs ordres de même nature, mais il
semblait que plus la cour de Vienne les réitérait, plus il trouvait de
moyens d'embrouiller la négociation. L'empereur voulait alors la finir,
croyant apparemment qu'il était bon d'engager les Hollandais à souscrire
à un traité dont il ne laissait pas de connaître les avantages, quelque
peine qu'il eût montrée à consentir à plusieurs de ses conditions.
Monteléon quoique habile avait cru lui-même que la cour de Vienne y
souscrirait difficilement, car il ne pouvait comprendre qu'elle
consentît à laisser au roi d'Espagne les moyens de rentrer en Italie. Il
s'échappa même jusqu'à dire, quand il sut que l'empereur acceptait le
projet, qu'enfin Sa Majesté Catholique remettrait le pied en Italie, et
qu'elle y serait soutenue par un bon et puissant ami. Monteléon se
flattait en effet que cette assistance ne pouvait manquer à l'Espagne de
la part de la France, et comme il avait jugé que la cour de Vienne en
penserait de même, il fut très surpris d'apprendre que, contre son
ordinaire, elle se rendit si facile. Il attribua ce changement au peu
d'espérance qu'elle avait apparemment de conclure la paix ou la trêve
avec les Turcs. Mais il se trompait encore, car alors la conclusion de
la paix était prochaine. Il crut aussi que l'empereur, voyant les
princes d'Italie las de ses vexations, prêts à s'unir ensemble pour
secouer le joug des Allemands, ne voulait pas s'exposer à soutenir une
guerre en Italie, pendant que celle de Hongrie durait encore\,; que
d'ailleurs il avait à craindre les mauvaises dispositions des peuples de
Naples et de Milan, qui seraient vraisemblablement fomentées par le roi
de Sicile, si la négociation que ce prince avait commencée secrètement à
Vienne ne finissait pas heureusement. Or il n'y avait pas lieu d'en
espérer un bon succès. Une des conditions préliminaires que le roi de
Sicile demandait était celle de conserver ce royaume\,; et l'empereur,
de son côté, ne trouvait rien de plus sensible et de plus avantageux
pour lui que d'en faire l'acquisition. La résistance des ministres
piémontais l'aigrit d'autant plus qu'il parut par leurs discours que
leur maître prétendait conserver la Sicile de concert et avec
l'assistance du roi d'Espagne. À la vérité ils faisaient paraître plus
de confiance en ce secours éloigné qu'ils n'en avaient en effet,
connaissant parfaitement la faiblesse de l'Espagne et le peu de réalité
des forces dont Albéroni faisait valoir les seules apparences. Mais
eux-mêmes les relevant se flattaient que, si l'empereur pouvait croire
avoir besoin de leur maître, il se rendrait plus facile sur le mariage
d'une archiduchesse qu'il désirait avec ardeur pour le prince de
Piémont.

Soit qu'ils crussent que le régent par des vues particulières
traverserait ce mariage, soit que ce fût dans leur pensée de faire un
mérite à la cour de Vienne de parler contre le gouvernement de France,
ils parlaient avec peu de circonspection de la personne de M. le duc
d'Orléans. La conclusion de leur discours était qu'il ne serait pas bien
difficile d'enlever le roi des mains de Son Altesse Royale. Un de ces
Piémontais, nommé Pras, se porta même jusqu'à dire que le projet en
était fait, et qu'il osait répondre de l'exécution. Le roi n'avait alors
d'autre ministre à Vienne qu'un nommé du Bourg, que le comte du Luc,
dont il était secrétaire, avait laissé à cette cour quand il en était
parti pour revenir en France. Pras s'imagina que du Bourg était opposé
aux intérêts de M. le duc d'Orléans, et plein de confiance ou pressé de
parler, il lui dit que le roi de Sicile avait des liaisons très intimes
avec le cardinal Albéroni, et que par le moyen de cette union secrète,
le roi d'Espagne avait prétendu prendre des mesures avec l'empereur pour
disposer ensemble, et de concert, du sort de toute l'Europe. Pras fit de
plus voir à du Bourg une lettre horrible contre M. le duc d'Orléans
qu'il supposa lui avoir été écrite de Paris. La même lettre fut
communiquée à l'empereur par l'intrigue des Piémontais, qui prétendirent
que ce prince en avait été fort ému. Ils ne réussirent cependant ni dans
leurs desseins ni dans les moyens dont ils se servirent pour y parvenir.
Le caractère du roi de Sicile était connu depuis longtemps. Il voulut à
son ordinaire frapper à toutes les portes. Il les trouva toutes fermées,
parce que l'expérience commune avait appris à tout le monde à se défier
également de lui\,; ainsi chacun se réjouissait de voir qu'il était la
victime de ses manèges doubles.

Dans ces circonstances, Monteléon zélé pour son maître, attaché
peut-être à l'Angleterre par quelque intérêt particulier, souhaitait
ardemment qu'il voulût demeurer uni avec le roi d'Angleterre. Il
prévoyait l'embarras où se trouverait l'Espagne si les choses en
venaient à une rupture, et connaissant qu'elle ne pouvait soutenir seule
un engagement contre les principales puissances de l'Europe, il eût
conseillé, s'il l'eût osé, de faire de nécessité vertu, de ne pas
mépriser le bénéfice offert, et de rendre grâces pour les offenses\,;
mais la crainte de déplaire au premier ministre le retenait\,; et
c'était avec peine qu'il osait confier à ses amis ce qu'il pensait sur
l'état des affaires. Il se contentait lorsqu'il en rendait compte en
Espagne de mettre, dans la bouche des autres une partie de ce qu'il
n'osait représenter comme de lui, et quand la nouvelle de l'acceptation
de l'empereur fut arrivée, il représenta que ce prince avilit beaucoup
gagné auprès de la cour d'Angleterre en prévenant par son consentement
celui qu'on attendait, et qu'on désirait ardemment du roi d'Espagne.

La France et l'Angleterre, unies et sûres de l'empereur, pressèrent
vivement la Hollande de souscrire au traité, et d'entrer avec elles dans
les mêmes liaisons\,; mais cette république dont les délibérations sont
ordinairement lentes, redoublait encore de lenteur, retenue par le
mauvais état de ses finances et par la mauvaise constitution de son
gouvernement. L'une et l'autre de ces raisons, obstacles invincibles à
la guerre, faisaient désirer ardemment la conservation de la paix. Ainsi
la république désapprouvait la précipitation de l'Angleterre, et
trouvait qu'elle avait tort de presser l'armement destiné pour la
Méditerranée. Les Hollandais, du moins ceux qui ne dépendaient pas
absolument de l'Angleterre, accusaient les Anglais d'une égale
imprudence, en donnant à l'empereur les moyens de se rendre
insensiblement maître de toute l'Italie.

Beretti soufflait le feu qu'il se flattait, et qu'il se vantait souvent
mal à propos d'avoir excité, et, pour se faire un mérite auprès
d'Albéroni, faisait des pronostics sur les troubles qu'on verrait
bientôt en Écosse, si le Prétendant, s'embarquant en Norvège, passait
dans ce royaume avec les secours du roi de Suède et du czar, comme on
supposait que les torys et les wighs mécontents, et les jacobites le
désiraient et le croyaient. Beretti avait ordre d'Albéroni de fomenter
l'exécution de ce projet, et de parler pour cet effet, soit à ceux qui
seraient dans la confidence du roi de Suède, soit aux ministres du czar
à la Haye. Il s'adressa donc aux uns et aux autres. Le roi de Suède
avait en Hollande un secrétaire nommé Preiss, mais ce prince se confiait
principalement à un officier polonais attaché au roi Stanislas nommé
Poniatowski. Beretti, suivant ses ordres, lui demanda si le roi de Suède
consentirait à recevoir quelques sommes d'argent du roi d'Espagne, et
s'il donnerait en échange des armes et des provisions nécessaires pour
la marine d'Espagne. La proposition ne parut pas nouvelle au Polonais.
Il dit qu'elle lui avait déjà été faite en secret à Paris par Monti\,;
que tout ce qu'il à voit pu lui répondre était que, se trouvant pressé
de se rendre, auprès du roi de Suède, il fallait laisser l'affaire à
traiter entre Beretti et Preiss. Il ajouta comme une chose très secrète,
et qu'il prétendait bien savoir, que l'amitié qui paraissait si vive
entre le roi d'Angleterre et le régent n'était que masquée\,; que, si la
paix qu'il croyait alors prête à se faire entre le roi de Suède et le
czar venait à se conclure, la France changerait de conduite, et qu'elle
se comporterait à l'égard de l'Angleterre d'autant plus différemment,
que le roi d'Angleterre s'éloignait chaque jour de plus en plus de
traiter avec le roi de Suède. Beretti, content des bonnes dispositions
que Poniatowski lui laissait entrevoir, le fut encore davantage de
celles de l'ambassadeur de Moscovie. Ce ministre lui dit que le temps
approchait où le roi d'Espagne pouvait tirer un grand avantage de
l'intelligence étroite qu'il établirait avec le czar et le roi de Suède,
qui de leur côté profiteraient de ces liaisons réciproques. Beretti
jugeait qu'elles étaient d'autant plus nécessaires, que, malgré
l'espérance que les agents, du roi de Suède lui avaient donnée que
l'union entre la France et l'Angleterre ne serait ni, solide ni de
durée, il voyait au contraire les ministres français et anglais agir
entre eux d'un grand concert, et presser unanimement les États généraux
de souscrire au projet du traité. On se flattait même alors que le
cardinal Albéroni deviendrait plus docile\,; on disait qu'il commençait
à mollir. Les Anglais faisaient usage de ces avis en Hollande, et s'en
servaient comme de raisons décisives pour engager la république à
convenir de ce qu'ils désiraient.

Toutefois Cellamare et Monti, mieux instruits des véritables sentiments
d'Albéroni, assurèrent toujours Provane qui était encore à Paris, de la
part du roi de Sicile, que certainement le roi d'Espagne rejetterait le
projet\,; qu'il ne se contenterait pas des compliments du roi
d'Angleterre ni de ses discours équivoques pendant qu'il travaillait par
des réalités à augmenter la puissance de l'empereur. Les discours de
Cellamare et de Monti étaient confirmés par les lettres qu'ils
montraient d'Albéroni. Cellamare pour lui plaire s'exhalait contre le
traité en plaintes et en réflexions à peu près les mêmes qu'on a déjà
vues. Mais il avait bon esprit, et les propos qu'il tenait ne
l'empêchaient pas de connaître parfaitement que le roi d'Espagne, en
rejetant le traité, exposait sa monarchie à de grands dangers. On voyait
clairement la liaison intime du roi d'Angleterre, prince de l'empire,
avec l'empereur chef de l'empire. Il était apparent que les Anglais
lèveraient incessamment le masque de médiateurs, et que, reprenant le
personnage de protecteurs de la maison d'Autriche, ils insulteraient
pour lui plaire les États d'Espagne en Europe et en Amérique. Cellamare
le prévoyait, mais, il aurait mal fait sa cour en Espagne, s'il eût
annoncé quelque suite fâcheuse des résolutions où le premier ministre
voulait entraîner son maître. Ainsi Cellamare se contenta de mettre dans
la bouche des personnes sensées ce qu'il n'osait dire comme son, propre
sentiment, encore usa-t-il de la précaution de rapporter ces réflexions
comme un effet de la terreur qui s'était emparée de tous les esprits, ou
d'une prostitution générale. C'était sous ces couleurs qu'il rapportait
les différents jugements qu'on faisait du parti que prendrait le roi
d'Espagne.

Cellamare inclinant à la paix, parce qu'il en voyait la nécessité,
disait que l'opinion commune était que Sa Majesté Catholique en
accepterait les conditions conditionnellement, c'est-à-dire qu'elle les
soumettrait à la discussion des ministres assemblés, et que cependant il
n'y aurait rien de conclu ni d'exécuté jusqu'à ce que toutes les parties
intéressées eussent été entendues. Son idée était de profiter du
bénéfice du temps propre à guérir les maladies les plus dangereuses, et
pour appuyer ce sentiment il citait l'autorité du comte de Peterborough,
qui lui avait dit que l'empereur était très éloigné de renoncer à ses
droits imaginaires\,; que ce prince ne consentait au projet que parce
qu'il était bien persuadé qu'il n'aurait pas lieu, que le roi d'Espagne
le rejetterait, et que l'empereur par sa docilité apparente se
concilierait l'amitié des médiateurs. Ainsi l'ambassadeur d'Espagne
conseillait à son maître de combattre ses ennemis par les mêmes armes
qu'ils prétendaient employer pour l'attaquer, et de contre-miner leur
artifice en affectant de faire paraître encore plus de penchant pour la
paix et plus de douceur qu'ils n'en témoignaient pour s'accorder sur les
conditions. Son but était de procurer une assemblée où les ministres de
toutes les parties intéressées conviendraient des conditions d'une paix
générale. C'était dans cette conjoncture que Cellamare jugeait que le
roi d'Espagne parviendrait à rompre le dangereux fil de cette trame mal
ourdie, qui réunissait tant de puissances contre Sa Majesté Catholique.
Jusqu'alors elle n'avait, selon lui, d'autre parti à prendre que de
prolonger la négociation, et pour y réussir, il conseillait de demander
premièrement une suspension d'armes, parce que le roi d'Espagne ne
pouvait seul, et par ses propres forces, établir et conserver
l'équilibre de l'Europe, malgré l'aveuglement universel de tous les
autres princes. La demande d'une suspension engagerait vraisemblablement
les alliés à demander aussi au roi d'Espagne de retirer ses troupes de
la Sardaigne, et de la remettre entre les mains d'un tiers pour la
garder en dépôt jusqu'à là conclusion du traité de paix. En ce cas,
Cellamare conseillait à son maître d'insister sur le dédommagement de
l'inexécution des traités que l'empereur avait faits peu d'années
auparavant pour retirer ses troupes de Catalogne, sans avoir satisfait
aux principales conditions de ces traités. Il prévoyait que les
prétentions réciproques sur ces matières donneraient lieu à de longues
contestations, et comme les Allemands pourraient cependant en venir aux
insultes, que même ils seraient peut-être soutenus par les Anglais,
l'avis de Cellamare était que le roi son maître, ne pouvant soutenir une
guerre déclarée contre toute l'Europe, devait s'armer assez puissamment
pour tenir dans le respect ceux qui songeraient à l'attaquer pendant le
cours de la négociation de paix. Comme l'Espagne avait principalement
besoin de forces maritimes, et qu'il fallait non seulement pour les
mettre sur pied, mais encore pour les faire agir et pour les commander,
des officiers expérimentés et capables, dont l'Espagne manquait
absolument, Cellamare crut donner une nouvelle agréable au roi d'Espagne
en lui annonçant qu'un Anglais nommé Camok, autrefois chef d'escadre en
Angleterre, était venu nouvellement lui réitérer les offres de services
qu'il avait déjà faites à Sa Majesté Catholique. Camok assurait
positivement que, si l'escadre Anglaise entrait dans la Méditerranée, il
engagerait sept ou huit capitaines de cette escadre à passer, avec leurs
navires et leurs officiers, au service d'Espagne, et ce qui est plus
étonnant, de semblables promesses étaient appuyées par le témoignage du
lieutenant général Dillon, homme de mérite et de probité. Les
préparatifs de guerre étoient d'autant plus nécessaires, qu'il
prétendait découvrir chaque jour de nouvelles intrigués et de nouveaux
moyens que l'empereur et le roi d'Angleterre employaient pour animer le
régent et pour l'engager à faire la guerre à l'Espagne.

Suivant cet ambassadeur, les ministres impériaux avaient confié à Son
Altesse Royale que le roi de Sicile offrait de céder la Sicile à leur
maître, à condition qu'il emploierait ses forces à placer le roi de
Sicile sur le trône d'Espagne, si le roi d'Espagne occupait celui de
France en cas d'ouverture à la succession à cette couronne. Les
Impériaux, disait-il, ajoutaient encore que, si, ce projet n'avait pas
lieu, le roi de Sicile consentirait à céder ce royaume en échange, de la
simple assurance des successions de Toscane et de Parme, dont il se
contenterait. Provane, que le roi de Sicile laissait encore à Paris,
traitait de faussetés et de calomnies inventées contre l'honneur de son
maître ces différents bruits de traités et de conventions entre
l'empereur et lui. Provane, au contraire, disait que toutes les
puissances de l'Europe, réunies ensemble, n'entraîneraient pas son
maître à s'immoler lui-même tranquillement et volontairement\,; que, si
elles voulaient se satisfaire, elles seraient obligées d'y employer la
forcé\,; qu'alors elles auraient affaire non à un agneau, mais à un
lion, qui se défendrait avec les ongles et avec les dents jusqu'au
dernier moment de sa vie. Enfin Provane disait que, si la France
réduisait le roi de Sicile au pied du mur, il ferait peut-être des
choses qu'elle n'aurait pas prévues, et qu'il pourrait contribuer encore
une fois à voir les étendards de la maison d'Autriche dans les provinces
de Dauphiné et de Provence.

Le nonce du pape n'était pas moins attentif que les ministres d'Espagne
et de Sicile à ce qui regardait le progrès de l'alliance, ni moins
ardent à relever et à faire valoir tout ce qu'il croyait contraire aux
intérêts de là France et aux vues de M. le duc d'Orléans. Sur ce
principe Bentivoglio regardait et répandait comme une bonne nouvelle
l'opposition du roi d'Espagne au projet de traité. Il assurait en même
temps comme une chose certaine que la ligue était faite entre le czar et
le roi de Suède\,; que les forces de ces deux princes étant réunies, le
roi de Suède s'embarquait pour aller faire une descente en Angleterre,
et rétablir le roi Jacques sur le trône de ses pères. Tout événement
capable de déranger les mesures du gouvernement lui paraissait d'autant
plus à souhaiter qu'il croyait, et qu'il tâchait de persuader au pape,
qu'il ne devait rien attendre de bon pour Rome de la France, etc.

Le pape était bien moins occupé et touché des affaires de la
constitution en France, qu'il ne l'était des affaires d'Espagne. Il
tremblait de voir la flotte et les troupes de cette couronne venir
fondre en Italie\,; et de la demande qu'elle lui avait faite de ses
ports pour son armée navale, à quoi il ne savait que répondre. Il était
bien plus en peine d'apaiser les Allemands qui, sans le croire,
l'accusaient d'intelligence contre eux avec l'Espagne, pour le tenir
sans cesse dans la frayeur et la souplesse à leur égard, et l'obliger
ainsi à n'oublier rien pour détourner l'orage qui les menaçait en
Italie, tandis que la Hongrie les occupait encore presque tous. Le pape
tâchait donc de toucher le roi d'Espagne par le souvenir de tant de
grâces qu'il lui avait faites, sans exiger de lui aucune satisfaction
pour les offenses qu'il en avait souffertes pendant huit ans. Sa
Sainteté voulait que Sa Majesté Catholique lui tînt compte d'avoir
détourné l'empereur de poursuivre ses prétentions par l'avoir engagé à
la guerre de Hongrie pendant tout le cours de laquelle il lui avait
promis qu'il ne serait point attaqué en Italie. Le pape se plaignit
amèrement de l'entreprise de Sardaigne, malgré ces engagements, du
mépris de ses représentations et de l'odieux soupçon que cette conduite
donnait aux Impériaux, qui l'accusaient d'intelligence avec l'Espagne
contre l'empereur. Une vive péroraison se termina par les plus fortes
menaces, si le roi d'Espagne ne cessait tous ses préparatifs. Le bruit
que fit l'empereur à Rome de l'accusation qu'on a vu plus haut qu'il y
avait fait porter contre Albéroni sur un prétendu traité qu'il avait
fait avec la Porte, fut vivement renouvelé\,; obligea le pape d'écrire
un bref très fort au roi d'Espagne, qui néanmoins se référait à ce que
lui dirait son nonce sur la gravité de l'affaire dont il s'agissait,
telle qu'il n'en était point arrivé qui approchât de celle-là, depuis
les dix-huit années de son pontificat, ni dont la gloire et la
conscience de Sa Majesté Catholique pussent être plus fortement
intéressées\,; ce bref plein d'autres expressions véhémentes était de la
main du pape, et devait être présenté au roi d'Espagne par Aldovrandi.
Ce nonce eut ordre de représenter en même temps à Sa Majesté Catholique
que son honneur et sa conscience exigeaient qu'il rétablît incessamment
sa réputation si horriblement attaquée, ce qu'il ne pouvait qu'en se
désistant de toute hostilité contre l'empereur, et tournant ses armes
contre les infidèles, et de menacer, en cas de refus de déférer à cet
avertissement, que Sa Sainteté ne pourrait se dispenser de prendre les
résolutions que son devoir lui suggérerait.

Ces résolutions étaient déjà méditées. Le pape, épouvanté de la colère
de l'empereur, se persuadait voir déjà les preuves de l'accusation que
ce prince avait fait porter par son ambassadeur à Rome contre Albéroni
sur son prétendu traité avec les Turcs. Ainsi le pape s'était proposé de
priver le roi d'Espagne des grâces que Rome avait accordées à lui et à
ses prédécesseurs telles que la \emph{crusade}, le
\emph{sussidio}\footnote{Ces mots, qui désignaient des impôts
  particuliers, ont été expliqués plus haut.}, et les millions
uniquement destinés à soutenir une guerre continuelle contre les
infidèles, et que Sa Sainteté, voyant le roi d'Espagne éloigné et sans
forces en Italie, ne croyait pas en conscience {[}devoir{]} laisser
subsister, pour être employés à faire une diversion à l'empereur, tandis
qu'il était occupé contre les Turcs. Le pape avait d'autres griefs
contre la cour de Madrid. Il se plaignait inutilement du trouble que
recevait en Espagne l'exercice de la juridiction ecclésiastique, et il
avait représenté avec aussi peu de succès qu'il n'appartenait pas à Sa
Majesté Catholique de disposer des revenus des églises de Tarragone et
de Vich, dont Albéroni s'était emparé, sous prétexte qu'ils étaient mal
administrés pendant l'absence de ces deux évêques rebelles, et s'était
mis peu en peine de satisfaire le pape là-dessus, persuadé que la
complaisance pour Rome est un mauvais moyen pour en obtenir les grâces
qu'on lui demande. Il sollicitait alors avec chaleur l'expédition de ses
bulles de Séville. Le pape alléguait qu'il ne voyait point de raison
pour autoriser une translation si prompte à Séville de l'évêché de
Malaga. Mais il ajoutait qu'étant à la tête du gouvernement d'Espagne,
il passait pour être l'auteur du bouleversement qui arrivait à la
prospérité des armes chrétiennes, et pour perturbateur public, accusé
publiquement d'intelligence avec la Porte, et d'être le directeur d'une
diversion qui produisait tant d'avantages à l'ennemi commun de la
chrétienté. Feignant de vouloir bien suspendre encore son jugement sur
une dénonciation si énorme, il ne pouvait pourtant la dissimuler ni
faire des grâces à celui qui était accusé jusqu'à ce qu'il en eût fait
voir la calomnie. Il revenait ensuite à ce prétendu soupçon de
l'empereur, si offensant pour Sa Sainteté, de sa prétendue intelligence
avec l'Espagne contre lui, coloré par le manquement horrible du roi
d'Espagne à sa parole sur son armement et sa destination, l'année
précédente.

Ces lamentations du pape n'eurent pas l'effet qu'il s'en était promis.
Acquaviva, au contraire, avait déclaré que, puisque Sa Sainteté n'avait
aucun égard aux instances du roi d'Espagne sur les bulles de Séville, ce
prince allait faire séquestrer les revenus des églises vacantes dans ses
États, et défendre à ses sujets de prendre aucune expédition en daterie.
À ces menaces Paulucci, principal ministre du pape, avait répondu que Sa
Sainteté espérait de la droiture du roi d'Espagne qu'il se laisserait
toucher des raisons qu'elle avait de suspendre la translation précipitée
d'Albéroni de Malaga à Séville, et que ce prince ne voudrait pas
augmenter par de nouvelles offenses l'embarras et la peine où elle se
trouvait, non seulement parce qu'il avait manqué à la parole qu'il lui
avait donnée l'année dernière, mais encore parce qu'il faisait de
nouveaux préparatifs pour continuer une guerre si pernicieuse à la
religion et à la tranquillité publique.

Le pape voulut que Paulucci écrivît à Albéroni dans le même sens, et à
peu près dans les mêmes termes qu'il avait parlé à Acquaviva. On ne
manqua pas de représenter à Albéroni ses devoirs comme créature du pape,
l'obligation où il était, par conséquent, d'employer son crédit à
travailler à la cause commune de la religion, bien loin de travailler à
la diversion des forces de l'empereur occupées contre les infidèles.
Paulucci l'excita par tout ce qu'il put de plus fort et de plus
touchant, l'assura que le pape le priait, comme bon père et comme
créateur (quel blasphème dans ces paroles romaines\,!) plein
d'affection, de penser que l'unique moyen de réparer sa réputation, et
de recevoir des marques de la reconnaissance de Sa Sainteté, était non
seulement de faire cesser ces hostilités, qui pouvaient retarder les
progrès des armes impériales, mais encore d'employer contre les
infidèles les mêmes forces que le roi d'Espagne prétendait faire agir
contre les princes chrétiens (difficilement vit-on jamais lettre si
parfaitement inepte). Comme Albéroni avait déjà reçu le plus grand
bienfait qu'il pût attendre du saint-siège, le pape, persuadé que
l'espérance fait agir les hommes plus que la reconnaissance, jugea que
le confesseur du roi d'Espagne montrerait plus d'ardeur de plaire à Sa
Sainteté, et peut-être agirait plus utilement qu'Albéroni, déjà revêtu
de la pourpre. Elle voulut donc que, le cardinal Albane écrivit au P.
Daubenton, et que, lui témoignant la confiance particulière qu'elle
avait en lui, il l'assurât qu'elle ne doutait point de sa sensibilité
pour ses peines, et qu'il ne fût plus en état que personne de faire
utilement au roi d'Espagne les représentations qui regardaient sa
conscience, trop exposée par le feu qu'il était sur le point d'allumer
en Italie, au préjudice de la religion. La lettre contenait de plus une
récapitulation de ce qui était arrivé depuis l'année précédente. Le pape
avait dicté les termes de la lettre\,; il avait employé, sous le nom de
son neveu, les expressions les plus pathétiques pour faire voir quels
étaient les devoirs du chef de l'Église en cette triste conjoncture, où
la religion (c'est le nom) et l'État ecclésiastique (c'est la chose) se
trouvaient également en danger. Il insistait sur l'obligation d'un
confesseur du roi d'Espagne, qui devait non seulement tirer Sa Sainteté
de l'affliction où elle était plongée, mais, de plus, avertir le roi
d'Espagne. Elle ne doutait pas même que ces avis n'eussent un plein
effet, puisqu'il s'agissait de faire souvenir ce prince qu'il était
assis sur un trône occupé avant lui par des rois à qui le saint-siège
(si libéral d'étendre sa puissance par des titres vains, qui ne lui
coûtent rien) avait accordé le titre de Catholiques à cause de la guerre
irréconciliable qu'ils avaient faite aux ennemis du nom de Jésus-Christ
(dont on ne voit ni commandement, ni conseil dans l'Évangile, ni dans
les apôtres, ni dans pas un endroit du Nouveau Testament. Guerre
d'ailleurs uniquement faite par Ferdinand et Isabelle pour réunir à
leurs couronnes toutes celles que les Maures occupaient dans le
continent de l'Espagne). De ces raisons, Albane tirait la conséquence
que le pape son oncle avait lieu d'espérer d'obtenir du roi d'Espagne
l'effet de l'offre que ce prince lui avait faite l'année précédente,
c'est-à-dire une suspension de guerre contre les chrétiens. Enfin,
c'était le moyen que le cardinal neveu proposait pour détruire
totalement les écrits que les ennemis du roi d'Espagne avaient imprimés
au désavantage de ce prince et de la nation espagnole. Comme les menaces
étaient jointes aux représentations, le pape, craignant de nouveaux
engagements, voulut que son nonce à Madrid usât de beaucoup de prudence
et de circonspection. Il souhaitait que le roi d'Espagne, frappé de la
crainte de voir les grâces que ses prédécesseurs avaient reçues du
saint-siège révoquées, prévint en le satisfaisant les effets du
ressentiment qu'il voulait lui faire appréhender, et comme il doutait si
les moyens qu'il employait pour faire agir Albéroni et Aubenton seraient
suffisants, il y employait encore le crédit que le duc de Parme avait
sur l'esprit de la reine d'Espagne et sur celui d'Albéroni.

On commençait à regarder en Italie ce prince comme l'auteur de la guerre
que l'Espagne méditait. Les Allemands de plus lui imputaient à crime
d'avoir contribué à la promotion d'Albéroni. Ils menaçaient de s'en
venger bientôt et facilement sur ses États, en sorte qu'ayant intérêt de
détourner l'orage qu'il voyait prêt à retomber sur lui, il paraissait un
agent très propre pour désarmer par sa persuasion le roi d'Espagne, prêt
à commencer une guerre qui ne pouvait être que fatale à l'Italie. Ses
représentations lui valurent vingt-cinq mille pistoles, que le roi
d'Espagne lui fit toucher pour mettre ses places en état de défense, et
le besoin que le pape crut avoir du P. Daubenton valut à son neveu le
gratis des bulles d'une abbaye que le régent lui avait donnée en
considération de son oncle.

Mais il eût fallu des moyens plus puissants pour adoucir le roi,
d'Espagne, ou plutôt son premier ministre, personnellement irrité du
refus de ses bulles de Séville. Albéroni voulut intéresser la nation
espagnole dans sa cause particulière, et, pour faire voir que c'était
une affaire d'État, il la fit renvoyer au conseil de Castille avec ordre
d'en dire son sentiment. Ceux, qui le composaient profitèrent d'une
occasion de signaler sans risque leur zèle pour le maintien des droits
de la couronne d'Espagne, donnèrent leurs voeux\,; et la consulte formée
sur leurs avis, très forte contre les prétentions de la cour de Rome,
fut rendue publique, et fut accompagnée d'une consultation signée de
plusieurs docteurs en théologie et en droit canon. Albéroni, comme
revêtu de ces armes, fit dépêcher un courrier à Rome pour intimer au
pape un temps fatal pour l'expédition des bulles de Séville, menaçant Sa
Sainteté que, si elle différait au delà de ce terme de les faire
expédier, le roi d'Espagne emploierait les moyens que le conseil de
Castille lui avait suggérés pour ranger la cour de Rome à son devoir.
Aldovrandi fut effrayé ou feignit de l'être de la réponse du conseil de
Castille. Il représenta donc au pape l'embarras où il se trouvait,
voyant augmenter un feu que Sa Sainteté avait intérêt d'éteindre,
surtout dans une conjoncture où elle voulait, par ses offices et par sa
médiation, tâcher de prévenir la guerre entre les princes chrétiens. Il
prévoyait qu'une rupture, même une simple froideur entre les cours de
Rome et de Madrid, l'empêcherait bientôt de traiter avec le ministre du
roi d'Espagne\,; qu'il demeurerait sans action, hors d'état d'exécuter
les ordres du pape, et par conséquent de faire valoir ses services.
Cette situation lui paraissait d'autant plus fâcheuse, que vers la fin
du mois d'avril où on était pour lors, on croyait voir quelque
disposition à un accommodement entre l'empereur et le roi d'Espagne.

\hypertarget{chapitre-ii}{%
\chapter{CHAPITRE II}\label{chapitre-ii}}

1718

~

{\textsc{Étrange caractère du roi de Sicile.}} {\textsc{- Entretien
curieux entre le secrétaire de son ambassade et Albéroni.}} {\textsc{-
Lascaris, envoyé de Sicile, malmené par Albéroni.}} {\textsc{- Plaintes
hypocrites d'Albéroni.}} {\textsc{- Il déclame contre le traité et tâche
de circonvenir le maréchal d'Huxelles.}} {\textsc{- Albéroni menace\,;
veut reculer le traité et gagner les Hollandais.}} {\textsc{- Caractère
de Beretti.}} {\textsc{- Embarras des ministres d'Espagne au dehors.}}
{\textsc{- La France et l'Angleterre communiquent ensemble le projet du
traité aux États généraux.}} {\textsc{- Conduite de Beretti.}}
{\textsc{- Son avis à Albéroni et sa jalousie contre Monteléon.}}
{\textsc{- La nation anglaise et la Hollande partagées pour et contre la
traité.}} {\textsc{- Triste prodige de conduite de la France.}}
{\textsc{- Conduite de Châteauneuf en Hollande.}} {\textsc{- Duplicité
des ministres d'Angleterre à l'égard du régent.}} {\textsc{- Hauteur de
Craggs à l'égard du ministre de Sicile.}} {\textsc{- Efforts du roi de
Sicile pour lier avec l'empereur et obtenir une archiduchesse pour le
prince de Piémont.}} {\textsc{- Conduite de la cour de Vienne.}}
{\textsc{- Artificieuse conduite des ministres anglais à l'égard du
régent.}} {\textsc{- Manèges de Penterrieder à Londres.}} {\textsc{-
L'Espagne voudrait au moins conserver la Sardaigne\,; mal servie par la
France.}} {\textsc{- L'Angleterre s'y oppose avec hauteur.}} {\textsc{-
Triste état de Monteléon.}} {\textsc{- Les ministres anglais plus
impériaux que les Impériaux mêmes.}} {\textsc{- Ministres espagnols
protestent dans toutes les cours que l'Espagne ne consent point au
traité.}} {\textsc{- Efforts de Beretti pour détourner les Hollandais
d'y souscrire.}} {\textsc{- Cris de cet ambassadeur contre la France.}}
{\textsc{- Ses plaintes.}} {\textsc{- Fâcheuse situation de la
Hollande.}} {\textsc{- Le roi d'Espagne rejette avec hauteur le projet
du traité communiqué enfin par Nancré, et se plaint amèrement.}}
{\textsc{- Conduite et avis de Cellamare.}} {\textsc{- Son attention aux
affaires de Bretagne.}}

~

L'opinion publique était fondée sur les traitements distingués et les
marques de confiance que Nancré recevait d'Albéroni\,; et comme
l'empereur avait accepté le traité, on jugeait que le roi d'Espagne ne
voudrait pas s'engager à soutenir seul la guerre contre la France et
contre les autres puissances principales de l'Europe. Toutefois les
préparatifs de guerre n'étaient point ralentis. L'Espagne pressait son
armement avec plus de chaleur que jamais\,: elle devait avoir vingt
navires de guerre, outre les brûlots et les galiotes à bombes\,; mais
les apprêts par mer et les forces par terre n'approchaient pas des
forces que le roi d'Espagne pouvait prévoir qu'il aurait à combattre\,;
car, en effet, il n'avait point d'alliés, et c'était sans fondement que
le public s'était figuré un traité entré Sa Majesté Catholique et le roi
de Sicile. Elle soupçonnait au contraire le roi de Sicile d'être
d'accord avec l'empereur, et croyait que la condition principale de leur
engagement était celle du mariage du prince de Piémont avec une
archiduchesse. Il y avait alors trois ministres piémontais à Madrid\,:
l'abbé del Maro était ambassadeur ordinaire\,; le roi son maître, peu
content de lui et se défiant du compte qu'il lui rendait, avait envoyé
Lascaris, soit pour découvrir les véritables sentiments d'Albéroni, soit
pour faire avec lui un traité secret\,; enfin, ce prince soupçonneux et
toujours en garde contre ses propres ministres, les faisait épier, l'un
et l'autre par le secrétaire de l'ambassade, nommé Corderi, et donnait
directement à ce dernier des ordres et des instructions dont la
connaissance était cachée à Lascaris comme à del Maro. Immédiatement
après l'arrivée de Lascaris à Madrid, Corderi fut chargé d'en aller
donner part à Albéroni. Ce premier ministre répondit qu'il était très
aise que cette voie lui fût ouverte pour donner au roi de Sicile des
preuves effectives d'une confiance très sincère, et pour le persuader de
l'attachement naturel qu'il avait pour la personne et pour les intérêts
de ce prince\,; il ajouta que, comme ils ne pouvaient être séparés dans
la conjoncture présente des intérêts de la couronne d'Espagne, il se
ferait un devoir d'en user à l'égard de Lascaris avec autant d'ouverture
et de confiance que les obligations de son ministère le lui pourraient
permettre. Les deux agents du roi de Sicile conçurent une merveilleuse
espérance d'une si favorable réponse.

Peu de jours après, le secrétaire Corderi retourna chez Albéroni\,; il
avait à l'instruire des intentions de son maître sur la mission de
Lascaris. Le cardinal avait demandé quelles étaient ses instructions,
afin de pouvoir traiter avec lui sur les affaires courantes, et Corderi,
ayant reçu les ordres du roi de Sicile sur cette question, lui dit que
ce prince répondait que, pour fixer les instructions qu'il donnerait à
son ministre, il était nécessaire en premier lieu qu'il fût lui-même
éclairci sur la diversité des sentiments entre la cour d'Espagne et les
cours de Franche et d'Angleterre\,; en second lieu, qu'il sût en détail
quels étaient les projets de guerre du roi d'Espagne, et surtout quels
moyens Sa Majesté Catholique avait d'en assurer le succès. Il ajoutait
que jusqu'alors le cardinal ne lui avait communiqué que des idées vagues
et générales, en sorte que ce prince était demeuré non seulement dans sa
première obscurité, mais tombé dans une autre plus grande encore
qu'auparavant, voyant la France et l'Angleterre plus déterminées que
jamais à procurer l'acceptation du projet qu'elles avaient formé pour la
paix générale. Albéroni répondit à cette espèce de reproché qu'il
s'était ouvert de reste sur les projets de l'Espagne, et soutint à
Corderi qu'il lui avait dit en détail tout ce qu'il pouvait lui confier
sur cette matière\,; souriant ensuite, il fit connaître qu'il
soupçonnait les doutes du roi de Sicile, et qu'il les regardait comme un
prétexte affecté pour colorer l'accommodement que ce prince avait fait
avec l'empereur. Corderi le nia\,: entre autres raisons qu'il employa
pour se défendre, il allégua la nomination que le roi de Sicile venait
de faire du comte de Vernon pour l'envoyer en Espagne\,: le cardinal
répondit qu'il n'avait rien à répliquer sur cette nomination\,; que
c'était toutefois une démonstration extérieure assez ordinairement
usitée en pareille conjoncture\,; qu'il avait d'ailleurs de bons avis et
réitérés par le ministère de France, qui l'avertissait particulièrement
de se garder de s'ouvrir aux ministres du roi de Sicile. Enfin,
Albéroni, se laissant aller aux mouvements de son impatience naturelle,
dit avec impétuosité que le roi de Sicile ne connaissait point d'autres
liens que ceux qui pouvaient convenir à ses intérêts, mais qu'un tel
avantage n'était pas de durée\,; que, si ce n'était pas le père, ce
serait un jour le fils qui serait obligé de supplier à genoux le roi
catholique de le secourir et de le délivrer de la tyrannie et de
l'oppression des Allemands. Corderi ne douta pas que la colère du
cardinal ne fût un prétexte pour couvrir ses desseins et pour manquer de
parole au roi de Sicile. Une telle conversation ne promettait pas à
Lascaris une audience plus favorable, et l'effet répondit aux
apparences. Il voulut représenter au cardinal les promesses qu'il avait
faites au roi de Sicile de lui communiquer ce qui se passerait dans les
négociations de la paix. Lascaris dit que son maître ne pouvait douter
qu'elle fût fort avancée, étant informé des longues conférences que
Nancré et le colonel Stanhope avaient avec le cardinal. Il répondit avec
chaleur qu'il n'était plus obligé à ses promesses, puisque le roi de
Sicile avait peut-être déjà signé son traité avec l'empereur, et que le
roi d'Espagne en avait des avis certains et positifs. Lascaris voulut en
vain combattre et détruire une opinion si injurieuse à son maître\,; il
soutint que ce prince n'avait fait aucune démarche contraire aux
derniers traités\,; qu'on ne devait donc ajouter, aucune foi à des avis
qui blessaient sa réputation. Ses répliques furent inutiles\,; Albéroni
rompit l'audience, et, se levant, dit qu'il était obligé de se rendre
auprès du roi d'Espagne. Lascaris en tira la conséquence que la, paix
était bien avancée et les intérêts de son maître sacrifiés.

Soit feinte, soit vérité, Albéroni déplorait avec ses amis la situation
où il se trouvait, la plus scabreuse, disait-il, et la plus critique
qu'il fût possible. Il se plaignait que sa fortune ne servait qu'à lui
faire passer de mauvais jours et de fâcheuses nuits\,; il voulait qu'on
le crût détrompé du monde, mais forcé d'y vivre pour se conformer et se
soumettre aux ordres de la Providence. Il était bien éloigné\,; comme
les Piémontais l'en soupçonnaient, d'entrer dans le traité de paix.
C'était sincèrement qu'il déclamait contre, et quoique le détail des
conditions secrètes n'eût pas encore été communiqué au roi d'Espagne,
Albéroni prétendait que Nancré s'était expliqué assez clairement pour ne
laisser aucune curiosité, pas même celle d'ouvrir et de lire les lettres
qu'il écrivait en France. Il protestait que le roi d'Espagne perdrait
plutôt quarante couronnes que de faire un pareil traité.

Il disait, que, si l'empereur possédait une fois les royaumes de Naples
et de Sicile, il serait maître quand il voudrait du reste de l'Italie,
et que, si jamais les garnisons espagnoles étaient, admises dans les
États de Toscane et de Parme, l'Espagne sentirait le préjudice de la
sortie des troupes qu'il faudrait tirer de chez elle sans aucune
utilité, parce que la supériorité des Allemands serait telle qu'ils
auraient envahi ces mêmes États avant que la nouvelle de leur entreprise
fût parvenue en Espagne. Ainsi, le roi d'Espagne perdrait inutilement
ses troupes et la dépense pour les transporter. Albéroni, persuadé que
le maréchal d'Huxelles n'approuvait pas un traité dont un autre que lui
avait été le promoteur et l'agent, chargea Cellamare de lui dire que le
roi d'Espagne connaissait trop son esprit, son jugement et sa probité
pour le soupçonner d'avoir parlé en cette occasion suivant sa pensée\,;
que si le maréchal convenait que la fraude et l'injustice avaient été
employées de manière à forcer Sa Majesté Catholique à s'accommoder à des
lois dures et barbares, il aurait raison\,; mais s'il disait qu'un
projet dont le fruit était d'agrandir l'empereur, et d'augmenter sa
puissance au delà de ses justes bornes, était un moyen capable d'établir
une paix solide, un tel discours répugnerait absolument au bon sens et
aux lumières de tout homme sage, instruit des affaires du monde\,; que
si Huxelles regardait cet ouvrage comme un pot-pourri, et comme une
trame de l'abbé Dubois, conforme à son génie et à sa personne, les gens
sages le croiraient\,; mais qu'ils ne se figureraient jamais qu'un homme
dont la probité et la réputation étaient suffisamment établies pût
approuver un projet préjudiciable à l'Espagne, fatal à la France,
déshonorant pour le nom du régent, en un mot, scandaleux au monde
entier, et capable d'exercer les galants discours qu'on ne manquerait
pas de tenir sur un si beau sujet.

Albéroni cependant proposa de former une assemblée pour examiner ce
projet, regardant cet expédient comme la seule voie à prendre pour ne se
pas éloigner de l'équité, et ne pas offenser la liberté des gens. Et
comme le colonel Stanhope le pressait d'entrer dans le traité, il lui
répondit seulement qu'il avait écrit en France, et qu'il en attendait
les réponses, mais qu'il s'expliquerait plus librement à d'autres. Sur
l'injustice prétendue du projet, il disait que les vues de ceux qui en
étaient les promoteurs étaient suffisamment connues\,; que le roi
d'Espagne en conserverait le souvenir, s'il était forcé à la dure
nécessité de subir la loi qu'on lui imposait\,; qu'il attendrait un
meilleur temps et des conjonctures plus favorables pour se dédommager,
et pourvoir lui-même à son indemnité. Comme il voyait les principales
puissances unies pour forcer l'Espagne à souscrire aux conditions de là
paix, il chercha l'appui de la Hollande, qui reculait à entrer dans le
traité. Il fit représenter à ceux qui passaient pour les meilleurs
républicains qu'ils devaient par honneur et par intérêt s'éloigner de
l'infamie qu'on leur proposait\,; que les Anglais, depuis quelques
années, se croyaient en droit comme en possession de partager le monde à
leur fantaisie, d'enlever les États à leurs légitimes possesseurs, et de
les distribuer à d'autres selon qu'il convenait à leurs intérêts\,; que
l'exécution de ce traité exécrable ne pouvait être que fatale à la
liberté de l'Europe, dont les Hollandais sentiraient les premiers
effets, parce que l'empereur, rejoignant la Sicile à Naples, aurait
bientôt une marine, et s'emparerait du commerce du Levant, et que les
puissances les plus éloignées se ressentiraient bientôt de l'esprit de
domination sans bornes de la maison d'Autriche, dès qu'elle se
trouverait en possession de l'Italie. Il fit espérer aux Hollandais
d'entrer dans les projets que leur compagnie des Indes occidentales lui
avait fait proposer pour le commerce de l'Amérique, et tâcha d'augmenter
leur jalousie et leur défiance des Anglais sur un article si
intéressant.

Beretti, tout occupé des intérêts du roi d'Espagne, et guère moins de se
vanter et de faire valoir jusqu'à ses moindres démarches, aurait voulu
qu'on lui sût gré à Madrid jusque de son inaction et de son silence. Il
trouvait qu'il ne recevait jamais d'ordres à temps, et véritablement
ayant à répondre à un ministre difficile, qui souvent désirait rejeter
la faute de l'obscurité de ses lettres sur l'exécution de ceux qui les
recevaient, Beretti, comme les autres ministres d'Espagne au dehors,
était souvent embarrassé du parti qu'il devait prendre autant pour
plaire à sa cour que pour le bien des affaires qui lui étaient commises.
Il se trouva dans cet embarras, lorsqu'à la fin d'avril l'ambassadeur de
France et l'envoyé d'Angleterre allèrent ensemble communiquer aux États
généraux le projet du traité de la quadruple alliance. Beretti n'avait
pas encore reçu des ordres suffisants, pour régler sa conduite\,; il
jugea qu'en cette conjoncture il ne pouvait rien faire de mieux que de
gagner du temps et d'empêcher la république de prendre aucun engagement.
Il demanda donc une conférence avec les députés des États, leur tint à
son ordinaire force verbiages, et parut content des assurances qu'il en
reçut de rapporter à leurs maîtres ce qu'il leur avait dit, et de leur
désir de conserver les bonnes grâces de l'Espagne. Beretti les trouvait
folles et générales\,; il crut agir prudemment d'avouer à Albéroni que
son inquiétude était extrême depuis que l'ambassadeur de France marchait
avec l'envoyé d'Angleterre. Il fit remarquer que cette cour gagnait la
supériorité dans le parlement, depuis qu'on savait que M. le duc
d'Orléans concourait avec elle. Qu'on avait bien prévu que les
Hollandais seraient invités d'entrer dans l'alliance\,; mais que de plus
on était persuadé que, s'ils y résistaient, ils seraient forcés d'y
souscrire. On ajoutait, disait-il, que le régent ferait une ligue avec
l'empereur\,; que, quoique la chose ne lui parût pas vraisemblable, tout
était possible, s'espaçait contre la France et le traité, et concluait
qu'en attendant qu'il reçût des ordres pour régler sa conduite, il
ferait tout son possible pour empêcher la république de s'engager. Il
supposa que ces ordres lui étaient d'autant plus nécessaires, qu'il
avait lieu de se défier des conseils que Monteléon lui donnait. Cet
ambassadeur était l'objet de sa jalousie, car, outre que Monteléon était
supérieur par son esprit et par son expérience, il avait encore paru que
le roi d'Espagne avait pour lui beaucoup de goût, et comme il était
Espagnol, il était vraisemblable que ce prince lui donnerait la
préférence pour les emplois sur un Italien, qui n'était pas né son
sujet. Ainsi Beretti profitait de toutes les occasions d'inspirer en
Espagne des soupçons sur la fidélité de Monteléon\,: la chose n'était
pas difficile, c'était faire sa cour au premier ministre de décrier
Monteléon. Beretti le représenta comme entrant dans toutes les vues de
l'Angleterre, jurant qu'elle n'avait nulle intention de favoriser
l'empereur\,; que séduit par elle, il voulait faire passer le projet de
paix comme un ouvrage avantageux au roi d'Espagne qui, par là,
remettrait le pied en Italie, et aurait des troupes dans les États de
Toscane et de Parme\,; que la cour de Vienne, qui en prévoyait les
conséquences et sentait bien les avantages que l'Espagne en retirerait,
n'eût jamais accepté le projet si elle n'avait regardé comme une
nécessité de prévenir en l'acceptant les liaisons qui se tramaient
contre elle entre la France et l'Angleterre. Ainsi Beretti, tournant en
ridicule la fausse politique de Monteléon, soutenait qu'en suivant ses
avis on faciliterait à l'empereur les moyens de tout envahir, dont déjà
son ministre triomphait.

Il paraissait en effet en Hollande une lettre de Londres de
Penterrieder, qui disait que le projet était tel que l'empereur le
pouvait jamais désirer, et que l'Angleterre enverrait vingt-six
vaisseaux dans la Méditerranée malgré l'opposition de la nation
Anglaise. En effet, bien des gens en Angleterre traversaient cette
expédition, les uns du parti contraire à la cour, les autres craignant
qu'entrant en guerre avec l'Espagne, et la Hollande résistant à se
déclarer ne profitât pour son commerce de la neutralité qu'elle
affectait de vouloir conserver pour l'Espagne, et véritablement cette
considération partageait la Hollande. Ceux qui depuis longtemps étaient
dévoués à l'Angleterre ne connaissaient que ses volontés. Les
républicains, au contraire, mettaient tous leurs soins à gagner du temps
pour éviter que leur État se mêlât d'une affaire commencée sans sa
participation par la France et l'Angleterre. Ils représentaient que les
sollicitations de ces couronnes n'étaient pas une preuve de leur
considération pour leur république, et qu'elles seraient certainement
demeurées à leur égard dans le silence si le roi d'Espagne eût souscrit
comme l'empereur au traité.

On vit alors ce qui n'aurait pas paru vraisemblable quelques années
auparavant\,: l'ambassadeur de France combattre, conjointement avec
l'envoyé d'Angleterre, pour terrasser, de concert avec le Pensionnaire
de Hollande, le parti républicain, et ramener aux volontés de
l'Angleterre ceux qui, ne regardant que l'intérêt de leur patrie et le
maintien du commerce, craignaient d'entrer en de nouveaux engagements
que la république serait obligée de soutenir par des dépenses qu'elle
était hors d'état de faire, et dont elle ne pouvait attendre pour fruit
que de nouveaux troubles et de nouveaux malheurs. Châteauneuf employait
cependant tout son crédit pour persuader ceux que lui-même avait
autrefois le plus exhortés à secouer le joug de la domination Anglaise.
Il agissait en cette occasion avec d'autant plus d'ardeur, que les
ministres d'Angleterre s'étaient déclarés hautement contre lui,
l'accusant d'être si prévenu des anciennes maximes de France, et, des
instructions que le feu roi lui avait données en l'envoyant en Hollande,
qu'il était impossible que jamais ils prissent confiance en lui.
Châteauneuf n'oublia donc rien pour détruire ces accusations, et y
réussit en partie, en forçant Widword, envoyé d'Angleterre à la Haye,
d'écrire à Stairs qu'il était content de la vigueur et de l'habileté de
l'ambassadeur de France dans la négociation présente. Les ministres du
roi d'Angleterre affectaient aussi de dire à Londres que leur maître ne
pouvait se défier de la bonne foi du régent, et qu'ils étaient persuadés
que l'union entre ces deux princes était parfaite\,: cette confiance
n'était qu'ostensible. Ils parlèrent avec moins de contrainte à La
Pérouse. Cet envoyé s'étant plaint de la manière injuste dont le roi de
Sicile était traité dans le projet d'alliance, Craggs lui demanda si ce
prince n'était entré dans nulle liaison pour détrôner le roi Georges\,;
l'étonnement, les protestations ne furent pas épargnés de la part de La
Pérouse\,; il promit de faire voir la fausseté de ces avis, si le
secrétaire d'État, à qui il parlait, voulait bien lui faire part de
quelques circonstances. Craggs lui répondit seulement qu'on avait averti
le roi Georges que le complot se tramait à Londres, qu'il n'était pas
impossible que l'avis fût sans réalité pour tirer quelque récompense, et
ne se mit pas en peine de dissiper autrement la crainte de l'envoyé de
Sicile, en sorte que ce dernier se figura que la cour de Londres
cherchait seulement un prétexte pour obliger le roi de Sicile de
révoquer, à l'occasion d'un nouveau traité, la protestation que la reine
de Sicile avait fait remettre au parlement d'Angleterre pour conserver
ses droits sur cette couronne.

Il y avait cependant encore une autre cause de mécontentement et de
jalousie entre la cour de Londres et, celle de Turin. La première
craignait les négociations du roi de Sicile à Vienne, et en traversait
le succès\,; et le roi de Sicile faisait tous ses efforts pour se lier
avec l'empereur et pour obtenir l'aînée des archiduchesses pour le
prince de Piémont\,; il offrit à l'empereur de le laisser maître des
conditions du traité\,; il avait su gagner le comte d'Althan, dont la
faveur auprès de l'empereur était grande. Il semblait que naturellement
il devait compter sur le prince Eugène\,; toutefois ce dernier s'était
déclaré contre la négociation des Savoyards. Quoi qu'il eût fait,
cependant on le soupçonnait d'avoir agi contre sa pensée, et bien des
gens croyaient qu'il souhaitait intérieurement que la négociation du roi
de Sicile réussît. Staremberg était un des ministres de l'empereur qui
s'opposait le plus fortement à ce mariage. La cour de Vienne, lente à
prendre ses résolutions, joignait à ce penchant naturel, beaucoup de
politique, non seulement à l'égard de la négociation de Savoie, mais
encore à l'égard de l'alliance négociée par l'Angleterre. L'empereur
faisait marcher l'une et l'autre du même pas, et comptait tirer de cette
lenteur un avantage considérable, car en même temps qu'il obligeait le
roi de Sicile de lui offrir la carte blanche, par le désir de ce prince
de prévenir, par un traité particulier, la conclusion de la quadruple
alliance, on en suspendait les expéditions que Schaub devait porter en
Angleterre.

Les ministres de Georges, voulant favoriser l'empereur, aiguisaient,
pour ainsi dire, le désir qu'on avait en France de voir cette
négociation incessamment finie. Ils représentaient qu'il était de la
dernière importance de conclure sans laisser à l'empereur le loisir de
changer de sentiment. Ils assument que jamais la cour de Vienne n'avait
eu plus de répugnance à aucune résolution qu'à la souscription de ce
traité. Ils protestèrent qu'ils ne pouvaient répondre de rien, si le
régent s'arrêtait à des bagatelles. Ils le pressèrent de conclure sans
perdre de temps, le moyen le plus sûr de faire échouer la négociation de
Savoie étant d'assurer la Sicile à l'empereur, sans qu'il eût besoin du
roi de Sicile. Il fallait encore pour appuyer les représentations des
Anglais faire voir que les affaires de Georges étaient en bon état. La
guerre du nord était pour lui l'affaire la plus importante, parce qu'il
était beaucoup plus sensible à ce qui regardait ses États d'Allemagne
qu'aux intérêts d'une couronne qu'intérieurement il regardait, sinon
comme usurpée, au moins comme incertaine sur sa tête, et peut-être
passagère. On eut donc soin de faire savoir au régent que le roi de
Suède était également disposé à s'accommoder avec Georges et avec le
czar, que l'animosité de la Suède tombait principalement sur les rois de
Danemark et de Prusse, mais que cette couronne était hors d'état de se
venger, faute de marine\,; que le roi d'Angleterre la tiendrait encore
en bride par une escadre avec laquelle l'amiral Norris allait passer
dans la mer Baltique. On assurait de plus que le czar avait
nouvellement, promis de ne faire point de paix séparée\,; qu'il avait
protesté qu'il n'avait pas eu la moindre pensée de marier une de ses
nièces au Prétendant, et que les bruits répandus sur ce sujet étaient
les effets des intrigues d'Erskin, son médecin. Il fallait joindre à ces
insinuations des apparences de ménagement, même de partialité pour les
intérêts du régent. Les Anglais connaissaient que la persuasion était
facile\,; ils croyaient aussi qu'il convenait à leurs intérêts de
préférer cette voie à d'autres plus dures\,; ils employèrent donc les
raisons personnelles qui pouvaient le toucher, et ne cessèrent de lui
représenter que le moment était favorable et qu'il ne devait pas le
laisser perdre. Quelquefois ils affectaient de condamner les prétentions
de la cour de Vienne\,; ils laissèrent entendre que, si cette cour après
tant de délais voulait apporter quelque changement aux conditions du
traité, le roi d'Angleterre ne le souffrirait pas. Ils savaient que ce
prince, bien sûr des intentions de l'empereur, ne s'engageait à rien. Un
jour ils assuraient que la négociation de Savoie était prête à échouer,
et que, si les Impériaux entretenaient encore les Piémontais par des
espérances vagues, ce n'était qu'artifice et dessein d'empêcher que ce
prince ne prît un parti de désespoir pendant que l'empereur avait peu de
forces en Italie. Un autre jour les Anglais faisaient entendre que la
négociation de Savoie s'avançait, et que le comte de Zinzendorff était
un des ministres qui l'appuyait le plus fermement auprès de l'empereur.

Penterrieder, de son côté, excita, étant à Londres, de nouveaux soupçons
sur cette alliance\,; il se servit du secrétaire de Modène pour entamer
une espèce de négociation avec La Pérouse à qui il fit dire que l'année
précédente, pendant que le roi d'Angleterre était en Allemagne, le comte
de Schullembourg lui avait offert, de la part du roi de Sicile, de céder
cette île à l'empereur\,; que Sunderland, Stanhope, Bernsdorff et l'abbé
Dubois étaient également instruits de cette offre. Penterrieder conclut
que les mêmes raisons qui l'année précédente engageaient ce prince à
cette cession subsistaient encore, et qu'il devait être également touché
des avantages qu'il envisageait alors et des périls où il s'exposerait,
s'il perdait l'occasion de regagner l'amitié de l'empereur.

Nonobstant ces insinuations, Penterrieder ménageait avec soin la
confiance des ministres d'Angleterre. Il était très content de les voir
persuadés que l'union et la vigueur des puissances contractantes était
le seul moyen de réduire l'Espagne à des sentiments plus modérés, et de
l'obliger à se relâcher sur les difficultés qu'elle apportait encore au
traité. Une des principales était la prétention du roi d'Espagne de
retenir la Sardaigne. Ce prince ayant demandé au régent de lui aider à
obtenir cette condition, Dubois dit à Monteléon qu'il en avait l'ordre
exprès de Son Altesse Royale, qu'elle voulait qu'il fît tous ses efforts
pour y réussir, qu'elle en avait même écrit au roi d'Angleterre, qu'il
craignait cependant que les instances qu'il ferait en exécution de ses
ordres ne fussent infructueuses. Monteléon s'étendit en représentations
sur l'excès de la puissance de l'empereur. Il les avait souvent faites
aux ministres d'Angleterre, mais ils répondaient seulement qu'ils
croyaient favoriser l'Espagne en contribuant à la paix. Monteléon
pensait de même\,; il le laissait entrevoir sans oser l'avouer. C'était
cependant un grand démérite pour lui en Espagne, et quand il faisait
entendre qu'il serait très fâché si les médiateurs, perdant toute
confiance pour l'Espagne, signaient enfin le traité entre eux, Albéroni
faisait passer cet aveu pour une preuve convaincante que Monteléon était
gagné par l'Angleterre.

Cette cour était très opposée à ce que l'Espagne exigeait de conserver
la Sardaigne. Les ministres confiaient à Penterrieder qu'ils croyaient
que le dessein d'Albéroni était non seulement d'embarrasser l'exécution
du traité par cette proposition, mais que, de plus, il voulait garder la
Sardaigne comme un entrepôt nécessaire pour les entreprises qu'il
méditait et qu'il espérait d'exécuter sur l'Italie, lorsque les temps et
les conjonctures seraient plus favorables. Ils envoyèrent au colonel
Stanhope de nouveaux ordres de renouveler ses instances auprès du roi
d'Espagne pour l'engager à faire cesser ses préparatifs pour la
campagne. L'objet des Anglais, de concert avec le ministre de
l'empereur, était de procurer à l'escadre Anglaise le loisir d'arriver
dans la Méditerranée avant que les Espagnols eussent le temps de
commettre aucune hostilité. Ils promirent donc à Penterrieder de
concerter avec lui les instructions qui seraient données au commandant
de cette escadre, et comme Penterrieder témoignait quelque inquiétude
des changements qu'on avait faits à Vienne à quelques expressions dans
les actes dressés en conséquence du traité, ils l'assurèrent que le
régent ne s'arrêterait pas à de simples formalités, l'empereur, en sa
considération, ayant passé avec tant de générosité sur l'essentiel des
points qui lui devaient paraître, si durs après qu'on s'était, sitôt
écarté du premier plan d'Hanovre.

Les difficultés de la part de l'empereur, augmentaient à proportion des
facilités que la cour d'Angleterre trouvait en France. Les ministres
d'Espagne dans les cours étrangères avaient ordre de se tenir sur leurs
gardes. Ils s'avertissaient mutuellement, et déclaraient en même temps à
ceux des princes d'Italie qui se trouvaient dans les mêmes cours qu'il
était absolument faux que le roi leur maître eût accepté comme on le
publiait le plan du traité, et que ce prince, convenant du projet
général, ne se rendît difficile que sur les conditions plus ou moins
avantageuses. Ils agissaient conformément à cette déclaration\,; car en
Hollande Beretti travaillait ouvertement à détourner les États
d'acquiescer à la proposition que les ministres de France et
d'Angleterre faisaient à la république d'admettre l'empereur dans la
triple alliance conclue l'année précédente. Après avoir exagéré
l'horreur de voir la France, oubliant ce qu'elle avait fait pour placer
un prince de la maison royale sur le trône d'Espagne, servir
actuellement de lien entre l'empereur et le roi d'Angleterre pour faire
la guerre à ce même prince, sorti du sang de ses rois, Beretti
conseillait aux principaux ministres de la république d'éluder au moins
les instances pressantes des puissances alliées s'ils ne se sentaient
pas assez forts, et peut-être assez fermes pour les rejeter ouvertement.
Il proposa donc au Pensionnaire comme un moyen de gagner du temps de
répondre que ses maîtres avant de prendre un parti décisif, voulaient
aussi faire des représentations au roi d'Espagne, et qu'ils enverraient
un ministre à Madrid pour essayer de résoudre Sa Majesté Catholique de
se rendre plus facile aux conditions qui lui étaient offertes. Beretti
croyait que, si cet expédient réussissait, il serait utile aux intérêts
du roi son maître d'avoir, avant que de se déterminer, un temps aussi
considérable qu'il le désirerait, puisqu'il serait maître de retarder
autant qu'il lui plairait la réponse qu'il aurait promise. Dans cette
vue Beretti s'attacha principalement à faire nommer un ambassadeur pour
Madrid. Il représenta que le roi son maître prendrait plus de confiance
en un seul Hollandais qu'en cinq cents ministres Anglais unis ensemble,
et pour ne rien omettre de ce qui pouvait animer la jalousie des deux
nations, il eut soin de rappeler le souvenir du traité que le comte de
Stanhope étant à Barcelone avait fait avec l'empereur, et dont les
conditions faisaient voir combien les Anglais étaient attentifs à
profiter de toutes les occasions favorables qu'ils croyaient avoir
d'obtenir quelque avantage pour leur commerce au préjudice de celui des
Hollandais. On dit que, partant pour Amsterdam, il porta ce traité,
comptant s'en servir comme d'une pièce excellente pour faire voir à
cette puissante ville, si jalouse du commerce qui est la base de sa
grandeur, ce qu'elle avait à craindre en tout temps de la part des
Anglais, ses rivaux irréconciliables. C'était le temps où elle donne des
instructions aux députés qu'elle a coutume d'envoyer aux états de la
province\,: ainsi Beretti regardait comme un point capital de prévenir
en faveur du roi d'Espagne une ville qui donne la règle et le mouvement
à la Hollande, comme la Hollande le donne aux six autres provinces de
l'Union.

Malgré ces diligences qu'il eut grand soin de faire valoir en Espagne,
il avoua cependant qu'il ne pouvait espérer rien de bon depuis que la
France et l'Angleterre, unies contre le roi d'Espagne, travaillaient et
réussissaient à réunir les deux partis de cette république, opposés l'un
à l'autre depuis tant d'années. Il semblait que cet ambassadeur n'eût de
ressourcé que de se plaindre comme d'une chose qui faisait, disait-il,
mal au coeur de voir l'ambassadeur de France aller de porte en porte
avec le ministre d'Angleterre, solliciter les députés aux États généraux
d'accepter un traité uniquement avantageux à l'empereur, et que ce
prince affectait de regarder avec indifférence. Toute vigueur semblait
éteinte dans la république, parce qu'elle était en effet dans une
situation très fâcheuse. La dernière guerre avait épuisé ses finances.
Pendant son cours les Anglais, dominant en Hollande, avaient profité de
la conjoncture pour usurper sur les Hollandais beaucoup d'avantages dans
le commerce, qu'ils avaient conservés après la paix. La sûreté que les
Provinces-Unies crurent trouver par leur Barrière en exigeant de la
France et de l'Espagne de laisser les Pays-Bas à l'empereur, les
assujettissait à dépendre des Impériaux, en sorte que cette république
dont les résolutions étaient autrefois d'un si grand poids dans les
affaires de l'Europe, paraissait réduite à suivre encore longtemps les
mouvements de l'Angleterre, et à recevoir la loi d'elle et de
l'empereur. Toutefois les ministres Anglais trouvaient plus de
difficulté qu'ils ne se l'étaient figuré à persuader les provinces,
surtout celle de Hollande, et particulièrement les villes d'Amsterdam et
de Rotterdam, d'entrer dans le traité de la quadruple alliance. Elles
espéraient que, si l'Angleterre rompait enfin avec l'Espagne, elles
profiteraient de cette rupture pour faire ensuite plus avantageusement
le commerce d'Espagne et des Indes. Elles craignaient en même temps de
perdre ce commerce si nécessaire, si la république prenait des liaisons,
et si elle entrait dans un projet désagréable au roi catholique. La
province de Frise, et ensuite celle de Gueldre, moins touchées de
l'intérêt du commerce, et plus accoutumé à suivre et à seconder les vues
des Anglais, résolurent les premières d'entrer dans le traité.

Si cette démarche donna de nouvelles espérances aux ministres
d'Angleterre, elle n'ébranla pas le roi d'Espagne. Le nombre des
puissances prêtes à signer l'alliance augmentait. Il se formait, par
conséquent, autant d'ennemis nouveaux prêts à se déclarer contre
l'Espagne, sous prétexte qu'elle seule s'opposait au bien commun de
l'Europe, en s'opposant à la paix générale. Nonobstant le péril dont le
roi catholique paraissait menacé, il rejeta avec hauteur le projet
entier du traité que Nancré avait eu enfin ordre de lui confier.
Plusieurs conditions de ce projet furent traitées, sous le nom du roi et
de la reine d'Espagne, de propositions violentes, injustes,
impraticables et pernicieuses. On eut soin de répandre que Leurs
Majestés Catholiques en avaient été scandalisées et irritées. Cellamare
eut ordre non seulement de s'en plaindre, mais il lui fut enjoint en
termes exprès de jeter les hauts cris aussi bien sur les propositions
que sur la manière artificieuse dont elles avaient été faites. Il
exécuta sans peine un tel ordre, et ne se contraignit pas en déclamant
contre les erreurs du gouvernement. Toutefois il crut apercevoir au
travers de tout le fief dont les lettres de la cour d'Espagne étaient
pleines, qu'elle ne s'éloignerait pas d'avaler la pilule, si elle était,
disait-il, mieux dorée et présentée en forme plus civile\,; mais quelque
parti que cette cour voulût prendre, Cellamare conseillait de ne pas se
relâcher sur les préparatifs de la guerre et de la marine, persuadé que
le moyen le plus sûr de réussir en toute négociation était de traiter
les armes à la main.

\hypertarget{chapitre-iii.}{%
\chapter{CHAPITRE III.}\label{chapitre-iii.}}

1718

~

{\textsc{La Sardaigne en achoppement à la paix.}} {\textsc{- Attention
de Cellamare aux affaires de Bretagne.}} {\textsc{- Adresse de l'avis de
Monteléon à Albéroni.}} {\textsc{- Manège du roi de Sicile.}} {\textsc{-
Penterrieder en profite.}} {\textsc{- Bassesse du roi de Sicile pour
l'Angleterre, qui le méprise et qui veut procurer la Sicile à
l'empereur.}} {\textsc{- Sage avis de Monteléon.}} {\textsc{- Erreur de
Beretti.}} {\textsc{- Cadogan le désabuse.}} {\textsc{- Intérêt
personnel de l'abbé Dubois.}} {\textsc{- Plaintes malignes des
Piémontais.}} {\textsc{- Cellamare déclare, tant qu'il peut, que
l'Espagne n'acceptera point le projet de traité.}} {\textsc{- Beretti et
Cadogan vont, l'un après l'autre, travailler à Amsterdam pour mettre
cette ville dans leurs intérêts contraires.}} {\textsc{- Nancré rend le
roi de Sicile suspect à l'empereur.}} {\textsc{- Albéroni raisonne
sainement sur la Sicile et sur le roi Georges\,; très malignement sur le
régent\,; artificieusement sur le roi de Sicile\,; déclame contre le
traité, contre lequel il fait faire partout les déclarations les plus
fortes\,; presse les préparatifs.}} {\textsc{- Secret impénétrable sur
la destination de son entreprise.}} {\textsc{- Continue à bien traiter
Nancré et à conférer avec lui et avec le colonel Stanhope.}} {\textsc{-
Le colonel Stanhope pense juste sur l'opiniâtreté d'Albéroni.}}
{\textsc{- Réponse de ce cardinal à une lettre du comte Stanhope, qui le
pressait d'accepter le traité.}} {\textsc{- Plaintes et vanteries
d'Albéroni.}} {\textsc{- Forces actuelles de l'Espagne.}} {\textsc{-
Crédit de ce premier ministre sur Sa Majesté Catholique.}} {\textsc{-
Albéroni menace Gallas, les Allemands et le pape.}} {\textsc{- Vanteries
de ce cardinal.}} {\textsc{- Vaines espérances de Giudice qui
s'indispose contre Cellamare.}} {\textsc{- Bassesses de ce neveu.}}
{\textsc{- Chimères attribuées à Giudice, qui font du bruit et du mal à
Madrid.}} {\textsc{- Il les désavoue et déclame contre les chimères et
le gouvernement d'Albéroni.}} {\textsc{- Fausse et basse politique du
pape.}} {\textsc{- Cellamare se fait bassement, gratuitement et mal à
propos l'apologiste d'Albéroni à Rome.}} {\textsc{- Il en reçoit de
justes reproches de son oncle.}} {\textsc{- Esprit de la pour de
Vienne.}}

~

On crut que le régent était embarrassé du refus du roi d'Espagne, et que
Son Altesse Royale s'était flattée que la reine d'Espagne aurait engagé
le roi son mari à signer un traité qui assurait aux enfants de cette
princesse la succession de deux États considérables en Italie. Il y
avait encore une voie pour satisfaire le roi catholique, c'était de lui
conserver la possession de la Sardaigne\,; mais la chose ne pouvait se
faire qu'au préjudice du duc de Savoie, à qui ce royaume était destiné
en dédommagement de celui de Sicile. Le régent dépêcha cependant un
courrier à Londres, portant ordre à l'abbé Dubois de le proposer au roi
d'Angleterre. Cellamare comptait que ce changement au traité apaiserait
le roi son maître et l'engagerait à signer. Il avertit Monteléon de
travailler sous main et sans paraître à faciliter le succès de cette
prétention nouvelle, sûr que, si elle ne réussissait pas, la signature
était inévitable. Peut-être la craignait-il\,; mais la prévoyant, il
donnait une attention très particulière à ce qui se passait en Bretagne,
et ne manquait pas d'avertir que, les affaires s'aigrissant, les
mouvements de cette province devenaient chaque jour plus considérables.
Le roi d'Angleterre ne goûta pas la proposition de laisser la Sardaigne
à l'Espagne\,; il jugea qu'un tel changement au projet de traité
exciterait non seulement de nouvelles disputes, mais produirait
peut-être des difficultés insurmontables. L'empereur voulait la Sicile à
quelque prix que ce fût. Georges voulait le satisfaire, et ne trouvait
déjà que trop de peines à réduire le duc de Savoie, sans lés augmenter
encore en rétractant l'offre de l'équivalent proposé à ce prince pour la
cession de la Sicile. Ainsi le courrier du régent étant arrivé à
Londres, le roi d'Angleterre tint pour la forme seulement deux conseils,
comme pour délibérer sur cette proposition nouvelle. Il y fut décidé
qu'il ne convenait pas d'altérer la substance du projet accepté par
l'une des parties\,; que ce serait s'exposer à des disputes inutiles
avec la cour de Vienne\,; qu'on pouvait même regarder ces contestations
comme dangereuses, après avoir eu tant de peine d'engager l'empereur à
consentir au projet.

Les ministres d'Angleterre instruisirent Monteléon de cette
délibération. Il avait bien jugé que la demande de retenir la Sardaigne
ne réussirait pas, mais il n'avait osé s'expliquer sur une proposition
dont le roi son maître désirait le succès, et que le premier ministre
avait particulièrement à coeur, parce que la Sardaigne était l'unique
fruit de tant de dépenses qu'il avait fait faire à l'Espagne. Il
fallait, pour combattre l'opinion du prince et du ministre, faire
semblant d'y acquiescer, leur en exposer toutefois les inconvénients
d'une manière si palpable qu'ils reconnussent clairement par eux-mêmes
ce que l'ambassadeur n'osait dire, de peur de s'exposer à déplaire.
C'est ce que Monteléon avait souvent pratiqué, mais le succès n'avait
pas répondu à ses intentions, non plus qu'à ses ménagements. Il avertit
Albéroni en cette dernière occasion que La Pérouse lui avait dit, après
l'arrivée d'un courrier dépêché de Turin, que le roi son maître ne se
laisserait pas dépouiller de son royaume, sans faire auparavant, pour le
conserver, tous les efforts que son honneur et ses droits demandaient.
Monteléon, donnant cet avis au cardinal, lui laissait en même temps
espérer qu'une résolution si ferme pourrait déconcerter l'exécution d'un
projet odieux au roi d'Espagne\,; mais après avoir fait entrevoir ce
rayon d'espérance, il essaya de le détruire lui-même en représentant
qu'il n'était pas permis de prendre confiance en la sincérité du roi de
Sicile, non seulement par la connaissance que tout le monde avait du
caractère de ce prince, mais encore parce que dans le temps même qu'il
se récriait si fort contre les dispositions du projet, il tenait à
Vienne un ministre caché, et sollicitait fortement l'empereur d'accorder
la seconde archiduchesse sa nièce en mariage au prince de Piémont.
Monteléon pouvait encore ajouter que Penterrieder continuait
d'entretenir une espèce de négociation à Londres avec La Pérouse, et
soit sincérité, soit dessein de l'amuser, Penterrieder l'assurait que,
si l'empereur avait voulu consentir à laisser la Sardaigne au roi
d'Espagne, Sa Majesté Catholique aurait sans hésiter promis d'unir ses
armes aux armes impériales pour enlever la Sicile au duc de Savoie, et
la donner à l'empereur. Penterrieder, faisant valoir ici l'équité de son
maître, et son attention aux intérêts du roi de Sicile, conclut que le
mieux pour l'un et pour l'autre serait de s'accommoder ensemble sans
l'intervention de la France ni de l'Angleterre.

Le roi de Sicile, attentif à ses intérêts et toujours agissant dans
cette vue, ne se reposait pas uniquement sur le succès incertain de la
négociation secrète qu'il avait entamée à Vienne. Il écrivit donc au roi
d'Angleterre pour lui demander pressement que le projet du traité lui
fût communiqué, n'ayant d'autre intention que de concourir et de
procurer la tranquillité publique autant qu'il serait en son pouvoir. Il
ajouta qu'il était persuadé que le principal fondement de ce projet,
était l'observation des traités d'Utrecht et leur garantie\,; qu'il
avait d'autant plus de raison de le croire que jamais il ne s'était
écarté de la volonté et des intentions de l'Angleterre, les ayant
toujours aveuglément suivies\,; qu'il protestait aussi que cette maxime
serait toujours la règle inviolable de sa conduite. Cette lettre demeura
longtemps sans réponse.

Monteléon fit usage de la connaissance qu'il en eut pour convaincre
encore le cardinal Albéroni, et du peu de fond qu'on devait faire sur le
roi de Sicile qui agissait si différemment de tous côtés, et de
l'opiniâtreté de la cour d'Angleterre à conserver toutes les conditions
du projet sans y faire le moindre changement\,; et comme il aurait
désiré sur toutes choses que le roi d'Espagne fût entré dans le traité
d'alliance, n'osant le dire ouvertement de peur de déplaire, il ne
perdit pas cette nouvelle occasion de représenter que, si le roi son
maître était contraint de céder à la dure nécessité du temps, et des
conjonctures, il était au moins à souhaiter qu'en s'y soumettant, il le
fit avec le moins de préjudice qu'il serait possible pour le présent, et
avec des dispositions favorables pour l'avenir. Monteléon était persuadé
qu'il était impossible de changer dans le moment présent aucune
condition d'une convention acceptée et signée par l'empereur\,; que, si
on pouvait espérer quelque modification, ce ne serait tout au plus que
dans la suite, par les offices qu'on emploierait avant son exécution, ou
plus certainement encore par les offres qu'on pourrait faire et les
sommes qu'on distribuerait à Vienne pour arracher le consentement de
cette cour. Il regrettait le temps qu'on avait perdu, et soutenait que,
si les ministres d'Espagne étaient entrés dans la négociation au moment
qu'elle avait commencé avec les ministres d'Angleterre et l'abbé Dubois,
le roi d'Espagne aurait peut-être obtenu ce qu'il désirait, et fait
changer en mieux les conditions du traité. Mais le nuage s'était formé
de manière qu'il n'était plus possible de le dissiper et d'espérer de
gagner au moins du temps\,; seule ressource qui aurait pu rendre
meilleure la condition de l'Espagne. Il ne comptait nullement sur
l'effet des offices que le régent avait promis d'interposer à Londres et
à Vienne, pour obtenir des modifications au traité telles que le roi
d'Espagne eût lieu d'être satisfait.

Beretti s'était flatté que de pareils offices seraient d'un grand poids,
et que la cour de Vienne, ayant tant de raisons particulières de marquer
sa considération pour le régent, ne pourrait se dispenser de déférer à
ses instances. Cadogan, nouvellement arrivé de Londres à la Haye, dit
avec beaucoup de franchise à Beretti qu'il devait se désabuser d'une
espérance si vaine\,; que, si le régent faisait quelque représentation,
il ne la ferait que pour la forme, pour sauver un reste d'honneur, mais
sans insister\,; qu'il ne le pouvait étant totalement engagé. Cadogan
poussant plus loin la confidence (c'est-à-dire le mépris de l'Espagne
livrée par la France, gouvernée et muselée par l'abbé Dubois qui ne
songeait qu'à son chapeau qu'il ne pouvait obtenir que par l'autorité de
l'empereur sur le pape, et par la recommandation forte du roi
d'Angleterre auprès de l'empereur), dit encore à cet ambassadeur
d'Espagne que l'Angleterre n'avait nul penchant pour le roi de Sicile,
parce que le souvenir des manèges qu'il avait faits pendant les guerres
passées était toujours présent\,; que, de plus, on savait à Londres que
ce prince avait à Madrid un ministre caché, dans le même temps qu'il
négociait à Vienne. Si les Anglais regardaient le roi de Sicile comme un
prince dont la foi devait toujours être suspecte, les Piémontais se
plaignaient réciproquement du régent et du roi d'Angleterre. Ils
disaient que Son Altesse Royale, de concert avec Stairs, jouait
également le roi d'Espagne et le roi de Sicile\,; qu'on faisait entendre
au roi d'Espagne, pour le porter à l'acceptation du traité, que le roi
de Sicile était près de faire son accommodement avec l'empereur\,; qu'on
disait en même temps au roi de Sicile que le roi d'Espagne accepterait
le plan, si les demandés qu'il faisait au préjudice de la maison de
Savoie lui étaient accordées.

Dans cette situation, Provane qui était encore à Paris, sous prétexte de
travailler au règlement des limites, se lia plus étroitement que jamais
avec Cellamare. Il l'assura que la répugnance que son maître avait à
souscrire au projet était invincible, et Cellamare ne manqua pas de le
fortifier dans ces sentiments. Ils étaient conformes aux intentions du
roi d'Espagne, car nouvellement encore il avait ordonné à cet
ambassadeur de déclarer qu'il trouvait le plan injuste et détestable\,;
que, si jamais il y souscrivait, ce ne serait jamais que forcé par la
violence et par la fatalité malheureuse d'être abandonné de tout le
monde. Cellamare fit voir à Provane et à beaucoup d'autres les ordres
qu'il avait reçus. Il crut d'autant plus nécessaire de s'en expliquer
qu'on répandait à Paris et à Londres que le roi d'Espagne consentait au
traité, en y changeant seulement quelques conditions. On donnait aux
nouvelles propositions que le roi d'Espagne avait faites le nom
d'acceptation limitée, et comme le régent avait envoyé à Nancré de
nouveaux ordres de presser le roi d'Espagne, plus que jamais, d'accepter
le projet, son ambassadeur à Paris, incertain du succès que ces
nouvelles instances pourraient avoir, croyait dans cet intervalle être
obligé de, rassurer ceux qui désiraient que le roi d'Espagne voulût
persister avec fermeté dans ses premières résolutions.

Beretti en usait de même en Hollande. Il fit un voyage à Amsterdam, où
il eut des conférences avec les deux pensionnaires Buys et Bassecourt,
et les bourgmestres Tropp, Pautras et Sautin. Outre les raisons pour les
empêcher d'accéder au traité, il employa les promesses\,; celles qui
regardaient le commerce firent assez d'impression pour empêcher la
régence de cette ville de prendre aucune résolution. Heureusement pour
Beretti, l'ambassadeur de France n'avait point reçu d'ordre depuis que
le courrier que le régent avait dépêché à Madrid était de retour à
Paris. Son silence favorisa les discours de l'ambassadeur d'Espagne. Les
ministres d'Angleterre s'en plaignirent, et Cadogan se crut obligé
d'aller à Amsterdam réparer le mal que Beretti y avait causé. Ce dernier
craignait Cadogan, persuadé que le roi d'Angleterre avait remis entre
ses mains des sommes très considérables pour gagner des suffrages en
Hollande. D'ailleurs il le regardait moins comme Anglais que comme
ministre de l'empereur, dont il avait la patente de feld-maréchal.

Les nouvelles représentations que Nancré fit en Espagne ne produisirent
pas plus d'impression que celles qu'il avait faites jusqu'alors. Il y
ajouta cependant de nouvelles raisons capables de rendre les intentions
du roi de Sicile très suspectes. Il avertit Albéroni qu'aussitôt que ce
prince avait appris que la France et l'Angleterre offraient la Sicile à
l'empereur, il avait dépêché à Vienne, pour l'offrir aussi, mais à
condition que la complaisance qu'il témoignait en cette occasion pour
l'empereur faciliterait le mariage du prince de Piémont avec l'une des
archiduchesses. Nancré dit de plus que l'offre n'était pas nouvelle\,;
que le même duc de Savoie, qui la renouvelait aujourd'hui, l'avait déjà
faite peu de temps avant la mort du feu roi\,; que d'autres difficultés
avaient empêché la conclusion du traité qu'il sollicitait à Vienne.

Albéroni était persuadé que l'empereur désirait ardemment la Sicile, et
que, depuis la paix d'Utrecht, il n'avait pensé qu'aux moyens de
l'acquérir pour s'assurer la conservation du royaume de Naples. Les
forces de mer étoient les seules qui manquaient à ce prince\,; ces deux
royaumes entre ses mains lui donnaient moyen d'avoir des forces
considérables dans la Méditerranée. Albéroni se vantait d'avoir jugé si
sainement des vues de la cour de Vienne, qu'il avait parié, dès qu'il
fut question du projet, que l'empereur l'accepterait. Il ne s'étonnait
pas, disait-il, que le roi Georges eut voulu faire un tel présent à la
maison d'Autriche, parce qu'étant Allemand, et voulant conserver
l'injuste acquisition de Brême et de Verden, il devait, pour y réussir,
acquérir par une autre injustice les bonnes grâces du chef de l'empire.
C'était par cette raison que le roi d'Angleterre, suivant le
raisonnement (en cela très juste) d'Albéroni, travaillait à
l'augmentation d'une puissance que les Français et les Anglais
trouvaient déjà trop grande, et qu'ils convenaient mutuellement qu'il
faudrait abaisser dans son temps. Toutefois il paraissait que la cour
d'Angleterre n'avait en vue que d'être invitée par l'empereur de rompre
avec l'Espagne. La preuve évidente de ce dessein était, selon le
cardinal, la résolution prise à Londres d'envoyer une escadre dans la
Méditerranée, le tout pour l'intérêt particulier du roi Georges.
Albéroni affectait de répandre que ces raisons secrètes et personnelles
avaient beaucoup plus de part aux changements projetés dans l'Europe que
les raisons d'État, et c'était à cette cause unique qu'il attribuait la
résolution surprenante que la France avait prise de concourir à
l'agrandissement de la maison d'Autriche. Quelque mauvaise opinion qu'il
eut du duc de Savoie\,; il voulut paraître invincible aux nouveaux
soupçons que Nancré essaya de lui inspirer des intentions et de la
conduite de ce prince. Il ne les rejeta pas entièrement, mais il dit que
le duc de Savoie le faisait assurer que la seule négociation qu'il eût à
Vienne était bornée au mariage du prince de Piémont, et que cette cour
elle-même lui avait offert une archiduchesse\,; qu'il déclarait en même
temps que jamais il ne consentirait à céder la Sicile, et qu'il priait
instamment le roi d'Espagne de s'y opposer. Le cardinal demanda
l'explication d'un pareil galimatias, qui ne pouvait servir qu'à couvrir
beaucoup de tromperies et de mauvaise foi\,; car en même temps qu'on
voulait persuader au roi d'Espagne que le duc de Savoie offrait
volontairement la Sicile, ce même prince conjurait Sa Majesté Catholique
de refuser son consentement à une condition si dure. On voulait donc,
disait Albéroni, tromper le roi d'Espagne, et le traiter comme un
enfant\,; on lui montrait de loin une babiole, et s'il ne l'acceptait
pas, on le menaçait de lui déclarer la guerre\,; mais il assurait que ce
prince était résolu de prendre patience, de ne céder que, dans le cas
d'une nécessité indispensable, et de se livrer aux partis les plus
extrêmes avant que d'entrer dans un projet, non seulement imaginaire,
mais dont l'exécution serait injuste, puisque les princes à qui on
désignait, malgré eux, des successeurs, déclaraient hautement qu'ils ne
consentiraient jamais à laisser entrer, tant qu'ils vivraient, des
garnisons espagnoles dans leurs places. Cette condition, étant une de
celles qu'on offrait au roi d'Espagne comme une sûreté de l'exécution du
traité, elle donnait aussi lieu à Albéroni de s'écrier que ce plan était
un pot-pourri infâme, qui disposait contre toutes les règles et
tyranniquement des biens et de l'état des souverains\,; que les Anglais
voulaient être les maîtres du monde pour le partager à leur fantaisie,
et que cette malheureuse France, concourant à des maximes si impies,
aidant elle-même à se forger des fers, oubliant ses maximes
fondamentales, rejetait absolument les résolutions qu'elle avait
constamment suivies jusqu'alors de réprimer la barbarie allemande et
l'insolence des Anglais.

Les ministres d'Espagne eurent ordre de s'expliquer à peu près dans les
mêmes termes en France et en Angleterre. Beretti devait parler de même
en Hollande, et déclarer au Pensionnaire, que, si le roi d'Espagne avait
à mourir, qu'il ne mourrait que l'épée à la main, et qu'il ne céderait
qu'à la dernière extrémité\,; qu'enfin Sa Majesté Catholique ferait
connaître que, si elle avait reçu la loi en souscrivant au traité
d'Utrecht, elle se l'était elle-même imposée par sa déférence
respectueuse pour les conseils du roi son grand-père. Beretti eut ordre
d'ajouter que, si la république de Hollande entrait dans un complot
aussi indigne que celui qu'on avait tramé, il dépendait d'elle de le
faire, mais qu'elle pouvait s'assurer que jamais le roi son maître
n'oublierait une telle injure. Les ministres d'Espagne eurent en même
temps soin de faire connaître que jamais le roi d'Espagne n'avait promis
de suspendre l'exécution des projets qu'il méditait. En effet on
pressait plus que jamais l'armement de la flotte, et vers le
commencement de mai, on disait à Madrid qu'elle serait prête à mettre à
la voile le 20 du même mois. Bien des gens croyaient le débarquement
destiné pour Naples, persuadés que le roi d'Espagne avait un parti
puissant dans ce royaume\,; d'autres assuraient que la reine d'Espagne,
en particulier, souhaitait qu'on introduisît des garnisons dans les
places du grand-duc et du duc de Parme. Il est certain que le secret
avait été gardé très exactement, et que les agents du roi de Sicile,
malgré leur activité, ne découvraient encore que ce que le public savait
du nombre et de la qualité des troupes qu'on faisait embarquer\,; mais
ils ignoraient absolument le but de l'entreprise, et se trompaient comme
les autres dans leurs conjectures.

Albéroni continuait d'avoir beaucoup d'égards pour Nancré. Ils avaient
souvent de longues conférences. Le colonel Stanhope était introduit à
quelques-unes. Il en avait aussi de particulières avec le cardinal. Les
courriers dépêchés continuellement de Paris à Madrid, et de Madrid à
Paris, donnaient lieu de croire que la France et l'Espagne agissaient de
concert\,; que, si ce n'était pour l'exécution du traité, ce serait pour
la guerre. Les ministres anglais, bien instruits de la manière dont le
régent pensait, ne témoignaient nulle jalousie de ses négociations à
Madrid\,; mais le colonel Stanhope était persuadé que ni les instances
des François ni les siennes n'apporteraient de changement à la
résolution que le roi d'Espagne avait prise de faire la guerre. Il remit
au cardinal une lettre qu'il avait reçue pour lui du comte de Stanhope,
son cousin, contenant de nouvelles instances pour l'acceptation du
projet. Albéroni y répondit dans les termes suivants\,:

«\,Si les prémisses que Votre Excellence établit dans sa lettre du 29 du
passé étaient vraies, les conséquences seraient infaillibles\,; mais il
est question que \emph{laboramus in principiis}. Enfin le roi catholique
est malheureux, puisque après avoir donné les dernières marques d'amitié
au roi de la Grande-Bretagne, et de sa bienveillance à la nation
anglaise, non seulement il ne peut tirer de l'un et de l'autre une juste
reconnaissance\,; mais l'état même, d'indifférence lui sera refusé. Je
me rapporte à tout ce que le marquis de Monteléon lui dira là-dessus de
ma part.\,»

Albéroni se récriait souvent sur l'ingratitude des Anglais\,; il voulait
faire, croire qu'il recevait souvent des reproches du roi et de la reine
d'Espagne, de la vivacité qu'il avait témoignée lorsqu'il avait été
question de conclure les deux derniers traités avec le roi Georges. Il
prétendait que Leurs Majestés Catholiques lui répétaient fréquemment
qu'il s'était laissé trop facilement séduire par les promesses des
Anglais. Il se consolait par l'espérance de faire bientôt éclater aux
yeux du monde la puissance où l'Espagne s'était élevée depuis le peu de
temps qu'il la gouvernait. On était à la veille de voir dans la
Méditerranée trois cents voiles sous pavillon d'Espagne, trente-trois
mille hommes de débarquement, cent pièces de canon de vingt-quatre,
vingt autres de campagne, vingt mille quintaux de poudre, cent mille
boulets, trois cent soixante-six mille outils à remuer la terre, des
bombes et des grenades à proportion. Il s'applaudissait en songeant
qu'on verrait en peu d'histoires un débarquement de trente-trois mille
hommes avec un train semblable, particulièrement six mille chevaux. Il
se flattait d'être absolument maître de ces troupes, parce qu'elles
avaient été payées avec profusion, et parce qu'il avait avancé plusieurs
officiers de mérite. Le trésor pour l'armée et pour la flotte montait à
un million et demi d'écus. Indépendamment de cette somme, Albéroni avait
encore fait remettre à Gênes vingt-cinq mille pistoles pour le duc de
Parme.

Tant de dispositions faites dans un temps où l'Espagne n'avait encore
donné nulle marque de sa nouvelle puissance, étaient pour son ministre
autant de sujets de croire que par son travail et par son industrie, en
élevant son maître, il s'était lui-même mis au-dessus de ses ennemis
personnels\,; qu'il n'avait rien à craindre de leurs traits\,; qu'en
vain ils s'efforçaient de le noircir, d'employer la calomnie pour le
rendre odieux, soit à l'Espagne, soit au duc de Parme\,; qu'ils ne
réussiraient pas à détruire le crédit et la réputation, que son mérite
confirmé par ses grands services lui avait acquis. Le roi et la reine
d'Espagne, dont il possédait alors la faveur et la confiance,
l'entretenaient dans la bonne opinion qu'il avait plus que personne et
de ses talents et de l'étendue de son génie. Comme il était maître
d'employer comme il voulait le nom de Leurs Majestés Catholiques, il ne
manqua pas de dire qu'elles avaient regardé avec autant d'indignité que
de mépris le libelle infâme divulgué contre lui par l'ambassadeur de
l'empereur à la cour de Rome. Albéroni promit de se venger du perfide
ministre de la cour de Vienne, accoutumé, disait-il, à se servir
d'impostures, et de faire la guerre aux Allemands de manière que cette
barbare nation s'en sentirait longtemps.

Il ne menaçait pas moins le pape que, l'empereur, quoique ce fût en
termes plus doux. Il déplorait le peu de courage que le chef de l'Église
montrait lorsqu'il s'agissait de défendre la religion. Albéroni, plein
de zèle, gémissait de voir les Allemands profiter de la faiblesse du
saint-père, et l'engager à faire chaque jour quelque demande contraire à
sa conscience et à son honneur. Il laissait entrevoir que Sa Sainteté
aurait lieu de se repentir de la manière dont elle en usait à son égard,
autant que de la partialité qu'elle témoignait pour l'empereur. Elle
suspendait encore les bulles de Séville\,; mais Albéroni, déjà pourvu de
l'évêché de Malaga, jouissait du revenu des deux églises. Il se vanta
qu'ils lui suffiraient pour vivre commodément à Madrid à la barbe de
Pantalon et pour aller en avant. Il voulut de plus faire connaître à la
cour de Rome qu'il pouvait compter sur les égards que la cour de France
aurait pour lui, et qu'il n'avait point à craindre que le régent
entreprît de le traverser\,; la preuve dont il se servit fut de révéler
à ses amis que le cardinal del Giudice s'étant adressé au régent pour se
justifier auprès du roi d'Espagne par l'intercession de Son Altesse
Royale, non seulement elle ne lui avait rendu aucun office, mais même
avait envoyé les lettres tout ouvertes de Giudice à Albéroni, sans les
accompagner de la moindre ligne ni pour lui ni pour Sa Majesté
Catholique.

Toutefois Giudice comptait beaucoup sur les offices de M. le duc
d'Orléans\,; il était même si persuadé qu'ils réussiraient, qu'en
attendant la réponse de Son Altesse Royale, il différait à exécuter les
ordres qu'il avait reçus d'ôter les armes d'Espagne de dessus la porte
de son palais. En vain Cellamare, son neveu, le pressait d'obéir, il
attribuait ses instances au désir lâche et bas de plaire au premier
ministre. Giudice lui reprocha plusieurs fois la déférence excessive
qu'il avait pour les folies furieuses d'Albéroni, et le peu d'attention
qu'il faisait aux représentations que le régent s'était chargé de faire,
dont il convenait par toutes sortes de raisons d'attendre le succès. Ces
reproches renouvelèrent d'autres plaintes plus anciennes que Giudice
croyait avoir lieu de faire de son neveu, et rappelant ce qui s'était
passé entre eux quelques années auparavant, il compara les insinuations
que Cellamare lui faisait alors à celles que ce même neveu, si zélé pour
son oncle, lui avait faites à Bayonne pour l'engager à signer l'infâme
projet d'Orry sans y changer un iota. Le bruit se répandit que Giudice
avait fait des projets et pris des mesures pour retourner en Espagne en
cas que le roi catholique vînt à mourir, comptant beaucoup sur la
tendresse, du prince des Asturies pour lui, et sur la faveur dont il
jouirait auprès de lui s'il montait sur le trône. Ces projets vrais ou
faux, et les soupçons des correspondances que ce cardinal entretenait en
Espagne, causèrent la prison d'un nommé don François d'Aguilar, que le
roi d'Espagne fit arrêter comme principal entremetteur de cette
correspondance. Giudice la désavoua, et, traitant de calomnie inventée
par Albéroni ce qu'on avait faussement publié de ses dangereuses
pratiques, il déclara à son neveu que, s'il ne pouvait espérer de le
guérir de la frayeur que le pouvoir d'un premier ministre lui inspirait,
et comme courtisan et comme ambassadeur, il le priait au moins et lui
conseillait d'épargner tant de ruses inutilement employées pour attirer
dans ses sentiments un oncle vieilli dans les affaires, assez instruit
du mérite d'Albéroni pour mépriser sa personne et sa toute puissance. En
même temps il tournait en ridicule les projets de l'Espagne\,; il disait
que tout le monde riait de voir que cette couronne prétendît donner la
loi quand elle était elle-même exposée et sur le point d'être forcée de
la recevoir\,; qu'il semblait par les discours de ses ministres à Rome
que le royaume de Naples fût déjà conquis, le Milanais englouti,
l'infant don Carlos grand-duc de Toscane et duc de Parme et de
Plaisance\,; qu'il ne manquait rien à ces progrès si rapides que la
petite circonstance qu'il n'y avait pas la moindre ombre de vérité\,;
qu'au lieu de ces fables, la monarchie d'Espagne était tellement ruinée
par des dépenses capricieuses et folles que le roi d'Espagne, trompé par
les espérances dont on l'amusait de recouvrer les domaines d'Italie,
emploierait seulement ses richesses à défendre et enrichir le duc de
Parme.

Cellamare, très attentif à sa fortune, voulait en même temps plaire à la
cour d'Espagne et ménager son oncle\,; l'événement lui fit voir que l'un
et l'autre ensemble était impossible\,; mais avant qu'il en eût fait
l'expérience entière, ne pouvant rien mander à son oncle d'agréable de
la part de l'Espagne, il essaya de le consoler et de l'adoucir en
l'assurant que la cour de France était très satisfaite de la conduite
qu'il tenait à l'égard de la constitution, etc.

Il est certain que le pape connaissait l'intérêt qu'il avait de ménager
les couronnes dans une conjoncture où il s'agissait de donner à
plusieurs États d'Italie une nouvelle face par le traité de paix qu'on
proposait de faire entre l'empereur et le roi d'Espagne. Les droits du
saint-siège étaient particulièrement intéressés dans les dispositions
projetées, et le pape prévoyait assez qu'il aurait à souffrir s'il
n'avait pour lui les princes dont le secours et la puissance pouvaient
le garantir du préjudice dont il était menacé. Sa Sainteté, connaissant
ses intérêts, se contentait cependant de simples paroles\,; elle faisait
dire qu'elle désirait sincèrement la paix entre l'empereur et le roi
d'Espagne\,; elle avertissait qu'une paix contraire à la justice ne
pouvait être bonne, mais loin de se concilier avec aucun des princes
intéressés à la conclu de ces grands différends. La seule règle de sa
politique était de faire par pure crainte tout ce que l'empereur
exigeait d'elle, pendant qu'elle montrait beaucoup de vigueur dans
toutes les affaires qui regardaient la France et l'Espagne.
Véritablement on aurait tort de condamner la fermeté que le pape fit
paraître aux instances réitérées fréquemment que le roi d'Espagne lui
fit d'accorder au cardinal Albéroni les bulles de l'archevêché de
Séville. Sa Majesté Catholique eut lieu de s'en repentir dans les suites
aussi bien que du cardinalat qu'elle avait procuré à cet étrange sujet.
Mais alors il gouvernait la monarchie d'Espagne, et les affaires d'un
tel ministre devenaient les intérêts les plus importants et du prince et
de la couronne. Après cette affaire principale, sollicitée vivement par
le cardinal Acquaviva, il y en avait encore une autre où Albéroni avait
intérêt\,; c'était celle de l'accusation que les Allemands avaient
intentée contre lui auprès du pape, fondée sur les négociations
prétendues de ce premier ministre avec la Porte.

Le prince de Cellamare, quoique dans un emploi qui ne l'engageait
nullement à prendre connaissance de ce que les Allemands faisaient à
Rome, encore moins de répondre aux invectives qu'ils y publiaient contre
Albéroni, crut cependant faire un trait de bon courtisan, et marquer son
zèle pour la gloire du premier ministre de son maître, en répondant à
l'écrit imprimé et publié par les Allemands. Il le fit par une lettre
qu'il écrivit à Acquaviva, et ce dernier, n'osant la rendre publique
sans en avoir demandé un ordre précis au roi, son maître, la fit voir au
pape, et ne lui en demanda pas le secret. Ce cardinal était
naturellement ennemi du cardinal del Giudice\,; et Giudice ne douta pas
un moment que, sous le faux prétexte de faire honneur à Cellamare,
Acquaviva n'eût été bien aise d'avoir une pièce entre les mains capable
d'irriter à jamais la cour de Vienne contre Cellamare, et d'empêcher
qu'il ne fût rétabli dans ses biens, que leur situation dans le royaume
de Naples soumettrait par la paix à la domination des Allemands. Il en
fit des reproches à son neveu, trouvant que, pour un homme sage, il
avait agi trop légèrement, et sans réflexion sur les conséquences
dangereuses d'accuser si souvent et si clairement les ministres
impériaux de fausseté et de supposition. Giudice ne s'était pas encore
déclaré pour l'empereur, mais vraisemblablement il en avait déjà pris la
résolution, et, l'écrit de Cellamare paraissant dans une pareille
conjoncture, en était d'autant plus désagréable à son oncle\,; car il
savait que le démérite d'un seul devient à la cour de Vienne celui de
toute une famille, que les Impériaux ne pardonnent jamais, et que le
ressentiment et la vengeance de leur part s'étendent à toute la race
tant que les générations subsistent. Giudice, mécontent du roi d'Espagne
et de son gouvernement, continuait à le décrier de toute son éloquence,
en séparant toujours avec respect le roi de son premier ministre.

\hypertarget{chapitre-iv.}{%
\chapter{CHAPITRE IV.}\label{chapitre-iv.}}

1718

~

{\textsc{Forces d'Espagne en Sardaigne.}} {\textsc{- Disposition de la
Sicile.}} {\textsc{- Le roi Jacques fait proposer au roi d'Espagne un
projet pour gagner l'escadre anglaise et tendant à son rétablissement.}}
{\textsc{- Le cardinal Acquaviva l'appuie en Espagne.}} {\textsc{-
Albéroni fait étaler les forces d'Espagne aux Hollandais.}} {\textsc{-
Albéroni continue ses déclamations contre le traité et contre le
régent\,; accuse Monteléon, qu'il hait, de lâcheté, de paresse\,; lui
fait d'autres reproches\,; en fait d'assez justes à l'Angleterre et au
régent.}} {\textsc{- Le roi d'Espagne veut demander compte aux États
généraux du royaume de la conduite du régent\,; ne se fie point aux
protestations du roi de Sicile.}} {\textsc{- Divers faux
raisonnements.}} {\textsc{- Malignité insultante et la plus partiale des
ministres anglais pour l'empereur sur la Sardaigne et sur les
garnisons.}} {\textsc{- Monteléon de plus en plus mal en Espagne.}}
{\textsc{- Friponnerie anglaise de l'abbé Dubois sur les garnisons.}}
{\textsc{- Maligne et insultante partialité des ministres anglais pour
l'empereur sur la Sicile.}} {\textsc{- Fausseté insigne d'Albéroni à
l'égard de la Sardaigne, ainsi qu'il avait fait sur les garnisons.}}
{\textsc{- Les Impériaux inquiets sur la bonne foi des ministres
anglais, très mal à propos.}} {\textsc{- Efforts de Cadogan et de
Beretti pour entraîner et pour détourner les Hollandais d'entrer dans le
traité.}} {\textsc{- Tous deux avouent que le régent seul en peut
emporter la balance.}} {\textsc{- Beretti appliqué à décrier Monteléon
en Espagne.}} {\textsc{- Ouverture et plainte, avis et réflexions du
grand-duc, confiés par Corsini à Monteléon pour le roi d'Espagne.}}
{\textsc{- Faible supériorité impériale sur les États de Toscane.}}
{\textsc{- Roideur des Anglais sur la Sardaigne, et leur fausseté sur
les garnisons espagnoles.}} {\textsc{- Mouvements de Beretti et de
Cellamare.}} {\textsc{- Fourberie d'Albéroni.}} {\textsc{- Sa fausseté
sur la Sardaigne.}} {\textsc{- Fureur d'Albéroni contre Monteléon\,;
aime les flatteurs\,; écarte la vérité.}} {\textsc{- Chimères, discours,
étalages d'Albéroni.}} {\textsc{- Friponnerie d'Albéroni sur les
garnisons.}} {\textsc{- Il fait le marquis de Lede général de l'armée,
et se moque de Pio et l'amuse.}}

~

Ce prince {[}Philippe V{]}, de son côté, très éloigné d'accepter les
conditions de la paix qu'on lui proposait, se préparait à l'exécution
d'une entreprise dont, en mai 1718, l'objet était encore ignoré de toute
l'Europe. On commençait véritablement à soupçonner qu'elle pouvait
regarder la Sicile. Les forces espagnoles étaient grandes\,; il y avait
en Sardaigne un corps de dix-sept mille hommes effectifs\,; dont trois
mille cinq cents hommes étaient cavalerie ou dragons, outre ce qui
devait être embarqué sur la flotte qu'on attendait d'Espagne. Les
troupes du duc de Savoie en Sicile se réduisaient à huit mille hommes,
composés en partie de gens du pays mal affectionnés à leur prince, et
disposés à se soulever dès que les vaisseaux d'Espagne paraîtraient à la
côte. On supposait alors qu'ils y arriveraient facilement longtemps
auparavant que la flotte qu'on préparait en Angleterre pût venir au
secours du roi de Sicile.

Cette disposition prochaine de nouvelles guerres rendit l'espérance au
roi Jacques. Il ne pouvait se flatter d'aucun secours tant que l'Europe
demeurerait tranquille. L'union de la France avec la Grande-Bretagne
assurait l'état de la maison d'Hanovre. Ce prince ne voyait donc de
ressource pour lui que de la part de l'Espagne, car il était évident que
l'empereur et le roi d'Angleterre demeureraient unis inviolablement,
moins pour satisfaire à leurs engagements réciproques, faible barrière
pour arrêter le roi Georges, que par la raison de leurs intérêts
communs. Le roi d'Espagne étant sur le point d'attaquer l'empereur, il
était comme impossible que l'Angleterre armant, ne prît et ne voulût
prendre part à la guerre. Ainsi le roi Jacques, attendant désormais son
salut de l'Espagne, s'empressa de lui rendre service autant qu'il
dépendait de son pouvoir, borné dans une sphère très limitée. Un
Anglais, officier de marine, dont ce prince prétendait connaître
parfaitement le courage et la fidélité, lui proposait d'aller par son
ordre à Madrid communiquer au cardinal Albéroni un projet dont le succès
presque sûr serait également avantageux aux deux rois. Commock était le
nom de cet officier.

Son plan était d'avoir des pouvoirs et du roi son maître et du roi
d'Espagne, pour traiter secrètement, soit avec l'amiral Bing commandant
l'escadre anglaise, soit avec d'autres officiers de cette escadre. Il
promettait de les engager à se déclarer en faveur du roi Jacques, et
pour le servir, à se joindre à la flotte d'Espagne. Commock demandait,
pour assurer l'effet de sa négociation, une promesse du roi d'Espagne
d'ouvrir ses ports et d'y donner retraite aux navires anglais, dont les
capitaines s'y rendraient à dessein de joindre la flotte d'Espagne et de
se déclarer en faveur de leur souverain légitime. Il désirait de la part
de son maître une lettre au chevalier Bing, écrite en termes obligeants,
avec promesse, si Bing y déférait, de cent mille livres sterling, et de
le revêtir du titre de duc d'Albemarle. Au refus de Bing, le négociateur
demandait le pouvoir de faire les mêmes offres à l'officier qui
commanderait sous les ordres ou au défaut de l'amiral. Il voulait de
plus une lettre circulaire à tous les capitaines de l'escadre, une
déclaration en faveur des officiers et des matelots, la permission de
promettre à chacun des récompenses proportionnées à son rang et à ses
services, à condition cependant que ceux qui voudraient les obtenir
s'expliqueraient dans le terme que cette déclaration prescrirait. La
récompense était vingt mille livres sterling, qui seraient payées par le
roi d'Espagne à chaque capitaine de vaisseau de ligne qui amènerait son
navire au service de Sa Majesté Catholique, et se déclarerait pour le
roi Jacques\,; de plus une commission d'officier général. Tout
lieutenant de vaisseau qui saisirait son capitaine refusant les offres,
et amènerait le navire dans un port d'Espagne, devait avoir la
commission de capitaine, le titre de chevalier, et cinq mille livres
sterling que le roi d'Espagne lui payerait. On promettait aux
subalternes un avancement proportionné à leur mérite, une médaille, et
deux mille livres sterling de récompense. Quant aux matelots, outre le
payement de la somme qui leur serait due, ils auraient encore cinq
livres sterling de gratification. Outre ces offres générales, Commock
demandait une lettre particulière du roi son maître pour un capitaine
nommé Scott dont il vantait fort le crédit, et pour l'engager, il
fallait lui promettre de le faire comte d'Angleterre, amiral de
l'escadre bleue, et lui, payer trente mille livres sterling quand il
joindrait la flotte d'Espagne, ou bien quand il entrevoit dans quelqu'un
des ports de ce royaume. Le point principal était le secret et la
diligence. Le roi Jacques ne risquait rien à tenter le succès des
visions de Commock\,; il adressa donc au cardinal Acquaviva le projet de
cet officier, le pria de le communiquer incessamment au roi d'Espagne,
ce plan intéressant Sa Majesté Catholique autant que lui-même\,; et
comme elle pouvait trouver que les dépenses proposées par Commock
monteraient à des sommes trop considérables, le roi Jacques offrit de
les rembourser quand il serait rétabli.

Acquaviva appuya ces vues, soit qu'elles lui parussent solides, soit
qu'il voulût faire plaisir à ce prince que la fortune persécutait depuis
qu'il était né. Le cardinal observa seulement que les gens attachés au
roi Jacques étaient gens abattus par leurs malheurs, presque au
désespoir, plus remplis de bonne volonté que de force pour exécuter\,;
qu'enfin ceux qui désirent voient pour l'ordinaire les choses plus
faciles que les indifférents. La conjoncture était favorable pour faire
écouter, même admettre à la cour de Madrid toute proposition capable de
faciliter au roi d'Espagne les moyens de soutenir la guerre. Ce prince,
déjà embarqué bien avant, voulait à quelque prix que ce fût, persister
dans l'engagement qu'il avait pris. Toutefois il était seul\,; les
puissances principales de l'Europe s'opposaient à ses desseins\,;
Albéroni déplorait leur aveuglement\,; il prévoyait que le succès de la
guerre serait au moins incertain.

Au défaut d'alliés, il fallait diminuer le nombre d'ennemis\,; et
quoique les neutres et les tièdes soient de la même classe, par
conséquent également rejetés, le premier ministre d'Espagne aspirait à
maintenir les Hollandais dans l'inclination qu'ils témoignaient pour la
neutralité. C'était donc en Hollande principalement qu'il faisait
publier et la résolution que le roi d'Espagne avait prise de ne pas
subir le joug des Anglais, et le détail des forces que ce prince avait,
et qu'il emploierait à soutenir son honneur aussi bien que ses intérêts.

Beretti eut ordre de déclarer à la Haye que son maître hasarderait tout
plutôt que de recevoir les conditions que l'Angleterre prétendait lui
imposer, et voir la Sicile entre les mains de l'empereur. Quant aux
forces de l'Espagne, l'ambassadeur devait dire qu'elles se montaient, à
l'égard des troupes, à quatre-vingt mille hommes\,; que le roi d'Espagne
avait trente navires de guerre, qu'on en construisait encore
actuellement onze dans les ports d'Espagne, chaque navire de
quatre-vingts pièces de canon. Suivant ce même récit, il y avait
trente-trois mille hommes de troupes réglées destinées pour le
débarquement, au lieu où il serait jugé à propos de le faire. Le
payement de ces troupes et de l'armée navale était assuré pour le cours
entier de l'année. Enfin on établissait comme chose certaine que Sa
Majesté Catholique n'avait encore consommé que sept mois de son revenu
des rentes générales et provinciales, et qu'elle attendait le retour de
soixante-treize vaisseaux qui revenaient des Indes. Avec ces belles
ressources, Albéroni concluait qu'il y aurait poltronnerie et bassesse à
céder, hors un cas de nécessité absolue\,; qu'il fallait auparavant
éprouver toutes sortes de contretemps\,; même s'il était nécessaire de
périr, périr les armes à la main\,; et qu'avant qu'être réduit à cette
extrémité, le roi d'Espagne verrait et connaîtrait ses véritables amis,
en sorte qu'après cette épreuve, il serait en état de prendre à leur
égard des mesures certaines\,; car il persistait toujours à conclure que
le projet était chimérique en ce qui regardait les conditions proposées
pour le roi d'Espagne, et qu'on devait le nommer monstrueux à l'égard
des avantages accordés à l'empereur\,; en sorte qu'il paraissait
clairement que la raison ni la justice n'avaient pas dirigé un tel
ouvrage, et qu'il était seulement forgé par la passion et par l'intérêt
particulier de ceux qui l'avaient imaginé. Voulant fortifier son avis
par le témoignage de tous les gens sensés, il assurait, qu'il n'y en
avait aucun qui ne fût surpris de voir les principales puissances de
l'Europe, comme conjurées ensemble, concourir aveuglément à
l'agrandissement d'un prince qu'elles devaient craindre par toutes
sortes de raisons, et tâcher, par conséquent, d'abaisser en cette
occasion. Il donnait aux bons Français le premier rang parmi les gens
sensés, soutenant qu'ils regardaient le projet avec horreur, et qu'ils
{[}étaient{]} pénétrés de douleur de voir la conduite du gouvernement,
si directement opposée aux anciennes maximes que la France avait suivies
et soutenues par de si longues guerres pour tenir en bride la puissance
autrichienne.

Albéroni, depuis longtemps ennemi de Monteléon, l'accusait de ne parler
que par l'organe de l'abbé Dubois. La lâcheté de cet ambassadeur, disait
le cardinal, allait jusqu'au point de dire que, considérant la fierté de
l'empereur, il était étonné qu'il eût accepté le projet. Enfin le roi,
la reine, ni le premier ministre d'Espagne, ne pouvaient lire ses
lettres sans indignation. Albéroni, dans ces dispositions à l'égard de
Monteléon, lui reprocha durement la tranquillité qu'il faisait paraître
en parlant du projet du traité. Il ne lui déguisa pas que Leurs Majestés
Catholiques avaient parfaitement reconnu qu'il se rendait l'organe de
l'abbé Dubois, pendant que les autres ministres détestaient son plan
comme abominable par les conséquences, fatal à la liberté des
souverains, totalement opposé à la raison d'État\,; renversant tout
principe d'établir un équilibre en Europe, et d'assurer le repos de
l'Italie, malheureusement ensevelie sous la dure servitude d'un prince
trop puissant et d'une nation insatiable\,: réflexion qu'un ministre né
en Lombardie devait faire encore, plus naturellement, que tout autre. À
ces reproches, il en ajouta d'autres fondés sur la lenteur de Monteléon
à faire savoir en Espagne ce qui regardait l'armement et la destination
de l'escadre anglaise, car il était persuadé que la cour de Londres,
ayant mis toute son étude à tromper le roi d'Espagne par un projet idéal
que le cardinal nommait un \emph{hircocerf}\footnote{Ce mot désigne un
  animal fantastique, un bouc-cerf.}, attendait seulement le moment de
se déclarer en faveur de l'empereur, afin de le mettre en possession de
la plus belle partie de l'Italie, et de lui donner ce nouveau moyen
d'usurper les autres États de cette partie de l'Europe sans que qui que
ce soit pût l'empêcher. Ainsi, disait-il, les Anglais traitent le roi
d'Espagne comme un roi de plâtre\,; ils croient pouvoir lui imposer
toutes sortes de lois\,; ils se figurent encore que, après bien des
vexations et des insultes, ils obligeront ce prince à leur rendre grâces
d'avoir forgé un projet chimérique, absolument impossible dans son
exécution. Les reproches d'Albéroni tombaient encore moins sur
l'Angleterre que sur le régent. Ce prince sollicitait fortement les
Hollandais d'entrer dans l'alliance. Albéroni déclara que ses instances
avaient achevé entièrement d'irriter le roi et la reine d'Espagne\,;
qu'elles prouvaient authentiquement que la conduite du régent n'était
pas celle d'un médiateur, mais celle d'une partie intéressée aux
avantages de l'ennemi irréconciliable des deux couronnes, celle enfin
d'un prince qui récemment avait assez fait voir le désir qu'il aurait de
les anéantir s'il en avait le pouvoir\,; et d'ailleurs, disait-il,
quelle raison pour les médiateurs de faire la guerre parce que le prince
à qui ils offrent des visions ne les accepte pas comme une proposition
réelle et avantageuse\,? Il ajoutait que le roi d'Espagne ne pouvait
donner ce caractère de solidité à l'offre qu'on lui faisait de mettre
des garnisons espagnoles dans Parme et dans Plaisance, parce que, si ces
garnisons étaient fortes et telles que le besoin le demandait, il serait
impossible que le pays pût fournir à leur subsistance\,; que si elles
étaient faibles, elles seraient sacrifiées d'un moment à l'autre, et
qu'autant de soldats et d'officiers dont elles seraient composées
deviendraient autant de prisonniers qui entreraient dans ces places à la
discrétion des Allemands.

Le roi d'Espagne, ayant donc bien examiné toutes choses, voulait voir si
la France lèverait le masque, et se porterait jusqu'au point de lui
déclarer la guerre ouvertement. Cellamare eut ordre de répandre dans
Paris que son maître ne recevrait la loi de personne, encore moins du
régent que de qui que ce soit\,; que Sa Majesté Catholique croyait
pouvoir s'adresser aux états généraux du royaume, et leur demander
compte de la conduite de M. le duc d'Orléans, les choses étant réduites
au point qu'elle pouvait désormais se porter aux plus grandes
extrémités. Tout expédient, tout tempérament devait être désormais
proscrit, parce que le coeur était ulcéré par la conduite que le régent
avait tenue, et par ses engagements si contraires aux intérêts
d'honneur, et {[}à{]} la réputation de Leurs Majestés Catholiques.
Albéroni était cependant embarrassé de la conclusion d'un traité entre
l'empereur et le roi de Sicile. On disait que ces princes étaient
convenus entre eux de l'échange du royaume de Naples avec les États
héréditaires de la maison de Savoie. Cette nouvelle vraisemblable était
regardée comme vraie parce que le caractère du duc de Savoie donnait
lieu d'ajouter foi à tout ce qu'on publiait de ses négociations
secrètes, quoiqu'on pût dire de contraire aux assurances que ses
ministres donnaient en même temps de sa fidélité envers les princes dont
il souhaitait de ménager l'amitié. Ainsi Lascaris, qui paraissait être
son ministre de confiance à Madrid, à l'exclusion de l'abbé del Maro,
son ambassadeur ordinaire, protestait que son maître était libre, et
qu'il n'avait fait aucun traité avec l'empereur\,; que, si jamais il
entrait en quelque accommodement avec ce prince\,; il ne perdrait point
de vue les traités qu'il avait signés avec le roi d'Espagne\,; qu'ils
seraient sa règle\,; qu'il ne prendrait aucun engagement qui leur fût
contraire\,; et qu'enfin il ne conclurait rien sans l'avoir auparavant
communiqué à Sa Majesté Catholique. Mais ces protestations étaient de
peu de poids, et le cardinal, persuadé que le ministre confident du roi
de Sicile serait le premier que ce prince tromperait pour mieux tromper
le roi d'Espagne, répondit seulement qu'il rendrait compte à Sa Majesté
Catholique des nouvelles assurances qu'il lui donnait de la part de son
maître\,; qu'il pouvait aussi lui écrire qu'elle ne conclurait rien avec
l'empereur sans la participation du roi de Sicile. Albéroni prétendit
que les avis de ces traités lui avaient été donnés comme certains par
les ministres de France et d'Angleterre\,; mais il ajouta qu'ils étaient
suspects, parce que le régent et le roi Georges, désiraient uniquement
pour leurs intérêts l'embrasement de toute l'Europe, et particulièrement
celui de l'Italie. Malgré les déclamations continuelles et publiques, et
le déchaînement d'Albéroni contre la France, on disait sourdement qu'il
y avait une intelligence secrète entre cette couronne et celle
d'Espagne. Bien des gens, à la vérité, croyaient que ces bruits étaient
artificieux, qu'ils étaient répandus par le premier ministre pour mieux
cacher ses entreprises et pour leur donner plus de crédit. Cette opinion
paraissait confirmée par la douceur qui régnait dans les conférences
fréquentes que le cardinal avait avec Nancré. On n'y découvrait pas la
moindre émotion ni le moindre commencement de froideur. On supposait
donc qu'il y avait dans le projet de traité des articles secrets
infiniment plus avantageux pour l'Espagne que ceux qu'on avait laissés
paraître. On ajoutait que la France ni l'Angleterre ne s'opposaient pas
au départ de la flotte espagnole. On allait jusqu'à dire que l'escadre
anglaise agirait de concert avec elle pour l'exécution du projet, dont
la connaissance n'était pas encore livrée au public. D'autres, moins
crédules et plus défiants, soupçonnaient également la foi de la cour de
France et de celle d'Espagne. Ils se persuadaient que toutes deux
voulaient sonder et découvrir réciproquement ce que l'autre pensait,
gagner du temps, et que ces manèges si contraires à la bonne
intelligence finiraient par une rupture. Ils étaient persuadés que la
cour de France était bien éloignée de souhaiter que le roi d'Espagne fit
des conquêtes\,; qu'elle désirait seulement de le voir engagé à faire la
guerre en Italie, et forcé de s'épuiser pour la soutenir. Comme le roi
d'Espagne avait frété un grand nombre de bâtiments français pour servir
au transport de ses troupes, ceux qui prétendaient que le régent verrait
avec plaisir commencer la guerre en Italie, regardèrent comme une preuve
de leur opinion, et comme une collusion secrète, la permission tacite
qu'il semblait donner aux sujets du roi, d'employer leurs vaisseaux au
service de Sa Majesté Catholique. Enfin chacun raisonnait à sa manière,
et peu de gens croyaient que l'Espagne, seule et sans certitude
d'alliés, voulût entreprendre la guerre.

On eut lieu de croire que le roi d'Espagne, paraissant difficile sur le
projet de traité en général, avait seulement en vue d'obtenir quelque
avantage particulier, car Albéroni dit clairement au colonel Stanhope
que ce prince accepterait le projet s'il obtenait de conserver la
Sardaigne. Le colonel ayant fait savoir en Angleterre la proposition qui
lui avait été faite, les ministres anglais assurèrent Monteléon que leur
maître était très affligé de ne pouvoir acquiescer à une demande si
raisonnable. Ils se plaignirent du silence que le roi d'Espagne avait
gardé jusqu'alors sur cette prétention, et feignirent d'en être d'autant
plus touchés que, selon eux, il y aurait eu moyen de satisfaire Sa
Majesté Catholique si elle eût déclaré plus tôt ses prétentions\,; que
l'argent aurait été bien employé pour y parvenir, et que l'Angleterre
aurait volontiers concouru avec la France pour assembler fine somme
telle qu'on eût obtenu ce que désirait le roi d'Espagne\,; mais
malheureusement cette conjoncture favorable était, disaient-ils, passée,
parce que l'engagement était pris avec l'empereur, qu'il était
impossible d'y rien changer, que ce prince se trouvait dans une telle
situation qu'il rejetterait avec hauteur toute proposition d'altérer la
moindre clause du traité\,; qu'il se voyait d'un côté sûr, et comme à la
veille de conclure la paix avec le Turc\,; que, d'un autre côté, le roi
de Sicile continuait de faire des propositions avantageuses à la maison
d'Autriche et que la cour de Vienne accepterait si l'Angleterre lui
donnait quelque occasion de retirer sa parole\,: inconvénients que le
roi d'Angleterre voulait surtout éviter par affection et par tendresse
pour le roi d'Espagne, car il prétendait que Sa Majesté Catholique
devait lui savoir beaucoup de gré de ce qu'il avait fait pour elle et
les ministres anglais feignaient de ne pouvoir comprendre l'injustice
que la cour de Madrid leur faisait, de les accuser de partialité pour
l'empereur, quand ils servaient réellement l'Espagne, et qu'ils
faisaient voir par les effets la préférence qu'ils donnaient à ses
intérêts sur ceux de la maison d'Autriche.

Monteléon se vanta d'avoir essuyé des reproches de leur part, et
prétendit qu'ils l'accusaient d'être auteur des soupçons injustes que le
roi son maître faisait paraître à leur égard. Mais ces accusations ne le
disculpaient pas à Madrid. Albéroni avait trop de soin de le représenter
au roi et à la reine d'Espagne comme vendu aux Anglais\,; et quand le
cardinal n'aurait pas eu le crédit et l'autorité d'un premier ministre
absolu, il aurait cependant persuadé d'autant plus aisément que la cour
d'Angleterre, donnant de grandes espérances au roi d'Espagne, ne tenait
rien de ce qu'elle avait promis quand il s'agissait de l'exécution.
C'est ainsi que les, ministres anglais promirent à l'abbé Dubois qu'il
serait permis au roi d'Espagne de mettre des garnisons espagnoles dans
les places des États du grand-duc et du duc de Parme. Monteléon fit des
instances pour obtenir que la déclaration d'une condition si
essentielle, qui n'était pas comprise dans le projet, lui fût donnée par
écrit. L'abbé Dubois lui promit de refuser sa signature au projet, si
cette condition n'était auparavant bien assurée. Nonobstant les
assurances et les promesses, les Anglais refusèrent de la passer, et
dans le temps qu'ils éludaient la parole donnée au roi d'Espagne, ils
assuraient son ambassadeur que l'objet du roi leur maître, en armant une
escadre pour la Méditerranée, était d'autoriser et d'employer ces
vaisseaux suivant les réponses dont il doutait, et qu'il attendait de la
cour de Vienne. Monteléon désirait que leurs intentions fussent droites.
Il était de son honneur et de son intérêt que la correspondance
s'établît parfaitement entre la cour d'Espagne et celle d'Angleterre, et
profitant de la disposition de son coeur, ne se contraignait pas
lorsqu'il était question de ménager d'autres princes au préjudice de Sa
Majesté Catholique. Les ministres d'Angleterre, pressés de conserver la
Sardaigne à ce prince, s'étaient excusés d'y travailler, alléguant pour
prétexte que l'empereur ne souffrirait jamais que le traité reçût la
moindre altération dans les conditions dont les parties intéressées
étaient convenues. La crainte d'un changement de la part de l'empereur,
était le motif qu'ils employaient pour autoriser le refus d'une
condition demandée par le roi d'Espagne, comme un moyen de lever toute
difficulté, et de conclure un, traité qu'on proposait comme la décision
du repos général de l'Europe. Mais en même temps qu'ils parlaient ainsi
à l'ambassadeur d'Espagne, Stanhope, impatient des reproches que lui
faisait le ministre de Savoie, répondit aux plaintes de cet envoyé que
le duc de Savoie, qui se plaignait d'être abandonné par l'Angleterre, ne
savait pas reconnaître les obligations qu'il avait à cette couronne\,;
qu'elle soutenait seule les intérêts de ce prince, bien résolue de ne se
pas relâcher sur un point qu'elle avait si fort à coeur\,; que le projet
serait accepté par le roi d'Espagne, si le roi d'Angleterre consentait à
lui laisser la Sardaigne\,; mais qu'il était trop attentif aux intérêts
du roi de Sicile pour y laisser donner quelque atteinte, nonobstant les
difficultés qu'il trouvait de tous côtés lors qu'il était question de
soutenir ces mêmes intérêts\,; et qu'actuellement sa plus grande peine à
Vienne était de vaincre la répugnance presque insurmontable, que
l'empereur montrait à renoncer à ses droits sur la monarchie d'Espagne
en faveur de la maison de Savoie.

Si les Anglais cherchaient à faire valoir en même temps leurs soins et
leurs peines pour des princes dont les intérêts étaient directement
opposés, la conduite d'Albéroni n'était pas plus sincère que celle de la
cour d'Angleterre, car il demandait au roi Georges la conservation de la
Sardaigne pour le roi d'Espagne\,; et pendant qu'il insistait sur cette
condition, comme sur un moyen sûr d'engager ce prince de souscrire au
traité, il donnait ordre à Cellamare de confier à Provane, qui était
lors encore à Paris de la part du roi de Sicile, que, nonobstant la
déclaration que Sa Majesté Catholique avait faite à l'égard de la
Sardaigne, elle n'avait nulle intention d'accepter le projet, quand même
cette condition lui serait accordée\,; qu'elle voulait seulement, par
une telle demande, exclure la proposition de l'échange de la Sicile.
Toutefois les ministres de l'empereur ne se croyaient pas encore assez
sûrs de la bonne foi du roi d'Angleterre pour demeurer tranquilles sur
les propositions nouvelles que faisait le roi d'Espagne, et sur les
conférences secrètes et fréquentes que l'abbé Dubois avait à Londres
avec Monteléon. Penterrieder était encore en cette cour de la part de
l'empereur. Il parut très inquiet de la demande faite par Sa Majesté
Catholique, et de la prétention qu'elle formait de mettre actuellement
des garnisons espagnoles dans les places de Toscane et de Parme. Il
était surtout alarmé de l'attention que le régent donnait à ces
nouveautés, que Penterrieder traitait d'extravagantes\,; et, pour en
trancher le cours, il disait que, si elles étaient écoutées, les ennemis
de la paix auraient le plaisir de la renverser et de l'étouffer dans sa
naissance. Quelque inquiétude qu'il fît paraître, les ministres anglais
ne lui donnaient aucun sujet de soupçonner ni leur conduite ni leurs
intentions en faveur de ce prince. Ils n'oubliaient rien pour consommer
l'ouvrage qu'ils avaient entrepris, et pour conduire à sa perfection le
projet de la quadruple alliance. Il fallait pour la rendre parfaite
persuader les Hollandais d'y souscrire\,; et la chose était encore
difficile, nonobstant l'habitude, que cette république avait contractée
depuis longtemps de suivre aveuglément les volontés de l'Angleterre.

Cadogan, alors ambassadeur d'Angleterre en Hollande, se donnait beaucoup
de mouvements pour entraîner les États généraux à se conformer aux
intentions de son maître. On prétendait qu'il répandait de l'argent que
le prince, naturellement aussi ménager que l'ambassadeur, n'épargnait
pas dans une occasion où il s'agissait de gagner les bourgmestres et les
magistrats d'Amsterdam. Cadogan s'était marié dans cette ville, et les
parents de sa femme agissaient pour contribuer au succès de sa
négociation. Beretti agissait de son côté pour le traverser\,; il
parlait mal à propos, donnait des mémoires mal composés, souvent peu
sensés. Toutefois la crainte que les Hollandais avaient de s'engager
dans une nouvelle guerre était si forte et si puissante, que Beretti
avait lieu de croire que son éloquence l'emporterait sur la dextérité de
Cadogan, sur ses libéralités, ses profusions, et sur le crédit de ses
amis. Les États de Hollande s'assemblaient, mais ils se séparaient sans
décider sur le point de l'alliance\,; en sorte que Cadogan,
reconnaissant que l'autorité de l'Angleterre était désormais trop faible
pour déterminer les États généraux, se voyait, chose nouvelle\,! réduit
à recourir aux offices de la France. Il craignait que le régent ne
laissât paraître quelque indécision dans ses résolutions. Il demandait
pressement que Son Altesse Royale ne se lassât point d'envoyer à
Châteauneuf, ambassadeur du roi en Hollande, des ordres clairs et
positifs, tels qu'il convenait de les donner pour assurer les États
généraux qu'il était incapable de changer\,; car il avouait qu'au
moindre doute les affaires seraient absolument ruinées, au lieu,
disait-il, que ses soins et ses diligences avaient si bien réussi à
Amsterdam que cette ville était prête à concourir avec les nobles et les
autres villes principales de la province à la signature de l'alliance\,;
en sorte que l'affaire serait conclue la semaine suivante, nonobstant
les représentations de Beretti et les raisonnements faibles et mal
fondés dont il prétendait les appuyer.

Ces deux ambassadeurs, directement opposés l'un à l'autre, convenaient
également que le régent seul pouvait entraîner la balance du côté qu'il
voudrait favoriser, et que les Hollandais, encore incertains du parti
qu'ils prendraient, seraient déterminés par le mouvement que Son Altesse
Royale leur donnerait. L'objet de Beretti était de gagner du temps et de
maintenir autant qu'il serait possible la Hollande neutre au milieu de
tant de puissances opposées. Mais un point encore plus sensible pour lui
était de décrier Monteléon en toutes occasions, de le rendre suspect à
son maître, et d'attribuer au dévouement qu'il avait pour les Anglais,
les conseils faibles et timides de s'accommoder au temps, de céder à la
nécessité, et de remettre à négocier aux conférences de la paix les
conditions que le roi d'Espagne ne pouvait se flatter d'obtenir avant le
traité, telle que celle de conserver la Sardaigne.

Il est certain que Monteléon, raisonnant politiquement sur la situation
où les affaires étaient alors, donnait lieu à son antagoniste de lui
porter secrètement des coups qui le ruinaient à la cour de Madrid,
d'autant plus sûrement, qu'en attaquant sa fidélité, on était sûr de
plaire au premier ministre. Toutefois la réputation du génie, de
l'expérience, de la capacité de Monteléon, étant mieux établie que celle
de Beretti, bien des gens, surtout les princes d'Italie, ne balançaient
pas à s'ouvrir à l'un plutôt qu'à l'autre, et confiaient à Monteléon ce
qu'ils voulaient faire savoir au roi d'Espagne. Ainsi l'envoyé du
grand-duc lui dit, de la part de son maître, que ce prince et son fils
auraient désiré tous deux, pour leur honneur et pour leur satisfaction,
qu'avant de faire un projet pour disposer de leur succession, on leur en
eût communiqué l'idée\,; ils auraient eu au moins la satisfaction de
faire connaître en concourant au même but leurs sentiments pour le roi
d'Espagne et pour la maison de France, et de découvrir sans crainte
l'inclination que les conjonctures des temps les avaient obligés de
tenir cachée au fond de leurs coeurs. Corsini ajouta que son maître et
le prince son fils, malheureusement privés de succession, ne pouvaient
recevoir de consolation plus touchante pour eux que de voir l'infant don
Carlos destiné, par le concours des principales puissances de l'Europe,
à recueillir après eux la succession de leurs États\,; qu'ils
prévoyaient les avantages que cette disposition apporterait à leurs
sujets. La satisfaction qu'ils en avaient était cependant troublée,
disait-il, par la loi nouvelle et dure\,; qu'on imposait à l'infant de
recevoir de l'empereur l'investiture de tous les États dont la maison de
Médicis était en possession. La liberté du domaine de Florence était
indubitable, et depuis Côme de Médicis il ne s'était fait aucun acte
capable d'y porter le moindre préjudice. La seule démarche que ce
prince, aussi bien qu'Alexandre son prédécesseur, eussent faite à
l'égard de l'empereur, avait été de recevoir la confirmation impériale
de l'élection que la république de Florence avait faite de leurs
personnes\,; mais les Florentins prétendaient que cet acte, reçu pour
d'autres fins, ne pouvait passer pour une investiture féodale. Ainsi le
prince et les sujets seraient également affligés de se voir assujettis
sous une loi si déshonorante\,; et comme il n'était ni juste ni
convenable que la Toscane, gouvernée par un prince de la maison de
France, devînt de pire condition qu'elle ne l'était sous le gouvernement
des Médicis, le grand-duc et son fils priaient le roi d'Espagne de
réfléchir sur les inconvénients qui retomberaient sur l'infant d'une
disposition si contraire à son honneur et à ses intérêts.

Ils représentaient en même temps ceux de l'électrice palatine
douairière, reconnue pour héritière des États de Toscane\,; et le
grand-duc disait qu'il ne pouvait croire que le roi d'Espagne, plein
d'équité, voulût s'opposer au droit de cette princesse, et empêcher
l'effet de la tendresse légitime d'un père envers une fille douée de
tant de mérite et de tant de vertu. D'ailleurs, si on jugeait par le
cours de nature, elle ne devait pas survivre à son frère, étant âgée de
quatre ans plus que lui. Mais quand même elle en hériterait, le
grand-duc représentait qu'il serait de l'intérêt du roi d'Espagne
d'établir le droit de succession en faveur des filles, parce qu'il
arriverait peut-être que l'infante, nouvellement née, profiterait un
jour de la loi que Sa Majesté Catholique appuierait pour la succession
des États de Toscane. Enfin le grand-duc regardait comme un déshonneur
pour lui qu'il fût stipulé dans le traité que le roi d'Espagne mettrait
des garnisons espagnoles dans les places de Toscane. C'était, disait-il,
douter de sa bonne foi que d'exiger de telles précautions lorsqu'il
aurait une fois consenti aux dispositions faites pour la succession de
ses États\,; et s'il était nécessaire d'augmenter les garnisons de ses
places, les moyens de les grossir ne lui manqueraient pas, sans troubler
le repos de ses sujets. Monteléon, instruit de l'opposition que le roi
d'Espagne et son premier ministre apportaient au projet du traité,
répondit à Corsini que tout ce qu'il savait des intentions de son maître
était qu'il trouvait ce projet impraticable, injuste et préjudiciable à
ses intérêts, parce qu'il était contraire à l'équilibre, au repos et à
la liberté de l'Italie.

Albéroni avait cependant laissé entendre en Angleterre que tant de
répugnance et tant d'opposition de la part du roi d'Espagne seraient
surmontées, s'il était possible de faire insérer dans le traité la
condition de lui laisser la Sardaigne, et d'introduire des garnisons
espagnoles dans les places du grand-duc et du duc de Parme. Mais la
première de ces conditions ne pouvait convenir aux vues des ministres
anglais, attentifs à plaire à l'empereur, et craignant la hauteur de la
cour de Vienne lorsqu'elle croyait avoir lieu de se plaindre. Ils
répondirent donc à l'égard de la Sardaigne, que, ni le roi leur maître
ni le régent ne pouvaient se départir du plan proposé tel qu'il avait
été accepté par l'empereur\,; que la résolution était prise de signer le
traité conformément à ce plan et sans y rien changer\,; que la moindre
variation renverserait absolument un projet qui avait coûté tant de
peine. Ils prétendirent que, si on faisait à l'empereur quelque
proposition sur ce sujet, ce prince regarderait toute négociation
nouvelle comme une rupture\,; que, se croyant affranchi des engagements
qu'il avait pris, il serait en état d'en prendre de contraires avec le
roi de Sicile, de qui il obtiendrait facilement cette île, conservant
lui-même ses droits et ses prétentions sur l'Espagne\,; que le fruit
d'une telle union serait de rendre l'empereur et le duc de Savoie
maîtres absolus en Italie, en sorte que l'Espagne, persistant à refuser
le projet du traité comme contraire au repos public, attirerait sur
elle-même et sur toute l'Europe le malheur que cette couronne semblait
appréhender de l'excès de puissance de la maison d'Autriche. La
conclusion de ce raisonnement était qu'il n'y avait de remède aux maux
qu'on craignait, que de lier les mains à l'empereur, et de profiter pour
cet effet du consentement qu'il y donnait lui-même\,; qu'il serait de la
dernière imprudence de lui laisser la liberté de se dégager, dans une
conjoncture où il était assuré de faire la paix avec le Turc, et maître
de traiter comme il voudrait avec le roi de Sicile.

Les Anglais ajoutèrent à ces raisons un motif d'intérêt et de
considération personnelle pour la reine d'Espagne et pour Albéroni. Ils
firent entendre à l'un et à l'autre que l'état incertain de la santé du
roi d'Espagne devait les porter tous deux à suivre en cette occasion les
conseils du roi d'Angleterre. Les ministres anglais se montrèrent plus
faciles sur l'article des garnisons espagnoles. Ils déclarèrent que le
roi d'Angleterre consentirait à la demande du roi d'Espagne d'introduire
ses troupes dans les places du grand-duc et du duc de Parme, pourvu
toutefois qu'il en obtînt le consentement de ces princes. Il fallait,
disaient-ils, ménager avec beaucoup d'attention une telle clause,
capable de renverser le traité si elle était mise en négociation avant
que l'empereur eût signé. Mais au fond, les Anglais savaient bien qu'ils
ne risquaient rien en donnant cette apparence de satisfaction au roi
d'Espagne, et que les deux princes dont ils exigeaient le consentement
préalable ne le donneraient jamais volontairement. Ils pouvaient compter
pareillement sur la disposition intérieure et véritable du roi
d'Espagne, résolu de tenter les hasards d'une guerre, et d'essayer s'il
pourrait profiter de la conjoncture qu'il trouvait si favorable et si
propre à réparer les pertes qu'il avait faites de ses États d'Italie.

Les ministres d'Espagne dans les cours étrangères ne permettaient pas de
douter de ses intentions. Cellamare à Paris, et Beretti en Hollande,
s'en expliquaient hautement, et déclamaient sans mesure contre le projet
du traité. Tous deux se flattaient de réussir. Beretti se vantait de
suspendre par sa dextérité l'accession des États généraux vivement
pressés par la France et l'Angleterre. Cellamare laissait entendre en
Espagne que le régent, touché de ses remontrances, pourrait bien faire
quelques pas en arrière pour sortir des engagements où il s'était
imprudemment jeté. Cet ambassadeur faisait valoir à sa cour les
démarches qu'il avait faites auprès des principaux ministres de la
régence. Il prétendait qu'ils étaient également touchés de ses
représentations, nonobstant la diversité, de leurs réponses\,; que
quelques-uns, plus courtisans que sincères, défendaient le projet, mais
si faiblement qu'il y avait lieu de croire qu'ils parlaient autrement
quand ils se trouvaient tête à tête avec le régent\,; que d'autres
approuvaient les réflexions qu'il leur faisait faire\,; que les Français
hors du ministère louaient ses raisonnements, et que la nation en
général, ennemie du nom autrichien, montrait ouvertement son respect et
son attachement pour le roi d'Espagne (et tout cela était parfaitement
vrai, mais parfaitement inutile).

Les ministres du roi de Sicile croyaient encore devoir faire cause
commune avec ceux d'Espagne, et Cellamare était persuadé qu'il était du
service de son maître de ne pas aliéner le seul prince qui parût disposé
à résister avec Sa Majesté Catholique aux desseins de leurs ennemis
communs. Albéroni voulait ménager encore les Piémontais, mais ses vues
étaient différentes de celles de Cellamare. Il fallait tromper le duc de
Savoie jusqu'à ce que le moment fût arrivé de faire éclater le véritable
objet de l'armement du roi d'Espagne. Son premier ministre se contenait
de dire qu'on verrait bientôt si le duc de Savoie, demandant à s'unir
avec l'Espagne, parlait sincèrement, et que le public connaîtrait
pareillement, avant qu'il fût peu de jours, que Sa Majesté Catholique
rejetait totalement le projet, sans laisser entendre qu'elle consentît
jamais à l'accepter, quelque offre avantageuse qu'on lui fît pour la
persuader\,; car il n'avait tenu qu'à elle, disait le cardinal,
d'obtenir des médiateurs la condition de conserver la Sardaigne, si elle
eût voulu, moyennant cette addition, souscrire aux engagements du
traité. Il prétendit même que le colonel Stanhope, lui offrant depuis
peu cette nouvelle condition, avait employé toute son éloquence pour le
convaincre que le roi d'Espagne devait se contenter de l'avantage qu'on
lui proposait, et qu'il ferait bien mieux de l'accepter que d'employer
inutilement ses trésors à faire armer tant de vaisseaux et transporter
tant de troupes en Italie.

Ces offres prétendues étaient bien opposées aux discours que les
ministres anglais avaient tenus à Londres à Monteléon. Les réponses, les
démarches et les insinuations dont ses lettres étaient remplies, toutes
tendantes à porter le roi son maître à la paix, déplaisaient tellement
au cardinal qu'il ne cessait de décrier la conduite d'un ambassadeur qui
depuis longtemps lui était odieux, peut-être parce qu'il trouvait en lui
trop de talents propres à bien servir son maître\,; et non content de
l'accuser souvent d'infidélité, il lui reprochait encore son incapacité,
jusqu'au point de dire que les réponses, qu'il faisait au sujet du
traité, étaient discours d'un homme ivre, et que le roi d'Espagne ne
pouvait avouer ce qui sortait de la bouche d'un ministre assez
indifférent pour traiter le projet avec tranquillité\,; pendant que les
autres le regardaient avec scandale et avec abomination. Celui qui a
tout pouvoir ne manque jamais de flatteurs et de complaisants prêts à
louer toutes ses vues, à applaudir à tous ses projets, et empressés
d'aplanir en lui parlant les difficultés qui semblent s'opposer à
l'exécution de ses desseins. Telles gens, dont l'espèce subsistera
toujours dans les cours, étaient écoutés avec plaisir par Albéroni\,;
d'autres plus sages, mais en moindre nombre, ne pénétraient pas jusqu'à
lui. On écartait avec soin ceux qui, pesant avec raison la qualité de
l'engagement que le roi d'Espagne prenait, et faisaient de tristes
réflexions sur le succès d'une entreprise prématurée, ne pouvaient, en
approchant du roi et de la reine, parler sincèrement, et découvrir à
Leurs Majestés Catholiques le péril où le royaume allait être exposé. La
nation, en général, était moins touchée de la crainte de l'avenir que de
l'espérance de se remettre en honneur et en crédit par le succès de
l'entreprise. Les Espagnols, jaloux de ce point d'honneur, se flattaient
de chasser les Allemands d'Italie, et d'en recouvrer les États qu'ils
regardaient toujours comme dépendants de la couronne d'Espagne.

Albéroni, sans alliés, se flattait que tous les événements seconderaient
ses desseins. Il se figurait que l'empereur serait obligé de faire
encore une campagne en Hongrie\,; et quoiqu'il n'eût pas lieu de douter
du désir que les Turcs avaient de conclure la paix, il voulait se
persuader qu'ils n'avaient demandé une suspension d'armes que pour
gagner du temps, résolus cependant d'attendre le succès de la descente
qu'on supposait alors que le roi de Suède ferait au premier jour dans le
Mecklembourg. Il espérait que les Hollandais, quoique dépendants depuis
un grand nombre d'années des volontés de l'Angleterre, secoueraient
enfin le joug qu'ils s'étaient laissé imposer, et que les menaces de la
France, jointes en cette occasion à celles des Anglais, n'ébranleraient
pas la fermeté des bons républicains, qui gémissaient de voir la France
et l'Angleterre unies pour forger des chaînes, à l'Europe, et
détestaient, disait-il, le régent, le regardant comme l'auteur des
pertes que leur patrie souffrirait, si elle permettait que la puissance
de l'empereur franchît les bornes où naturellement elle devait être
renfermée pour le bien commun de toutes les nations de l'Europe. Flatté
de cette idée, Albéroni croyait que, lors qu'il serait question de faire
déclarer la guerre à l'Espagne au nom de là France, le régent y
penserait plus d'une fois, nonobstant les vues secrètes qu'il attribuait
à Son Altesse Royale, car il ne feignait pas de dire que c'était se
tromper que de croire que le régent et le roi d'Angleterre fissent la
moindre attention à l'équilibre de l'Europe et à la sûreté de l'Italie.
L'un de ces princes, disait-il, songe à se maintenir roi, l'autre à le
devenir\,: tous deux croient avoir besoin de l'empereur, et tous deux
sont prêts, pour leurs fins particulières, à sacrifier le tiers et le
quart. Non seulement ils ne pensent pas à retirer Mantoue des mains des
Allemands, mais ils concourront encore à les introduire en d'autres
places d'Italie. Albéroni prétendait le prouver par le concours de la
France et de l'Angleterre, unies l'une et l'autre à procurer à
l'empereur la Sicile, unique objet de ses désirs. Il osait enfin traiter
de visionnaire l'abbé Dubois, qu'il nommait l'instrument de toutes les
mauvaises intentions du régent (mais c'était le régent qui était
l'instrument de toutes les mauvaises intentions de l'abbé Dubois\,;
souvent entraîné, contre ses propres lumières et contre sa volonté, par
l'ascendant qu'il avait laissé prendre sur lui à l'abbé Dubois,
l'Albéroni de la France, qui pour soi n'était rien moins que
visionnaire, et qui, \emph{sciens et volens}, sacrifiait la France,
l'Espagne, la réputation de son maître à son ambition de se faire
cardinal, par les voies que j'ai déjà expliquées, d'être tout Anglais et
tout impérial). Comme Albéroni ne pouvait susciter assez d'opposition
aux succès des vues du régent, il employait l'ascendant qu'il croyait
avoir sur l'esprit du duc de Parme pour lui persuader de protester qu'il
ne recevrait jamais de garnison espagnole dans ses places.

Il n'est pas difficile d'inspirer aux petits princes la crainte de
cesser d'être maîtres chez eux en admettant dans leurs places les
troupes de quelque grande puissance. Celle d'Espagne devenait
formidable, si on en croyait l'énumération qu'Albéroni faisait de ses
forces tant de terre que de mer. Il en répandait de tous côtés un détail
magnifique. Il publiait que l'armée navale du roi d'Espagne était
composée de trente-trois navires ou frégates\,; que le moindre de ces
vaisseaux portait quarante-cinq pièces de canon\,; que la flotte était
fournie d'argent et de vivres pour plus de cinq mois. Les troupes, selon
lui, formaient trente-trois mille hommes effectifs, payés jusqu'au
moment de leur embarquement, habillés de neuf et bien armés,
l'artillerie en bon état, et dix-huit mille fusils de réserve prêts à
distribuer aux gens de bonne volonté, s'il s'en trouvait qui offrissent
de servir le roi d'Espagne et la cause commune de l'Italie. Albéroni,
satisfait de tant de grandes dispositions dont il croyait le succès
infaillible, disait en s'applaudissant que la flotte et l'armée de terre
marchaient avec les fiocques\footnote{Le mot italien \emph{fiocchi},
  dont Saint-Simon a fait \emph{fiocques}, signifie littéralement
  \emph{flocons}, et métaphoriquement \emph{pompe, magnificence. Marcher
  avec les fiocques} veut dire \emph{marcher avec pompe et
  magnificence}. C'est le sens de cette expression dans la phrase de
  Saint-Simon.}. Il avouait cependant que Dieu était sur tout, et que
sans son aide tous les soins deviendraient inutiles. Le marquis de Lede
fut nommé général de cette armée, et la flotte partit de Cadix pour
Barcelone le 15 mai. Le prince Pio, alors vice-roi de Catalogne, s'était
flatté d'être chargé de l'exécution de l'entreprise dont il s'agissait.
Albéroni, pour l'en consoler, lui fit dire que Leurs Majestés
Catholiques avaient besoin de garder en Espagne un homme tel que lui,
dans une conjoncture si critique, et qu'il verrait par la destination
qu'elles avaient faite \emph{in petto} sur son sujet, si les choses
prenaient un certain pli, l'opinion qu'elles avaient de son mérite et de
ses talents. Le cardinal voulait que Pio reçût ces assurances
enveloppées comme des marques certaines de la franchise de coeur et de
la sincérité dont il usait en lui parlant.

\hypertarget{chapitre-v.}{%
\chapter{CHAPITRE V.}\label{chapitre-v.}}

1718

~

{\textsc{Riche prise de contrebandiers de Saint-Malo dans la mer du
Sud.}} {\textsc{- Albéroni inquiet de la santé du roi d'Espagne.}}
{\textsc{- Adresse d'Aldovrandi pour servir Albéroni à Rome.}}
{\textsc{- Faiblesse singulière du roi d'Espagne\,; abus qui s'en
fait.}} {\textsc{- Frayeur du pape.}} {\textsc{- Cellamare fait des
pratiques secrètes pour soulever la France contre le régent.}}
{\textsc{- Sentiment de Cellamare sur le roi de Sicile.}} {\textsc{- Il
confie à son ministre l'ordre qu'il a de faire une étrange déclaration
au régent.}} {\textsc{- Forte déclaration de Beretti en Hollande.}}
{\textsc{- Scélératesse d'Albéroni à l'égard du roi de Sicile.}}
{\textsc{- Audace des Impériaux, et sur quoi fondée.}} {\textsc{-
Nouvelle difficulté sur les garnisons.}} {\textsc{- Scélératesse de
Stairs.}} {\textsc{- Fausseté et pis des ministres anglais à l'égard de
l'Espagne.}} {\textsc{- Le czar s'offre à l'Espagne.}} {\textsc{-
Intérêt et inaction des Hollandais.}} {\textsc{- Vanteries, conseils,
intérêt de Beretti.}} {\textsc{- Succès des menées de Cadogan en
Hollande.}} {\textsc{- Menteries, avis, fanfaronnades, embarras de
Beretti qui tombe sur Cellamare.}} {\textsc{- Le duc de Lorraine demande
le dédommagement promis du Montferrat.}} {\textsc{- Manèges de
Beretti.}} {\textsc{- Sa coupable envie contre Monteléon.}} {\textsc{-
Manèges et bas intérêt de Beretti qui veut perdre Monteléon.}}
{\textsc{- Audace des ministres impériaux.}} {\textsc{- Abbé Dubois bien
connu de Penterrieder.}} {\textsc{- Embarras du roi de Sicile et ses
vaines démarches et de ses ministres au dehors.}} {\textsc{- Monteléon
intéressé avec les négociants anglais.}} {\textsc{- Ses bons avis en
Espagne lui tournent à mal\,; il s'en plaint.}} {\textsc{- Superbe de
l'empereur.}} {\textsc{- Partialité des ministres anglais pour lui.}}
{\textsc{- Leur insigne duplicité à l'égard de l'Espagne.}} {\textsc{-
Les ministres anglais pensent juste sur le traité d'Utrecht, malgré les
Impériaux.}} {\textsc{- L'Angleterre subjuguée par le roi Georges.}}
{\textsc{- Les ministres anglais contents de Châteauneuf.}} {\textsc{-
Conduite et manèges de Beretti.}} {\textsc{- Conduite, avis et manèges
de Cellamare.}} {\textsc{- Vagues raisonnements.}} {\textsc{- Monteléon
en vient aux menaces.}} {\textsc{- Stanhope emploie en ses réponses les
artifices les plus odieux\,; lui donne enfin une réponse par écrit,
devenue nécessaire à Monteléon.}} {\textsc{- Surveillants de Monteléon à
Londres\,; sa conduite avec eux.}}

~

Avant le départ de la flotte, on reçut à Madrid la nouvelle de la prise
que Martinet, officier français, servant le roi d'Espagne dans sa
marine, avait faite aux Indes occidentales de quelques vaisseaux de
Saint-Malo. Le vice-roi du Pérou écrivit que le produit des vaisseaux
pris montait à deux millions huit cent mille pièces de huit, tant en
argent comptant qu'en marchandises d'Europe et de la Chine, qu'il avait
fait mettre dans les magasins de Lima. Un tel secours venait fort à
propos pour fournir aux frais de l'expédition. Outre l'argent le roi
d'Espagne profitait encore des vaisseaux pris. Il en choisit les trois
meilleurs pour les joindre à deux autres qu'il avait dans la mer du Sud,
et pour en former ensemble une escadre destinée à empêcher la
contrebande. Ce succès, et l'espérance d'en obtenir de plus grands en
Italie, ne contrebalançaient pas la peine et l'inquiétude que le
dérangement de la santé du roi d'Espagne causait à Albéroni. Il
prévoyait ce qu'il aurait à craindre si ce prince, attaqué depuis
quelque temps d'une fièvre dont les médecins semblaient ignorer la cause
et la nature, venait à manquer. Il pouvait juger que les Espagnols lui
demanderaient un compte sévère de son administration, et qu'il lui
serait peut-être difficile de se justifier d'avoir engagé témérairement
la nation dans une guerre dont on ne pénétrait pas encore l'objet ni
l'utilité. L'ambassadeur de Sicile à Madrid ne fut pas le seul qui
avertit son maître de prendre garde aux desseins du roi d'Espagne. Le
nonce, qui les ignorait, avertit aussi le pape de prendre ses
précautions, parce qu'il pourrait arriver que le débarquement des
troupes d'Espagne se ferait en quelque endroit de l'État ecclésiastique.
Il l'écrivit, peut-être pour servir Albéroni, en intimidant le saint
père, comme un moyen sûr de vaincre le refus des bulles de Séville. Le
nonce dépeignait donc la nation espagnole comme également irritée de ce
refus. Il représenta qu'il était essentiel dans ces circonstances
d'apporter toutes les précautions nécessaires pour prévenir le mal qui
pourrait arriver\,; qu'il fallait user d'une extrême vigilance, d'autant
plus que le pape ne pouvait espérer de personne de recevoir des avis
sûrs et certains\,; que le duc de Parme, qui aurait pu lui en donner,
ignorait lui-même les desseins du roi d'Espagne\,; et qu'enfin Sa
Majesté Catholique, irritée vraisemblablement par les instigations de
son ministre, venait de mettre en séquestre les revenus des églises de
Séville et de Malaga, et d'établir un économe pour les percevoir à
l'avenir et les régir. Une telle résolution devint dans la suite un des
chefs principaux des plaintes et des poursuites que le pape fit contre
le cardinal Albéroni. En effet c'était à lui seul qu'il pouvait
attribuer un séquestre, qu'il regardait comme une violence faite aux
privilèges et immunités ecclésiastiques, étant bien assuré que les
intentions du roi d'Espagne étaient très éloignées des voies que son
ministre lui faisait prendre.

Ce prince avait donné une preuve singulière de ses sentiments à l'égard
des biens d'Église, car ayant des scrupules de conscience qu'il ne
pouvait surmonter sur l'usage qu'il avait été forcé de faire des revenus
de quelques églises vacantes de son royaume, pendant les temps
malheureux de la dernière guerre, il avait fait demander secrètement au
pape l'absolution de l'excommunication qu'il croyait avoir encourue pour
avoir appliqué aux besoins de l'État les revenus de ces églises pendant
ces conjonctures fâcheuses. La cour de Rome ne s'était pas rendue
difficile, et tout pouvoir d'absoudre ce prince avait été envoyé au P.
Daubenton son confesseur. Le pape avait, de plus, remercié par une
lettre particulière, et loué ce religieux, en des termes capables de lui
faire espérer les plus hautes récompenses du zèle qu'il avait fait
paraître en cette occasion. Il y avait donc lieu de croire qu'un roi si
pieux, dont la conscience était si timorée qu'il avait demandé
secrètement l'absolution d'une résolution prise et exécutée dans une
nécessité pressante et pour sauver son État, ne se porterait jamais de
lui-même à toucher de nouveau, et sans nul besoin., aux biens et aux
revenus de l'Église. Avant que le pape sût le séquestre des revenus de
Séville et de Malaga, il voulut s'informer de deux circonstances
seulement, pour la sûreté de la conscience du roi d'Espagne. Sa Sainteté
demanda au P. Daubenton\,: premièrement, quelles raisons il avait eues
de restreindre l'absolution, dont le pouvoir lui avait été envoyé de
Rome, et de la réduire au seul cas de l'appropriation des revenus
vacants. Le pape prétendait qu'il y avait bien d'autres cas où le roi
d'Espagne n'avait pas moins offensé l'immunité ecclésiastique et
l'autorité du saint-siège\,; et par conséquent il ne comprenait pas
pourquoi le P. Daubenton n'avait pas usé de l'ample faculté qui lui
avait été donnée d'absoudre de tous ces cas. Sa Sainteté se plaignait en
second lieu qu'il ne l'eût pas informée de ce qu'il avait réglé avec Sa
Majesté Catholique, au sujet des satisfactions dues à la chambre
apostolique. Le pape ne pouvait croire qu'il se fût avancé à donner
l'absolution sans cette condition, à laquelle la faculté d'absoudre
était littéralement limitée. Ces plaintes, au reste, ne diminuaient en
rien son affection pour ce jésuite. Il crut même lui donner une preuve
distinguée de sa confiance, en s'adressant uniquement à lui, pour avoir
ces éclaircissements sans les demander à son nonce à Madrid, à qui il ne
voulut pas en écrire, pour mieux observer le secret que le roi d'Espagne
avait demandé. Sa Sainteté exigea cependant du confesseur de communiquer
à ce ministre ce qui s'était passé, et de plus, d'envoyer à Rome un
témoignage authentique du concordat que le confesseur devait avoir fait
avec le roi d'Espagne, soit avant, soit après l'absolution donnée selon
les facultés qu'il en avait reçues. Cette cour, si sûre du roi
d'Espagne, craignait seulement son premier ministre, nonobstant la
dignité de cardinal qui devait l'attacher plus particulièrement aux
intérêts du saint-siège.

L'opinion publique était que le pape craignait moins encore les
entreprises qu'Albéroni méditait, que Sa Sainteté ne craignait le
ressentiment de l'empereur, persuadé ou faisant semblant de croire que
le projet du roi d'Espagne était concerté avec elle. Le pape désirait
donc, comme une grâce principale, que Sa Majesté Catholique lui fit
quelque honneur à. la cour de Vienne de la paix qu'on disait prête à se
conclure entre ces deux princes\,; et le nonce Aldovrandi eut ordre de
représenter au roi d'Espagne que ce serait, faire à Sa Sainteté un
plaisir, qui ne coûterait guère à Sa Majesté Catholique, que de répondre
à la lettre que Sa Sainteté lui avait écrite de sa main, et de marquer
dans cette réponse que les remontrances paternelles du chef de l'Église
avaient engagé ce prince à faciliter la conclusion de la paix avec
l'empereur, dans la vue de ne point mettre d'obstacle aux progrès des
armes chrétiennes en Hongrie. Une telle réponse, que le devoir et la
bienséance seuls semblaient exiger, était cependant si désirée de Sa
Sainteté qu'elle déclara que, dans son esprit, elle tiendrait lieu de la
satisfaction qu'elle avait jusqu'alors inutilement demandée pour le
manquement, disait-elle, de l'année précédente, dont le souvenir
demeurerait toujours profondément gravé dans sa mémoire.

Les arrêts opposés du conseil et du parlement, qui faisaient alors du
bruit, firent croire à ceux qui, comme le nonce Bentivoglio, désiraient
le désordre, qu'ils étaient au moment de voir leurs souhaits réussir.
Cellamare, qui travaillait alors dans cette vue, ne manqua pas d'avertir
le roi son maître que, s'il y avait en France des flambeaux pour allumer
le feu, l'affaire de la monnaie pourrait exciter un incendie funeste au
royaume. Il est ordinaire à ceux qui sont occupés d'une affaire
principale de croire qu'elle occupe également tous les esprits.
Cellamare était donc persuadé que généralement toute la nation française
songeait uniquement à l'alliance que le régent négociait, et que
généralement aussi toute la nation, à la réserve de peu de personnes
admises dans le cabinet de Son Altesse Royale pour seconder ses maximes,
désapprouvait cette négociation, même au point de prendre des partis
extrêmes pour en prévenir le succès. Sur ce fondement, il s'était
émancipé dans ses discours\,; et quoique jusqu'alors il n'eût agi que
secrètement, il s'était donné la liberté de parler de manière qu'il
avait aigri le régent. Il voulut réparer auprès de lui ce qu'il avait
dit, mais toutefois il n'abandonna pas les pratiques secrètes qu'il
avait commencées\,; et pendant qu'il voulait faire croire au régent
qu'il ne désirait que l'union et la bonne intelligence entre Sa Majesté
Catholique et Son Altesse Royale, il conjurait le roi son maître de
croire qu'à Londres et à Paris on persisterait dans les résolutions
prises, l'intention des deux princes étant d'établir sur les fondements
de la paix générale, l'un ses espérances, l'autre sa sûreté sur le
trône.

La foi du roi de Sicile, quoique douteuse, ne la paraissait plus à
Cellamare, parce qu'étant persuadé que le roi d'Espagne, ayant besoin de
ce prince, ne devait rien oublier pour ménager ses bonnes dispositions,
ainsi la confiance était grande entre l'ambassadeur d'Espagne et le
comte de Provane, chargé pour lors à Paris des affaires du roi de
Sicile. Cellamare lui apprit qu'il avait reçu par un courrier un ordre
positif de déclarer au régent qu'il était inutile de laisser plus
longtemps Nancré auprès de Sa Majesté Catholique, parce qu'elle ne
voulait accepter ni le projet ni tel autre qu'on pourrait lui proposer,
quand même la cession du royaume de Naples y serait comprise\,; qu'elle
voulait uniquement se venger de ceux qui osaient prétendre lui imposer
des lois et disposer de sa volonté à leur fantaisie\,; qu'elle tâcherait
en même temps d'ouvrir les yeux aux bons François, et leur faire
connaître le mauvais usage que M. le duc d'Orléans faisait de l'autorité
de sa régence, combien, par conséquent, leur fidélité était intéressée à
ne plus tolérer de semblables abus.

L'ambassadeur d'Espagne en Hollande eut en même temps ordre de déclarer
que son maître ne recevrait jamais la loi barbare, que ses plus grands
amis, et ceux qui avaient reçu de lui plus de bienfaits prétendaient lui
imposer\,; que le seul cas de la dernière extrémité pourrait le réduire
à cette nécessité\,; mais qu'il mettait sa confiance en Dieu, et que la
Providence divine saurait ouvrir à la monarchie espagnole les chemins
pour parvenir à la plus grande gloire, et pour obliger au repentir ceux
qui refusaient aveuglément de profiter de l'amitié que Sa Majesté
Catholique leur offrait. À cette déclaration, {[}Beretti{]} ajouta que
le but de Georges et du régent était connu de toute l'Europe\,; qu'au
reste, l'Espagne n'était plus une puissance si faible et si abattue
qu'elle dût souffrir le manquement de foi et les mortifications qu'elle
avait essuyés en d'autres conjonctures\,; qu'elle pouvait enfin faire
respecter ses résolutions, et le parti qu'elle choisirait, de quelque
côté qu'elle voulût faire pencher la balance.

Pendant qu'Albéroni tâchait d'éblouir les nations étrangères par l'éclat
de la puissance nouvelle où il prétendait avoir élevé l'Espagne, il
voulut endormir le roi de Sicile par de fausses confidences. Ainsi, en
même temps qu'on dépêcha de Madrid un courrier au prince de Cellamare,
avec l'ordre de parler si décisivement au régent, le cardinal fit partir
un autre courrier pour avertir le roi de Sicile que le roi d'Espagne
faisait partir sa flotte\,; que l'intention de Sa Majesté Catholique
était de faire tous ses efforts pour, garantir ce prince des insultes de
l'empereur et de ses alliés. L'armement d'Espagne ne causait nulle
alarme à la cour de Vienne. Si elle en eût eu la moindre inquiétude\,;
il dépendait de l'empereur de s'assurer des secours de France et
d'Angleterre, en acceptant le traité que ces deux couronnes lui
offraient. Il était si avantageux à ce prince que le public était
persuadé qu'il y souscrirait, non seulement sans balancer, mais encore
avec l'empressement que produit ordinairement la crainte de perdre une
conjoncture heureuse, qu'on ne retrouve pas après l'avoir laissée mal à
propos échapper. Toutefois les ministres de l'empereur, bien persuadés
que les ministres d'Angleterre, encore moins le roi leur maître, ne leur
manqueraient pas, et que, par le moyen des Anglais, l'empereur
obtiendrait de la France ce qu'il désirerait, firent des difficultés,
même des changements, sur le projet que le Suisse Schaub leur avait
présenté. Il revint en France rendre compte de sa négociation, et des
obstacles qui suspendaient encore la conclusion du traité. Stairs,
ambassadeur d'Angleterre à Paris, ne trouva pas qu'ils fussent
considérables. Toutefois l'empereur demandait, par un nouvel article
qu'il avait ajouté au projet, que les alliés consentissent qu'il mît des
garnisons impériales dans les places des États de Toscane et de Parme\,;
et le seul adoucissement qu'il apportait à cette proposition dure était
qu'au moins on convînt de toutes parts qu'il n'entrerait dans ces places
ni garnisons françaises, ni espagnoles, ni soudoyées au nom du prince à
qui l'expectative des États de Toscane et de Parme devait être donnée.

Stairs et Schaub insistèrent, pour la satisfaction de l'empereur, sur ce
second point, dans une audience que le régent leur donna et qui dura
trois heures. Son Altesse Royale convint avec eux que les garnisons ne
seraient ni françaises ni espagnoles. Il proposa des troupes neutres\,;
il lui vint même en pensée de prier le roi d'Angleterre de garder par
des troupes à lui les places dont il était question. En attendant que la
contestation fût réglée, ces troupes auraient prêté serment au grand-duc
et au duc de Parme. Stairs se chargea d'en écrire au roi d'Angleterre,
et le régent dit qu'il attendrait la réponse avant d'en faire la
proposition à Vienne, Cependant Stairs n'oublia rien pour lui faire
craindre que l'empereur, bien disposé à souscrire le traité, ne changeât
de sentiment si l'expédition préparée par l'Espagne venait à échouer.
Les avantages offerts pour l'agrandissement de l'empereur ne suffisaient
pas, si l'on en voulait croire Stairs. Pour borner les désirs de ce
prince, il demanderait de nouvelles conditions, et ne se croirait pas
obligé aux premières, si l'entreprise du roi d'Espagne, dont le succès
était très incertain, venait à échouer. L'empereur prétendait aussi de
nouvelles renonciations de la part du roi d'Espagne. Stairs trouvait
tant de justice dans toutes ses demandes, tant de dispositions en France
à les passer, qu'il regardait le traité comme fait, puisque la
conclusion ne dépendait que d'un seul article, peu important suivant son
opinion, tel, enfin, que le régent ne pouvait refuser de l'admettre, non
plus que les autres demandes de la cour de Vienne, toutes si évidemment
raisonnables. C'était un triomphe pour un ministre anglais que d'obliger
la France et l'Espagne à demander des troupes anglaises pour garder les
placés des États de Toscane et de Parme. Il était vraisemblable que
l'empereur, sûr de la cour d'Angleterre, ne récuserait pas de pareils
gardiens. Ainsi, Stairs était personnellement flatté de la pensée que M.
le duc d'Orléans avait eue, de proposer lui-même à l'empereur de confier
ces places aux Anglais, et d'y laisser leurs garnisons jusqu'à ce qu'on
fût convenu d'un projet pour les relever par des troupes neutres
choisies à la satisfaction des parties intéressées. Mais il n'eut pas
longtemps le plaisir d'espérer que cette idée serait suivie de la
réalité. Le régent, au lieu de troupes anglaises, proposa des Suisses,
et pour ôter toute ombre de soupçon, il ajouta qu'ils seraient payés par
le corps helvétique, et que chaque canton recevrait des parties
intéressées un subside suffisant pour le payement de cette solde.

Une proposition si juste ne pouvait être rejetée. Stairs n'osa la
condamner en elle-même\,; mais il fit entendre au roi son maître qu'elle
était dangereuse, en ce qu'elle prolongerait la négociation, et que les
délais pourraient faire échouer le traité\,; que tout devait être
suspect de la part des ministres de France\,; qu'ils étaient les auteurs
de la proposition nouvelle des garnisons suisses\,; et que, quoiqu'on ne
pût la dire mauvaise en elle-même, ces ministres donnaient, disait-il,
dans ce qu'il y avait de plus mauvais sans en faire semblant\,; qu'on
pouvait porter ce jugement de leurs intentions secrètes sans blesser la
charité, puisqu'ils avaient saisi toutes les occasions de s'opposer au
traité dès le commencement\,; qu'ils différeraient le plus qu'il serait
possible d'envoyer à Londres la résolution du régent, pour la faire
passer à Vienne si elle était approuvée du roi d'Angleterre, et que
peut-être ils donneraient pour motif de retardement l'embarras survenu à
Paris au sujet de l'affaire de la monnaie. Cet incident, que les
ministres étrangers regardaient comme un commencement de brouillerie
éclatante entre le régent et le parlement, était pour eux un sujet
important de réflexions et d'attention sur les suites qu'un tel démêlé
pouvait avoir.

Le roi d'Angleterre, soit par ce motif, soit par l'intérêt capital qu'il
avait de conserver à ses sujets la liberté du commerce d'Espagne,
essayait de maintenir un reste de bonne intelligence avec le roi
d'Espagne, quoique la flotte anglaise fût déjà sortie de la Manche,
envoyée à dessein et avec des ordres exprès de traverser les entreprises
que l'armée d'Espagne pourrait tenter en Italie. Les ministres anglais
tâchaient de justifier par des paroles la conduite que leur maître
tenait à l'égard de l'Espagne\,; mais l'apologie en étant difficile, ils
se plaignaient d'Albéroni, attribuant au procédé de ce ministre
l'aigreur déraisonnable que le roi d'Espagne faisait paraître à l'égard
du roi d'Angleterre. Comme il était au moins douteux que ces plaintes
réussissent à Madrid, et que le roi d'Espagne se laissât persuader de
l'amitié des Anglais malgré les preuves qu'il recevait de leur inimitié,
les ministres anglais avaient soin d'avertir leurs marchands à Cadix et
dans les autres ports d'Espagne de se tenir sur leurs gardes, et de
prendre des mesures pour mettre à couvert leurs effets en cas de
rupture\,: toutes choses y paraissaient disposées, et cependant le roi
d'Espagne manquait absolument d'alliés. Un prince, dont la puissance
était grande, mais trop éloignée pour être utile à l'Espagne, s'offrit à
la seconder. Le czar fit dire à Cellamare qu'il était prêt de
reconnaître le roi d'Espagne pour médiateur des différends du Nord\,;
que, de plus, il ferait dire clairement au régent qu'étant mal satisfait
des Autrichiens et du roi d'Angleterre, il était résolu d'appuyer les
intérêts du roi d'Espagne. Il eût été plus utile pour ce dernier
monarque que les Provinces-Unies en eussent entrepris la défense\,; mais
l'objet principal de cette république était alors de conserver la paix
et de se ménager également envers toutes les puissances, dont les
intérêts différents pouvaient rallumer la guerre en Europe.

Cette république demeurait dans une espèce d'inaction, et paraissait
également sourde aux instances de la France et de l'Angleterre, et à
celles de l'Espagne. On attendait de temps à autre quelque effet de
différentes députations des villes de la province de Hollande, des
assemblées des états de la même province. Mais il n'en sortait aucune
résolution. Beretti s'applaudissait d'une lenteur qu'il croyait
insupportable aux cours de France et d'Angleterre. Il attribuait à sa
dextérité, la longue incertitude des Hollandais, et pour se rendre
encore plus agréable à Albéroni, il renchérissait par de nouvelles
invectives sur celles dont ce cardinal usait familièrement en parlant de
la conduite de la France. Beretti, non content de parler, faisait encore
agir le résident de Sicile à la Haye, et démentait par cet homme qu'il
envoyait de porte en porte le bruit qui s'était répandu d'un
accommodement déjà fait entre l'empereur et le duc de Savoie. Il
assurait en même temps que le roi d'Espagne se défendrait jusqu'à la
dernière extrémité\,; que plutôt que céder, il mettrait l'épée à la
main, résolu toutefois d'écouter et d'admettre les bons offices, que la
république interposerait pour la paix quand ils seraient, disait-il,
portés dans les termes et avec la possibilité convenables. Il se croyait
assuré, ou peut-être feignait-il de l'être pour se rendre plus agréable
à Madrid, que, si la république employait ses offices, elle userait de
phrases telles que la France et l'Angleterre et la cour de Vienne en
seraient également satisfaites, sans toutefois que les États généraux
prissent le moindre engagement sur la matière du projet que le roi
d'Espagne n'accepterait ni ne voulait accepter. Ainsi ce prince,
admettant seulement les offices d'une république zélée pour la
conservation de la paix, devait, suivant l'idée de son ambassadeur,
faire le beau personnage de prince pacifique sans se lier, sans
discontinuer s'il le voulait ses entreprises, libre et maître de faire
ce qu'il lui plairait dans la situation avantageuse d'attendre les
offices, de répondre comme il le trouverait à propos, et de dire non
quand bon lui semblerait.

Beretti conseillait, de plus, de rendre des réponses plausibles,
d'amuser le tapis et de gagner du temps, excellent moyen pour exciter
les soupçons et la division entre les puissances qui se liguaient contre
l'Espagne, car il croyait que la France se défierait des promesses du
roi d'Angleterre, dès qu'elle s'apercevrait que ce prince, qui avait
répondu que les Hollandais entreraient dans l'alliance, n'avait pas eu
en Hollande le crédit dont il s'était vanté, ou bien qu'il manquait à sa
parole. Pour appuyer ces conseils Beretti représenta que si le roi
d'Espagne refusait sèchement le projet sans ajouter comme un lénitif que
la Hollande pourrait employer ses offices, le parti français, anglais,
autrichien, celui des ignorants et des autres qui veulent tout savoir
tomberaient tous ensemble sur l'Espagne, au lieu que le torrent serait
détourné par le moyen qu'il proposait\,; que la conjoncture était
d'autant plus favorable que Cadogan, par ses bravades et par ses
menaces, avait irrité les bourgmestres d'Amsterdam, aussi bien que les
membres des États de Hollande, et qu'enfin quatre des principales villes
de cette province demandaient déjà des grâces au roi d'Espagne pour le
commerce, s'engageant de procurer en ce cas la neutralité des États
généraux.

Cadogan, de son côté, paraissait très content du succès de ces mêmes
négociations que Beretti disait échouées, et pendant que ce dernier se
donnait à Madrid comme le promoteur des dégoûts qu'il supposait que son
antagoniste recevait en Hollande, Cadogan écrivait à Londres que, par sa
dextérité et par le crédit de ses amis dans la province de Hollande, il
avait réussi à persuader les villes d'Amsterdam, Dorth\footnote{Cette
  ville de Hollande est désignée ordinairement sous le nom de Dordrecht.},
Harlem, Tergaw et Gorcum de prendre enfin la résolution de signer le
projet\,; que la plus grande partie des villes de la même province
suivrait l'exemple de ces premières, en sorte que, lorsque chaque ville
aurait donné son consentement particulier, rien ne retarderait plus la
résolution unanime de la province, et la chose paraissait d'autant plus
sûre que le Pensionnaire et les amis de l'Angleterre, alors très
nombreux, y travaillaient de tout leur pouvoir avec espérance de réussir
avant la séparation de l'assemblée des États de Hollande. La province
d'Utrecht donnait les mêmes espérances. Déjà ses ecclésiastiques et ses
nobles consentaient au projet, et on ne doutait pas que la ville
d'Utrecht n'y consentit aussi dans l'assemblée qui devait se tenir le 26
juin. Mais malgré ces dispositions Beretti, persuadé que la voie la plus
sûre de plaire était de rapporter des choses agréables, persistait à
assurer le roi son maître que les Hollandais ne feraient aucune démarche
qui pût lui déplaire. Il prétendait le savoir en confidence des députés
les plus graves. C'était selon lui l'effet des ménagements qu'il avait
eus à l'égard de ceux de la république capables de rendre de bons
services\,; mais en vantant son attention pour eux et le fruit qu'il
tirait de son industrie, il voulut aussi laisser croire que le dernier
mémoire qu'il avait délivré aux États généraux avait fait sur l'esprit
de l'assemblée une impression si heureuse qu'on devait attribuer à ce
rare ouvrage une partie principale du succès.

Beretti relevait l'utilité de ce mémoire avec d'autant plus de soin
qu'il s'était avancé sans ordre de promettre que le roi d'Espagne
accepterait les bons offices de la république. Il n'était pas sans
inquiétude des suites que pourrait avoir à Madrid une démarche faite
sans la participation du premier ministre, jaloux à l'excès de son
autorité, très éloigné d'approuver de pareilles licences, et de
permettre aux ambassadeurs d'Espagne de les prendre à son insu. Ainsi
Beretti n'oublia rien pour faire comprendre au cardinal Albéroni que,
s'il s'était émancipé, il ne l'avait fait que parce qu'il avait connu
clairement qu'une telle déclaration était, disait-il, le moyen unique de
mettre une digue au torrent impétueux des instances de la France et de
l'Angleterre, et qu'en effet par cet expédient employé à propos, il
avait obtenu les délais et le bénéfice du temps dont Cadogan paraissait
actuellement désespéré\,: car il était arrivé à la Haye en figure de
dictateur, accompagné de pompes, de festins, de livres sterling en
quantité prodigieuse. Il se trouvait, chose singulière, secondé par les
François et les Autrichiens. Outre l'argent, il faisait agir les
prédicants, et remuait par leur moyen, ajoutait Beretti, les passions du
bigotisme protestant, de manière que les peuples étaient persuadés que
la religion de l'État ne pouvait être en sûreté, si la république
n'adhérait en tout aux sentiments du roi Georges. Il semblait donc aux
ministres français et anglais qu'ils devaient commander à baguette à la
république de Hollande. Telles étaient les relations que l'ambassadeur
d'Espagne faisait à la cour de Madrid. Il les ornait de temps en temps
de quelques nouvelles découvertes. Il supposait que les alliés avaient
gagné de certains magistrats d'Amsterdam. Souvent il taisait leurs noms,
se faisant honneur de l'espèce de discrétion que l'ignorance des faits
ne lui permettait pas de violer. Quelqu'un lui, dit que Paneras,
bourgmestre d'Amsterdam, et Buys, pensionnaire de la même ville, avaient
été gagnés par l'argent d'Angleterre\,; il fut moins discret à leur
égard. Il chargea surtout Buys, le nommant l'orateur des Anglais. Malgré
ses ennemis, il se vantait de faire face à tout. Comme il doutait
cependant du succès de ses assurances et de ses prédictions, il ne
voulait pas s'en rendre absolument garant envers le roi son maître. Il
avertit ce prince qu'il était impossible de répondre du parti que
prendrait la république depuis que la France était entrée en danse,
rejetant indirectement sur Cellamare le démérite de n'avoir pas empêché
l'union entre le régent et le roi d'Angleterre.

Beretti, fertile en expédients bons ou mauvais, conseilla à Albéroni de
faire courir le bruit qu'il serait ordonné aux négociants espagnols de
remettre à ceux que Sa Majesté Catholique commettrait un registre fidèle
de tous les effets confiés à ces négociants appartenant aux Anglais et
aux Hollandais. Il représenta que cette simple formalité donnerait lieu
à bien des réflexions, et que la démarche pouvait être utile, parce que
Buys soutenait en Hollande que les négociants espagnols étaient si
fidèles que jamais ils ne découvriraient les effets appartenant à leurs
correspondants. Enfin la principale vue de Beretti étant toujours de
gagner du temps, il souhaitait comme une chose avantageuse au roi son
maître que les États généraux, sans en être sollicités de la part de ce
prince, lui écrivissent pour lui proposer non seulement d'être
médiateurs, mais encore arbitres des différends présents, car il serait
facile en ce cas de laisser écouler deux mois entre la proposition et la
réponse\,; et pendant cet intervalle, comme on était alors au mois de
juin, le roi d'Espagne aurait éprouvé le succès de son entreprise. S'il
était heureux, disait Beretti, Sa Majesté Catholique serait en état de
soutenir ses droits et ses prétentions, et s'il était malheureux, plus
on approcherait de la fin de la campagne, et plus on aurait le temps de
négocier. Ce ministre, de son côté, prétendait ne rien négliger, soit
pour détourner les villes de Hollande de prendre aucun engagement
contraire aux intérêts du roi son maître, soit pour semer la défiance,
source de discorde, entre les puissances liguées ou prêtes à se liguer
ensemble contre l'Espagne.

Comme le duc de Savoie n'avait pris encore aucun engagement, Beretti
crut faire beaucoup d'inspirer à l'agent que ce prince avait en Hollande
des soupçons sur les desseins que l'alliance prête à éclater pouvait
former au préjudice de la maison de Savoie. Le duc de Lorraine avait
écrit au roi d'Angleterre, et pareillement aux États généraux,
représentant à l'une et à l'autre puissance que, pendant la guerre
terminée par le traité d'Utrecht, les alliés lui avaient promis de
l'indemniser de ses prétentions sur le Montferrat donné au duc de Savoie
sans autre raison que celle du bien de la cause commune. Le roi
d'Angleterre avait déjà répondu qu'il fallait attendre un temps plus
favorable, la conjoncture présente ne permettant pas d'agir pour les
intérêts du duc de Lorraine, si le duc de Savoie n'y donnait occasion
par sa résistance à souscrire au traité.

La Hollande, plus lente dans ses réponses, n'en avait fait aucune au duc
de Lorraine. Le public ignorait même que ce prince lui eût écrit quand
Beretti révéla cette espèce de secret à l'agent de Sicile à la Haye, et
prétendit par cette confidence lui donner une preuve de l'attention que
le roi d'Espagne aurait toujours aux intérêts du roi de Sicile quand ce
dernier aurait un procédé sincère à l'égard de Sa Majesté Catholique.
Beretti, voulant toujours pénétrer les motifs secrets, dit à l'agent de
Sicile que comme le duc de Lorraine ne remuait pas la prunelle sans la
volonté de l'empereur, on devait regarder les lettres qu'il avait
écrites en Angleterre et en Hollande comme une insinuation procédant de
quelque stratagème politique de la cour de Vienne, soit pour faire peur
au roi de Sicile, soit pour se venger de lui, supposé qu'elle crût que
ce prince se conduisît de bonne foi à l'égard du roi d'Espagne. Beretti,
content de tout ce qu'il remarquait d'ingénieux dans sa propre conduite,
satisfait de son zèle et de son attention à profiter des moindres
occasions de servir utilement son maître, et, persuadé que la cour, de
Madrid ne pouvait lui refuser la justice qu'il se faisait à lui-même,
croyait aussi qu'il ne lui manquait pour posséder toute la confiance du
roi d'Espagne dans les affaires étrangères que de décrier et de vaincre
Monteléon, son ancien ami, mais qu'il haïssait alors, parce que tous
deux couraient la même carrière, et que, dans l'esprit du public,
Monteléon avait sur lui de grands avantages\,: c'en était un pour
Beretti de savoir que son émule était mal dans l'esprit du roi et de la
reine d'Espagne et d'Albéroni. Avec une pareille avance, il ne doutait
pas de perdre un compétiteur si dangereux, et pour y parvenir, il ne
cessait de se plaindre des lettres qu'il recevait de Monteléon,
contenant des avis si superficiels et si obscurs qu'après les avoir lus,
il n'en était pas plus instruit. Beretti l'accusait de faire l'avocat
perpétuel des Anglais, si changés à son égard qu'ils célébraient ses
louanges après en avoir dit beaucoup de mal, il n'y avait pas encore
longtemps. Beretti se vantait d'être devenu, au contraire, l'objet de
leur haine et de celle des François, nonobstant les civilités feintes et
affectées qu'il recevait de leur part.

Il est certain que les ministres de la cour d'Angleterre décriaient ou
élevaient alors ceux de France et d'Espagne, selon qu'ils pliaient ou
qu'ils résistaient aux volontés du roi d'Angleterre. Nancré était alors
regardé comme absolument gagné par Albéroni\,; l'abbé Dubois était
célébré quoique Penterrieder, alors ministre de l'empereur à Londres,
eût très mauvaise opinion de lui et que même il ne se mît pas en peine
de cacher ce qu'il en pensait\,: car il suffisait d'être agent de
l'empereur pour se croire en droit de parler avec autorité, de trancher
et de décider souverainement sur toutes les difficultés d'une
négociation, même sur le mérite du négociateur. Penterrieder trouva
mauvais que l'abbé Dubois eût proposé à la cour d'Angleterre d'essayer
les moyens de douceur pour fléchir le roi d'Espagne et lui persuader de
souscrire au traité moyennant la promesse que les alliés lui feraient de
permettre qu'il mit des garnisons espagnoles dans les places de Toscane.
Une telle proposition choquait la cour de Vienne, et Penterrieder, sans
attendre de nouveaux ordres, déclara que, s'il en était question, il ne
fallait plus parler de sociétés, son maître étant résolu de se porter à
toutes sortes d'extrémités plutôt que d'admettre de telles conditions\,;
il ajouta que ces complaisances ne servaient qu'à augmenter la fierté
d'Albéroni\,; que son but était de retrancher aux ministres anglais la
connaissance des affaires d'Espagne, et que, bien loin de se rapprocher
de leur manière de penser, on apprenait par les dernières lettres de
Madrid qu'il demandait pour le roi d'Espagne la Sicile et la Sardaigne,
et qu'il prétendait encore prendre le duc de Savoie sous sa protection.
Ainsi cet homme n'ayant en vue que de renverser la disposition des
traités, il fallait, suivant le raisonnement de Penterrieder, agir avec
vigueur pour le prévenir et pour détruire ses projets. La conséquence de
ce raisonnement était la nécessité de faire partir au plus tôt l'escadre
anglaise destinée pour la Méditerranée. Les instances de l'envoyé de
l'empereur étaient favorablement écoutées\,; le roi d'Angleterre lui
promit à la fin de mai que cette escadre partirait avant la fin de la
semaine, et que le commandant, qui avait reçu des instructions conformes
aux engagements de l'Angleterre, promettait de faire le voyage en quinze
jours si le vent était favorable.

Il n'y a {[}pas{]} pour les souverains de situation plus embarrassante
que celle d'un prince faible, dont les États sont enviés par des
puissances supérieures à la sienne, ennemies entre elles, mais désirant
également l'une et l'autre s'enrichir de ses dépouilles. Le duc de
Savoie se trouvait dans cette situation à l'égard de l'empereur et du
roi d'Espagne\,; il ne pouvait espérer d'empêcher par la force
l'exécution de leurs desseins\,; sa seule ressource était celle de la
négociation\,; il l'avait employée à Vienne et à Madrid, mais sa
dextérité ne pouvait suppléer à l'opinion que toute l'Europe avait de sa
foi, et comme il n'y avait point de cour où elle ne fût également
suspecte, ses ministres étaient plus souvent occupés à faire des
apologies qu'ils ne l'étaient à négocier. Ils ne réussirent pas à
Vienne, et leurs justifications à Madrid n'eurent pas un meilleur
succès. Ils avouèrent au roi d'Espagne que leur maître avait négocié à
Vienne, mais ils soutinrent que Sa Majesté Catholique n'avait pas lieu
de s'en plaindre puisque ce prince lui avait donné part et de l'objet et
du peu de succès de sa négociation. L'objet en avait été le mariage du
prince de Piémont avec une des archiduchesses filles du défunt empereur
Joseph. Le roi de Sicile prétendait encore de s'assurer par le même
traité la possession de la Sicile, ou tout au moins d'en obtenir un
équivalent juste et raisonnable si l'échange était jugé absolument
nécessaire au repos de l'Europe ainsi qu'à l'accomplissement des vues
des puissances engagées dans l'alliance. Il donnait comme une marque de
sa bonne foi le soin qu'il avait eu de communiquer à ces mêmes
puissances ainsi qu'au roi d'Espagne le peu de succès de cette
négociation\,; mais, prévoyant qu'on douterait de la `sincérité de ses
expressions, il y ajouta que, si quelque puissance le voulait attaquer
il repousserait la force par la force, qu'il mettait la Sicile en état
de faire une résistance ferme et vigoureuse, et qu'il en usait de même à
l'égard des places de Piémont\,; qu'il avait fait la revue de ses
troupes, qu'il était résolu de tout risquer si quelque ennemi
l'attaquait, et qu'enfin la défense qu'il ferait serait digne de lui. Ce
fut en ces termes que le marquis du Bourg, un de ses principaux
ministres, déclara les intentions du roi son maître au marquis de
Villamayor, alors ambassadeur d'Espagne à Turin.

Monteléon, instruit de cette déclaration par Villamayor, et croyant
savoir les intentions du roi d'Espagne, jugea que Sa Majesté Catholique
et le roi de Sicile ayant une égale horreur du traité proposé, il ne
risquait rien en s'ouvrant à La Pérouse, résident de ce prince à
Londres, comme au ministre d'un prince qui pensait comme le roi
d'Espagne, et qui, par conséquent, devait avoir le même intérêt, ayant
le même objet. Il lui dit donc qu'il avait reçu un ordre précis
d'Albéroni de déclarer et de prouver que le roi d'Espagne ne pouvait
accepter les propositions qui lui étaient faites par la France et par
l'Angleterre. La Pérouse remarqua une sorte d'affectation de la part de
Monteléon à ne pas dire que Sa Majesté Catholique ne voulait pas
accepter les propositions. Tout est suspect à un ministre chargé des
affaires de son maître, et les soupçons souvent contraires au bon succès
des négociations sont permis quand on traite dans une cour dont les
intentions sont au moins douteuses, et avec gens qu'on a raison de
croire gagnés et conduits par leur intérêt particulier. La Pérouse était
persuadé que, si jamais le ministère anglais procurait quelque avantage
au roi de Sicile, ce ne serait que par hasard, par caprice et par
passion de la part des ministres\,; mais que, lorsqu'ils agiraient de
sang-froid et de propos délibéré, ils travailleraient directement contre
les intérêts de ce prince et à son désavantage. Il n'était pas plus sûr
de l'ambassadeur d'Espagne, car enfin Monteléon avait acheté des
actions\,; il était lié intimement avec les principaux négociants
anglais\,; sa partialité pour eux paraissait en toutes occasions. Son
union était grande avec l'abbé Dubois. Il différait autant qu'il lui
était possible à déclarer les intentions du roi son maître au sujet du
traité, et lorsqu'il avait déclaré à La Pérouse les derniers ordres
qu'il avait reçus de Madrid, la conclusion de son discours avait été
qu'il ne pouvait se promettre un heureux succès du parti que prenait le
roi d'Espagne, et qu'il n'y avait rien à espérer de pareilles
entreprises si la France ne faisait quelque chose de plus que de
demeurer neutre.

Les lettres de Monteléon en Espagne étaient de même style, et comme
elles contrariaient directement la résolution du roi catholique, non
seulement, elles déplaisaient, mais elles fortifiaient les soupçons
qu'Albéroni avait conçus, que Beretti avait augmentés, et que tant de
circonstances semblaient confirmer au sujet de la fidélité de
l'ambassadeur. Il n'était pas difficile à Monteléon de reconnaître par
les lettres qu'il recevait les fâcheuses idées que la cour de Madrid
avait prises à son égard. Il s'en plaignait, persuadé qu'il avait bien
servi son maître, et lui représentait les inconvénients que le refus du
traité entraînerait, les difficultés de soutenir longtemps un semblable
refus, enfin, indiquant les mesures qu'il était nécessaire de prendre,
et dont l'omission était cause du mauvais état où se trouvait
actuellement l'Espagne, car il craignait tout pour sa flotte, celle
d'Angleterre étant prête à mettre à la voile pour la Méditerranée, et le
roi Georges ayant donné de nouveaux ordres pour en hâter le départ.
Malgré les injustices dont il prétendait que ses services étaient payés,
il se vantait de se comporter en homme d'honneur et en ministre fidèle
de son maître, lorsqu'il était question pour satisfaire à ses ordres de
parler avec fermeté aux ministres d'Angleterre, même à l'abbé Dubois,
car il témoignait également à tous la juste indignation que Sa Majesté
Catholique ressentait et du projet de traité et de la conduite tenue
dans le cours de la négociation\,; mais se plaindre et menacer était
pour l'Espagne crier dans le désert.

La cour de Londres n'avait d'attention que pour l'empereur. Il se
faisait solliciter pour accepter les avantages qu'elle voulait lui
procurer. Ses ministres faisaient des difficultés, non sur des choses
essentielles, car ils étaient satisfaits, mais sur les termes les plus
indifférents de la traduction du traité. Les ministres anglais
attendaient que ces difficultés fussent levées pour faire partir la
flotte, et témoignaient la même impatience de les voir aplanies, que si
l'empereur en eût attendu la décision pour appuyer de toute sa puissance
le roi d'Angleterre et conquérir en faveur de ce prince une nouvelle
couronne. Toutefois ils ne négligeaient pas le roi d'Espagne, et pendant
qu'on armait dans les ports d'Angleterre pour le combattre, le colonel
Stanhope recevait des ordres précis d'assurer Albéroni que Georges avait
soutenu les intérêts de l'Espagne comme les siens propres\,; que les
peines qu'il s'était données pour amener la cour de Vienne à la raison
ne se pouvaient exprimer, et qu'il ne pouvait dire aussi les difficultés
sans nombre qu'il avait essuyées et surmontées de la part de l'empereur
pour le fléchir et le réduire à peu près au point que Sa Majesté
Catholique le désirait, chose d'autant plus difficile, que la paix avec
la Porte était comme assurée, et que l'empereur n'était pas moins sûr de
conclure un traité avec le roi de Sicile en tel temps et à telles
conditions qu'il conviendrait aux intérêts de la maison d'Autriche.
Ainsi l'envoyé, d'Angleterre devait faire voir que, sans les bons
offices du roi son maître, le roi d'Espagne n'aurait pas eu le moindre
lieu d'espérer qu'il trouverait tant de docilité de la part de la cour
de Vienne.

Le roi d'Angleterre prétendait aussi qu'il n'aurait pu se flatter de
réussir, s'il n'eût fait naître dans l'esprit de l'empereur ces bonnes
dispositions, en lui faisant voir que lui-même était réciproquement
disposé à lui donner toutes sortes de secours contre les perturbateurs
du repos public. C'était le motif que les ministres anglais alléguaient
pour justifier l'armement de l'escadre prête à faire voile au premier
vent. Ils décidaient en même temps que quelques changements que
l'empereur désirait au projet lui devaient être accordés\,; qu'aucun ne
devait faire la moindre peine, même à l'égard de la forme, ni à la
France ni à l'Angleterre. Ils jugèrent seulement que la France pourrait
avoir quelque répugnance à consentir à l'idée que les ministres de
l'empereur avaient d'exiger du roi une renonciation nouvelle à ses
droits sur la couronne d'Espagne et sur les États qui en dépendent, et
de faire assembler les états du royaume pour autoriser cette
renonciation. Ces ministres Anglais s'objectaient eux-mêmes qu'un tel
acte fait par un prince mineur serait nul\,; que s'il paraissait qu'on
eût, quelque doute sur la solidité du traité d'Utrecht, l'incertitude
sur la foi qui faisait la base de tout l'édifice affaiblirait toutes les
précautions nouvelles qu'on prendrait pour les soutenir\,; qu'il était
enfin plus à propos de s'abandonner à la disposition de ce traité, et de
croire que la clause insérée en faveur de la maison de Savoie, valait
une renonciation du roi et du régent que de troubler la France en lui
demandant une assemblée d'états, dangereuse et principalement odieuse
dans un temps de minorité. Ainsi rien ne les embarrassait, pas même les
murmures de la nation, qui voyait avec peine les apprêts d'une guerre
prochaine avec l'Espagne. Les négociants, uniquement touchés de
l'intérêt du commerce, ne dissimulaient pas à quel point leur déplaisait
une rupture sans prétexte, sans avantage pour les Îles Britanniques,
uniquement utile aux intérêts de l'empereur, et par conséquent aux vues
d'agrandissement et d'affermissement qu'un roi d'Angleterre, duc de
Hanovre, pouvait avoir en Allemagne. De telles vues paraissaient très
dangereuses, bien loin d'être conformes à l'intérêt et à la liberté de
la nation\,; mais étant assujettie, et n'ayant d'autre pouvoir que de
former des voeux, elle souhaitait et elle espérait qu'une guerre si mal
entreprise produirait la ruine du ministère, consolation et ressource
ordinaire des Anglais.

Les ministres d'Angleterre parurent alors aussi contents du mouvement
que Châteauneuf se donnait en Hollande pour engager la république à
souscrire à l'alliance, qu'ils avaient paru précédemment mal satisfaits
de la mollesse et de la partialité dont ils avaient accusé plusieurs
fois cet ambassadeur dans les plaintes qu'ils en avaient portées au
régent. Ils commencèrent à louer son zèle, sa vigilance, son industrie,
sa sincérité à leur égard, la vigueur qu'il faisait paraître dans ses
discours. Ils lui donnèrent ces louanges comme à dessein de réparer ce
qu'ils en avaient dit précédemment à son préjudice, et comme un effet de
la justice qu'ils croyaient devoir à ses bonnes intentions présentes et
à son activité. Ce nouveau langage tenu par les Anglais fut une raison
nouvelle à Beretti de changer de style à l'égard de Châteauneuf. Beretti
avait assuré plusieurs fois en Espagne qu'il ferait si bien par ses
manèges, que la Hollande ne souscrirait pas au projet proposé par
l'Angleterre. Il voyait qu'il ne pouvait plus parler si affirmativement,
et que chaque fois que les états de la province de Hollande
s'assemblaient, il avait lieu de craindre qu'ils ne prissent la
résolution de souscrire au traité. Il fallait donc pour son honneur
préparer la cour d'Espagne à un événement qui pouvait arriver d'un jour
à l'autre, et comme c'était pour lui une espèce de rétractation que
d'annoncer ce qu'il craignait, le seul moyen d'éviter de se rendre
garant de ce qu'il avait avancé était d'attribuer le changement des
Hollandais aux sollicitations impétueuses, disait-il, de la France,
assurant que, si cette couronne ne s'était mêlée de la négociation
commencée par les Anglais, jamais leurs propositions n'auraient été
écoutées, qu'elles n'auraient pas même été mises en délibération, car
outre que les États généraux étaient bien résolus d'éviter tout
engagement capable d'entraîner une rupture avec le roi d'Espagne, et de
causer, par conséquent, un préjudice extrême à leur commerce, la
défiance qu'ils avaient depuis longtemps des Anglais augmentait tous les
jours.

Beretti prétendait qu'elle était montée d'un nouveau degré depuis qu'il
avait découvert aux députés de la province de Hollande que l'Angleterre
offrait au roi d'Espagne de lui remettre Gibraltar. Une telle offre
faisait juger que le roi d'Angleterre obtiendrait de nouvelles
prérogatives pour le commerce de la nation\,; que même il était déjà sûr
des avantages que le roi d'Espagne lui accorderait, puisqu'il n'était
pas vraisemblable que sans cette considération, un prince tenace
désirant toujours d'acquérir, ayant à répondre à des peuples également
avides, voulût abandonner et céder gratuitement une acquisition que la
couronne d'Angleterre avait faite sous le règne précédent. Le mystère de
cette négociation inconnue aux Hollandais fournit encore à Beretti
matière à leur faire soupçonner des embûches, et d'exciter de leur part
la jalousie si facile et si naturelle entre deux nations si intéressées
au commerce. Toute défiance sur cet article est un moyen sûr d'inquiéter
et d'alarmer la république de Hollande. Ainsi, Beretti fit répandre le
bruit dans les provinces maritimes que le roi d'Espagne prenait déjà des
mesures pour découvrir dans son royaume les effets appartenant aux
négociants nationaux des royaumes et pays qui avaient abusé des grâces
que Sa Majesté Catholique accordait pour la facilité de leur commerce.
Mais, malgré l'industrie dont Beretti se vantait, il s'apercevait que,
les moyens qu'il employait étaient de faibles ressources. Il avouait
donc que la cabale contre l'Espagne était trop forte, et ne trouvait en
quelque façon de consolation que dans la honte qui rejaillissait,
disait-il, sur la France des démarches que son ambassadeur faisait à la
Haye, démarches si basses, disait-il, qu'elle avait été obligée de les
dénier dans le temps même qu'elles se faisaient. Il les attribuait à
l'abbé Dubois, grand moteur de la machine, dont il prétendait connaître
parfaitement la manoeuvre et le mauvais esprit, et avoir averti
plusieurs fois Cellamare de prendre garde aux intentions et à la
conduite de la France.

Cellamare, de son côté, assura le roi son maître que, suivant ses
ordres, il avait parlé très fortement au maréchal d'Huxelles\,; qu'il
n'avait pas ménagé les termes\,; qu'il avait clairement fait connaître
les sujets que le roi d'Espagne avait de se plaindre des instances que
la France faisait pour engager la république de Hollande dans une
alliance, et vraisemblablement dans une guerre contre Sa Majesté
Catholique, instances plus vives et plus pressantes que ne l'étaient
celles que l'Angleterre même faisait à cette république. À ces
représentations l'ambassadeur d'Espagne avait ajouté quelque espèce de
menaces\,; mais il ne comptait nullement sur l'effet que ses plaintes,
ses protestations et ses clameurs pourraient produire. L'engagement
était pris, et Cellamare comprenait que, quoi qu'il pût dire pour
décrier la quadruple alliance, ses discours n'obligeraient pas le régent
à faire le moindre pas en arrière\,; qu'en vain les ministres d'Espagne
répandraient de tous côtés qu'un tel traité scandalisait toute l'Europe,
Son Altesse Royale suivrait toujours son objet\,; qu'elle travaillait
constamment à l'affermissement d'une paix qui assurait ses intérêts
particuliers, et qu'elle ne s'embarrasserait que des moyens de faire
réussir ses vues. Il y avait peu detemps qu'on avait reçu avis en France
que Martinet, Français, officier de marine, actuellement au service
d'Espagne avait pris dans la mer du Sud six vaisseaux français qui
faisaient le commerce de la contrebande. Il paraissait impossible
d'obtenir la restitution de ces vaisseaux. Cellamare avertit le roi
d'Espagne que les particuliers intéressés en cette perte, jugeant bien
que toute négociation sur un point si délicat pour l'Espagne serait
absolument inutile, prenaient le parti d'armer en Hollande et en
Angleterre quatre frégates, qu'ils enverraient sous le pavillon de
l'empereur au-devant des vaisseaux espagnols chargés des effets pris, et
qu'après avoir enlevé leurs charges, ces frégates les rapporteraient
dans les ports de France. Si l'ambassadeur d'Espagne servait fidèlement
son maître en lui donnant de pareils avis, il s'en fallait beaucoup
qu'il ne rendit des services aussi utiles à ce prince, lorsque, croyant
lui faire sa cour, il l'assurait que les Français, presque généralement,
détestaient la conduite du régent\,; qu'ils ne pouvaient souffrir qu'il
n'eût pas pris le parti sage, et seul convenable, de s'unir à l'Espagne,
et d'agir de concert avec elle et le roi de Sicile contre la maison
d'Autriche. Les suites firent voir que Cellamare ne s'en tint pas à ces
simples assurances. Toutefois il se défiait lui-même de ce qu'il
avançait à la cour de Madrid, dans la seule vue vraisemblablement de
plaire et de flatter\,; car en même temps il exhortait son oncle à Rome
à demeurer dans une espèce de neutralité, persuadé que toute
détermination serait dangereuse d'un côté ou d'autre, jusqu'à ce que le
sort douteux de la Sicile fût décidé.

On ignorait encore si l'armement d'Espagne avait pour objet la conquête
de cette île. Ceux des ministres du roi de Sicile, qui croyaient avoir
plus lieu de le craindre, se flattaient que l'empereur s'opposerait au
succès d'un e pareille entreprise, et que les forces qu'il avait en
Italie suffiraient pour l'empêcher. D'ailleurs on ne comptait point à
Turin sur l'assistance de la France\,; et Provane, qui était à Paris, ne
cessait d'assurer son maître que le régent sacrifierait sans peine les
intérêts de la maison de Savoie, quand il le croirait nécessaire,
persuadé qu'il n'avait rien à craindre ni à espérer d'elle. Toutefois
Provane demeura longtemps incertain des véritables sentiments de Son
Altesse Royale. Il crut qu'elle était inquiète des menaces personnelles
que l'ambassadeur d'Espagne laissait entendre qu'il lui avait faites du
ressentiment du roi d'Espagne, et qu'alarmée des suites, elle désirerait
n'avoir pas pris d'engagement sur le plan proposé par la cour
d'Angleterre. Il y avait même des gens qui assuraient Provane qu'elle
s'en dégagerait volontiers si elle trouvait quelque bon expédient pour
rompre cette liaison fatale, parce qu'elle commençait à connaître que
c'était en vain qu'elle s'était flattée d'obliger le roi d'Espagne de
souscrire au projet, et qu'enfin ni l'espérance de la succession des
États de Parme et de Toscane, ni la crainte de la quadruple alliance, ni
celle de l'accommodement prétendu du roi de Sicile avec l'empereur, que
le régent avait regardé comme un moyen, infaillible de persuader Sa
Majesté Catholique, ne suffisaient pas pour faire impression sur son
esprit.

Mais Provane, et ceux qui lui donnaient dés avis, se trompaient
également, et dans le temps qu'ils supposaient quelque incertitude dans
l'esprit du régent, Stairs louait, au contraire, la fermeté de Son
Altesse Royale, étant sûr qu'elle était résolue à signer le traité, dès
le moment que Penterrieder aurait reçu l'ordre de le signer au nom, de
l'empereur, événement d'autant plus important que les ministres
d'Angleterre étaient alors persuadés que l'objet principal de la reine
d'Espagne et d'Albéroni était de ménager et de se conserver toujours une
ouverture à la succession de la couronne de France, se flattant l'un et
l'autre que la branche d'Espagne avait un grand parti dans le royaume\,;
que, cultivant ceux qui lui étaient attachés, et se faisant de nouveaux
amis, elle y serait un jour assez puissante pour exclure M. le duc
d'Orléans, et y placer un des fils du roi d'Espagne, système absolument
opposé aux dispositions que l'Angleterre et la Hollande avaient faites
pour empêcher à jamais l'union des deux couronnes, même la trop grande
intelligence entre les deux branches de la maison royale, et maintenir
en les divisant l'équilibre de l'Europe, objet que le ministère
d'Angleterre présentait pour faire valoir aux autres nations ce que le
roi Georges, prince d'Allemagne, porté par les vues de son intérêt
particulier à ménager l'empereur, faisait aux dépens des Anglais pour
agrandir la puissance de la maison d'Autriche\,; car en même temps qu'il
protestait au roi d'Espagne que ses intentions et ses vues concouraient
toutes au véritable intérêt de Sa Majesté Catholique, les Anglais
déclaraient, avec beaucoup de franchise, que l'escadre armée dans leurs
ports était destinée à s'opposer à toutes entreprises que les Espagnols
tenteraient en Italie. En vain les ministres d'Espagne en France et en
Hollande tâchaient de profiter au moins du bénéfice du temps, leurs
ménagements, leurs instances, les représentations réitérées qu'ils
faisaient, lorsqu'ils croyaient que quelque difficulté survenue à la
négociation pouvait en interrompre le cours, rien de leur part ne
produisait l'effet qu'ils désiraient\,; et Cellamare avouait qu'il
regardait comme absolument inutiles les sollicitations les plus fortes
qu'il faisait, parce que le régent était tellement aheurté à mettre
l'Espagne en, paix, malgré qu'elle en eût, que ni promesses, ni menaces
de la part du roi d'Espagne ne pouvaient détourner Son Altesse Royale du
projet qu'elle avait formé.

Les instances de l'ambassadeur d'Espagne en Angleterre ne furent pas
plus heureuses. Monteléon, pressé par les ordres réitérés qu'il recevait
de la cour de Madrid, fut enfin obligé, malgré lui, d'en venir aux
menaces. Il déclara donc au comte de Stanhope que, si l'escadre Anglaise
destinée pour la Méditerranée faisait la moindre, hostilité, ou si elle
causait le moindre dommage à l'Espagne, toute la nation Anglaise
généralement s'en ressentirait, et que le prochain parlement de la
Grande-Bretagne vengerait Sa Majesté Catholique. Stanhope, facile à
prendre feu, n'écouta pas tranquillement les menaces de l'Espagne\,; il
suivit son penchant naturel, et renchérit, par un emportement qui ne lui
coûtait rien, sur les discours que Monteléon lui avait tenus. Tous deux
se calmèrent, l'un plus facilement que l'autre\,; et Stanhope, revenu
avec peine, tâcha de faire voir que le roi son maître, plein de bonnes
intentions pour le roi d'Espagne, agissait pour le véritable bien de Sa
Majesté Catholique en faisant passer une escadre dans la Méditerranée.
Pour soutenir un tel paradoxe, il établit, comme un principe
incontestable, que le projet du traité était ce qu'on pouvait imaginer
de mieux pour le roi d'Espagne\,; qu'il était indubitable par cette
raison que l'empereur s'opposerait à sa conclusion, et que cette opinion
n'était que trop bien fondée, puisque ce prince hésitait encore à
souscrire à l'alliance. Comme elle était tout à l'avantage de l'Espagne,
suivant les principes de Stanhope, le roi d'Angleterre avait
essentiellement travaillé pour les véritables intérêts du roi d'Espagne
en armant une escadre et la faisant actuellement passer dans la
Méditerranée, uniquement à dessein de s'opposer à la mauvaise volonté de
l'empereur, et d'empêcher le trouble que ce prince apporterait à
l'exécution des vues formées pour l'avantage du roi d'Espagne, si les
Allemands avaient la liberté d'agir, et s'ils n'étaient retenus par une
puissance telle que serait celle que l'Angleterre ferait agir par mer.
Mais comme il était juste que cette couronne tînt une balance à peu près
égale entre l'empereur et le roi d'Espagne, Stanhope ajouta que ce
serait abuser Sa Majesté Catholique que de lui laisser croire que
l'Angleterre, faisant autant qu'elle faisait pour la maison royale
d'Espagne, pût demeurer dans l'indifférence, si les armes espagnoles se
portaient à quelque entreprise contraire à la tranquillité des États que
l'empereur possédait en Italie. On croit que Stanhope poussa le
raisonnement jusqu'à vouloir prouver à Monteléon que c'était servir
réellement le roi d'Espagne que de traverser et faire échouer toutes les
entreprises de cette nature, parce qu'elles rallumeraient la guerre en
Italie, et qu'il était de l'intérêt essentiel de ce prince d'y maintenir
la paix.

Monteléon, persuadé ou non, demanda une réponse par écrit. Elle lui fut
promise\,; et quelques jours après, ayant réitéré la même demande dans
une conférence qu'il eut avec les trois ministres principaux du roi
d'Angleterre, Stanhope, Sunderland et Craggs, la réponse par écrit lui
fut remise, mieux digérée et disposée avec plus d'ordre qu'il ne l'avait
reçue de Stanhope. Monteléon désira de l'avoir pour sa justification
personnelle auprès du roi son maître, car Albéroni ne cessait de lui
reprocher une tranquillité coupable sur les intérêts de Sa Majesté
Catholique, et une confiance outrée aux paroles et aux conseils de
l'abbé Dubois. Il fallait donc faire voir, par un écrit des ministres
d'Angleterre, que les comptes qu'il rendait de leurs, sentiments et de
leurs expressions était exact et fidèle. Il avait d'ailleurs à Londres
des surveillants très attentifs à sa conduite, observant jusqu'à la
moindre de ses démarches. L'un était l'agent de Sicile, l'autre celui du
duc de Parme. Tous deux l'interrogeaient sur chaque pas qu'il faisait et
sur les ordres qu'il recevait. Il se croyait obligé de ménager le
ministre de Parme, dans la vue de se conserver la protection du duc de
Parme auprès de la reine\,; mais quelque inclination qu'il eût pour le
roi de Sicile, il était un peu plus réservé à l'égard de son ministre.
Toutefois Monteléon, affectant à son égard une apparence de confiance,
l'informait des choses qu'il ne pouvait lui cacher. Il y ajoutait
souvent que, pourvu que le roi de Sicile tînt ferme avec l'Espagne, on
pourrait enfin dissiper le nuage\,; mais cette apparente cordialité
n'alla pas jusqu'au point de lui communiquer la réponse par écrit des
ministres d'Angleterre. Monteléon se fit un mérite auprès d'Albéroni de
sa discrétion. Il assura le premier ministre qu'il avait voulu le
laisser maître de communiquer cette réponse à l'ambassadeur de Sicile à
Madrid, ou de lui en dérober la connaissance suivant qu'il le jugerait
plus à propos\,; et pour se justifier du reproche de trop de confiance
en l'abbé Dubois, il assura qu'il évitait de le voir, chose aisée, parce
qu'alors l'abbé Dubois demeurait renfermé dans sa maison à Londres, et
ne se montrait ni à la cour ni ailleurs.

\hypertarget{chapitre-vi.}{%
\chapter{CHAPITRE VI.}\label{chapitre-vi.}}

1718

~

{\textsc{Départ de l'escadre anglaise pour la Méditerranée.}} {\textsc{-
Fourberie de Stanhope à Monteléon.}} {\textsc{- Propos d'Albéroni.}}
{\textsc{- Maladie et guérison du roi d'Espagne.}} {\textsc{- Vanteries
d'Albéroni.}} {\textsc{- Secret du dessein de son expédition.}}
{\textsc{- Défiance du roi de Sicile de ceux même qu'il emploie au
dehors.}} {\textsc{- Leurs différents avis.}} {\textsc{- Ministres
d'Espagne au dehors déclarent que le roi d'Espagne n'acceptera point le
traité.}} {\textsc{- Détail des forces d'Espagne fait en Angleterre avec
menaces.}} {\textsc{- Albéroni déclame contre le roi d'Angleterre et
contre le régent.}} {\textsc{- Albéroni se loue de Nancré\,; lui impose
silence sur le traité\,; peint bien l'abbé Dubois\,; menace\,; donne aux
Espagnols des louanges artificieuses.}} {\textsc{- Il a un fort
entretien avec le colonel Stanhope, qui avertit tous les consuls anglais
de retirer les effets de leurs négociants.}} {\textsc{- Inquiétude des
ministres de Sicile à Madrid.}} {\textsc{- Fourberie insigne
d'Albéroni.}} {\textsc{- Forte et menaçante déclaration de l'Espagne aux
Hollandais.}} {\textsc{- Avis contradictoire d'Aldovrandi au pape sur
Albéroni.}} {\textsc{- Plaintes du pape contre l'Espagne qui rompt avec
lui, sur le refus des bulles de Séville pour Albéroni.}} {\textsc{-
Conduite de Giudice à l'occasion de la rupture de l'Espagne, avec
Rome.}} {\textsc{- Il ôte enfin les armes d'Espagne de dessus sa
porte\,; craint les Impériaux et meurt d'envie de s'attacher à eux\,;
avertit et blâme la conduite de Cellamare à leur égard.}} {\textsc{- Le
pape menacé par l'ambassadeur de l'empereur.}} {\textsc{- Malice
d'Acquaviva contre les Giudice.}} {\textsc{- Dangereuses pratiques de
Cellamare en France.}} {\textsc{- Secret et précautions.}} {\textsc{-
Ses espérances.}} {\textsc{- Embarras domestiques du régent, considérés
différemment par les ministres étrangers à Paris.}} {\textsc{-
Koenigseck, ambassadeur de l'empereur à Paris, gémit de la cour de
Vienne et de ses ministres.}} {\textsc{- Garnisons.}} {\textsc{-
Conduite insolente de Stairs.}}

~

Enfin le moment du départ de l'escadre anglaise destinée pour la
Méditerranée arriva. Comme elle était prête à mettre à la voile,
Stanhope dit à Monteléon que l'amiral Bing, qui la commandait, avait
ordre d'user d'une bonne correspondance avec l'Espagne. Monteléon
demanda si le cas fatal aux deux rois et aux deux nations arriverait, et
si l'Angleterre s'opposerait aux desseins du roi d'Espagne. Stanhope
répondit, en termes généraux, qu'il espérait que cette occasion ne se
présenterait pas\,; que le roi d'Angleterre et son ministère avaient
toujours devant les yeux combien il leur importait de maintenir l'amitié
et la bonne correspondance avec l'Espagne, aussi bien que les
inconvénients et le préjudice d'une rupture\,; que le temps et les
effets dissiperaient, les mauvaises impressions et l'opinion sinistre
qu'on avait à Madrid de leurs intentions. En effet, cette opinion ne
pouvait être plus mauvaise. Le roi d'Espagne était non seulement
persuadé de la partialité du roi d'Angleterre pour l'empereur, mais de
plus Sa Majesté Catholique déplorait le malheur général de l'Europe et
l'esclavage dont plusieurs nations étaient menacées, si les projets que
la France et l'Angleterre soutenaient avec tant d'efforts réussissaient
en faveur de la maison d'Autriche.

Albéroni, pour lors arbitre absolu des sentiments et des décisions de
son maître, protestait que jamais ce prince ne subirait la dure loi que
ceux qui se disaient ses meilleurs amis voulaient lui imposer\,; que
s'il cédait, ce ne serait que lorsqu'il y serait forcé par la nécessité
et qu'il ne serait plus maître d'agir contre ses propres intérêts\,;
qu'il adorait les jugements impénétrables de Dieu, et qu'il prévoyait
que quelque jour les mêmes puissances, qui travaillaient à augmenter
celle d'un prince dont elles devaient redouter les desseins ambitieux,
regretteraient amèrement les secours qu'elles lui donnaient avec tant de
zèle pour s'élever à leur préjudice. Le cardinal prétendait que Nancré
même, venu à la cour d'Espagne comme ministre confident du régent, était
honteux de sa commission\,; que, ne pouvant répondre aux justes plaintes
que le roi d'Espagne faisait de la conduite et des démarches de ce
prince, il se contentait de lever les épaules et de dire qu'il était
trop engagé pour reculer, et d'avouer en même temps qu'il avait bien
prévu que son voyage en Espagne aurait un triste succès.

Cette cour, ou pour mieux dire la reine et le premier ministre, avaient
eu de grands sujets d'alarme causés par une maladie opiniâtre du foi
d'Espagne, dont les médecins auguraient mal et ne pouvaient le guérir.
Sa santé se rétablit enfin d'elle-même sans remèdes, et la fièvre le
quitta après beaucoup d'accès et différentes rechutes. On ne manqua pas
de publier avec soin sa guérison\,; et Albéroni réitéra, surtout en
Italie, les descriptions magnifiques qu'il avait déjà faites de l'état
de la flotte espagnole, de celui de l'armement destiné à faire une
descente, des provisions de vivres, d'artillerie, et généralement de
toutes les précautions qu'il avait prises pour assurer le dessein dont
il gardait encore le secret. Enfin il voulait que le monde vît que
l'Espagne n'était plus un cadavre, et que l'administration d'un ministre
habile, pendant un an et demi, avait mis ce royaume en état d'armer et
habiller soixante-cinq mille hommes effectifs, et de former une marine,
de construire actuellement douze navires chacun de quatre-vingts pièces
de canon, de fondre cent cinquante pièces d'artillerie, et de bâtir à
Barcelone une des plus belles citadelles de l'Europe. Il envisageait
comme un moyen de fournir à tant de dépenses le retour prochain de
quatorze vaisseaux envoyés en Amérique pour le compte seul du roi
d'Espagne, et ce qui marquait à quel point la puissance de ce prince
imposait au dehors était l'empressement que le duc de Savoie témoignait
de s'unir à Sa Majesté Catholique, offrant d'envoyer exprès à Madrid un
ministre muni de pouvoirs pour traiter. Il aurait été le quatrième de
ceux que ce prince avait à la cour d'Espagne. L'abbé del Maro, son
ambassadeur, quoique rappelé, n'en était pas encore parti. Il y avait
envoyé quelque temps auparavant Lascaris comme ministre de confiance,
dont il n'avait cependant que l'apparence. Un nommé Corderi, secrétaire
d'ambassade, paraissait être plus du goût du roi son maître\,; toutefois
il n'avait pas encore son secret. Aucun de ces ministres et agents du
roi de Sicile n'avait pu pénétrer quel était le véritable objet de
l'armement d'Espagne. Del Maro, mécontent de cette cour, assurait depuis
longtemps que l'entreprise regardait la Sicile\,; Lascaris, espérant
encore de réussir où l'ambassadeur avait échoué, assurait son maître que
c'était Naples. Il élevait le bon état et la puissance de l'Espagne, et
par ses relations il insinuait à son maître que le meilleur parti qu'il
eût à prendre était de traiter avec cette couronne. Corderi, souhaitant
de prolonger son emploi, écrivait douteusement. Il représentait le roi
d'Espagne comme encore indéterminé dans ses résolutions\,; il répandait
des doutes sur l'état de la négociation de Nancré\,; et n'étant pas
informé de ce qu'il s'y passait, il croyait utile à ses vues
particulières de laisser entrevoir à son maître qu'Albéroni et Nancré
étaient entre eux plus d'accord que le public n'avait lieu de le
croire\,; il était d'ailleurs l'espion de Lascaris. Moyennant les
différentes affections de ces trois ministres, le roi de Sicile était
très mal informé d'un projet dont la connaissance était si importante à
ses intérêts.

Si la bonne foi d'Albéroni eût été moins suspecte, qui que ce soit
n'aurait douté de la résolution ferme et constante, que le roi d'Espagne
avait prise, de rompre toute négociation et d'entrer incessamment en
guerre\,; car il n'y avait pas d'occasion où le cardinal ne déclarât
nettement les intentions de Sa Majesté Catholique sur ce sujet. Ses
ministres au dehors avaient ordre d'en parler avec la même franchise.
Monteléon, peut-être parce qu'il était plus suspect, reçut des ordres
plus précis qu'aucun autre de déclarer que le roi son maître ne
consentirait jamais à l'indigne projet qu'on lui proposait, son honneur
exigeant qu'il pérît plutôt que de recevoir une loi dont sa dignité et
l'intérêt de sa couronne souffriraient un égal préjudice, loi très
fatale d'ailleurs au bien général de l'Europe. Monteléon devait dire
aussi que Sa Majesté Catholique attendait de savoir quels ordres le roi
d'Angleterre donnerait à l'escadre qu'il faisait passer dans la
Méditerranée, afin de régler de son côté les mesures, qu'elle aurait à
prendre\,; que, si elle, n'avait pu gagner l'amitié du roi Georges, elle
voulait au moins gagner son estime. Pour appuyer une telle déclaration,
Albéroni fit une nouvelle énumération des forces d'Espagne. Cette
couronne, disait-il, réveillée de sa léthargie, fait ce que nulle
puissance n'a fait encore. Elle a plus de trois cent soixante voiles,
trente-trois mille hommes effectifs de débarquement, cent pièces de
canon de vingt-quatre, trente de campagne, quarante mortiers, trente
mille bombes et grenades, le reste à proportion\,; vingt mille quintaux
de poudre, quatre-vingt mille outils à remuer la terre, dix-huit mille
fusils de réserve, des vivres pour l'armée de terre et de mer jusqu'à la
fin du mois d'octobre, toutes les troupes armées, montées et vêtues de
neuf\,; enfin deux millions de pièces de huit embarquées, c'est-à-dire,
un million trois cent mille pièces en monnaie d'or et d'argent, le reste
en lettres de change sur Gènes et sur Livourne. Outre ces troupes, il
demeure quarante-deux mille hommes en Espagne. C'est en ces termes
qu'Albéroni s'expliquait à Monteléon au commencement de juin 1718,
avouant cependant que les hommes ayant fait ce qu'ils pouvaient, le
succès dépendait de la bénédiction de Dieu\,; mais ces dispositions
suffisaient, disait le cardinal, pour faire voir au roi d'Angleterre
qu'il se trompait s'il croyait traiter un roi d'Espagne à l'allemande\,;
car enfin Sa Majesté Catholique se mettait en état de faire de temps en
temps de ces sortes de coups qui devraient donner à penser à quelqu'un,
et si, plutôt que de porter ses forces en Italie, elle les eût fait
passer en Écosse sous le commandement de ce galant homme pour lors
relégué à Urbin et demandant secours à tout le monde, peut-être que le
roi Georges eût fait ses réflexions avant que d'envoyer une escadre dans
la Méditerranée\,; mais il paraissait que Dieu aveuglait ce seigneur,
permettant qu'il travaille contre son propre bien, et comme conduit par
un esprit d'erreur qui ne lui permettait pas de se laisser persuader par
les raisons les plus claires, les plus convaincantes et les plus
conformes à ses véritables intérêts.

Albéroni ne traitait pas le régent plus favorablement que le roi
d'Angleterre\,: tous deux selon lui ne pensaient qu'à leurs intérêts
particuliers, et tous deux prenaient, disait-il, de fausses routes pour
arriver à leur but. L'un, selon lui, sacrifiait à cet objet la nation
Anglaise, et, l'autre la française. Enfin, sortant des bornes du simple
raisonnement, il se porta jusqu'à dire à Nancré, de la part du roi
d'Espagne, de cesser absolument de parler du projet à Sa Majesté
Catholique, pour ne pas obliger sa patience royale à sortir des règles
usitées à l'égard des ministres étrangers. Cette espèce de menace ne
regardait pas personnellement Nancré, car Albéroni déclara souvent qu'il
avait lieu d'être content de sa conduite\,; qu'elle ne pouvait être plus
sage ni plus mesurée, ayant une mauvaise cause à défendre. Il ajoutait à
cet éloge un parallèle peu obligeant pour l'abbé Dubois, qu'il traitait
de nouveau ministre, d'artisan de chimères, agent des passions d'autrui
(point du tout, mais des siennes), d'homme qui avait mis tout son génie
à vendre et à débiter ses artifices par cabale et par mille menteries
(c'était bien là le vrai portrait de tous les deux), mais dont
l'orviétan trouvait peu de débit, parce que tout homme d'honneur était
persuadé que ses manèges n'aboutiraient qu'à décréditer son maître et à
l'engager dans le précipice. La conséquence et la conclusion de tous ces
discours étaient que ceux qui se donnaient pour amis du roi d'Espagne
avaient enfui poussé son flegme au point de jouer à jeu découvert, et de
prendre en main toutes les armes qu'il croirait utiles à la défense de
son honneur et de la monarchie\,; qu'il serait vaillamment secondé par
la nation espagnole généralement occupée du désir de contribuer de son
sang, de son bien, enfin de tout ce qu'elle possédait, pour servir le
roi son maître, qu'elle était transportée de joie de voir une narine et
tant de forces, que Sa Majesté Catholique avait mises sur pied\,; que
les Espagnols disaient unanimement\,: si l'on avait tant fait en peu de
temps, que pourrait-on faire à l'avenir\,? que le moindre d'entre eux se
croyait conquérant de nouveaux mondes\,; que l'Espagne enfin était en
pleine mer, et qu'il fallait ou périr ou parvenir au port. Albéroni
s'expliqua dans le même sens et dans les mêmes termes à peu près avec le
colonel Stanhope.

Cet envoyé avait reçu de Londres l'ordre de représenter les raisons qui
empêchaient le roi d'Angleterre d'acquiescer à la proposition que le roi
d'Espagne avait faite, de garder la Sardaigne en souscrivant au projet
du traité. Stanhope crut adoucir ce refus en l'ornant de toutes les
expositions que le roi son maître lui avait prescrites, pour persuader
le cardinal que ce prince était plus touché que personne de l'honneur et
des intérêts de Sa Majesté Catholique, et que c'était même en cette
considération qu'il croyait important de ne rien innover au projet de
traité, parce qu'il fallait éviter de fournir à l'empereur le moindre
prétexte de changer de sentiment, au moment qu'il dépendait de lui de
faire la paix avec les Turcs. Albéroni ne parut point touché de ces
marques de considération., que Stanhope lui voulait faire valoir. Il
répondit qu'il regardait toujours le plan comme désavantageux,
déshonorant pour l'Espagne, et comme dressé avec beaucoup de partialité
en faveur de l'empereur\,; que, si le roi d'Angleterre et le régent
étaient résolus à refuser tout changement, le roi d'Espagne l'était
aussi de rejeter tout l'ouvrage, et que, par cette raison, il était
inutile de traiter davantage\,; qu'il attaquerait l'empereur avec toute
la vigueur possible, quand même toute l'Europe le menacerait de lui
déclarer la guerre, qu'il en attendait l'effet avant que de changer de
résolution\,; que, si les événements lui étaient contraires, il se
retirerait auprès de sa cheminée, et tâcherait de s'y défendre, n'étant
pas assez don Quichotte pour attaquer tout le genre humain\,; mais aussi
qu'il aurait l'avantage de connaître ses ennemis, et que peut-être il
trouverait le temps et l'occasion de faire sentir sa vengeance\,; qu'il
préférait donc un parti honorable à celui de se soumettre à des
conditions infâmes. Cette déclaration fut soutenue d'une description
pompeuse des forces d'Espagne. Si le pouvoir de cette couronne était
demeuré comme éclipsé pendant plusieurs siècles, la faute\,; dit
Albéroni, devait en être imputée, à ceux qui, se trouvant à la tête des
affaires, les avaient follement et pitoyablement administrées. Mais au
moment présent les finances du roi d'Espagne étaient dans un état
florissant. Ce prince ne devait rien, son bonheur ayant été de manquer
de crédit pour emprunter dans les conjonctures fatales où il aurait
regardé comme un bien les moyens de se ruiner. Il pouvait donc, disait
le cardinal, soutenir désormais la guerre sans le secours de personne,
et déjà les fonds étaient réglés pour les dépenses d'une seconde
campagne.

L'ostentation d'un pouvoir, dont il était permis aux étrangers de
douter, aurait peut-être fait peu d'impression sur les Anglais. Comme il
fallait les toucher par quelque intérêt plus sensible et plus pressant
pour la nation, Albéroni déclara nettement à l'envoyé d'Angleterre que
le roi d'Espagne ne permettrait pas à la compagnie Anglaise du Sud
d'envoyer dans le cours de cette même année le vaisseau qu'elle avait
droit de faire passer tous les ans dans les Indes espagnoles, en vertu
du traité d'Utrecht. Ce refus n'était ni l'effet ni l'apparence d'une
rupture prochaine. Albéroni prit pour prétexte l'excès des marchandises
d'Europe portées aux Indes en contrebande, et promit qu'au lieu d'un
vaisseau les Anglais auraient l'année suivante permission d'en envoyer
deux dans la mer du Sud. Mais en même temps qu'il relevait l'avantage
que la nation Anglaise retirerait de ce changement, il ne put s'empêcher
de laisser échapper avec colère, soit malgré lui, soit à dessein\,; que
l'Espagne n'aurait plus d'égard aux traités faits avec l'Angleterre\,;
que Stanhope ne recevrait désormais aucune réponse favorable sur les
mémoires qu'il pourrait donner, parce que, dans la situation où se
trouvaient les affaires, le roi catholique n'avait que trop de sujets de
regarder le roi d'Angleterre comme ennemi. Stanhope, étonné de
l'emportement du cardinal, et persuadé que les menaces qu'il laissait
échapper seraient suivies de l'effet prochain, crut à propos de lui
représenter qu'au moins, en cas de rupture, les traités fixaient un
temps aux marchands des deux nations pour retirer leurs personnes et
leurs effets. Albéroni répondit avec encore plus de chaleur
qu'auparavant, que sitôt que l'escadre Anglaise paraîtrait dans la
Méditerranée, les Anglais devaient s'attendre à être maltraités dans
toutes les circonstances imaginables. Les vivacités d'Albéroni furent
mêlées de mots entrecoupés du prétendant, de dispositions que le
parlement prochain de la Grande-Bretagne témoignerait vraisemblablement
à l'égard de la guerre d'Espagne, de raisonnements et de pronostics sur
la nécessité où l'Espagne et l'Angleterre se trouveraient
indispensablement réduites de périr l'une ou l'autre\,; enfin de tant de
mouvements de colère, et si vifs, de la part du premier ministre, que
Stanhope, au sortir de l'audience, dépêcha sur-le-champ des courriers
aux consuls Anglais de tous les ports d'Espagne pour leur enjoindre de
mettre sous leur garde tous les effets appartenant aux marchands de leur
nation. On doutait cependant encore à Madrid des intentions du roi
d'Espagne. Quelques ordres donnés pour différer de quelques jours le
départ de la flotte fit croire que Sa Majesté Catholique pourrait enfin
accepter le projet, malgré tant de démonstrations contraires qu'elle
avait données au public. Les ministres de Sicile parurent plus inquiets
et plus alarmés du soupçon qu'ils eurent d'une intelligence prochaine du
roi d'Espagne avec l'empereur, que de la crainte qu'ils avaient eue que
la Sicile ne fût effectivement l'objet de l'entreprise. Lascaris, entre
autres, observa qu'Albéroni ne donnait que le titre de duc de Savoie au
roi de Sicile, dans une lettre que ce premier ministre lui communiqua,
et qu'il écrivait au prince de Cellamare. C'était un grand sujet de
réflexions pour les ministres d'un prince défiant, qui d'ailleurs
soupçonnaient avec beaucoup de raison la bonne foi et la sincérité du
cardinal.

Il était parvenu à persuader au nonce Aldovrandi que c'était contre son
avis et contre son sentiment que le roi d'Espagne s'engageait dans la
guerre. Il se lit même honneur d'avoir disposé ce prince à
l'accommodement\,; mais il prétendit que toutes ses mesures avaient été
rompues par l'opiniâtreté de la reine, si entêtée du projet de guerre,
et des avantages particuliers qu'elle se proposait d'en tirer, qu'il y
avait eu à cette occasion une contestation très vive entre le roi et
elle\,; que, se regardant elle-même, elle ne pouvait renoncer aux vastes
espérances qu'elle avait conçues du succès, et que, quoique tout le
monde le regardât comme impossible, elle persistait cependant dans
l'idée qu'elle avait formée dès le commencement\,; qu'elle se fiait en
la force des armées de terre et de mer jusqu'au point de croire que la
France ne pressait la paix que poussée par la crainte qu'elle avait des
succès et du pouvoir du roi d'Espagne. C'était à cette raison que le
cardinal attribua l'inutilité des dernières instances de Nancré, qui
avait déclaré formellement que la France et l'Angleterre s'opposeraient
de toutes leurs forces aux entreprises de l'Espagne. L'autorité de la
reine avait tout entraîné sans laisser le moindre crédit aux, avis
contraires au sien. Albéroni, voulant flatter Rome, laissa croire qu'il
avait proposé au roi d'Espagne de faire passer sa flotte en Afrique,
d'employer ses troupes à faire la conquête d'Oran, à délivrer Ceuta, et
ruiner Alger par les bombes. Il demanda cependant un profond secret d'un
projet qui pouvait réussir encore si le roi d'Espagne faisait la paix
avec l'empereur. Albéroni savait bien qu'un tel mystère serait de peu de
durée, car en même temps il fit savoir aux ministres d'Espagne employés
au dehors qu'il n'était plus question de parler d'un traité si contraire
à l'honneur du roi d'Espagne, et si fatal à ses intérêts\,; qu'il ne
céderait donc qu'au seul cas de la dernière extrémité, et que, se
conformant alors à la nécessité des temps, il attendrait des
conjonctures plus favorables pour reprendre les délibérations, et les
mesures qui conviendraient le mieux à son honneur.

Beretti eut ordre de déclarer particulièrement aux États-généraux les
sentiments du roi d'Espagne. Ce prince voulut qu'il leur dît en termes
clairs que jamais il ne se soumettrait à la loi dure et inique que la
France et l'Angleterre prétendaient lui imposer\,; qu'il n'admettait ni
n'admettrait jamais les conditions honteuses d'un projet qui blessait
également son honneur et sa satisfaction. Sa Majesté Catholique voulut
que son ambassadeur avertît les États généraux, comme puissance amie,
des engagements où le roi d'Angleterre et le régent avaient dessein de
les entraîner\,; qu'il ouvrît les yeux à ceux qui gouvernaient la
république, afin de leur découvrir et de leur faire éviter le piège où
on voulait les faire tomber, d'autant plus dangereux que ces deux
princes prétendaient pour leurs fins particulières conduire
effectivement cette république à sa ruine, sous l'apparence trompeuse de
ne vouloir point de guerre aux dépens même d'une paix de peu de durée.
Beretti eut ordre d'ajouter que le roi son maître serait affligé, même
offensé, si les États généraux se conduisaient en cette occasion d'une
manière contraire au bien public et à la continuation de l'amitié et de
la bonne correspondance\,; car ils forceraient Sa Majesté Catholique à
faire usage des conjonctures que le temps et la justice de sa cause lui
fourniraient, et ce serait à regret qu'elle se verrait obligée de
prendre les mesures et les résolutions qui lui conviendraient davantage.

La flotte avait déjà mis à la voile pour faire le trajet de Cadix à
Barcelone, lorsque ces déclarations furent faites. Aldovrandi avait déjà
employé son industrie à persuader le pape que les intentions d'Albéroni
étaient bonnes, et que, si les effets n'y répondaient pas, on devait
l'attribuer à la situation présente de l'Espagne, qui ne permettait pas
au premier ministre de faire généralement tout ce qu'il voulait, car il
avait à combattre les préventions de la reine, persuadée que son intérêt
`et celui de ses enfants était que la guerre se fît en Italie. Mais
lorsque la flotte fut partie, Aldovrandi, désabusé trop tard, changea de
sentiment à l'égard d'Albéroni. L'objet de l'entreprise était encore un
secret\,; mais le nonce ne douta plus que, quel que fût le dessein du
roi d'Espagne, l'Italie n'en sentît, le principal dommage, et tel que la
paix qui ne pouvait être éloignée ne réparerait pas les pertes, et
peut-être la destruction totale que la guerre lui aurait causée. Il
avertit le pape qu'il ne fallait compter ni sur la piété, ni sur les
bonnes intentions du roi d'Espagne, parce que ce prince souvent malade
était hors d'état de s'appliquer aux affaires, et qu'elles étaient
souverainement gouvernées par un premier ministre plein de ressentiment,
et vivement piqué des refus qu'il essuyait de la cour de Rome. Tout
était à craindre de sa vengeance, et le pape, naturellement porté à
s'alarmer facilement, avait lieu d'être encore plus intimidé par les
prédictions fâcheuses que lui faisait son ministre à Madrid, et par les
avis réitérés qu'il lui donnait de veiller sur toutes choses à prévenir
les premières tentatives que les troupes espagnoles pourraient faire sur
l'État ecclésiastique. Albéroni, de son côté, n'oubliait rien pour
augmenter les frayeurs du nonce et celles du pape. Il faisait dire à Sa
Sainteté que c'était elle qui servait plutôt que le roi d'Espagne, en la
pressant d'accorder les bulles de Séville, lui laissant assez entendre
ce qu'elle avait à craindre d'un plus long refus. Elle y persistait
cependant, et le cardinal Acquaviva, ayant inutilement insisté pour
vaincre sa résistance, se crut enfin obligé d'exécuter les ordres qu'il
avait reçus à Madrid, de rompre ouvertement avec la cour de Rome. Avant
que d'en venir à cette extrémité, il avait pris toutes les voies qu'il
croyait propres à persuader au pape de l'éviter\,; un accommodement avec
l'Espagne ne convenait pas à Sa Sainteté\,; elle était moins alarmée des
effets incertains du ressentiment du roi d'Espagne, qu'elle n'était
effrayée de la vengeance prochaine et facile dont les Allemands la
menaçaient continuellement, soit que l'empereur fût véritablement
persuadé d'une intelligence secrète entre la cour de Rome et celle de
Madrid, soit que ce prince crût de son intérêt de conserver longtemps un
pareil prétexte, dont il se servait utilement pour intimider
lepapeetpour le tenir dans une dépendance continuelle.

Les vues de l'empereur réussirent si bien qu'Acquaviva devint l'objet de
toute la colère de Sa Sainteté. Il ne reçut d'elle que des réponses
dures. Lorsqu'il insistait sur les bulles de Séville, il demandait des
réparations publiques et authentiques de tous les affronts et de tout le
préjudice que l'immunité ecclésiastique avait reçus en Espagne. Un des
principaux chefs sur cet article était le séquestre et l'emploi que le
roi d'Espagne avait fait pour son usage des revenus des églises vacantes
de Vich et de Tarragone, et la jouissance des revenus de celles de
Malaga et de Séville qu'Albéroni s'était en même temps attribués.
Toutefois, ne voulant pas que la rupture vînt de sa part, et suivant en
cette occasion son caractère incertain et indécis, {[}le pape{]} dit à
Acquaviva de conférer avec le cardinal Albane. Mais ces conférences ne
conduisirent à rien de certain, en sorte que les ordres du roi d'Espagne
étant précis et pressants, Acquaviva jugea qu'il devait enfin les
exécuter, et pour cet effet, il fit dire à tous les Espagnols qui
étaient à Rome d'en sortir incessamment. Ils obéirent tous, et leur
soumission surprit la cour de Rome. Le pape parut embarrassé, et laissa
voir qu'il n'aurait jamais cru que le roi d'Espagne prît une telle
résolution, et qu'il croyait encore moins que les ordres de Sa Majesté
Catholique fussent exécutés et suivis avec autant d'exactitude.

Le cardinal del Giudice, moins prompt à obéir, voulut tourner en
ridicule, et la résolution prise à Madrid, et l'effet qu'elle avait eu à
Rome. Il dit que cette expédition éclatante avait fait rire tout le
monde\,; que ceux qui voulaient flatter le conseil d'Espagne disaient
qu'elle avait été concertée avec le pape, et que le véritable dessein
était de tromper les Allemands et de leur déguiser l'intelligence
secrète que. Sa Sainteté avait avec le roi d'Espagne\,; qu'il serait
cependant difficile de les abuser longtemps, et que, si le nonce
demeurait encore à Madrid sous quelque prétexte et sous quelque figure
que ce pût être, son séjour en cette cour découvrirait la vérité.
Giudice, tournant en dérision l'obéissance des Espagnols envers le roi
leur maître, croyait justifier le refus qu'il faisait depuis quelque
temps d'obéir à l'ordre qu'Acquaviva lui avait fait présenter de la part
du roi d'Espagne de faire ôter le tableau des armes d'Espagne qu'il
avait sur la porte de son palais, ainsi que les cardinaux et les
ministres des princes étrangers ont coutume d'élever sur la porte des
leurs les armes des princes qu'ils servent ou à qui ils sont attachés
véritablement. Il avait espéré que le régent intercéderait pour lui
auprès du roi d'Espagne, et que ses puissants offices procureraient la
révocation d'un ordre qu'il attribuait au crédit absolu de son plus
mortel ennemi\,; mais l'ordre n'ayant pas été révoqué, il fallut enfin
se soumettre. Le pape même le pressa de prendre ce parti nécessaire, un
particulier ne pouvant longtemps tenir tête à un grand roi. Giudice, en
obéissant, protesta que jamais il n'arborerait les armes d'une couronne
qui rejetait ses services, et se félicitant d'être libre désormais, il
paraissait résolu d'éviter tout commerce avec les Allemands\,; mais,
soit désir de les servir, soit qu'il craignît effectivement les effets
de leur ressentiment à l'égard de sa famille, il avertit souvent
Cellamare, son neveu, de songer sérieusement aux mauvais offices qu'on
lui avait rendus à Vienne, et de prévenir les suites qu'ils pourraient
avoir.

Cette cour avait envoyé au comte de Gallas, ambassadeur de l'empereur à
Rome, plusieurs pièces, dont on disait que les unes étaient originales
et les autres légalisées, toutes servant à prouver une intelligence
secrète entre le roi d'Espagne et le Grand Seigneur, liée et contractée
par le moyen de Cellamare. Le bruit courait que, parmi, ces pièces, il y
avait plusieurs lettres originales de lui et du prince Ragotzi. Gallas,
en les communiquant au pape, lui avait dit en forme de menace que
l'empereur serait attentif à la conduite de Sa Sainteté, et qu'elle
servirait de règle aux mesures qu'il croirait devoir prendre. C'en était
assez pour faire trembler Rome, et plus qu'il n'en fallait pour faire
trembler en particulier un Italien dont les biens étaient situés dans le
royaume de Naples, sous la domination de l'empereur. Cellamare avait
encore ajouté un autre motif à la colère de ce prince. Il avait écrit
une lettre où, rejetant comme calomnie ce que les Allemands avaient
publié de ses négociations avec la Porte, il s'était répandu en
invectives sur la mauvaise foi de la cour de Vienne. Acquaviva
communiqua cette lettre au pape, en distribua différentes copies, et
pour la rendre plus intelligible aux Romains, il la fit traduire en
italien. Il dit même qu'il la ferait imprimer\,; en sorte que, sous
prétexte de relever et de faire valoir le zèle de l'ambassadeur
d'Espagne pour son maître, il suscitait en effet, et faisait retomber
toute la vengeance de l'empereur sur la famille des Giudice. Le
cardinal, persuadé que tout ce que faisait Acquaviva n'était que par
malignité, avertit son neveu de prendre garde aux conséquences fâcheuses
qu'il devait craindre d'un pareil écrit, le danger étant pour lui
d'autant plus grand que le roi d'Espagne venait d'ordonner à son
ministre à Rome de mépriser les vains discours des Allemands. Ainsi
l'ambassadeur d'Espagne paraissait en quelque façon abandonné du roi son
maître, et livré à ce que voudraient faire contre lui les ministres de
l'empereur qui trouveraient également à satisfaire et leur vengeance et
leur avidité, en retenant, lors d'un traité de paix, les biens
confisqués dont ils étaient en possession dans le royaume de Naples\,;
mais cet ambassadeur était alors moins occupé de ses propres intérêts du
côté de l'Italie qu'il ne l'était d'animer et de fortifier les intrigues
et les cabales secrètes qu'il entretenait depuis quelque temps à la cour
de France, sous l'espérance de secours infaillibles et puissants de la
part du roi d'Espagne.

Cellamare se flattait que, s'il réussissait dans l'affaire du monde qui
touchait le plus sensiblement le roi d'Espagne, et qui satisfaisait en
même temps le goût et la vengeance de son premier ministre, la
récompense qu'il tirerait d'un pareil service le dédommagerait
abondamment des pertes qu'il comptait avoir déjà faites dans le royaume
de Naples. Il travaillait donc, et connaissant parfaitement la nécessité
du secret, il aimait mieux laisser le roi son maître quelque temps dans
l'ignorance du progrès de ses manèges que s'en expliquer autrement que
par des voies bien sûres, telles par exemple que les voyages que
quelques officiers espagnols ou wallons avaient occasion de faire à
Paris et à Madrid, et c'était ordinairement par les mêmes voies qu'il
recevait les réponses et les ordres de Sa Majesté Catholique. Il se
défiait même des courriers, en sorte que, lorsqu'il était obligé
d'écrire par cette voie, il ne s'expliquait jamais clairement\,; mais,
enveloppant ses relations de voiles, il disait, par exemple, qu'il
préparait les matériaux nécessaires et qu'il s'en servirait en cas de
besoin, que les ouvriers contribuaient cordialement à les lui fournir.
Quelquefois il laissait entendre qu'il se défiait de quelques-uns de
ceux qui entraient dans ces intrigues. Enfin il cachait le mieux qu'il
lui était possible, sous différentes expressions figurées ce qu'il
voulait et ce qu'il n'osait exposer clairement aux yeux de son maître.
Deux circonstances flattaient alors l'ambassadeur d'Espagne, et lui
faisaient espérer un succès infaillible des intrigues qu'il avait
formées. L'une était la division qui éclatait ouvertement entre le
régent et le parlement de Paris. Cellamare, persuadé du poids que
l'exemple et l'autorité de cette compagnie devait avoir dans les
affaires publiques, traitait de héros les officiers qui la composaient.
Il assurait que leur constance surpassait toute croyance\,; que ceux
d'entre eux qui souffraient quelque mortification s'en réjouissaient
comme s'ils étaient couronnés parla gloire du martyre\,; que jusqu'alors
ils n'étaient soutenus que par la bienveillance et par les
applaudissements du public, mais que bientôt l'intérêt commun et le bien
de l'État unirait les autres parlements du royaume à celui de Paris, et
que cette union mutuelle causerait immanquablement des nouveautés
imprévues.

L'autre circonstance dont l'ambassadeur d'Espagne espérait profiter pour
les intérêts du roi son maître était celle de la division que la bulle
\emph{Unigenitus} excitait plus fortement que jamais, non seulement dans
le clergé, mais encore dans tous les états du royaume. Il semblait que
l'expédition des bulles nouvellement accordées, par le pape devait
calmer pour quelque temps cette agitation. Mais le nonce Bentivoglio
était le premier à détruire le bon effet que cette démarche sage du pape
aurait dû produire, et les déclamations imprudentes de ce ministre
rallumaient le feu dans le temps que son maître témoignait avoir
intention de l'apaiser. Ainsi les partisans de Rome qui désiraient le
véritable bien de cette cour commençaient à craindre les résolutions que
la France serait obligée de prendre pour prévenir celles du Vatican. Ils
ne doutaient pas que le régent ne consentît enfin à l'appel général de
la nation, etc.

D'un autre côté, le régent avait sur les bras des affaires qui pouvaient
devenir très sérieuses, et l'embarrasser de manière qu'il se trouverait
dans un triste état, s'il avait en même temps à soutenir des démêlés
avec la cour de Rome. Ces affaires étaient celles qui survinrent alors à
l'occasion des monnaies. Le nonce, ajoutant foi aux bruits de ville\,;
croyait, ainsi que les autres ministres étrangers, que la cour et le
parlement prenaient réciproquement des engagements dont les suites
seraient considérables. Ces ministres en attendaient l'événement avec
différentes vues. L'ambassadeur d'Espagne se flattait que l'opposition
du parlement aux résolutions que le régent prenait sur la monnaie
donnait à penser à Son Altesse Royale sur la négociation du traité
d'alliance, et que la réflexion qu'elle faisait sur la disposition
générale des esprits ne contribuait pas moins que les représentations de
la cour d'Espagne à ralentir l'ardeur qu'on avait fait voir en France
pour la conclusion de ce traité. Les agents du roi d'Angleterre
jugeaient, au contraire, que les embarras suscités au régent par le
parlement le persuaderaient encore davantage du besoin qu'il avait de se
faire des amis\,; qu'il comprendrait qu'il ne pouvait en avoir de plus
puissants que l'empereur et le roi d'Angleterre\,; que ce serait, par
conséquent, une nouvelle raison pour lui de s'unir avec ces princes,
trouvant chez lui si peu de satisfaction.

Le comte de Kœnigseck, ambassadeur de l'empereur, suivant le génie des
ministres autrichiens, voulait, quoique d'ailleurs honnête homme,
trouver à redire et donner un tour de mauvaise foi à toute la conduite
du régent. Le style de la cour de Vienne, et le moyen de lui plaire est
depuis longtemps d'interpréter à mal toutes les démarches de la France,
et la suprême habileté d'un ministre de l'empereur est de croire,
d'écouter de fausses finesses et de secondes intentions dans les
résolutions les plus simples. Ainsi Koenigseck prétendait avoir
découvert que le régent commençait à changer de langage\,; que Son
Altesse Royale ne lui parlait plus avec la franchise et la vivacité qui
faisaient juger quelque temps auparavant la prompte conclusion du
traité. Il remarquait, comme une preuve indubitable de ce changement et
du désir de ralentir la négociation, les différentes propositions que ce
prince avait faites pour assurer les principales conditions de
l'alliance. Comme un des articles les plus essentiels était celui de la
succession des États de Parme et de Toscane, Son Altesse Royale avait
proposé que la garde des places fortes de ces deux États fût commise à
des garnisons suisses. Rien n'était moins du goût des ministres de
l'empereur. Koenigseck crut avoir pénétré par les discours de Stairs
que, les garnisons suisses rejetées, on proposerait de substituer en
leur place des garnisons Anglaises et Hollandaises. L'empereur, qui n'en
voulait aucune, ne s'en serait pas mieux accommodé\,; mais son
ambassadeur lui conseilla de l'accepter, persuadé que la France
elle-même n'y consentirait jamais. Les variations de la cour au sujet de
l'alliance étaient, selon lui, le triomphe des anciens ministres
toujours opposés à ce projet\,; mais il prévoyait que le régent serait
la victime de la victoire qu'ils remportaient, et que ces mêmes
ministres, dévoués à l'Espagne, l'entraîneraient insensiblement en de
tristes affaires.

Il y avait alors grand nombre de gens, et principalement les étrangers,
qui regardaient comme un abîme ouvert sous les pieds du régent les
brouilleries que l'affaire des monnaies excitait entre la cour et le
parlement, et ces mêmes gens étaient persuadés que les autres parlements
du royaume suivraient incessamment l'exemple de celui de Paris. Stairs,
de son côté, paraissait mécontent de quelque refroidissement qu'il avait
cru remarquer dans la confiance que le régent lui avait témoignée
jusqu'alors. Son Altesse Royale lui avait communiqué un mémoire qu'elle
voulait envoyer en Angleterre\,; comme il y fit quelques remarques, elle
eut égard à ses représentations et promit de s'y conformer. Il prétendit
qu'elle lui avait promis de lui faire voir une seconde fois le projet
quand il serait changé. Toutefois les changements faits, elle envoya ce
projet en Angleterre, même avec quelques additions, sans le communiquer,
et ce ne fut qu'après le départ du courrier que Stairs en reçut la
copie. Il s'en plaignit. Le régent lui répondit qu'il avait apostillé
chaque article du mémoire de sa propre main. Stairs, peu satisfait de la
réponse, fit partir sur-le-champ un courrier pour informer son maître de
ce qu'il s'était passé, et de plus, il obligea Schaub, l'homme de
confiance de Stanhope, de passer lui-même en Angleterre pour instruire
plus particulièrement les ministres de cette cour de la situation et du
véritable état des affaires de France.

\hypertarget{chapitre-vii.}{%
\chapter{CHAPITRE VII.}\label{chapitre-vii.}}

1718

~

{\textsc{Avis peu uniformes de Monteléon en Espagne sur l'escadre
anglaise.}} {\textsc{- Forfanteries de Beretti.}} {\textsc{- Les
ministres d'Angleterre veulent faire rappeler Châteauneuf de Hollande.}}
{\textsc{- Comte de Stanhope à Paris, content du régent, mécontent des
Hollandais.}} {\textsc{- Le czar se veut réunir aux rois de Suède et de
Prusse contre l'empereur et l'Angleterre.}} {\textsc{- Conférence de
Monteléon avec les ministres d'Angleterre sur les ordres de l'escadre
anglaise, qu'il ne lui déguise pas.}} {\textsc{- Ils résistent à toutes
ses instances.}} {\textsc{- Faux et odieux discours du colonel Stanhope
à Albéroni.}} {\textsc{- Opinion des Anglais du régent, de ceux qu'il
employait et d'Albéroni.}} {\textsc{- Albéroni tente de surprendre le
roi de Sicile et de le tromper cruellement, en tâchant de lui persuader
de livrer ses places de Sicile à l'armée espagnole.}} {\textsc{-
Artificieuses lettres d'Albéroni à ce prince.}} {\textsc{- Albéroni
compte sur ses pratiques dans le nord, encore plus sur celles qu'il
employait en France contre le régent.}} {\textsc{- Il les confie en gros
au roi de Sicile.}} {\textsc{- Albéroni envoie à Cellamare la copie de
ses deux lettres au roi de Sicile.}} {\textsc{- Il propose
frauduleusement au colonel Stanhope quelques changements au traité pour
y faire consentir le roi d'Espagne, et, sur le refus, éclate en
menaces.}} {\textsc{- Lui seul veut la guerre et a besoin d'adresse pour
y entraîner le roi et la reine d'Espagne, fort tentés d'accepter le
traité pour la succession de Toscane et de Parme.}} {\textsc{- Albéroni
s'applaudit au duc de Parme d'avoir empêché la paix, et lui confie le
projet de l'expédition de Sicile et sur les troubles intérieurs à
exciter en France et en Angleterre.}} {\textsc{- Artifices et menaces
d'Albéroni sur le refus des bulles de Séville.}} {\textsc{- Aldovrandi,
malmené par Albéroni sur le refus des bulles de Séville, lui écrit\,;
n'en reçoit point de réponse\,; s'adresse, mais vaguement, à Daubenton
sur un courrier du pape, et ferme la nonciature, sans en avertir.}}
{\textsc{- Sur quoi il est gardé à vue, et Albéroni devient son plus
cruel ennemi, quoiqu'il l'eût toujours infiniment servi.}} {\textsc{-
Étranges artifices d'Albéroni sur Rome et contre Aldovrandi.}}
{\textsc{- Reproches réciproques des cours de Rome et de Madrid.}}
{\textsc{- La flotte espagnole arrivée en Sardaigne\,; crue aller à
Naples.}} {\textsc{- Triste état de ce royaume pour l'empereur.}}

~

L'escadre Anglaise était alors partie des ports d'Angleterre\,; elle
avait mis à la voile le 13 juin\,; on comptait quinze jours environ de
navigation pour arriver au détroit, et peut-être quatre semaines en tout
pour se rendre au port Mahon. Monteléon, avec le secours des amis dont
il se vantait, ne put pénétrer les ordres de l'amiral Bing qui la
commandait. Il se flattait, et même il en assura le roi d'Espagne, que
les Anglais éviteraient tout engagement avec la flotte espagnole. Il
prétendit savoir que les ministres autrichiens étaient bien loin
d'espérer que les vaisseaux d'Angleterre allassent à toutes voiles
chercher et combattre ceux d'Espagne. Toutefois, en habile ministre, il
ne devait compter que jusqu'à un certain point sur les avis qu'il
recevait. Il écrivit au roi son maître que, suivant les conjonctures, le
roi d'Angleterre pouvait envoyer de nouveaux ordres. Monteléon
s'apercevait alors du changement de cette cour par les traitements qu'il
y recevait très différents de ceux qu'il y avait précédemment reçus, et
comme les ministres d'Angleterre avaient peu de communication avec lui,
celui de France (Dubois) encore moins, il avouait qu'il ne pouvait plus
découvrir leur intrigue ni leurs intentions.

Beretti se flattait de servir l'Espagne avec plus de succès en Hollande.
Chaque fois que les états de la province se séparaient sans avoir pris
de résolution sur l'alliance proposée, Beretti l'attribuait à ses
pratiques secrètes et aux ressorts qu'il savait faire jouer à propos
pour traverser les ennemis de son maître. Si quelque député donnait sa
voix pour l'alliance\,; Beretti assurait aussitôt qu'il avait été gagné
par argent. Cadogan, de son côté, se moquait de la vanité de Beretti, et
triomphait quand quelqu'une des villes de la province de Hollande
paraissait disposée à l'acceptation de l'alliance\,; chacun des deux se
croyait assuré de ses partisans, et si Cadogan comptait sur les villes
de Leyde et de Rotterdam, Beretti se vantait d'avoir persuadé les
députés de Delft, d'autant plus difficiles à ramener qu'ils avaient paru
les plus empressés pour l'alliance. Comme il ne convenait pas de se
borner à la seule province de Hollande, Beretti voulut gagner le baron
de Welderen, tout puissant, croyait-il, dans la province de Gueldre. Il
lui promit un présent considérable si, par son crédit, il empêchait les
États généraux d'entrer dans l'alliance, et persuadé qu'il ne pouvait
faire une meilleure acquisition pour le service du roi son maître, il
écrivit à Albéroni qu'il vendrait son bien pour satisfaire la promesse
qu'il avait faite si le roi d'Espagne désapprouvait l'engagement qu'il
avait pris pour son service. Le bruit se répandit alors que ce prince
avait donné ordre à ses sujets négociants, sous peine de la vie, de
remettre un registre exact et fidèle des effets qu'ils avaient entre les
mains appartenant à des étrangers de quelque nation qu'ils fussent. Une
telle nouvelle causa quelque alarme à la Haye. Beretti se flatta d'en
avoir profité, et d'avoir utilement augmenté la frayeur que les
apparences d'une guerre prochaine et de, la ruine du commerce
produisaient déjà dans les esprits, mais son zèle et l'attention qu'il
avait à le faire valoir à la cour de Madrid y réussissait mal. Il eut
plusieurs fois lieu de se plaindre de la manière dont il était traité
par Albéroni. Il gémissait donc, mais inutilement, d'essuyer mille
dégoûts de la cour d'Espagne, ou pour mieux dire du premier ministre de
cette cour, pendant qu'il se donnait tout entier au service de son
maître, et que, sans en recevoir aucun secours, il employait uniquement
ses talents, son industrie, ses manèges, comme les seules armes qu'il
eût pour combattre l'ambassadeur d'Angleterre, soutenu par de puissants
amis et répandant l'or avec profusion pour gagner ceux qu'il savait être
autorisés dans la république. Beretti comprenait dans ce nombre Pancras,
bourgmestre, régent d'Amsterdam, et Buys, pensionnaire de la même ville.
Le dernier, disait-il, menait l'autre par le nez. La liste des
magistrats et députés gagnés par l'Angleterre était bien plus nombreuse
si on ajoutait foi à un écrit imprimé qu'on distribuait sous main à la
Haye, spécifiant par nom et par surnom tous ceux qui recevaient des
pensions ou des gratifications de cette couronne. Beretti se vantait
que, malgré tant de dépenses faites et continuées par les ennemis de Sa
Majesté Catholique, il était parvenu par son activité et par ses amis à
faire en sorte que la province de Hollande avait déjà séparé cinq fois
ses assemblées sans rien résoudre au sujet de l'alliance. Cadogan
parlait en même temps très différemment, car il dit avec plus de vérité
que les états de cette province avaient pris unanimement la résolution
d'entrer dans le traité. Il est vrai cependant que les députés des
principales villes déclarèrent à l'assemblée que leur instruction
portait de consentir à la quadruple alliance quand l'affaire serait mise
en délibération\,; mais le temps de cette délibération fut prolongé.

Les ministres d'Angleterre, se défiant toujours de Châteauneuf,
ambassadeur de France en Hollande, pressaient plus que jamais son rappel
et l'envoi du successeur qui lui était désigné. Ils comptaient de tout
obtenir du régent par le moyen du comte Stanhope nouvellement arrivé à
Paris. Son Altesse Royale lui avait fait un accueil très favorable\,;
elle avait pris soin de lui persuader qu'elle souhaitait ardemment la
conclusion du traité et qu'elle n'oublierait rien pour en faciliter la
signature. Ainsi les Anglais comptaient qu'elle ne serait désormais
retardée qu'autant de temps qu'il en fallait pour traduire le traité en
latin. Ils approuvaient quelques changements que le régent demandait, et
comptaient que la cour de Vienne ne pourrait avec raison y refuser son
approbation. Il s'en fallait beaucoup que les ministres d'Angleterre
fussent aussi contents de la conduite des Hollandais. On commençait à
dire que la république, après avoir longtemps biaisé\,; après avoir
laissé entrevoir exprès une diversité apparente de sentiments entre les
villes de la province de Hollande, terminerait ces incertitudes
affectées par une offre simple et toujours inutile d'interposer ses
offices pour mettre en paix les principales puissances de l'Europe. Une
telle offre aurait été un, refus honnête d'accéder au traité, et les
ministres Anglais avaient un intérêt personnel de faire voir à la nation
Anglaise que le projet de la quadruple alliance était un projet sage,
solide, approuvé généralement des principales puissances de l'Europe et
de celles qui pouvaient donner le plus de poids aux affaires.

Une telle opinion était pour eux d'autant plus nécessaire à établir
qu'il était alors vraisemblable que le czar, cherchant à faire un
personnage dans les affaires de l'Europe, animé d'ailleurs contre le roi
d'Angleterre, voulût s'opposer à la quadruple alliance et secourir le
roi d'Espagne par quelque diversion puissante. On assurait déjà que la
paix était faite entre la Suède et la Moscovie et le roi de Prusse\,;
que les mesures étaient prises entres ces princes pour s'opposer de
concert aux desseins de l'empereur et du roi Georges. Ce qui n'était
encore que bruits incertains parut se confirmer et devenir réel suivant
un discours que le ministre du czar à Paris, tint à Cellamare. Le
Moscovite l'assura que son maître, voulant s'opposer aux desseins de
l'Angleterre, avait fait sa paix avec le roi de Suède\,; qu'il ménageait
celle du roi de Prusse, et qu'une des principales conditions du traité
serait une ligue offensive et défensive contre l'empereur et contre le
roi Georges. Il ajouta qu'il sollicitait actuellement le régent d'entrer
dans la ligue ou tout au moins de demeurer neutre. Ce ministre ne se
contenta pas de ce qu'il avait dit à l'ambassadeur d'Espagne, il crut le
devoir dire encore au comte de Provane, chargé pour lors des affaires du
roi de Sicile à Paris. À son récit il ajouta des réflexions sur
l'utilité que le roi de Sicile tirerait de la diversion que le czar
ferait des forces de l'empereur. Il pressa Provane de lui découvrir les
intentions du roi son maître au sujet de l'alliance, et les liaisons
qu'il avait prises avec le roi d'Espagne. Ce discours ne servit qu'à
faire voir quelles étaient alors les dispositions du czar.

Son animosité contre le roi d'Angleterre n'empêcha pas les ministres de
cette cour de suivre le plan qu'ils avaient formé pour traverser
l'entreprise que le roi d'Espagne était sur le point de tenter en
Italie. Ils jugeaient alors qu'elle regardait le Milanais et
qu'apparemment il agirait de concert avec le roi de Sicile. Comme
l'escadre Anglaise était partie des ports d'Angleterre. L'ambassadeur
d'Espagne, suivant les ordres qu'il en avait reçus du roi son maître,
demanda une conférence aux ministres d'Angleterre pour savoir d'eux
positivement quelles étaient les instructions que l'amiral Bing,
commandant de l'escadre, avait reçues avant son départ. La conférence
fut tenue le 24 juin\,; Stanhope n'était pas encore parti pour la
France\,; ainsi Monteléon le vit aussi bien que Sunderland et Craggs, et
leur dit que ce serait apparemment une des dernières fois qu'il leur
parlerait d'affaires puisqu'il se croyait à la veille d'aller à Douvres
s'embarquer, prévoyant quelque hostilité imminente quand l'escadre
Anglaise paraîtrait dans la Méditerranée. Ayant ensuite demandé quels
étaient les ordres dont l'amiral Bing était chargé, Stanhope lui
répondit que les instructions données à Bing lui prescrivaient
d'observer toute la bonne correspondance que le roi son maître
prétendait entretenir avec l'Espagne\,; qu'il avait ordre de donner
toutes sortes de marques d'attention à l'égard des officiers du roi
d'Espagne, soit de terre, soit de mer\,; que, s'il trouvait quelque
convoi faisant voile en Sardaigne, à Portolongone, même en Sicile, il
n'en troublerait pas la navigation mais s'il arrivait que la flotte
espagnole entreprît de débarquer des troupes dans le royaume de Naples
ou sur quelque autre terre, dont l'empereur était en possession en
Italie, en ce cas l'amiral anglais déclarerait aux commandants espagnols
qu'il s'opposerait à leur entreprise, le roi d'Angleterre ne pouvant
permettre qu'il s'en fît aucune au préjudice de la neutralité d'Italie
dont il s'était rendu garant envers l'empereur. Stanhope ajouta, de
plus, à cet aveu que, si les bonnes raisons ne suffisaient pas, les
Anglais emploieraient la force et qu'ils s'opposeraient ouvertement à
l'entreprise de l'Espagne. Monteléon, peu content de cette explication,
voulut cependant pousser les questions plus loin\,: il supposa que la
flotte d'Espagne eût mis le débarquement à terre avant que l'escadre
Anglaise fût arrivée, et demanda si Bing traiterait en ce cas les
vaisseaux espagnols comme ennemis. Stanhope répondit à cette question
nouvelle qu'il était impossible de prévoir tous les accidents qui
pouvaient arriver\,; et, revenant à son principe, il dit que l'ordre
général donné à l'amiral Bing était de s'opposer à toute entreprise que
l'Espagne ferait contre l'Italie.

L'explication était claire et nette\,: ainsi Monteléon, suffisamment
instruit des intentions de la cour d'Angleterre, ne trouva de ressources
pour les faire changer que dans son éloquence\,; mais il l'employa
vainement. Les raisons, quand le parti est pris, sont d'un faible
secours, et l'ambassadeur d'Espagne s'étendit assez inutilement sur
l'aveuglement et l'ingratitude de l'Angleterre qui renonçait aux
avantages du commerce d'Espagne, perdait en un moment le souvenir de
ceux que le roi catholique lui avait nouvellement accordés, le tout pour
agrandir l'empereur sans utilité pour la nation Anglaise, même au
préjudice du roi Georges intéressé comme électeur de l'empire à modérer
la puissance de la maison d'Autriche\,; il reprit en détail tout le
projet de l'alliance et efforça ça de faire voir qu'elle était
absolument contraire au but d'établir le repos public et l'équilibre
nécessaire pour le maintenir, comme on affectait de se le proposer, car
il n'y avait rien de si opposé à la tranquillité générale qu'une rupture
entre l'Espagne et l'Angleterre, et les facilités que le roi
d'Angleterre donnait à l'empereur de subjuguer l'Italie. Monteléon ne
garda pas le silence sur l'état de la France et la conduite du régent\,;
il insista sur le changement des ordres donnés à Bing\,; il demanda
qu'il lui fût défendu de faire la moindre hostilité ou tout au moins
qu'il fût averti que, si les Espagnols avaient débarqué leurs troupes
avant leur arrivée, le sujet de sa mission étant fini, l'intention du
roi son maître était qu'il évitât tout engagement, surtout la
déclaration d'une guerre ouverte contre l'Espagne. L'ambassadeur essaya
de flatter les ministres d'Angleterre de la gloire qui reviendrait au
roi leur maître de faire le personnage d'arbitre dans une négociation
prochaine pour la paix. Il tenta même de les piquer contre les ministres
de Hanovre accusés, dit-il, par les Anglais, d'être les instigateurs de
la partialité que le roi d'Angleterre témoignait pour l'empereur, même
de sa dépendance pour la cour de Vienne. Mais enfin la conférence finit
sans se persuader de part ni d'antre, comme il arrive en semblables
conjonctures, et les ministres Anglais, n'acceptant aucune des
propositions de Monteléon, protestèrent seulement que l'intention du roi
leur maître était de faire ce qui dépendrait de lui pour ne pas rompre
avec l'Espagne.

Le colonel Stanhope eut ordre de parler dans le même sens à Albéroni, et
de joindre aux plaintes et même aux menaces des reproches tendres de
l'ingratitude que l'Espagne témoignait à l'égard de l'Angleterre. Le roi
Georges prétendait avoir travaillé si puissamment pour procurer au roi
d'Espagne une paix avantageuse, que l'empereur était mécontent des
efforts qu'il avait faits pour la satisfaction de Sa Majesté Catholique,
et qu'ils avaient été regardés à Vienne comme une marque évidente de
partialité\,; que cette cour se plaignait encore amèrement des délais du
roi d'Angleterre à satisfaire aux conditions principales du traité, et
des prétextes dont il s'était servi jusqu'alors pour éviter d'envoyer le
secours qu'il avait promis\,; condition que l'Espagne n'ignorait pas,
puisque la copie de ce même traité lui avait été communiquée de bonne
foi par l'envoyé d'Angleterre. Ce ministre eut ordre de se plaindre du
peu de retour que l'Angleterre trouvait de la part de l'Espagne à tant
de marques d'attention et d'amitié qu'elle recevait de la part du roi
d'Angleterre et de la nation Anglaise\,; car, au lieu de témoignages
réciproques d'amitié et de confiance, le roi d'Espagne se conduisait
comme envisageant une rupture prochaine entre les deux couronnes. Il
semblait même qu'elle était déjà résolue dans son esprit, puisqu'il
refusait d'exécuter les derniers traités de paix, et que les Anglais
étaient presque regardés comme ennemis dans les, ports et dans les îles
de la domination d'Espagne. La cour d'Angleterre établissait pour
premier sujet de plaintes le refus que le roi d'Espagne faisait
d'accorder la permission stipulée par le traité d'Utrecht pour le
vaisseau Anglais qui devait être envoyé tous les ans à la mer du Sud. Il
n'appartenait pas à l'Espagne, disaient les Anglais, de décider si le
traité devait être accompli ou son exécution suspendue, et d'en juger
par la seule raison de ce qui convenait ou non aux intérêts de cette
couronne. Les Anglais se plaignaient encore des poursuites injustes et
dures, disaient-ils, que l'on faisait en Espagne contre les négociants
de leur nation. Ils ajoutaient que nouvellement le roi d'Espagne avait
fait enlever dans les ports de son royaume un grand nombre de bâtiments
Anglais, qui depuis avaient été employés, par ses ordres, à transporter
ses troupes en Italie. Enfin les Espagnols venaient de s'emparer, dans
les Indes occidentales, de l'île de Crab, dont l'Angleterre était en
possession\,; ils en avaient chassé les habitants, enlevé plusieurs
bâtiments Anglais, soit à l'ancre, soit en pleine mer. Ils menaçaient
encore plusieurs autres îles de traitements semblables.

Malgré tant de griefs le colonel Stanhope eut ordre de protester que le
roi son maître voulait maintenir la paix, et qu'il l'observerait
ponctuellement, si malheureusement l'Espagne ne le forçait à la
rompre\,; qu'il oublierait les sujets particuliers qu'il avait de se
plaindre\,; qu'il garderait le silence sur l'entreprise faite contre
l'empereur au préjudice de la neutralité de l'Italie, dont l'Angleterre
était garante, pourvu que le roi d'Espagne voulût, de son côté, renoncer
au dessein de troubler l'Europe, et donner à un roi qui voulait cultiver
avec Sa Majesté Catholique la plus sincère amitié les témoignages qu'il
devait attendre d'une confiance et d'une amitié réciproques\,; que, s'il
en arrivait autrement, il saurait conserver la dignité de sa couronne,
la sûreté de ses sujets et la foi des traités\,; que jusqu'alors il
avait souffert, et que ses sujets recevant tout le dommage de la part de
l'Espagne, il n'avait causé aucun mal à cette couronne\,; qu'il avait
prié pendant qu'il était menacé\,; que l'événement ferait peut-être
connaître que le langage qu'il avait tenu était dicté par l'amitié et
non par la crainte\,; et qu'enfin, ne manquant ni de raisons de rupture
ni de moyens de se venger, il n'appartenait pas au cardinal Albéroni de
croire et de se vanter qu'il pouvait intimider un roi d'Angleterre, de
qui l'inimitié pouvait être fatale à ceux qui se flatteraient vainement
de pouvoir aider ses ennemis. Les ministres d'Angleterre étaient
persuadés que si celui d'Espagne menaçait l'Angleterre des entreprises
du prétendant, l'empereur était à l'égard de l'Espagne un prétendant au
moins aussi dangereux, et que l'état présent de ces deux monarchies
donnait à celle d'Angleterre une supériorité bien marquée sur celle
d'Espagne. On ne craignait à Londres aucune traverse de la part de la
France mais en même temps qu'on était persuadé de la sincérité du
régent, on se défiait des ministres qu'il employait. Nancré surtout
était suspect. Stanhope fut averti de veiller sur sa conduite comme sur
celle d'un homme qu'Albéroni avait gagné, car il passait pour constant
que rien ne coûtait au premier ministre d'Espagne\,; qu'il était maître
en l'art de séduire et de tromper\,; il s'en faisait lui-même honneur,
et, persuadé de sa supériorité en cet art, il amusait depuis longtemps
le roi de Sicile sous la feinte apparence d'une négociation qu'il jugea
nécessaire pour surprendre ce prince, et pour l'empêcher de veiller à la
conservation du royaume dont il était alors en possession.

Le roi de Sicile, prince très éclairé, très attentif à ses intérêts,
facilita cependant à Albéroni les moyens de le surprendre. Ce prince,
accoutumé à se défier de ses ministres, en employait souvent plusieurs
de différents ordres dans la même cour. Lascaris était le dernier qu'il
avait envoyé à Madrid, pour lier à l'insu de son ambassadeur, une
négociation secrète qu'il n'avait peut-être pas envie de conclure. On ne
pénétra pas le détail des propositions faites par Lascaris, mais il est
certain qu'elles ne convinrent pas aux desseins d'Albéroni. Comme il ne
se rapportait pas absolument au compte que Lascaris rendait à son maître
de cette négociation secrète, il écrivit lui-même au roi de Sicile que
les offres faites par son ministre éclaircissaient un peu l'état des
affaires présentes\,; qu'elles donnaient lieu d'embarrasser le projet de
l'alliance, et de faire voir à tout le monde l'injustice et la tromperie
de ceux qui voulaient pour leur intérêt particulier s'ériger en maîtres
de partager l'univers à leur fantaisie, et sans autre raison que celle
de leur volonté se rendre arbitres du sort des princes, et les
dépouilles des États qu'ils avaient reçus de leurs ancêtres.

Albéroni assura ce prince que le roi d'Espagne ne recevrait la loi de
personne, qu'il se défendrait jusqu'à la dernière extrémité, ajoutant
qu'une bonne union avec Sa Majesté Catholique obligerait peut-être le
roi Georges et le régent à changer de pensée, l'un et l'autre
connaissant ce qu'ils auraient à craindre d'une telle liaison. Albéroni
conclut de ce principe qu'il n'y avait point de temps à perdre, et qu'il
était nécessaire de prendre et d'exécuter au plus tôt les mesures
proposées en conséquence. Il pressa le roi de Sicile de remettre
incessamment quelques places de ce royaume, on n'a pas su lesquelles,
entre les mains du roi d'Espagne\,; car alors rien n'empêcherait de
passer sur-le-champ dans le royaume de Naples, dont la conquête serait
prompte et facile par le moyen des intelligences pratiquées dans ce
royaume qui seraient appuyées d'une grosse armée abondamment pourvue de
tout l'attirail et de toutes les provisions nécessaires pour assurer le
succès de l'entreprise. La remise des places de Sicile entre les mains
des Espagnols étant donc la base et le fondement du traité proposé,
Albéroni promit au roi de Sicile que, s'il consentait à cette condition
essentielle, et s'il voulait envoyer au plus tôt ses ordres aux
gouverneurs de ses places, et les remettre sans délai au commandant de
l'armée espagnole, on profiterait non seulement de l'alarme et de la
confusion où cet événement jetterait les Allemands dans le royaume de
Naples, mais que de plus Sa Majesté Catholique ne perdrait pas un
instant à faire passer un corps considérable de ses troupes, en tel
endroit de Lombardie que le roi de Sicile jugerait à propos\,; qu'elles
y seraient payées aux dépens de l'Espagne, et quant aux places de Sicile
que le roi d'Espagne les recevrait comme un dépôt sacré qu'il garderait
à telles conditions que le roi de Sicile voudrait prescrire, ne les
demandant que pour assurer le succès du projet, puisque tous les États
que les Allemands possédaient en Italie étaient incertains et vacillants
entre leurs mains s'ils ne s'emparaient de la Sicile dont la conquête
les mettrait en état de subjuguer le reste\,; mais il ne fallait pas,
dit-il, perdre un instant\,; tout moment était précieux, et le moindre
délai pouvait devenir fatal\,; parce que le moyen de rendre inutile la
dépense que l'Angleterre avait faite pour armer sa flotte, était de
débarquer promptement l'armée d'Espagne en Sicile, et d'occuper
incessamment la place de Messine.

Albéroni pratiquait depuis longtemps des alliances dans le nord. Il
tramait des intelligences en France, un grand royaume fournissant
toujours et des mécontents et des gens qui n'ayant rien à perdre se
repaissent d'espérances chimériques d'obtenir de grands avantages dans
un changement produit par le trouble et la confusion. Cette seconde
ressource était celle qui flattait le plus Albéroni\,; il était persuadé
que le roi d'Espagne avait en France un parti puissant très affectionné
aux intérêts de Sa Majesté Catholique\,; qu'il n'y avait pas le moindre
lieu de douter des bonnes intentions de ceux qui le composaient. Comme
le cardinal s'applaudissait de l'avoir heureusement ménagé, il fit
valoir au roi de Sicile l'importance dont il était de pouvoir compter
sur un tel secours, et de se trouver en état de donner au régent une
occupation si sérieuse, qu'il penserait plus d'une fois à s'engager à
faire une guerre ouverte à l'Espagne pour une cause, ajoutait Albéroni,
si injuste et si peu honorable à Son Altesse Royale. Il espérait, de
plus, que les Hollandais, instruits des dispositions intérieures de la
France, craindraient moins les menaces que cette couronne et celle
d'Angleterre ne cessaient de leur faire pour les obliger d'approuver le
traité d'alliance, et de s'engager à le soutenir. Enfin, il comptait
tellement sur les mouvements que ses négociations secrètes exciteraient
dans le nord, qu'il n'était plus question, selon lui, que de seconder et
d'aider de la part du roi d'Espagne les sages dispositions que ce
ministre avait faites. Il se proposait pour en assurer le succès
d'employer présentement à lever des Suisses l'argent qu'il attendait des
Indes. Il assura le roi de Sicile que la seule représaille faite depuis
peu sur les François dans la mer du Sud, avait produit plus d'un million
d'écus. Ce secours, casuel n'était qu'un commencement, Albéroni comptait
que la monarchie d'Espagne lui fournirait d'autres assistances
pareilles, et que le bon usage qu'il en ferait lui donnerait les moyens
de prouver aux alliés du roi son maître que ce prince voulait agir de
bonne foi, avec sincérité, honneur et probité\,; ainsi, que chaque
démarche de générosité que ferait le roi de Sicile, le roi d'Espagne y
répondrait avec une générosité égale et réciproque, avec reconnaissance,
et Sa Majesté Catholique, suivant les assurances de son ministre, ferait
fidèlement tous ses efforts pour procurer les avantages, l'honneur et la
gloire des deux rois également offensés, également intéressés à ne
consentir jamais que les Allemands maintinssent leur autorité en Italie,
au préjudice du repos et de la liberté de cette partie de l'Europe.

Ces projets et ces espérances dont le cardinal fit part au roi de Sicile
par une lettre qu'il lui écrivit de sa main, le 22 mai, furent
nouvellement confirmés par une seconde lettre de ce ministre au même
prince du 30 du même mois. Mais il développa ses intentions dans cette
seconde lettre plus clairement que dans la première. L'une avait été
écrite pour donner une grande idée des forces du roi d'Espagne, et pour
faire envisager à ceux qui s'uniraient à Sa Majesté Catholique, les
avantages singuliers qu'ils devaient se promettre de son alliance. La
seconde lettre fit voir que le roi d'Espagne avait besoin du concours du
roi de Sicile, et que les projets du cardinal ne pouvaient réussir, si
les places principales de la Sicile n'étaient confiées à la garde des
commandants et des troupes d'Espagne. Il n'était pas aisé de faire
goûter une pareille proposition à un prince aussi défiant que le roi de
Sicile. Toutefois Albéroni, s'appuyant apparemment sur la supériorité de
son génie, entreprit de persuader à ce prince qu'un acte de confiance
aussi opposé à son caractère qu'il l'était à la prudence, devenait une
démarche nécessaire et conforme à ses intérêts. Il employa toute son
éloquence à convaincre ce prince que l'unique moyen de délivrer l'Italie
de l'oppression des Allemands, était qu'il s'abandonnât lui-même avec
une confiance généreuse à la bonne foi, sincérité, probité du roi
d'Espagne, n'ayant d'autres vues que d'assurer la liberté de l'Italie.
Une fin si glorieuse était impossible, disait le cardinal, sans cette
pleine confiance. Il avouait même que, si elle manquait, on serait forcé
d'accepter le parti proposé par les médiateurs, car il fallait
nécessairement être sûr d'une retraite avant que d'exposer les troupes
espagnoles, et là retraite n'était sûre qu'autant qu'elles seraient en
possession des places de Sicile. Le roi d'Espagne les demandait, non
pour en demeurer le maître et pour recouvrer un État qu'il avait perdu,
mais par la seule nécessité d'assurer ses projets, dont l'exécution
serait encore plus avantageuse au roi de Sicile qu'à l'Espagne. Ce
prince, suivant le raisonnement d'Albéroni, contribuerait infiniment à
les avancer s'il déclarait par la remise de ses places son union avec
l'Espagne, car il donnerait une telle inquiétude aux Allemands, qu'ils
n'oseraient dégarnir l'État de Milan pour envoyer du secours à Naples\,;
et suivant le plan d'Albéroni, le soulèvement entier et subit de ce
royaume était indubitable, si les Napolitains voyaient les armes
d'Espagne et de Sicile, et les places de cette île entre les mains du
roi d'Espagne qui promettait de les garder purement et simplement comme
un dépôt, et de les rendre fidèlement au roi de Sicile après la lin de
la guerre. Naples soumise, le roi d'Espagne détacherait un gros corps de
ses troupes et l'enverrait en Lombardie en tel lieu que le roi de Sicile
le jugerait à propos, l'intention de Sa Majesté Catholique étant de
travailler autant pour l'intérêt d'un prince qu'elle aimait, et qui
faisait la première figure en Italie, que par la gloire de rendre à
cette partie de l'Europe son ancienne liberté. Albéroni attribuait à ces
deux motifs détachés de tout désir de faire des conquêtes, l'armement
que le roi d'Espagne avait fait, et comme le succès de l'entreprise
serait apparemment utile au roi de Sicile, il voulait persuader à ce
prince qu'il était le premier obligé à faciliter une expédition dont il
retirerait le plus grand avantage. Son union, disait Albéroni, et l'aveu
public de ses liaisons avec le roi d'Espagne, ne laisserait pas
d'étourdir et de rompre les mesures de ceux qui s'étaient figurés qu'ils
étaient les maîtres de couper le monde en morceaux.

Comme ces exhortations générales ne suffisaient pas pour persuader un
prince attentif à ses intérêts, qui pesait lés engagements avant de les
prendre, Albéroni ne voulant peut-être pas lui faire par écrit des
offres précises, ajouta que, si le roi de Sicile voulait envoyer à
Madrid quelque personne de confiance munie de pouvoirs nécessaires pour
conclure et signer un traité, le roi d'Espagne ne ferait aucune
difficulté de lui accorder tout ce qu'il pourrait prétendre et
désirer\,; que Lascaris, bien informé des forces d'Espagne et du
gouvernement actuel de cette monarchie, ne lui aurait pas laissé ignorer
qu'elle était en état de faire figure dans le monde\,; que certainement
il l'aurait informé des conférences que le cardinal et lui avaient eues
ensemble, et qu'enfin le temps était passé où les affaires qu'on
traitait à Madrid étaient affaiblies ou déchirées par la longueur des
conseils\,; que le roi d'Espagne les examinait présentement par
lui-même\,; que la décision de celles qui regarderaient le roi de Sicile
serait également prompte\,; que la même diligence se trouverait dans
l'exécution, parce que le succès en dépendait, et, par cette raison, Sa
Majesté Catholique priait le roi de Sicile d'avertir de ce qu'il ferait
Patino, intendant de l'armée d'Espagne, en sorte qu'on évitât de faire
plusieurs débarquements, surtout d'artillerie, et que l'armée d'Espagne
pût au plus tôt descendre au royaume de Naples. Ainsi le roi d'Espagne,
ne doutant pas que le roi de Sicile ne profitât des dispositions où Sa
Majesté Catholique se trouvait à son égard, avait, par avance, ordonné à
Patino de se conformer aux avis qu'il recevrait de ce prince, et de les
suivre comme la règle la plus sûre des mouvements que l'armée aurait à
faire.

Le cardinal chargea Lascaris d'envoyer cette lettre à son maître, priant
Dieu, dit-il, de persuader ce prince de faire attention à des
insinuations dont le seul objet était de l'agrandir et de pourvoir à sa
gloire et à la sûreté de l'Italie. Il ajouta que jamais l'occasion ne
serait si belle\,; que, si le roi de Sicile, prudent et politique, la
laissait échapper, il ne devait pas compter de retrouver en d'autres
temps un roi qui voulût bien employer ses forces et son argent dans un
pays où lui-même n'avait nulle prétention, ni de trouver auprès de ce
même roi un ministre italien transporté de l'amour de sa patrie, et
résolu de faire tous ses efforts pour seconder les intentions de son
maître. La copie de ces deux lettres fut envoyée par Albéroni à
Cellamare\,; car, alors, le cardinal avait une attention particulière à
bien instruire l'ambassadeur d'Espagne en France des projets et des
résolutions du roi son maître, l'assurant toujours que jamais ce prince
n'accepterait la proposition de la quadruple alliance, qu'il traitait de
projet inique en sa substance et indigne en sa manière. Il parut
toutefois que le roi d'Espagne, quoique déterminé à le rejeter, voulait
cependant avoir un prétexte assez spécieux pour justifier envers le
public le refus qu'il faisait de concourir à la tranquillité de
l'Europe, et fit proposer au colonel Stanhope quelques changements
{[}afin{]}, dit Albéroni, d'adoucir Sa Majesté Catholique, et de la
porter à souscrire aux engagements que la France et l'Angleterre avaient
déjà pris ensemble. Le colonel en ayant rendu compte en Angleterre
répondit, suivant les ordres qu'il en reçut, que son maître n'avait pas
osé faire savoir à Vienne que l'Espagne voulût altérer une seule syllabe
dans le projet. Sur cette réponse Albéroni déclara que le roi d'Espagne
rejetait entièrement le plan du traité, et qu'il attaquerait l'empereur
avec toute la vigueur possible. Il dit, de plus, au colonel Stanhope que
les marchands Anglais établis en Espagne étaient comme entre les bras de
l'escadre de leur nation, parce que, si elle faisait la moindre
hostilité, les effets de ces négociants seraient arrêtés sans égard au
temps que le dernier traité leur donnait pour se retirer en cas de
rupture entre les deux couronnes. Malgré tant de menaces, et malgré ces
déclarations si souvent répétées de la fermeté du roi d'Espagne,
Albéroni n'avait pas été sans inquiétude et sans crainte au sujet de
l'offre faite au roi d'Espagne des États de Parme et de Toscane, dont la
succession devait être assurée à l'infant don Carlos. Il avoua que la
tentation avait été grande, et que l'espérance d'un tel héritage,
destiné au fils de la reine d'Espagne, avait fait une impression très
vive sur l'esprit de cette princesse. Il confia ses alarmes au duc de
Parme, mais s'applaudissant en même temps d'avoir si habilement et si
heureusement travaillé, qu'il avait fait connaître à Leurs Majesté
Catholiques que l'idée était chimérique, l'offre trompeuse et sans
fondement. Après les avoir entraînés dans son sentiment, craignant
apparemment quelque changement de leur part, il avait protesté en France
et en Angleterre que le roi d'Espagne ne consentirait jamais à laisser
la Sicile entre les mains de l'empereur\,; enfin il avait établi comme
un principe de politique dont Sa Majesté Catholique ne devait jamais
s'écarter, que la paix avec l'empereur lui serait toujours
préjudiciable, qu'une guerre éternelle était au contraire conforme aux
véritables intérêts de l'Espagne, ces événements ne pouvant jamais nuire
à cette couronne, au lieu qu'il en pouvait arriver de tels que
l'empereur en recevrait un préjudice considérable.

Le temps approchait, et le secret de l'entreprise depuis longtemps
méditée par le roi d'Espagne allait être dévoilé. On était près de la
fin du mois de juin, et la flotte était prête à mettre en mer. Albéroni,
sujet du duc de Parme, et parvenu par sa protection à la fortune où il
était monté, ne lui avait pas jusqu'alors confié l'objet de l'armement
d'Espagne. Il ne lui en donna part que le 20 juin, et il lui apprit que
la foudre allait tomber sur la Sicile. La raison que le roi d'Espagne
avait de s'en emparer était que, s'il ne s'en rendait maître, il ne
pouvait le devenir du royaume de Naples, ni se promettre d'éviter les
pièges et les tromperies ordinaires du duc de Savoie. Si Sa Majesté
Catholique se faisait un ennemi de plus, elle croyait en être dédommagée
par une conquête facile à conserver, et qui donnerait le temps de semer
pendant l'hiver la discorde en France et en Angleterre\,; c'est ainsi
qu'Albéroni s'en expliquait, persuadé qu'il trouverait dans l'un et dans
l'autre royaume des dispositions favorables au succès de ses intrigues,
et prévenu que les mouvements dont il entendait parler, soit en France
soit en Angleterre, produiraient des révolutions.

Sur ce fondement, il pria le duc de Parme de vivre en repos et sûr qu'il
ne recevrait pas le moindre préjudice tant qu'Albéroni subsisterait\,;
il promit pareillement de faire valoir en temps et lieu ses droits sur
le duché de Castro. Le cardinal comptait déjà les Allemands chassés
d'Italie, convaincu que sans leur expulsion totale cette belle partie de
l'Europe ne jouirait jamais de la paix et de la liberté. Il se donnait
pour désirer ardemment de procurer l'une et l'autre à sa patrie,
nonobstant les raisons générales et personnelles qu'il avait de se
plaindre des traitements que le roi d'Espagne et lui recevaient du
pape\,; car il unissait autant qu'il était possible les intérêts de
Leurs Majestés Catholiques aux siens, et leurs plaintes étaient selon
lui plus vives que les siennes sur le refus des bulles de Séville. Le
roi et la reine d'Espagne étaient, disait-il, persuadés que ce refus
n'était qu'un prétexte à de nouvelles offenses que la cour de Rome
voulait leur faire pour plaire à celle de Vienne. Ainsi Leurs Majestés
Catholiques, lasses de se voir sur ce sujet l'entretien des gazettes,
avaient résolu de garder désormais le silence et d'employer les moyens
qu'elles jugeraient à propos à maintenir les droits de la royauté et de
leur honneur, ayant toutefois peine à comprendre que le pape vît avec
tant de sérénité d'esprit une rupture entre les deux cours. Sa Sainteté,
disait le cardinal, refusait quatre baïoques\footnote{Petite monnaie de
  cuivre. 1 baïoque = 5 centimes.} et voyait tranquillement la
confiscation de tous les revenus des églises vacantes en Espagne et de
ce qu'on appelle le \emph{spoglio}\footnote{Ce mot italien signifie
  \emph{dépouille} dans le sens de \emph{meubles}. On appelait autrefois
  en France \emph{droit de dépouille} un usage qui donnait à l'évêque ou
  à l'archidiacre le lit, la soutane, le cheval et le bréviaire du curé
  décédé. Cet usage avait commencé par les monastères, où les prieurs et
  autres religieux n'ayant un pécule que par tolérance, tout revenait à
  l'abbé après leur mort. Les évêques s'attribuèrent ensuite le droit de
  \emph{dépouille} sur les prêtres et les clercs. Les rois l'exercèrent
  aussi pendant plusieurs siècles dans quelques églises. Enfin
  l'antipape Clément VII, à l'époque du schisme d'Avignon, prétendit que
  le pape devait être le seul héritier de tous les évêques, et il obtint
  en effet le droit de dépouille en Italie et en Espagne. Voy. Fleury,
  \emph{Institution au droit ecclésiastique} (Paris, 1687, 2
  vol.~in-12.)} des évêques chassés du royaume, sûr que, quelque
accommodement qu'il se fît à l'avenir, la chambre apostolique n'en
retirerait pas un maravedis\footnote{Petite monnaie de cuivre. 1
  maravedis = 75 cent.}. Le scandale d'une rupture ouverte était trop
éminent\,; la patience du roi et de la reine d'Espagne éprouvée pendant
huit mois était enfin à son dernier période\,; la modération chrétienne
avait suffisamment éclaté de leur part\,; il était temps que Leurs
Majestés Catholiques prissent les résolutions nécessaires pour défendre
leurs droits, les souverains étant obligés en honneur et en conscience
d'employer à les soutenir les moyens que Dieu leur avait mis en main.
C'est ce qu'Albéroni disait et qu'il écrivait en même temps à Rome pour
intimider cette cour, toutefois avec la précaution de se représenter
lui-même au pape comme un instrument de paix, de protester qu'il n'avait
rien omis de ce qui pouvait dépendre de lui pour éviter les maux qu'il
prévoyait, et que la cour de Rome s'était trompée quand elle avait
regardé comme un effet d'impatience excessive les démarches qu'il avait
faites dans la seule vise de conserver l'union entre le saint-père et le
roi catholique.

Albéroni savait que le P. Daubenton, très attentif à se faire un mérite
à Rome des saintes dispositions du roi d'Espagne, assurait fréquemment
le pape que ce prince ne prendrait jamais de résolution contraire à la
soumission qu'il devait à Sa Sainteté. Le cardinal voulait détruire
cette confiance, et comme, il fallait une action d'éclat, il résolut et
menaça de chasser de Madrid le nonce Aldovrandi\,; c'était par une telle
voie qu'il voulait, disait-il, mériter à l'avenir, de la part du pape,
l'estime due à un cardinal et à un gentilhomme (il était public qu'il
était de la dernière lie du peuple et fils d'un jardinier) alors à la
tête des affaires d'une monarchie qui pouvait se rendre arbitre des
cours de l'Europe, puisqu'il n'avait pu mériter par ses services
(quels\,?) la moindre attention de la part de Sa Sainteté (qui l'avait
fait cardinal). Le pauvre nonce était à plaindre, mais ces termes de
compassion furent les seules marques qu'il reçut de la reconnaissance
d'Albéroni. La principale affaire de ce premier ministre était non
seulement de se venger des refus qu'il essuyait de la part du pape, mais
encore de faire voir à Sa Sainteté qu'elle s'était absolument trompée en
appuyant ses espérances à la cour d'Espagne sur la correspondance et sur
le crédit d'Aubenton\,: car il était essentiel au cardinal d'établir à
Rome qu'il n'y avait à Madrid qu'une unique source pour les affaires, et
que toutes les cours de l'Europe étaient instruites de cette vérité par
la pratique et par les négociations conduites à leur fin sans qu'il en
eût été parlé à âme vivante, hors à une seule.

Les dispositions du premier ministre ne laissaient pas espérer au nonce
beaucoup de succès des raisons que le pape lui avait ordonné d'employer
pour autoriser le refus des bulles de Séville. En effet, Albéroni reçut
si mal ces représentations, et la conférence entre eux fut si vive, que
depuis, Aldovrandi, homme sage, ne jugea pas à propos de retourner à la
cour. Il fallait cependant savoir quelle résolution le roi d'Espagne
prendrait après avoir su celle du pape. Le nonce écrivit au cardinal,
mais inutilement\,; la lettre demeura sans réponse. Ce silence fut un
pronostic de ce qui devait bientôt arriver. Le nonce, s'y préparant,
avertit le pape que, s'il était chassé de Madrid, il irait directement à
Rome, suivant les ordres de Sa Sainteté\,; qu'il croyait cependant
convenable à son service de laisser une personne de confiance à portée
d'entendre les propositions que la cour d'Espagne pourrait faire, et
capable d'entrer dans les expédients propres à réunir les deux cours,
car il regardait les conséquences d'une rupture comme plus fatales à la
religion qu'on ne le pensait peut-être à Rome, et sur ce fondement il
était persuadé que rien ne serait plus dangereux que de fermer toute
voie à la conciliation. Il s'était plaint déjà plusieurs fois du peu
d'égards que Rome avait eu à ses représentations. Il enchérit encore sur
les plaintes précédentes, assurant que, si la cour de Madrid en venait
aux démarches violentes qu'il prévoyait, bien des gens verraient clair
sur les fausses suppositions qu'ils avaient faites, en attribuant ses
représentations à des motifs d'intérêt personnel\,; qu'il n'avait rien à
espérer d'Albéroni, et que, lorsqu'il avait ménagé et cultivé sa
confiance, il n'avait eu d'autres vues que le service du saint-siège\,;
que l'autorité était tout entière entre les mains de ce ministre, et son
pouvoir augmenté considérablement depuis que le roi d'Espagne, attaqué
par de fréquentes maladies, était hors d'état de s'appliquer aux
affaires\,; que ce serait désormais mal raisonner que de compter sur la
piété et sur la religion du roi catholique\,; que tout dépendait d'un
premier ministre vindicatif et irrité\,; que les ordres qu'il donnerait
seraient les seuls que les troupes d'Espagne recevraient\,; que le
secret en était observé si exactement, qu'on ne les savait qu'après
qu'ils étaient exécutés, et qu'enfin les dispositions étaient telles
qu'il ne serait pas surpris si les Espagnols, débarqués en Italie\,;
faisaient quelque entreprise au préjudice de l'État ecclésiastique. La
rupture prévue parle nonce arriva, et, malgré la sagesse de ses
conseils, Rome et Madrid firent tomber sur lui toute l'iniquité d'un
événement qu'il avait tâché de prévenir. La nouvelle du refus des bulles
de Séville fut confirmée par les lettres du cardinal Acquaviva apportées
par un courrier extraordinaire. Le nonce en reçut en même temps un du
pape, et comme ce ministre n'avait point eu de réponse à la lettre qu'il
avait écrite à Albéroni, la cour étant alors à Balsaïm, il demanda une
audience au P. Daubenton, qui était demeuré à Madrid. Il dit seulement à
ce religieux que, quoique ses lettres de Rome ne fussent pas encore
déchiffrées, il en voyait assez pour juger qu'il serait obligé
d'exécuter des ordres peu avantageux à la cour d'Espagne et à la
personne, du cardinal Albéroni. En effet, dès le lendemain, il fit
fermer le tribunal de la nonciature sans en donner auparavant le moindre
avis et sans faire paraître aucune marque d'égards et de respects pour
le roi d'Espagne.

Albéroni affecta de répandre que ce prince était aussi vivement que
justement indigné de la conduite du nonce, et, pour en donner une
démonstration publique, Sa Majesté Catholique commanda qu'il fût gardé à
vue jusqu'à ce qu'elle eût consulté le conseil de Castille, son tribunal
supprimé, sur les mesures qu'elle avait à prendre pour repousser les
entreprises téméraires du ministre de la cour de Rome. Le conseil de
Castille consulté fut d'avis que le roi d'Espagne démit faire arrêter le
nonce, fondé sur ce que ce ministre du pape, n'ayant pas l'autorité par
lui-même d'ouvrir le tribunal de la nonciature et ne pouvant le faire
sans la permission du roi d'Espagne, né pouvait aussi le fermer sans la
connaissance et la permission de Sa Majesté Catholique. On ne douta plus
à la cour d'Espagne que la rupture, dont, cette cour faisait retomber la
haine sur le pape, ne fût depuis longtemps préméditée comme le seul
moyen que Sa Sainteté et ses ministres eussent imaginé de persuader les
Allemands qu'elle n'avait aucune liaison secrète avec l'Espagne, et, par
conséquent, nulle part aux entreprises de cette couronne en Italie. On
disait qu'il y avait plus de trois mois que le nonce faisait emballer ce
qu'il avait de plus précieux dans sa maison, et, qu'étant dans
l'habitude de faire valoir son argent, il avait pris depuis quelque
temps ses mesures pour retirer des mains des négociants les sommes qu'il
leur avait données à intérêt\,; on ajoutait que le courrier, dépêché de
Rome au nonce, avait eu l'indiscrétion, en passant à Barcelone, de dire
au prince Pio que le cardinal Albane l'avait fait partir avec un extrême
secret, qu'il lui avait donné deux cents pistoles pour sa course, le
chargeant de dire au nonce qu'ils se verraient bientôt, et de l'assurer
qu'il serait content, parce qu'il trouverait de bons amis à Rome. Le
même courrier avait dit aux domestiques de ce prélat que les nouvelles
de Rome étaient bonnes pour leur maître, et qu'il serait bientôt élevé à
la pourpre.

Albéroni chargeait encore sur ces bruits dont il était le secret auteur.
Il ajoutait que les Allemands avaient reconnu qu'ils devaient gagner
Aldovrandi comme un agent nécessaire pour engager le pape à rompre avec
l'Espagne, et qu'Aldovrandi, de son côté, persuadé que toute sa fortune
dépendait de se réconcilier avec la cour de Vienne, avait oublié
facilement tout ce qu'il devait au cardinal et au confesseur, aussi bien
que les protestations qu'il avait tant de fois faites d'une
reconnaissance éternelle, jusqu'au point de dire qu'étant assuré de
l'amitié et de la protection du cardinal il se moquait de ses ennemis à
Rome, et ces ennemis n'étaient pas des personnages de peu de
considération, car il avait attaqué directement le cardinal Albane, il
l'avait traité de vil mercenaire des Allemands, d'homme ingrat et sans
foi qui trahissait l'honneur de l'Église et celui du pape, son oncle,
pour l'intérêt sordide d'une pension de vingt-quatre mille écus assignée
sur les revenus du royaume de Naples, dont le payement était suspendu
toutes les fois qu'il ne servait pas les ministres de l'empereur à leur
fantaisie. Cette accusation n'était ni secrète ni portée au pape par des
voies obscures. Albéroni prétendait savoir que le nonce l'avait écrite
dans une lettre signée de lui et envoyée à Rome à dessein qu'elle fût
montrée à Sa Sainteté. Il concluait qu'un homme, si déclaré contre le
cardinal neveu, n'aurait pas osé renoncer à la protection du roi
d'Espagne, et tenir à son égard une conduite indigne, s'il n'était sûr
que la protection de l'empereur ne lui manquerait pas au défaut de celle
de Sa Majesté Catholique. C'était donc en se déclarant contre l'Espagne,
disait le cardinal, qu'Aldovrandi s'était réconcilié avec la cour de
Vienne, et le pape, au moins aussi timide que le nonce, essayait de
regagner les bonnes grâces de l'empereur en refusant les bulles de
Séville.

Ces sortes de refus étaient les voies que les ministres impériaux
traçaient à Sa Sainteté pour plaire à leur maître. Ils s'étaient
précédemment opposés à l'expédition des bulles qu'Albéroni avait
demandées pour l'évêché de Malaga. Leurs oppositions ayant été inutiles,
ils avaient fait des instances si pressantes pour empêcher que les
bulles de Séville ne fussent données, que le pape, timide, mais
toutefois ne voulant pas paraître céder aux menaces des Allemands, avait
cherché des prétextes pour autoriser le refus d'une grâce toute simple
que le roi d'Espagne lui demandait. Ces prétextes, traités à Madrid de
frivoles, étaient que les évêques de Vich et de Sassari étaient chassés
de leurs sièges et privés de leurs revenus\,; que ceux de l'église de
Tarragone étaient confisqués, et qu'Albéroni en jouissait\,; que ce
ministre, revêtu de la pourpre, oubliait les intérêts de la chrétienté
jusqu'au point de négocier une ligue entre le roi son maître et le Grand
Seigneur. C'était sur ces reproches que le refus des bulles de Séville
était fondé. Le pape avant de les accorder voulait que le roi d'Espagne
rétablît les évêques de Sassari et de Vich sur leurs sièges. Il jugeait
bien que les conjonctures ne permettaient pas qu'il rétablît deux
prélats manifestement rebelles. Les ministres d'Espagne lui avaient
souvent exposé les raisons du roi leur maître à l'égard de l'un et de
l'autre, et quant aux revenus confisqués de Tarragone, Albéroni
s'étonnait des reproches que Sa Sainteté lui faisait sur cet article,
elle qui n'avait jamais rien dit sur la confiscation des revenus de
l'église de Valence, dont plusieurs particuliers jouissaient, entre
autres le cardinal Acquaviva, à qui le roi d'Espagne avait donné une
pension de deux mille pistoles sur cet archevêché. Ainsi Albéroni
faisant tomber sur la cour de Rome toute la haine de la rupture, dit que
cette cour avait cru faire un sacrifice à celle de Vienne en ordonnant
au nonce d'y procéder d'une manière offensante pour Leurs Majestés
Catholiques\,; qu'elles étaient indignées de la manière dont ce prélat
s'était conduit, et que son imprudence avait forcé lé roi d'Espagne à
suivre l'avis que le conseil de Castille avait donné de le faire
arrêter.

L'ordre fut envoyé en même temps au cardinal Acquaviva de signifier
généralement à tous les Espagnols qui étaient à Rome d'en sortir
incessamment. L'une et l'autre cour croyait avoir également raison de se
tenir vivement offensée. Si celle de Madrid se plaignait, Rome
prétendait, de son côté, que les menaces et la conduite du roi d'Espagne
ne justifiaient que trop le pape sur les délais qu'il avait prudemment
apportés à la translation que le cardinal Albéroni demandait de l'église
de Malaga en celle de Séville. C'était à ces mêmes menaces que Sa
Sainteté attribuait la résolution qu'elle avait prise de refuser
absolument la grâce que le cardinal prétendait arracher d'elle en
l'intimidant\,; car il serait, disait-elle, pernicieux à l'autorité
apostolique, aussi bien qu'aux lois les plus sacrées de l'Église,
d'admettre et de couronner un tel exemple de violence, et la conquête de
l'église de Séville était si différente de celle de Sardaigne, que les
moyens qui avaient été bons pour l'une étaient exécrables pour l'autre.
Le pape s'expliquant ainsi protestait qu'il n'oublierait jamais la
manière terrible dont la cour d'Espagne avait abusé de sa crédulité
l'année précédente, ni le préjudice que le saint-siège et la religion en
avaient reçu. Sa Sainteté plus attentive alors aux affaires d'Espagne,
et surtout aux desseins de cette couronne sur l'Italie, qu'à toute autre
affaire de l'Europe, différait de s'expliquer encore sur celles de
France, et par ses délais excitait l'impatience du nonce Bentivoglio,
etc.

Cependant la flotte d'Espagne était en nier, et le 15 juin elle entra
dans le, port de Cagliari. Toute l'Italie était persuadée que la
conquête du royaume de Naples était l'objet de l'entreprise du roi
d'Espagne. On supputait le temps nécessaire pour l'exécution, et on
comptait que les Espagnols ne seraient pas en état d'agir avant le 20
juillet. Les agents du roi d'Angleterre en Italie se flattaient que la
flotte du roi leur maître ferait une navigation assez heureuse pour
arriver avant ce terme aux côtes du royaume de Naples, et s'opposer aux
desseins de l'Espagne. Le secours des Anglais était d'autant plus
nécessaire que les Allemands ne paraissaient pas assez forts pour,
s'opposer avec succès au grand nombre de troupes que le roi d'Espagne
avait fait embarquer. Le comte de Thaun, vice-roi de Naples, ayant
rassemblé dans un même camp toutes celles que l'empereur avait dans ce
royaume, il s'était trouvé seulement six mille fantassins et quinze
cents chevaux qu'il avait ensuite distribués dans Capoue et dans Gaëte
pour la défense de ces deux places. On remarqua même à cette occasion
l'indifférence que la noblesse du royaume témoigna pour la domination de
l'empereur, qui que ce soit de ce corps ne s'étant fait voir au camp.

Fin des six premiers mois de l'année 1718.

\hypertarget{chapitre-viii}{%
\chapter{CHAPITRE VIII}\label{chapitre-viii}}

1718

~

{\textsc{Scélératesses semées contre M. le duc d'Orléans.}} {\textsc{-
Manèges et forte déclaration de Cellamare.}} {\textsc{- Manège des
Anglais pour brouiller toujours la France et l'Espagne, et l'une et
l'autre avec le roi de Sicile.}} {\textsc{- Cellamare se sert de la
Russie.}} {\textsc{- Projet du czar.}} {\textsc{- Son ministre en parle
au régent et lui fait inutilement des représentations contre la
quadruple alliance.}} {\textsc{- Cellamare s'applique tout entier à
troubler intérieurement la France.}} {\textsc{- Le traité s'achemine à
conclusion.}} {\textsc{- Manèges à l'égard du roi de Sicile.}}
{\textsc{- Le régent parle clair au ministre de Sicile sur l'invasion
prochaine de cette île par l'Espagne, et peu confidemment sur le
traité.}} {\textsc{- Convention entre la France et l'Angleterre de
signer le traité sans changement, à laquelle le maréchal d'Huxelles
refuse sa signature.}} {\textsc{- Cellamare présente et répand un peu un
excellent mémoire contre le traité, et se flatte vainement.}} {\textsc{-
Le ministre de Sicile de plus en plus alarmé.}} {\textsc{- Folie et
présomption d'Albéroni.}} {\textsc{- Efforts de l'Espagne à détourner
les Hollandais de la quadruple alliance.}} {\textsc{- Albéroni tombe
rudement sur Monteléon.}} {\textsc{- Succès des intrigues de Cadogan et
de l'argent de l'Angleterre en Hollande.}} {\textsc{- Châteauneuf non
suspect aux Anglais, qui gardent là-dessus peu de mesures.}} {\textsc{-
Courte inquiétude sur le nord.}} {\textsc{- Le czar songe à se
rapprocher du roi Georges.}} {\textsc{- Intérêt de ce dernier d'être
bien avec le czar et d'éviter toute guerre.}} {\textsc{- Ses
protestations sur l'Espagne.}} {\textsc{- Les Anglais veulent la paix
avec l'Espagne, et la faire entre l'Espagne et l'empereur, mais à leur
mot et au sien.}} {\textsc{- Monteléon y sert le comte Stanhope outre
mesure.}} {\textsc{- Le régent, par l'abbé Dubois, aveuglément soumis en
tout et partout à l'Angleterre, et le ministère d'Angleterre à
l'empereur.}} {\textsc{- Embarras de Cellamare et de Provane.}}
{\textsc{- Bruits, jugements et raisonnements, vagues instances et
menées inutiles.}} {\textsc{- Menées sourdes du maréchal de Tessé avec
les Espagnols et les Russes.}} {\textsc{- Le régent les lui reproche.}}
{\textsc{- Le régent menace Huxelles de lui ôter les affaires
étrangères, et le maréchal signe la convention avec les Anglais, à qui
Châteauneuf est subordonné en tout en Hollande.}} {\textsc{- Efforts de
Beretti à la Haye.}} {\textsc{- Embarras de Cellamare à Paris.}}

~

Pendant que le pape aussi bien que toute l'Europe, donnait sa principale
attention aux desseins de l'Espagne prêts à éclore, et aux succès
qu'auraient les entreprises de cette couronne, Bentivoglio, nonce de Sa
Sainteté à Paris, occupé des affaires de la constitution, condamnait le
silence de Sa Sainteté, et ne cessait de lui représenter, etc.

La conservation si précieuse de la personne sacrée du roi était aussi ce
qui servait de prétexte aux discours que les malintentionnés répandaient
sans beaucoup de ménagements pour alarmer le public et pour l'animer
contre M. le duc d'Orléans. Les faux bruits qu'ils suscitaient étaient
fomentés par Cellamare, ambassadeur d'Espagne à Paris. Son but apparent
était d'empêcher la conclusion de la quadruple alliance\,; et, pour y
réussir, il se croyait tout permis. Il crut qu'il n'avait pas un moment
à perdre quand il vit arriver à Paris le comte Stanhope, secrétaire
d'État et ministre confident du roi d'Angleterre. Comme il devait
ensuite passer à Madrid, Cellamare se donna de nouveaux mouvements, non
seulement auprès des ministres étrangers, mais encore dans l'intérieur
du royaume, pour traverser l'union et la consommation des projets du
régent et du roi d'Angleterre. Cellamare, immédiatement après l'arrivée
du comte de Stanhope, déclara que, si le régent entrait dans les
propositions de cette couronne au sujet de la quadruple alliance ou dans
quelque autre engagement contraire aux dispositions du roi d'Espagne,
les liaisons que prendrait Son Altesse Royale produiraient une rupture
ouverte entre Sa Majesté Catholique et elle, des maux infinis à la
couronne de France, aussi bien qu'à celle d'Espagne, et certainement un
préjudice égal aux intérêts particuliers et personnels de l'un et de
l'autre de ces princes. Provane, ministre de Savoie, excité par
Cellamare, fit ses représentations, avec tant de force que tous deux se
flattèrent que le régent s'était borné à donner à Stanhope de bonnes
paroles, et que Son Altesse Royale sans rien conclure gagnerait du
temps, remettant à décider jusqu'à ce qu'elle eût reçu les réponses de
Vienne, et vu quel serait le succès de l'arrivée de la flotte d'Espagne
aux côtes d'Italie, et du débarquement des troupes espagnoles. Il ne
tenait qu'à Cellamare de se détromper de ces idées. Stanhope qu'il vit
ne lui dissimula pas ses sentiments\,; il parut défenseur très âcre du
projet de la quadruple alliance, regardée pour lors comme le moyen
infaillible de maintenir la paix de l'Europe.

Cellamare déploya son éloquence pour combattre ce plan et pour en faire
voir l'injustice\,; il ne réussit qu'à s'assurer que Stanhope, ainsi que
les autres ministres Anglais, s'étudiait à semer la jalousie entre les
cours de France et d'Espagne, et que, dans la vue de les priver l'une et
l'autre des secours du roi de Sicile, ses artifices tendaient à rendre
ce prince également suspect à Paris et à Madrid. Il en avertit Provane,
qui d'ailleurs parut alarmé par les discours positifs que tenait le
ministre d'Angleterre, car il assurait sans le moindre doute que le roi
d'Espagne accepterait sans hésiter le projet qu'il allait incessamment
lui porter. Stanhope prétendait le savoir certainement de l'envoyé du
roi son maître à Madrid. Il ajoutât avec la même certitude que Sa
Majesté Catholique abandonnerait les intérêts du roi de Sicile, et que
pour le dépouiller de son nouveau royaume elle unirait ses armes à
celles des alliés, si le roi d'Angleterre se relâchait sur l'article de
la Sardaigne. Cellamare fit encore agir l'envoyé de Moscovie. Le czar,
impatient de faire figure en Allemagne, et de se mêler des affaires de
l'empire, prétendait réussir en son dessein en se liant au roi de Suède,
et prenant pour prétexte de soutenir les droits du duc de Mecklembourg.
Il étendait encore ses vues plus loin\,: son intention était de se
venger du roi d'Angleterre, en faisant valoir les droits du roi Jacques.
Il voulait porter ce prince à la guerre en Écosse, le soutenir par une
armée de soixante mille hommes, pendant que le czar maintiendrait pour
l'appuyer une flotte de quarante navires de ligne dans la mer Baltique
et plusieurs galères.

Ce projet étant concerté avec le roi de Suède qui n'était pas moins
irrité contre le roi Georges, et qui ne désirait pas moins se venger de
sa perfidie que le czar, Cellamare avait, par ordre de son maître, fait
passer un émissaire secret à Stockholm, et cependant l'union était
intime entre le ministre d'Espagne et celui de Moscovie résidant tous
deux à Paris. Ce dernier parla donc au régent dans les termes que lui
prescrivit Cellamare, et pour appuyer les représentations qu'il fit à
Son Altesse Royale contre la quadruple alliance, il l'assura que tout
était disposé à former incessamment une alliance entre les princes du
nord, qui serait également utile à la France et au maintien de la paix,
puisqu'elle empêcherait également et l'empereur et le roi d'Angleterre
de troubler l'une et l'autre\,; qu'il serait, par conséquent, plus utile
au roi et plus avantageux de favoriser ces liaisons et d'y entrer, que
de persister à soutenir le projet proposé par le roi d'Angleterre. Ces
représentations inutiles furent éludées par une réponse douce et honnête
du régent, dont l'envoyé de Moscovie ne fut pas content. Il pria
Cellamare d'en informer le roi d'Espagne, et de lui demander des ordres
positifs aussi bien que des pouvoirs, pour traiter ensemble quand les
réponses du czar arriveraient, et pour former une ligue capable de tenir
tête à celle des François et des Anglais, puisqu'on ne pouvait plus
douter que le projet pernicieux de la France et de l'Angleterre n'eût
incessamment son exécution. Les Hollandais commençaient même à se
montrer plus faciles, et les ministres de la régence, voyant la conduite
de l'ambassadeur de France à la Haye, semblaient se laisser entraîner au
torrent.

Cellamare commençait donc à réduire et à fonder ses espérances
uniquement sur les dispositions qu'il croyait voir en France en faveur
du roi d'Espagne. Il ramassait les discours qu'on tenait dans le public,
et, soit pour plaire à Sa Majesté Catholique, soit pour faire sa cour à
Albéroni, il assurait que les François parlaient avec autant de joie que
d'étonnement de la flotte que l'Espagne avait mise en mer, que les
voeux, publics étaient pour le succès heureux de cette entreprise, et
que, si la cour pensait différemment, les intérêts particuliers de ceux
qui gouvernaient n'empêchaient pas la nation de faire voir ses
sentiments. Dans ces favorables dispositions, Cellamare continuait,
disait-il, de cultiver la vigne sans toutefois porter la main à cueillir
les fruits qui n'étaient pas encore mûrs. On vendait déjà publiquement
les premiers raisins destinés à adoucir la bouche de ceux qui devaient
tirer le vin, on se disposait ensuite à porter chaque jour au marché les
autres qui demeuraient sur la paille. C'était sous ces expressions
figurées que Cellamare cachait ses manèges secrets, mais, il ne
dissimulait pas l'espérance qu'il avait conçue d'une division prochaine
entre la cour et le parlement, dont il se persuadait que les suites
éclatantes produiraient de grands changements. Il comptait que le
parlement était appuyé par le duc du Maine, le comte de Toulouse et les
maréchaux de Villeroy et de Villars, et qu'enfin, dans la disposition où
les esprits étaient, le régent craindrait au moins autant que les
Anglais d'en venir à une rupture ouverte avec l'Espagne, événement que
les ministres de Sa Majesté Catholique croyaient que le roi d'Angleterre
éviterait avec la dernière attention, persuadés même que le voyage du
comte de Stanhope à Madrid était une preuve du désir que la cour
d'Angleterre avait de trouver quelque expédient pour n'en pas venir à
une rupture qui certainement déplairait fort à la nation Anglaise.

Cette crainte faisait peu d'impression sur l'esprit du régent et du roi
Georges. Stanhope régla les articles du traité\,; les difficultés qui
suspendaient son exécution s'aplanirent. La principale était celle qui
regardait les garnisons qui seraient mises dans les places de Toscane.
Le ministre d'Angleterre le dressa de manière qu'il ne douta plus
qu'elle ne dût passer au moyen des ménagements qu'il se flattait d'y
avoir apportés. L'ambassadeur de l'empereur en parut content, et comme,
la satisfaction de ce prince était le point de vue du roi d'Angleterre,
Stanhope crut tout achevé si le traité plaisait à la cour de Vienne. Il
s'embarrassait beaucoup moins de celle d'Espagne, et si Albéroni
prétendait exécuter les menaces qu'il avait faites de se porter aux
dernières violences à l'égard des Anglais, négociants en Espagne,
l'expédient dont le ministre d'Angleterre prétendait user pour réprimer
ces violences était d'en informer sur-le-champ l'amiral Bing. Il fallait
aussi rompre toute intelligence entre le roi d'Espagne et le roi de
Sicile, car il était assez incertain quelles liaisons ces princes
pouvaient avoir prises ensemble.

Le roi de Sicile aimant toujours à négocier, avait eu à Madrid des
ministres avec caractère public, et plusieurs agents secrets. Provane
était encore à Paris sans caractère, mais très attentif à toutes les
démarches de Stanhope, et très exact à faire savoir à son maître ce
qu'il pouvait en découvrir. Il croyait encore que l'intérêt de ce prince
et celui du roi d'Espagne était le même, et par cette raison, il
cultivait avec soin l'ambassadeur d'Espagne. Ce dernier était persuadé
de son côté que le roi son maître devait ménager le roi de Sicile, et
sur ce fondement, il n'oubliait rien pour fortifier Provane dans les
sentiments qu'il témoignait, et pour le mettre en garde contre les
artifices qu'il disait que la France et l'Angleterre employaient pour
semer les soupçons, et faire naître la mauvaise intelligence entre la
cour de Madrid et celle de Turin. Il fit donc voir à Provane la réponse
nette et décisive qu'Albéroni avait rendue au colonel Stanhope au sujet
du projet du traité. Cette preuve toutefois ne fut pas assez forte pour
déraciner les défiances d'un ministre du duc de Savoie, et Provane,
persuadé qu'il convenait aussi au roi d'Espagne d'être parfaitement uni
avec le roi de Sicile, douta néanmoins si Sa Majesté Catholique
s'intéresserait pour lui vivement et sincèrement. Stanhope ne manqua pas
d'ajouter par ses discours de nouvelles inquiétudes à celles que Provane
lui fit paraître. Il lui dit que ce prince devait craindre les promesses
trompeuses d'Albéroni\,; que le roi d'Espagne aurait déjà souscrit au
projet de paix si la cession eût été ajoutée en sa faveur aux conditions
proposées à Sa Majesté Catholique. Stanhope ajouta qu'Albéroni en avait
fait la confidence au colonel Stanhope, son cousin, envoyé d'Angleterre
à Madrid, offrant même d'accepter encore, nonobstant le débarquement que
la flotte d'Espagne avait peut-être fait alors en Italie\,; qu'il avait
dit de plus que cette flotte se joindrait à l'escadre Anglaise pour
faire ensemble la conquête de la Sicile. Provane étonné combattit le
discours de Stanhope, en disant que Cellamare lui avait communiqué les
lettres d'Albéroni, directement contraires aux relations du colonel
Stanhope. Le comte de Stanhope répondit qu'Albéroni tenait deux
langages\,; qu'il tromperait les Anglais si la flotte réussissait\,;
que, si l'entreprise manquait, le roi de Sicile serait sacrifié\,; que
d'ailleurs un prince si prudent, si éclairé, devait connaître qu'il ne
pouvait espérer aucun avantage solide en Italie de l'union qu'il
formerait avec l'Espagne, parce que l'année suivante l'empereur se
vengerait des liaisons prises à son préjudice\,; que l'unique voie
d'obtenir des avantages dont la durée serait sûre était d'entrer dans
l'alliance proposée.

Le régent parla plus clairement encore à Provane, et voyant qu'il
flottait encore entre les derniers discours du comte de Stanhope et les
assurances contraires d'Albéroni, lui offrit de parier que la flotte
d'Espagne faisait voile vers la Sicile, et qu'elle débarquerait sur les
côtes de cette île. Ce prince ajouta qu'on soupçonnait le roi de Sicile
d'être en cette occasion de concert avec le roi d'Espagne, et même
disposé de remettre entre les mains des Espagnols quelques places de
Sicile pour la sûreté du traité. Provane, surpris, voulut effacer un tel
soupçon comme injurieux à son maître. Il assura que ce prince
seconderait de toutes ses forces l'opposition que le régent apporterait
aux desseins du roi d'Espagne si Son Altesse Royale voulait en concerter
les moyens\,; mais elle répondit qu'elle réglerait ses démarches suivant
les événements que produirait l'entreprise de la flotte d'Espagne, la
paix de l'empereur avec les Turcs, et la ligue du nord\,; que, jusqu'au
dénouement de ces grandes affaires, il ne convenait pas aux intérêts du
roi de prendre aucun parti décisif\,; que, sur ce fondement, elle venait
de déclarer au comte de Stanhope qu'elle ne signerait la quadruple
alliance qu'après que l'empereur se serait désisté de la difficulté
qu'il formait sur le projet de la paix, et qu'après que les Hollandais
se seraient engagés dans l'alliance comme garants des promesses du roi
d'Angleterre\,; elle ajouta qu'elle prévoyait qu'ils auraient peine à
s'en charger, et que, d'un autre côté, elle trouverait les Anglais
opposés à rompre les premiers avec l'Espagne, et retenus par la crainte
d'exposer leur commerce. Tout était cependant réglé entre les cours de
France et d'Angleterre, on s'obligeait de part et d'autre à signer une
convention portant que le roi et le roi d'Angleterre ne souffriraient
aucun changement au projet du traité de paix. Il devait être inséré de
mot à mot dans la convention, aussi bien que la promesse de le signer
dès que le ministre de l'empereur à Londres aurait pouvoir de le signer
pareillement au nom de son maître.

Ce fut à cette occasion que le maréchal d'Huxelles, président du conseil
établi pour les affaires étrangères, refusa sa signature. Le comte de
Cheverny, conseiller du même conseil, qui subsistait encore, se montra
plus facile. L'ambassadeur d'Espagne, persuadé des dispositions du
premier, comptait toujours que les sollicitations de Stanhope seraient
infructueuses, et que la cour de France était encore éloignée de
souscrire à la quadruple alliance. Il voyait cependant, disait-il, un
nuage épais et noir, qu'il fallait dissiper\,; mais se confiant en son
éloquence, il se flatta d'éclaircir les ténèbres par un mémoire qu'il
fit pour combattre les oppositions d'Angleterre, et la négociation qu'il
s'agissait alors de conclure. On disait à Paris qu'elle l'avait été peu
de jours auparavant dans un souper que le régent avait donné à Stanhope
au château de Saint-Cloud. Cellamare ne le pouvait croire, persuadé que
Son Altesse Royale attendait le retour d'un courrier dépêché à Vienne,
et que jusqu'à son arrivée les instances de Stanhope n'ébranleraient pas
la volonté du régent. Ainsi le moment lui parut propre à communiquer à
Son Altesse Royale, ensuite aux maréchaux d'Huxelles et de Villeroy, le
mémoire qu'il avait fait contre les propositions du ministre
d'Angleterre. Outre la force des raisons contenues dans ce mémoire,
Cellamare espérait beaucoup des ministres de Moscovie et de Sicile. Le
premier s'opposait ouvertement à la quadruple alliance jusqu'au point
d'avoir présenté un mémoire au régent pour la combattre. Le second
n'avait rien oublié pour détourner Son Altesse Royale de s'unir si
étroitement avec les Anglais. Il avait peint le génie et les maximes de
la nation avec les couleurs qui convenaient le mieux pour détourner tout
Français de prendre confiance en elle\,; mais la ferveur de Provane se
ralentissait, il ne savait plus quel langage il devait tenir, et depuis
quelques jours, il paraissait tout hors de lui, et consterné d'avoir
appris de Stairs que la flotte d'Espagne faisait voile vers la Sicile.

Cellamare n'avait pu opposer aux assurances certaines de Stairs que des
raisonnements vagues et des présomptions, que les forces d'Espagne
n'agiraient que de concert avec le roi de Sicile, avouant au reste qu'il
ignorait absolument les ordres dont les commandants de la flotte et des
troupes étaient chargés. Il était vrai qu'Albéroni ne l'en avait pas
instruit\,; mais il lui avait communiqué, sous un grand secret et par
des voies détournées, les propositions dures que le roi d'Espagne avait
faites au roi de Sicile, et Cellamare avait pénétré que, nonobstant le
secret qui lui était recommandé, le régent avait eu connaissance de ces
propositions. Ce ne pou voit être par la cour de Turin, car alors le roi
de Sicile se flattait encore de réussir dans sa négociation à Madrid\,;
il croyait avoir fait toutes les offres que le roi d'Espagne pouvait
attendre et désirer de sa part, et si le roi d'Espagne avait gardé si
longtemps le silence, le roi de Sicile ne semblait l'attribuer qu'au
désir qu'il avait de voir, avant conclure, quel serait le succès de ses
premières expéditions. Il était persuadé, et même plusieurs ministres
d'Espagne croyaient pareillement que, sans une union intime avec lui,
l'Espagne ne réussirait pas dans ses projets\,; que, si l'intelligence
était bien établie, et les entreprises faites de concert, le Milanais
serait bientôt enlevé aux Impériaux, qui déjà même songeaient à retirer
leurs troupes à Pizzighittone et à Mantoue. Mais Albéroni prévenu de ses
propres talents, enivré de ce qu'il croyait avoir fait pour l'Espagne,
comptait de pouvoir se passer de l'alliance et des secours de tous les
potentats de l'Europe\,; sûr du succès de ses projets, il n'était plus
occupé que de savoir ce qu'on disait de lui dans les pays étrangers. Il
espérait que sa curiosité serait payée par les louanges qu'on donnerait
de toutes parts à ses lumières, à sa vigilance, à son activité, et par
la comparaison flatteuse que chacun selon lui devait faire de la misère
précédente où les rois d'Espagne s'étaient vus depuis longtemps réduits,
avec l'état de splendeur, de force et de puissance où ses soins avaient
enfin fait remonter le roi Philippe. C'était aux talents d'un tel
ministre, infiniment supérieur dans sa pensée à tous ceux qui l'avaient
précédé en de pareils postes, que Sa Majesté Catholique devait,
disait-il, le bonheur d'être désormais regardée avec respect et non
traitée comme un petit compagnon.

Il voulait que ces hautes idées fussent principalement données en
Hollande, parce que l'accession de la république à la quadruple alliance
était toujours douteuse. Ainsi, Cellamare, Monteléon et Beretti, comme
étant les ministres du roi d'Espagne qui se trouvaient le plus à portée
d'agir utilement auprès des États généraux, soit par écrit, soit par
leurs discours, reçurent des ordres nouveaux et pressants d'employer
tout leur savoir-faire pour exciter toute l'attention de la république
sur les suites funestes qu'elle devait craindre pour son gouvernement,
si elle se laissait entraîner aux sollicitations qu'on ne cessait de lui
faire d'entrer dans la quadruple alliance. Ces ministres devaient en
parler sans ménagement comme d'un projet injuste, abominable, criminel,
dont l'unique but était de soutenir les intérêts particuliers et
personnels du roi Georges et ceux du régent\,; projet si détestable,
disait Albéroni, que l'univers était étonné que la Hollande l'eût
seulement écouté\,; que bientôt elle s'en repentirait et confesserait
humblement qu'en l'écoutant seulement elle se mettait la corde au cou.
Ces invectives, et tant d'épithètes que la passion dictait à Albéroni,
seraient cependant tombées, même de son aveu, si les Anglais eussent
offert la restitution de Gibraltar\,; mais, pour l'obtenir, il fallait,
suivant la pensée d'Albéroni, un ambassadeur à Londres plus fidèle à son
maître que Monteléon ne l'était au roi d'Espagne. Le cardinal l'accusait
de faire en Angleterre le métier de marchand bien plus que celui de
ministre. Il lui reprochait de dire que l'air de Londres lui était
mauvais, que sa santé y dépérissait, prétexte qu'il cherchait pour aller
jouir quelque part en repos de ses gains illicites, aussi condamnable
dans sa sphère que l'était dans la sienne Cadogan, insigne voleur,
fripon achevé, qui avait enlevé de Flandre plus de deux cents mille
pistoles, indépendamment des autres vols ignorés, enfin vrai ministre
d'iniquité.

Pendant qu'Albéroni déclamait à Madrid, Cadogan agissait en Hollande, et
pour engager cette république à souscrire à la quadruple alliance, il
n'épargnait ni présents ni promesses. Les parents de sa femme, puissants
à Amsterdam, travaillaient à rendre utiles les moyens qu'il mettait en
usage pour assurer le succès de ses négociations. Les personnes privées,
les magistrats mêmes, touchés de l'appât d'un gain que peut-être ils ne
croyaient pas contraire aux intérêts de leur patrie, se permettaient
sans scrupule d'agir et de conseiller au préjudice de l'Espagne.
Beretti, malgré sa vivacité, cédait à la nécessité du temps\,; il
conseillait à son maître de dissimuler, de suspendre tout ressentiment,
et de remarquer seulement ceux qui, dans ces temps difficiles, feraient
paraître de bonnes intentions. Il mettait dans ce nombre Vander Dussen,
chef de la députation de la province de Zélande, qui tout nouvellement
l'avait assuré que cette province désirait toutes sortes d'avantages au
roi d'Espagne, et que l'expérience ferait voir comment elle se
comporterait. Beretti s'appuyait encore sur l'éloignement et sur la
crainte que la province de Hollande et la ville d'Amsterdam en
particulier avaient témoignée jusqu'alors, d'engager la république à
soutenir une partie des frais de la guerre que le traité proposé
pourrait entraîner, d'autant plus que ces dépenses retomberaient
principalement sur la ville et sur la province, qui, dans les
répartitions, supportent toujours le poids le plus pesant des charges de
l'État.

En effet, il s'était tenu quelque temps auparavant une conférence entre
les deux ministres d'Angleterre en Hollande, Paneras, bourgmestre
régent, et Buys, pensionnaire de la ville d'Amsterdam. Ce dernier avait
représenté aux Anglais qu'une des clauses du projet de l'alliance
portait\,: «\,Que, si malheureusement toutes les conditions n'étaient
pas acceptées, les alliés prendraient les mesures convenables pour en
procurer l'accomplissement et le rétablissement du repos de l'Italie.\,»
Qu'une telle clause causait une juste inquiétude aux Provinces-Unies en
leur donnant lieu de craindre qu'elles ne fussent liées et forcées
d'entrer dans toutes les mesures que l'Angleterre proposerait dans la
suite. Pancras et Buys protestèrent qu'un pareil scrupule venait moins
d'eux que des autres députés, mais qu'il était absolument nécessaire de
le lever. Les ministres Anglais condescendirent à la proposition des
deux magistrats, et pour dissiper l'alarme des Provinces-Unies, ils
assurèrent qu'elles ne seraient engagées, en cas de refus, qu'à réunir
leurs soins, leurs instances, leurs démarches, avec les alliés, et
concerter avec eux les mesures qui seraient jugées les plus
convenables\,; qu'elles auraient, par conséquent, une entière liberté
d'agréer ou de rejeter les mesures qu'on leur proposerait, aussi bien
que de proposer celles qu'ils croiraient plus conformes, soit à
l'intérêt de leur État, soit à l'accomplissement du principal objet du
traité. Unetelle déclaration, faite verbalement aux députés des affaires
secrètes, parut suffisante pour calmer les soupçons d'esprits faibles et
difficultueux, et pour engager la province de Hollande à souscrire au
traité. Ce pas fait, les Anglais se promettaient que les États généraux
se trouveraient trop engagés pour reculer. Ils étaient contents de la
franchise et de la bonne volonté de Pancras et de Buys\,; ils ne le
furent pas moins de celle de Duywenworde, appelé depuis à la
consultation de la même affaire. Tous convinrent unanimement qu'il ne
suffisait pas que l'Angleterre seule fît la déclaration proposée\,;
qu'il était nécessaire que la France la fît en même temps par son
ambassadeur. Ils crurent que Châteauneuf ne répugnerait pas à la faire
telle qu'ils la désiraient, parce qu'il avait déjà dit aux députés
d'Amsterdam l'équivalent de ce qu'on lui demandait. Mais, s'agissant de
faire une déclaration au nom du roi, ils comprirent que le ministre de
Sa Majesté avait besoin d'un ordre particulier et précis, pour s'en
expliquer avec les députés aux affaires secrètes, et pour obtenir cet
ordre du régent, ils avertirent les ministres du roi d'Angleterre à
Londres qu'il était nécessaire d'engager l'abbé Dubois d'en écrire
fortement à Son Altesse Royale. Les intentions et la conduite de
Châteauneuf leur étaient fort suspectes\,; ils observaient jusqu'à ses
moindres démarches. S'il dépêchait un courrier en France, ils
l'accusaient de travailler secrètement à séduire la cour par de fausses
représentations. Il parut en Hollande un écrit contre l'alliance\,; le
nommé d'Épine, agent du duc de Savoie auprès des États généraux, passa
pour en être l'auteur\,; les ministres anglais répandirent qu'il avait
été composé de concert avec l'ambassadeur de France, et que son neveu
jésuite avait eu part à l'ouvrage. Ils se plaignirent ouvertement des
discours que Châteauneuf avait tenus au greffier Fagel, prétendant que
ce ministre avait dit que les changements étaient si fréquents en
Angleterre que le régent ne pouvait compter sur les secours de cette
couronne, et qu'il serait contre la, prudence d'entrer en des
engagements qui certainement conduiraient là France à la guerre, si les
États généraux ne se liaient avec elle. Châteauneuf leur avait dit à
eux-mêmes que le roi comptait que la république entrerait ouvertement et
franchement dans la dépense et les risques, et comme le régent devait
donner son bon argent, il s'attendait aussi que l'État en devait faire
de même quant à sa proposition\,; que jamais Son Altesse Royale ne se
serait embarquée en cette affaire si elle n'avait été positivement
assurée qu'il en serait ainsi. Sur de tels discours les Anglais se
crurent en droit de dire que Châteauneuf avait prévariqué, car enfin
c'était un crime, à leur avis, de presser les États généraux de
consentir à ce qui devait être réservé pour faire la matière des
articles secrets, avant que la république eût pris sa résolution sur
l'alliance\,; c'était agir contre les mesures prises, c'était gâter les
affaires en Hollande, où le moyen infaillible de les perdre était de les
précipiter\,; un négociateur habile et sincère devait savoir qu'on ne
pouvait amener l'État que par degrés à consentir au projet du traité\,;
il devait agir sur ce principe, et par conséquent Châteauneuf n'était
pas excusable, puisqu'il savait, que les députés d'Amsterdam entendaient
que leurs signatures les engageaient à prendre part à toutes les mesures
qu'on jugerait nécessaires pour l'exécution du traité, toutefois autant
que leurs divisions et le mauvais état de leurs finances le pourraient
permettre. Nonobstant cette clause qu'on pouvait effectivement regarder
comme un moyen, que le roi d'Angleterre laissait aux Hollandais de
s'exempter de toute contribution aux frais de la guerre que le traité
pouvait exciter, les ministres de ce prince ne pouvaient pardonner à
Châteauneuf d'avoir laissé entendre au régent que les États généraux,
entrant dans le traité, ne seraient tenus qu'à la simple interposition,
de leurs bons offices. C'était à leur avis un crime à l'ambassadeur de
France d'avoir donné lieu par sa conduite et par ses discours aux
soupçons injurieux formés contre la pureté des intentions du régent\,;
ils assurèrent le roi leur maître que la déclaration demandée par
quelques députés était un acte qui n'engageait ni la France ni
l'Angleterre, qu'il n'en avait pas même été fait mention sur le registre
des états\,; que le Pensionnaire avait seulement spécifié dans ses notes
particulières, au bas du registre, en quels termes les députés
désiraient que la déclaration fût conçue. Les termes étaient les
suivants\,: «\, Que si, contre toute attente, les rois d'Espagne et de
Sicile refusaient d'accepter les conditions stipulées pour eux dans
ledit traité et qu'il fût nécessaire de prendre des mesures ultérieures,
les États généraux seraient dans une entière liberté de délibérer par
rapport auxdites mesures, comme ils étaient avant que d'avoir signé le
traité.\,»

Ainsi, disaient Cadogan et Widword, c'était une malice noire et un
dessein formé d'embrouiller le traité que le retardement que Châteauneuf
apportait à s'expliquer comme eux aux députés des affaires secrètes\,;
qu'un tel retardement pouvait faire naître des jalousies incroyables\,;
et, sur ce fondement, ils pressèrent le roi leur maître de solliciter
vivement cette déclaration de la part de la France, comme un moyen
nécessaire pour fixer enfin l'incertitude de quelques provinces qui
hésitaient encore de signer le projet de l'alliance, quoique la plus
grande partie des députés des principales villes de Hollande fussent
autorisés à consentir au traité. Le pensionnaire Heinsius et les autres
ministres de Hollande qu'on avait toujours regardés comme amis et
partisans de l'Angleterre, employaient tous leurs soins à vaincre le
répugnance de quelques magistrats d'Amsterdam, trop persuadés que, le
principal bien de la république consistant à demeurer en repos, il ne
lui convenait pas de s'engager dans les nouveaux embarras que le projet
dont il s'agissait pouvait produire. Quelques autres magistrats des
autres grandes villes de la province de Hollande étaient aussi de la
même opinion. Il fallait ramener ces esprits difficiles, et leur
inspirer avant l'assemblée des États de la province l'unanimité de
sentiments pour concourir tous à l'acceptation du traité.

Chaque jour la chose devenait plus pressante\,: car alors le czar
inquiétait toutes les puissances du nord par les mouvements qu'il
faisait faire à sa flotte. Le roi d'Angleterre et les Hollandais étaient
également alarmés des apparences qu'ils croyaient voir à une paix
prochaine, suivie de liaisons secrètes entre le roi de Suède et le
Moscovite. Quelques voyages du baron de Gœrtz, ministre confident du roi
de Suède, autorisaient les soupçons qu'on avait d'une alliance entre ces
deux princes, et de la jonction de leurs flottes. L'ambassadeur
d'Espagne en Hollande se flattait plus que personne d'une diversion du
côté du nord, et s'attribuait tout le mérite de ce qu'elle produirait de
favorable aux intérêts de son maître, se donnant aussi la gloire de
l'incertitude et même de la répugnance que la province de Hollande
témoignait à l'acceptation du traité, chaque fois que les États de la
province se séparaient sans avoir de résolution sur ce sujet. Mais
l'inquiétude que les négociations secrètes entre le roi de Suède et le
czar avaient causée cessa bientôt. Le czar ne voulait pas abandonner le
roi de Prusse, et le roi de Suède refusait alors de traiter avec les
amis du czar. La conjoncture n'était pas favorable pour retirer ce que
le roi de Prusse avait acquis en Poméranie. Le roi de Suède, attendant
un moment heureux, ne put s'accorder avec les Moscovites. Ainsi le czar,
changeant de pensée, fit quelques démarches pour se réconcilier avec le
roi d'Angleterre. Rien n'était plus à souhaiter pour le roi Georges. Il
n'y avait qu'à perdre pour lui et pour les Anglais dans une guerre
contre la Moscovie\,; les conséquences en pouvaient être fatales à ses
États d'Allemagne, et quant aux Anglais, elle ruinait sans profit un
commerce avantageux à la nation. Il était d'ailleurs de l'intérêt de ce
prince de conserver la paix en Europe, et la guerre pouvait donner lieu
à des révolutions dans la Grande-Bretagne. Persuadé de cette vérité, il
témoignait un désir ardent d'éviter toute rupture avec l'Espagne. Il
vantait les bons offices qu'il avait rendus à cette couronne pour
établir la paix générale en Europe. Il se plaignait des mauvais
traitements qu'il recevait de la cour d'Espagne, en échange de ses
attentions et de ses empressements pour elle. Mais il s'en plaignait
tendrement, et Stanhope eut ordre de mesurer les discours qu'il
tiendrait à Madrid, et de faire ses représentations de manière que le
roi d'Espagne, persuadé des bonnes raisons et de l'amitié du roi
d'Angleterre, voulût bien, se porter à changer de conduite à son égard.
Nancré était suspect aux ministres d'Angleterre. Stanhope eut ordre de
le prier d'être témoin des représentations qu'il ferait, et de
l'accompagner à l'audience d'Albéroni. Monteléon, ami de Stanhope,
soupçonné même d'être intéressé à plaire au roi d'Angleterre et à ses
ministres, n'avait rien oublié pour préparer au négociateur un accueil
favorable à la cour de Madrid, persuadé d'ailleurs qu'il se ressentirait
à Londres de la manière dont ce comte, ministre confident du roi
d'Angleterre, serait reçu en Espagne. Il assura donc, sur sa propre
connaissance, que le comte de Stanhope avait toujours été
particulièrement porté pour les intérêts de l'Espagne, qu'il les
regardait comme inséparables de ceux de l'Angleterre, et sur la foi de
Craggs, l'autre secrétaire d'État d'Angleterre, il répondit hardiment
que le motif du voyage de Stanhope à Madrid était de porter à Sa Majesté
Catholique non seulement des assurances, mais des preuves de l'amitié
que le roi d'Angleterre avait pour elle, et de l'attention très
particulière de ce prince aux intérêts de l'Espagne. Ainsi, dans cette
vue, Stanhope tenterait tous les moyens possibles pour établir la
tranquillité publique par une paix stable entre l'empereur et le roi
d'Espagne\,; autrement un ministre de cette sphère demeurerait
tranquillement auprès de son maître et ne s'exposerait pas aux risques
d'une longue absence, simplement pour être porteur de propositions peu
convenables à l'honneur et à la satisfaction d'un grand roi tel que le
roi d'Espagne, et par ces considérations Monteléon conclut que ce voyage
ne pouvait causer aucun préjudice à l'Espagne. Toutefois, exagérant
l'affection singulière du roi Georges aussi bien que son zèle et la
droiture de ses intentions pour la paix, il avait dit très clairement,
et comme une preuve incontestable des sentiments de ce prince, qu'il se
déclarerait ennemi de celui qui refuserait d'accepter la proposition
qu'il avait faite.

Le public avait lieu de juger que le refus ne viendrait pas de la part
de l'empereur, et Monteléon, bien instruit de l'état des affaires de
l'Europe, aurait eu peine à penser différemment. Mais comme il lui
convenait que le roi son maître fût persuadé de la sincérité du roi
d'Angleterre et de ses ministres, il assura que la menace de ce prince
regardait uniquement la cour de Vienne, fondé sur ce que Craggs avait
dit que cette cour était inflexible sur les conditions du projet,
qu'elle refusait opiniâtrement les sûretés demandées pour les
successions de Parme et de Toscane, qu'elle rejetait avec une hauteur
égale les changements proposés, enfin les autres conditions jugées si
nécessaires, que sans elles les médiateurs ne pouvaient se charger de
faire exécuter les traités\,; mais que, si elle se rendait trop
difficile, flattée par l'espérance d'une paix prochaine avec les Turcs,
ses prétentions étant connues, le plan serait facile à changer\,;
qu'alors le roi d'Espagne connaîtrait l'injustice de ceux qui lui
dépeignaient le ministère d'Angleterre comme partial pour l'empereur. Il
y a des moments où les princes les plus liés d'intérêt pensent
différemment, mais l'union entre eux est intime. Cette diversité de
sentiments n'est qu'un nuage qui obscurcit la lumière du soleil pendant
quelques instants sans l'éteindre. Le conseil de Vienne avait fait
plusieurs changements au projet envoyé de Londres. Les ministres Anglais
avaient désapprouvé cette contradiction de la part des Allemands, mais
les ratures faites ensuite par les ministres d'Angleterre ne pouvaient
altérer l'union entre les deux cours\,; et celle de Londres, travaillant
uniquement pour la grandeur et les avantages de la maison d'Autriche,
était bien assurée que l'empereur serait docile à ses décisions\,: elle
n'était pas moins sûre de la docilité de la France. L'abbé Dubois avait
déclaré qu'elle ferait tout ce que voudrait le roi d'Angleterre, que le
régent lui commandait de signer tout ce que Sa Majesté Britannique
jugerait à propos de lui prescrire. Ainsi les ministres d'Angleterre,
maîtres de la conclusion, ne la différaient que pour essayer d'amener
l'empereur à se désister des conditions qu'il avait ajoutées au projet,
ou pour se faire honneur des tentatives, même inutiles, qu'ils feraient
encore à Vienne\,; mais qui que ce soit ne croyait que cette cour
consentît à la condition que la France demandait, comme condition
capitale, de mettre dans les places des duchés de Toscane et de Parme
des garnisons suisses entretenues et payées aux dépens de la France et
de l'Angleterre. Monteléon disait lui-même que, si l'empereur y
consentait, le roi d'Espagne ne pouvait se dispenser d'accepter le
projet. Ces raisonnements incertains ne faisaient rien au fond de
l'affaire. L'union était intime entre le roi d'Angleterre et le régent,
et Stanhope avec Stairs trouvaient à Paris les mêmes dispositions, les
mêmes sentiments, les mêmes facilités dont l'abbé Dubois à Londres ne
cessait de renouveler les assurances. Le régent et le maréchal
d'Huxelles évitaient encore d'avouer aux ministres étrangers l'état
véritable de la négociation. Cellamare importunait par ses
représentations et par ses questions pressantes\,: on lui répondait
sèchement que le traité de la quadruple alliance n'était pas encore
signé, mais qu'il fallait prendre les mesures nécessaires pour assurer
le repos de l'Europe. C'en était assez pour instruire un homme d'esprit
du fait qu'il voulait pénétrer. Il conclut donc sans peine qu'on
travaillait vivement à finir le traité\,; faute de ressources, il
attendait du secours du bénéfice du temps ou des inégalités de la
Hollande, enfin des succès que l'armée d'Espagne aurait peut-être en
Italie. Albéroni lui laissait ignorer l'objet de cette expédition\,;
mais les nouvelles publiques de la route que tenait la flotte
commençaient à dissiper les doutes, et on jugeait, avec apparence de
certitude, que le dessein du roi d'Espagne regardait la Sicile. On
croyait le roi de Sicile de concert avec Sa Majesté Catholique, parce
qu'il ne paraissait pas vraisemblable qu'elle entreprît une guerre
éloignée sans alliés, qu'il fallait soutenir par mer, et qu'elle voulût
attaquer en même temps la maison d'Autriche et celle de Savoie. On
supposait donc, des traités secrets entre le roi d'Espagne et le roi de
Sicile, parce que la prudence et la raison d'État le voulait ainsi. Le
récent dit à Provane qu'il savait sûrement que le roi de Sicile avait
retiré ses troupes du château de Palerme, de Trapani, de Syracuse, pour
y laisser entrer apparemment les troupes espagnoles. Provane, de son
côté, mettait toute son application à pénétrer les intentions et le
dessein du régent, et remarquant seulement des contradictions fréquentes
dans les discours et dans les démarches de ce prince, il en inférait que
la vue principale, même l'unique vue de Son Altesse Royale, était
d'assurer la paix à la France pour s'assurer à lui-même la couronne.
Fondé sur ce principe, Provane avertit son maître que le roi
d'Angleterre pour se maintenir tranquillement sur le trône, et M. le duc
d'Orléans pour y monter, procureraient de tout leur pouvoir les
avantages du roi d'Espagne\,; qu'ils sacrifieraient à leurs desseins les
intérêts, du roi de Sicile, s'ils pouvaient à ce prix engager Sa Majesté
Catholique à l'alliance proposée. Comme la conclusion en demeurait
encore secrète, les ministres intéressés à la traverser continuaient
d'agir auprès du régent pour en représenter les inconvénients à ce
prince. L'envoyé du czar réitéra ses instances, et lui dit qu'en vain
son maître s'était proposé de mettre l'équilibre dans l'Europe, si Son
Altesse Royale renversait par les conditions dont elle convenait les
dispositions que le czar avait faites pour empêcher que la paix générale
ne fût troublée par l'ambition des princes dont la puissance n'était
déjà que trop augmentée. Le régent répondit qu'il n'avait pas signé la
quadruple alliance\,; que la ligue qu'il avait faite avec l'Angleterre
ne l'empêchait en aucune manière de s'unir avec le czar, et de concourir
aux bonnes intentions de ce prince. Son Altesse Royale ajouta qu'elle
souhaiterait de le voir dès ce moment réuni parfaitement avec les rois
de Suède et de Prusse, la triple alliance entre eux signée, et ces
princes déjà prêts à entrer en action\,: discours qui ne coûtaient rien
à tenir, mais si peu conformes aux dispositions où se trouvait alors le
régent, qu'il reprocha au maréchal de Tessé d'avoir formé les entrevues
secrètes entre le prince de Cellamare et le ministre moscovite\,; et ces
reproches, dont le comte de Provane fut bientôt instruit, parvinrent
bientôt à la connaissance du roi de Sicile. Toutefois l'attention que
Provane apportait à découvrir ce {[}qui{]} se passait dans une
conjoncture si critique et si délicate pour son maître, ses liaisons
avec les ministres étrangers résidant lors à Paris, ses soins, ses
peines, ses intrigues, ses amis, tous les moyens enfin qu'il employait
pour pénétrer la vérité et la situation des affaires, étaient moyens
inutiles pour lui apprendre certainement et l'objet véritable de
l'armement d'Espagne et l'état du traité d'alliance entre la France et
l'Angleterre. Il ignorait encore l'un et l'autre le 15 juillet. Il
inclinait à croire avec tout Paris que l'alliance était signée. Mais le
régent l'assurait si positivement du contraire qu'il se réduisait à
penser que Son Altesse Royale avait simplement signé une convention
particulière avec Stanhope pour assurer la garantie de la France, en
faveur des États que le roi Georges possédait en Allemagne, clause omise
dans le traité fait avec ce prince deux ans auparavant. L'expédition de
deux courriers extraordinaires dépêchés en même temps, l'un à Londres
par Stanhope, l'autre à Vienne par Koenigseck, confirmait le mouvement
qui paraissait dans les affaires, mais dont la qualité ne se démêlait
pas encore\,; Cellamare crut que le régent attendrait, pour signer
l'alliance, le retour du courrier dépêché à Vienne. On disait qu'elle
l'avait été après un souper que le régent avait donné à Stanhope à
Saint-Cloud, mais on en doutait, et les politiques assuraient que le
régent mesurerait un peu plus ses pas, surtout après l'éclat que le
maréchal d'Huxelles avait fait en refusant de signer. Le bruit que fit
ce refus cessa bientôt et ne produisit nul effet. Les deux ministres
Anglais eurentla satisfaction de voir le régent, excité par leurs
plaintes, prendre feu et ordonner au maréchal d'Huxelles de signer ou de
se démettre de son emploi, et le maréchal signer. Ils obtinrent aussi
des ordres précis à Châteauneuf de se conformer à ce que les ministres
d'Angleterre feraient à la Haye, et jugeraient à propos qu'il fît
lui-même auprès des États généraux. Ainsi les ministres d'Espagne se
flattaient inutilement de quelque résolution favorable et de quelque
secours du côté de la Hollande. Ils interprétaient à leur avantage les
délais que cette république apportait à s'expliquer. Le soin qu'elle
avait de gagner du temps était, selon eux, une marque évidente du désir
qu'elle avait de se retirer du labyrinthe dangereux où on tâchait de
l'engager. Cellamare excitait Beretti à continuer de représenter aux
États généraux qu'il était de leur prudence autant que leur intérêt
d'observer une neutralité parfaite, et d'éviter non seulement les
dépenses, mais de plus le danger où on voulait les entraîner uniquement
pour favoriser et pour soutenir les vues et les intérêts de deux
princes, dont l'un voulait monter sur le trône, l'autre se maintenir sur
celui où la fortune l'avait élevé. Les Hollandais différaient à se
résoudre\,; mais la crainte seule les retenant, on jugeait assez que le
côté où elle serait la plus forte serait celui où la balance pencherait.
Les instructions manquaient aux ambassadeurs d'Espagne dans les cours
étrangères. Albéroni, persuadé que le moyen le plus sûr de garder son
secret était de ne le communiquer à personne, les laissait dans une
ignorance totale des desseins, même des résolutions du roi leur maître.
Cellamare, mécontent des Anglais, surtout de Stairs, était réduit à le
rechercher, à l'inviter à des repas chez lui, à demander à ce même
Stairs à dîner dans sa maison de campagne, espérant par un tel commerce
pouvoir au moins découvrir quelque circonstance de ce qu'il se passait,
plus certaine que les nouvelles qu'on en répandait dans le public. Le
mois de juillet s'avançait, et tout ce que Cellamare savait encore de la
flotte d'Espagne était qu'on avait appris par des lettres de Marseille
qu'elle était arrivée à Cagliari le 23 juin\,; que l'opinion commune
était qu'elle ferait le débarquement des troupes espagnoles en Sicile.

\hypertarget{chapitre-ix.}{%
\chapter{CHAPITRE IX.}\label{chapitre-ix.}}

1718

~

{\textsc{Albéroni confie à Cellamare les folles propositions du roi de
Sicile au roi d'Espagne, qui n'en veut plus ouïr parler.}} {\textsc{-
Duplicité du roi de Sicile.}} {\textsc{- Ragotzi peu considéré en
Turquie.}} {\textsc{- Chimère d'Albéroni.}} {\textsc{- Il renie Cammock
au colonel Stanhope.}} {\textsc{- Albéroni dément le colonel Stanhope
sur la Sardaigne.}} {\textsc{- Éclat entre Rome et Madrid.}} {\textsc{-
Raisons contradictoires.}} {\textsc{- Vigueur du conseil d'Espagne.}}
{\textsc{- Sagesse et précautions d'Aldovrandi.}} {\textsc{- Ses
représentations au pape.}} {\textsc{- Sordide intérêt du cardinal
Albane.}} {\textsc{- Timidité naturelle du pape.}} {\textsc{- Partage de
la peau du lion, avant qu'il soit tué.}} {\textsc{- Le secret de
l'entreprise demeuré secret jusqu'à la prise de Palerme.}} {\textsc{-
Déclaration menaçante de l'amiral Bing à Cadix, sur laquelle Monteléon a
ordre de déclarer l'artificieuse rupture en Angleterre et la révocation
des grâces du commerce.}} {\textsc{- Sentiments d'Albéroni à l'égard de
Monteléon et de Beretti.}} {\textsc{- Albéroni, dégoûté des espérances
du nord, s'applique de plus en plus à troubler l'intérieur de la
France\,; ne peut se tenir de montrer sa passion d'y faire régner le roi
d'Espagne, le cas arrivant.}} {\textsc{- Aventuriers étrangers dont il
se défie.}} {\textsc{- Rupture éclatante entre le pape et le roi
d'Espagne.}} {\textsc{- Raisonnements.}}

~

Enfin, Albéroni s'ouvrit à cet ambassadeur, et lui confiant les
propositions que le roi de Sicile avait faites au roi d'Espagne, il
étendit la confiance jusqu'à lui apprendre que Sa Majesté Catholique ne
voulait plus en entendre parler. Ces propositions étaient, que le roi
d'Espagne attaquerait le royaume de Naples, ferait en même temps passer
dix mille hommes en Lombardie pour y agir sous les ordres du roi de
Sicile. Il demandait que dans les places qui seraient prises, et dans le
royaume de Naples, et dans l'État de Milan, les garnisons fussent
composées moitié de troupes espagnoles, moitié de troupes savoyardes
sous le commandement d'un officier savoyard à qui la garde de la place
serait confiée\,; qu'après la conquête du royaume de Naples, le roi
d'Espagne fit passer vingt mille hommes en Lombardie, que Sa Majesté
Catholique payerait\,; que, pour suppléer à l'artillerie et, aux
munitions, qu'elle ne pouvait envoyer dans le Milanais, elle payerait
les sommes d'argent, dont on conviendrait pour en tenir lieu. Le roi de
Sicile exigeait de plus un million d'avance pour faire marcher son
armée, et par mois soixante mille écus de subsides tant que la guerre
durerait. Il voulait commander également toutes les troupes, celles de
l'Espagne aussi absolument que les siennes, disposer pleinement des
quartiers d'hiver. Il consentait à partager les contributions qui se
lèveraient sur le pays ennemi, et se contentant de la moitié, il
laissait l'autre à l'Espagne. Des conditions si dures, dictées en
maître, irritèrent le roi d'Espagne et son premier ministre, d'autant
plus qu'ils savaient que, pendant que le roi de Sicile les faisait à
Madrid, il travaillait à Vienne, et, pressait vivement la conclusion
d'une ligue avec l'empereur. Les Anglais même en avertirent Albéroni, et
le ministre de Sicile à Madrid, ne pouvant nier une négociation entamée
à Vienne, se défendit en assurant qu'elle ne roulait que sur les
propositions de mariage d'une archiduchesse avec le prince de Piémont\,;
que d'ailleurs il n'était nullement question de la Sicile, comme de
fausses nouvelles le supposaient. Ainsi l'Espagne, mécontente du roi de
Sicile, entreprenait, sans alliés, de chasser les Allemands de l'Italie.
Le roi d'Espagne ne pouvait même se flatter de l'espérance d'aucune
diversion favorable au succès de ses desseins. Albéroni était désabusé
des projets et des entreprises du czar et du roi de Suède. Il en avait
reconnu la chimère aussi bien que celle qu'il s'était faite de susciter
à l'empereur de dangereux ennemis par le moyen et par le crédit du
prince Ragotzi à la Porte\,; car, au lieu de la considération que
Ragotzi s'était vanté qu'il trouverait auprès des Turcs, il avait été
obligé de dire, pour se relever auprès du Grand Seigneur et de ses
ministres, que le roi d'Espagne lui proposait de quitter la Turquie, et
de venir prendre le commandement des troupes espagnoles que Sa Majesté
Catholique voulait lui confier. Pour autoriser la supposition, il avait
fait croire qu'un nommé Boischimène, envoyé véritablement auprès de lui
par Albéroni, était venu exprès lui faire cette proposition. Il avait
affecté de persuader à la Porte qu'il entretenait une correspondance
avec la cour de Madrid, assez vive pour y dépêcher des courriers\,; et
pour y réussir, il avait nouvellement profité, de la bonne volonté ou
plutôt de l'empressement et de l'impatience qu'un officier français eut
de sortir pour jamais de Constantinople, où il s'était rendu avec un
égal empressement, attiré et persuadé par l'espérance qu'il s'était
formée de s'élever à une haute fortune par la protection de Ragotzi. Cet
officier, nommé Montgaillard, lui offrit de porter en Espagne les
lettres qu'il voudrait écrire au cardinal Albéroni. L'offre acceptée,
l'officier partit bien résolu de ne rentrer jamais dans un tel
labyrinthe, et, pour n'y plus retomber, il se mit au service du roi
d'Espagne, et prit de l'emploi dans un régiment d'infanterie wallonne.

Le roi d'Espagne, dénué d'alliés, persista cependant dans la résolution
qu'il avait fortement prise d'essayer une campagne, déclarant que,
quelque succès qu'eussent ses armes, il serait également porté à
recevoir des propositions de paix lorsqu'elles seraient honorables pour
lui, et telles que le demandait la sûreté de l'Europe, dont il voulait
maintenir le repos et la liberté. C'est ce qu'Albéroni répondit aux
instances du colonel Stanhope, l'assurant en même temps que le plan
proposé à Sa Majesté Catholique par la France et par l'Angleterre, pour,
un traité, était si contraire à son idée, que jamais elle n'accepterait
un tel projet. Malgré tant de fermeté le colonel ne laissait pas de
remarquer que le cardinal sachant la flotte Anglaise à la voile parlait
avec plus de modération et de retenue sur l'article des Anglais
négociants en Espagne. «\,Leur sort, disait-il, dépendra des ordres que
l'amiral Bing a reçus du roi d'Angleterre.\,» Ce ministre était persuadé
qu'ils étaient bornés à traverser le passage et le débarquement des
troupes espagnoles en Italie. L'un et l'autre étant exécutés suivant son
calcul, il supposait que l'Angleterre croirait, en envoyant sa flotte,
avoir satisfait aux engagements qu'elle avait pris avec l'empereur sans
être obligée de les étendre plus loin, et de faire de gaieté de coeur la
guerre à l'Espagne. Il voulait ménager la cour d'Angleterre et la nation
Anglaise\,; il conservait l'espérance d'y réussir, dans le temps même
qu'il voyait les forces navales de cette couronne couvrir les mers pour
soutenir les intérêts de l'empereur, et lui porter de puissants secours
contre les entreprises du roi d'Espagne. Un officier de marine Anglais
s'était donné à Sa Majesté Catholique. Son nom était Camock, et le
projet dont il avait flatté le cardinal était de corrompre environ
quarante officiers de la flotte Anglaise, de les faire passer au service
d'Espagne, quelques-uns même avec les vaisseaux qu'ils commandaient.
Stanhope se plaignit qu'une telle proposition eût été acceptée dans un
temps de paix et d'union entre les couronnes d'Espagne et d'Angleterre.
Albéroni répondit à ces plaintes en niant qu'elles fussent légitimes\,;
il traita Camock de visionnaire, dit que son projet était celui d'un fou
et d'un enragé\,; que le roi d'Espagne avait actuellement à son service
plus d'officiers de marine qu'il ne pouvait en employer. Il assura que
jamais il n'avait eu de correspondance avec ce Camock\,; qu'il ne le
connaissait pas, quoique véritablement il eût reçu de Paris plusieurs
lettres en sa faveur, et que Cellamare le lui eût recommandé
particulièrement. Il n'avait point encore le projet du roi d'Espagne, et
le mois de juillet s'avançait sans que le colonel Stanhope sût autrement
que par les conjectures et par les raisonnements vagues du public quelle
était la destination de l'escadre espagnole. On jugeait qu'elle
aborderait aux côtes de Naples ou de Sicile, et on jugeait par les
conférences fréquentes que le ministre de, Sicile avait avec le
cardinal, apparences d'autant plus capables de tromper, qu'il était
vraisemblable que le roi d'Espagne, voulant porter la guerre en Italie,
aurait apparemment pris ses liaisons, et concerté ses projets avec le
seul prince de qui l'union, la conduite et les forces pouvaient assurer
le succès de l'entreprise, et rendre inutile l'opposition des Allemands.
C'était pour le cardinal un sujet de triomphe, non seulement de cacher
ses desseins, mais de tromper par de fausses avances ceux même qu'il
désirait le plus de ménager. Le colonel Stanhope l'avait éprouvé, et
pour lors il avait eu besoin de tout le crédit du comte de Stanhope son
cousin pour se justifier auprès du roi d'Angleterre d'avoir écrit trop
légèrement que le roi d'Espagne accepterait le traité si la Sardaigne
lui était laissée. Il citait Nancré comme témoin de l'aveu que le
cardinal leur en avait fait. Nancré, de son côté, convenait qu'ils
avaient souvent, Stanhope et lui, rebattu cet article avec Albéroni, que
jamais ce ministre n'avait rien dit qui pût tendre à désavouer la
proposition qu'il en avait précédemment approuvée\,; mais Albéroni nia
le fait absolument\,: sa confiance était dans les événements qu'il se
flattait d'avoir préparés avec tant de prudence, qu'il serait difficile
que le succès ne répondît pas à son attente, et comme la décision en
était imminente, il comptait d'être incessamment débarrassé des
instances importunes du roi d'Angleterre, des ménagements qu'il se
croyait obligé de garder avec ce prince aussi bien que délivré de toute
crainte des menaces du pape. Il espérait enfin de se venger, avant qu'il
fût peu, du refus absolu de sa translation à Séville, et de venger le
roi son maître des ordres rigides que Sa Sainteté venait d'envoyer à son
nonce à Madrid.

En vertu de ces ordres, dont Rome menaçait depuis longtemps la cour
d'Espagne, le nonce Aldovrandi fit fermer, le 15 juin, le tribunal de la
nonciature. Il avertit les évêques du royaume par des écrits, portant le
nom de monitoires, que le pape suspendait toutes les grâces qu'il avait
accordées au roi d'Espagne. La cause de cette suspension était l'usage
que Sa Majesté Catholique avait fait des sommes qu'elle en retirait,
très différent de l'exposé qu'elle avait fait en obtenant ces grâces et
très opposé aux intentions de Sa Sainteté. Car elle prétendait qu'en
permettant au clergé d'Espagne d'aider de ses revenus le roi catholique,
c'était afin de le mettre en état d'armer l'escadre qu'il avait promis
d'envoyer dans les mers du Levant pour la joindre à la flotte
vénitienne, et faire ensemble la guerre contre les Turcs\,: au lieu que,
sous le faux prétexte du secours promis, l'Espagne avait effectivement
armé et fait partir sa flotte pour porter la guerre en Italie. Albéroni
prétendait que le roi son maître ne méritait en aucune manière les
reproches que le pape lui faisait. «\,Ils sont injustes, disait-il,
puisque Sa Majesté Catholique soutient actuellement contre les Maures
d'Afrique les sièges de Ceuta et de Melilla\,; qu'en défendant ces deux
places comme les dehors de l'Espagne, elle préserve le royaume de
l'irruption des infidèles, que de plus une de ses escadres est en course
contre les corsaires d'Alger.\,» Ces raisons dites, Albéroni jugea qu'il
fallait employer d'autres moyens pour soutenir l'honneur du roi son
`maître, et maintenir en Espagne son autorité contre les entreprises de
la cour de Rome\,; elle ne pouvait être mieux défendue que par le
premier tribunal du royaume. Ainsi le premier ministre fit décider par
le conseil de Castille que le nonce, en fermant la nonciature en
conséquence des ordres du pape, s'était dépouillé lui-même de son
caractère\,; qu'après cette abdication, il ne devait plus être souffert
en Espagne\,; que tolérer plus longtemps son séjour, ce serait offenser
Sa Majesté et causer un notable préjudice à son service. Le même conseil
décréta que tous monitoires répandus en Espagne par le nonce seraient
incessamment retirés des mains de ceux qui les avaient reçus, et que la
prétendue suspension des grâces accordées par le saint-siège à Sa
Majesté Catholique serait déclarée \emph{insuffisante}. Tout commerce
entre Rome et l'Espagne étant ainsi rompu, on résolut de former une
junte, de la composer de conseillers du conseil de Castille et de
canonistes, et de les charger d'examiner l'origine de plusieurs
introductions et pratiques prétendues abusives et aussi avantageuses à
la cour de Rome que contraires au bien du royaume d'Espagne. Leurs
Majestés Catholiques voulurent elles-mêmes parler en secret à quelques
ministres, en sorte qu'il parut que cette affaire très sérieuse, et dont
les suites deviendraient considérables, était leur propre affaire, non,
celle du cardinal Albéroni\,; et, soit qu'il voulût alarmer le pape par
des avis secrets, soit qu'il écrivît naturellement la vérité telle qu'il
croyait la voir, il confia au duc de Parme que le feu était allumé de
manière que sans la main de Dieu on ne verrait pas sitôt la fin de
l'incendie.

Quelques agents de Rome à Madrid, ou séduits par le cardinal, ou formant
leur jugement sur les discours qu'ils entendaient, pensaient aussi que
les engagements que le roi d'Espagne prenait pourraient faire une plaie
considérable à l'Église\,; ils condamnaient la précipitation du pape,
très opposée à la patience, si convenable au père commun, et très
dangereuse pour le saint-siège et pour l'Espagne, qu'elle exposait
également, au lieu que Sa Sainteté temporisant, comme elle le pouvait
aisément et comme elle le devait, jusqu'à la fin de la campagne, aurait
pris sûrement les résolutions qu'elle aurait jugé à propos de prendre
selon sa prudence et selon les événements. Ils l'accusaient d'avoir trop
écouté et suivi les mouvements de sa vengeance contre le cardinal
Acquaviva, car le pape se plaignait amèrement de lui, persuadé qu'il lui
avait manqué de parole, et sur ce fondement Sa Sainteté avait déclaré
qu'elle ne traiterait jamais avec lui d'aucune affaire.

Aldovrandi, homme sage, et nonce aimant la paix, assez expérimenté pour
prévoir qu'une division entre les cours de Rome et de Madrid serait
encore plus fatale à sa fortune particulière qu'elle ne la serait aux
affaires publiques, voulut ménager les choses, de manière qu'en
obéissant fidèlement à son maître, il prévînt, s'il était possible,
l'éclat d'une rupture entre le pape et le roi d'Espagne. Deux grands
princes se réconcilient, mais le ministre de la rupture demeure souvent
sacrifié. Aldovrandi ferma donc la nonciature suivant ses ordres, et
envoya les lettres monitoires dont on a parlé pour avertir tous les
évêques d'Espagne de la suspension des grâces accordées au roi d'Espagne
par le pape. Le nonce observa d'employer différentes mains pour écrire
les inscriptions de ces lettres, persuadé que toutes, et certainement
celles des ministres étrangers, étaient ouvertes à Madrid, et que le
passage libre n'était accordé qu'à celles qui n'intéressaient pas la
cour\,; il fit porter à Cadix, par un homme sûr, celles qui étaient
adressées aux évêques des Indes. Ces précautions prises, après avoir
obéi à son maître, il lui représenta vivement les inconvénients d'une
rupture et l'embarras où Sa Sainteté se jetait par les engagements
qu'elle venait de prendre. Elle voulait se venger du roi d'Espagne et de
son ministre, non de la nation espagnole dont le saint-père n'avait
point à se plaindre, et, par l'événement, la vengeance tombait
uniquement sur les Espagnols. Les revenus de la Crusade et des autres
grâces de Rome étaient affermés\,; le roi d'Espagne en était payé
d'avance, et les fermiers attendaient, sans beaucoup d'inquiétude, que
la querelle, qui ne pouvait durer longtemps, finît. Mais un grand nombre
de particuliers avaient payé pour jouir des grâces du saint-siège\,; par
exemple, pour obtenir pendant le cours d'une année les dispenses
accordées par la bulle de la croisade, l'argent était donné, les
dispenses et autres grâces étaient révoquées. Le nonce appuya beaucoup à
Rome sur les plaintes que cette révocation subite et inopinée lui avait
attirées\,; il différa, d'ailleurs, le plus qu'il lui fut possible son
départ de Madrid, et, soit vérité, soit artifice employé à bonne
intention, il excusa ce retardement sur ce que le roi d'Espagne lui
avait fait proposer d'attendre encore et d'examiner s'il ne serait pas
possible de trouver quelque expédient pour conduire les affaires à la
paix. Un tel délai parut au nonce moins dangereux et moins contraire aux
intentions du pape que ne le serait un départ trop précipité, capable de
fermer la porté à tout accommodement\,; mais s'il jugeait sainement des
intentions de Sa Sainteté, il y a lieu de croire qu'il n'était pas assez
bien informé de tous les ressorts que les Allemands faisaient agir
auprès d'elle pour l'intimider au point de la forcer à rompre totalement
avec l'Espagne.

Le pape avait résisté aux menaces de Gallas, ambassadeur de
l'empereur\,; Sa Sainteté ne put résister à celles de son neveu, le
cardinal Albane, plus foudroyantes que celles du ministre allemand. Ce
cardinal ne cessait, depuis longtemps, de dire au saint-père que la cour
de Vienne avait des sujets très légitimes de se plaindre de la conduite
ou partiale ou tout au moins molle que Sa Sainteté tenait à l'égard du
roi d'Espagne. Il avait promis d'envoyer ses vaisseaux dans la mer du
Levant\,; il avait manqué de parole, et Sa Sainteté, insensible à un tel
affront, n'avait rien fait encore ni contre ce prince ni contre son
ministre. Albane représentait à son oncle ce qu'il devait craindre d'un
gouvernement tel que celui de Vienne, justement irrité, qui donnait des
marques terribles de son ressentiment et de sa vengeance, quand même les
prétextes de se plaindre lui manquaient. Un tel solliciteur servait
mieux l'empereur que ses ministres, et les biens que ce prince lui
faisait dans le royaume de Naples l'assuraient de sa fidélité. Le roi
d'Espagne ne pouvait pas et peut-être n'aurait pas voulu lui accorder
des bienfaits supérieurs à ceux qu'il recevait de Vienne\,; c'était
l'unique moyen de le faire changer de parti. L'amitié ni la haine ne le
conduisaient pas\,; l'intérêt présent le déterminait, et d'un moment à
l'autre il embrassait, suivant ce qu'il croyait lui convenir davantage,
des sentiments contraires à ceux qu'il avait suivis précédemment. Son
intérêt, ses espérances pour sa famille, l'attachaient à l'empereur.
Aucune autre puissance ne combattant ces motifs par d'autres plus forts
et de même nature, le cardinal Albane travaillait avec succès pour le
parti qu'il avait embrassé\,; il réussissait moins par la confiance que
le pape avait en lui, que parce que le caractère d'esprit de Sa Sainteté
était timide, et qu'il était facile de l'obliger par la crainte à faire
les choses même qui paraissaient le plus opposées à sa manière de
penser. Ce moyen, employé à propos, força Sa Sainteté de rompre avec
l'Espagne, et cependant elle écrivit au roi catholique une lettre où,
mêlant les plaintes aux menaces, laissant entrevoir des sujets
d'espérance, évitant de s'engager, il paraissait qu'elle craignait les
suites de la démarche qu'on lui faisait faire, et que, si elle eût,
suivi son génie, elle aurait simplement taché de gagner du temps pour
voir quels seraient les événements de la campagne et se déterminer en
faveur du plus heureux.

Il y avait alors lieu de douter de quel côté la fortune se déclarerait.
L'Italie était persuadée que le roi d'Espagne était secrètement d'accord
avec le roi de Sicile, parce qu'il n'était pas vraisemblable que le roi
d'Espagne entreprît, seul et sans alliés\,; une guerre difficile, et que
les Allemands, maîtres de Naples et de Milan, les soutiendraient
aisément avec les forces qu'ils avaient dans ces deux États. On croyait
à Rome que la ligue était signée\,; le nonce l'avait écrit de Madrid au
pape. Les partisans de la couronne d'Espagne commençaient à donner des
conseils sur la conduite qu'elle devait tenir pour se réconcilier avec
les Italiens, et regagner leur affection qu'elle avait perdue en faisant
précédemment la guerre conjointement avec la France. Deux moyens selon
eux suffisaient pour y parvenir. Le premier était de délivrer le pape
des vexations qu'il essuyait de la part des Allemands, l'une au sujet de
Comachio que l'empereur avait usurpé sur l'Église, et qu'il retendit,
injustement\,; l'autre en faveur du duc de Modène que les Impériaux
protégeaient aux dépens de la ville et du territoire de Bologne, à
l'occasion des eaux dont le Bolonais courait risque d'être inondé. Les
amis de l'Espagne comptaient qu'il lui serait facile de faire restituer
au saint-siège la ville et les dépendances de Comachio, encore plus aisé
de ranger à son devoir un petit prince tel que le duc de Modène\,; qu'un
tel service rendu à l'Église, dans le temps même que le pape en usait si
mal à l'égard de Sa Majesté Catholique, ferait d'autant plus éclater sa
piété\,; qu'il augmenterait les soupçons que les Allemands avaient déjà
des intentions de Sa Sainteté, au point qu'elle n'aurait plus d'autre
parti à prendre que de se jeter entre les bras d'un prince qui se
déclarait son protecteur, lorsqu'il avait le plus de sujet de se
plaindre de la partialité qu'elle témoignait pour ses ennemis.

Selon ces mêmes conseils, rien n'était plus facile que de s'emparer de
l'État de Modène, de forcer le duc à restituer l'usurpation qu'il avait
faite de la Mirandole\,; et comme le prince qu'il avait privé de ce
petit État était alors grand écuyer du roi d'Espagne, on supposait que
le duc de Modène, privé de son pays, irait à son tour à Vienne briguer
la charge de grand écuyer de l'empereur. On intéressait dans ces projets
la reine d'Espagne, et pour la flatter, on voulait aussi que le duc de
Modène rendît au duc de Parme quelque usurpation faite sur le Parmesan.
Les restitutions ne coûtaient rien à ceux qui les conseillaient\,; ainsi
rien ne les empêchait de les étendre encore eu faveur du duc de
Guastalla, et de forcer l'empereur à lui rendre Mantoue comme le
patrimoine de la maison Gonzague, usurpé et retenu très injustement par
les Allemands. Le roi d'Espagne devenu le protecteur non seulement des
princes d'Italie, mais le réparateur des pertes et des injustices qu'ils
avaient souffertes, les engagerait aisément dans son alliance, et le
même intérêt les unirait pour fermer à jamais aux Allemands les portes
de l'Italie. Pour achever sans inquiétude de telles entreprises
proposées somme un moyen sûr d'établir solidement la paix et l'équilibre
du monde, on demandait seulement que, pendant que les troupes d'Espagne
s'ouvriraient un chemin en Lombardie, le roi d'Espagne fit croiser
quelques vaisseaux de sa flotte dans les mers de Naples, afin d'empêcher
le transport des secours que les Impériaux ne manqueraient pas d'en
tirer pour la défense du Milanais, si le passage demeurait libre. On se
promettait, de plus, que la ville de Naples, bientôt affamée, serait
obligée de se rendre à son souverain légitime sans être attaquée. Enfin
ceux qui désiraient de voir le roi d'Espagne engagé à faire la guerre en
Italie, soit par zèle pour le bien public, soit par des raisons
d'intérêt particulier, lui représentaient et l'assuraient que les
Allemands étaient consternés, qu'ils ne doutaient pas que l'orage ne
tombât sur l'État de Milan\,; mais ne sachant pas certainement où ils
auraient à se défendre, que leurs commandants n'avaient d'autres ordres
que de se tenir sur leurs gardes, et lorsque l'entreprise serait
déterminée, de secourir l'État que les Espagnols attaqueraient.

L'opinion publique était que l'armée d'Espagne devait attaquer cet État.
Un des ministres de Savoie à Madrid assura son maître que, malgré le
secret exact et rigoureux qu'on observait encore sur la destination de
l'armée d'Espagne, il savait qu'elle débarquerait à Saint-Pierre-d'Arena
et à Final. Albéroni lui avait cependant confié que depuis qu'il était
appelé au ministère, il avait écrit et chiffré de sa main tout ce qui
concernait les négociations et les affaires secrètes. Le cardinal ne fut
pas trahi, en cette occasion. C'était le 11 juillet que le ministre du
roi de Sicile avertit son maître que le débarquement se ferait à
Saint-Pierre-d'Arena, et le 16 du même mois on sut à Turin par un
courrier dépêché de Rome, que les Espagnols descendus en Sicile avaient
pris la ville de Palerme.

Environ le même temps, l'amiral Bing commandant la flotte anglaise,
arriva à Cadix. Aussitôt il déclara de la part du roi d'Angleterre que
ses ordres étaient d'insister auprès du roi d'Espagne, pour en obtenir
une suspension d'armes, et cessation de toutes hostilités, comme un
moyen nécessaire pour avancer la négociation de la paix\,; que, si le
débarquement des troupes espagnoles était déjà fait en tout ou en partie
en Italie, il avait ordre d'offrir le secours de la flotte qu'il
commandait pour les retirer en toute sûreté\,; qu'il offrait aussi la
continuation de la médiation du roi son maître, pour concilier le roi
d'Espagne avec l'empereur\,; que, si Sa majesté Catholique la refusant,
attaquait les États que l'empereur possédait en Italie, ses ordres en ce
cas l'obligeraient d'employer pour la défense de ces mêmes États et pour
le maintien de la neutralité, les forces qu'il avait sous son
commandement. Bing prétendait qu'une telle déclaration était fondée sur
le traité signé à Utrecht, pour la neutralité de l'Italie, aussi bien
que sur le traité de Londres signé le, 25 mai, entre l'empereur et le
roi d'Angleterre. Les offres ni les menaces des Anglais n'ébranlèrent
point le roi d'Espagne. Son ministre répondit que Bing pouvait exécuter
les ordres dont il était chargé, et regardant comme rupture la
déclaration que cet amiral avait faite, il écrivit à Monteléon qu'il
était juste et raisonnable que tout engagement pris par le roi d'Espagne
avec le roi d'Angleterre, fût rompu réciproquement\,; que Sa Majesté
Catholique cessait donc d'accorder aux négociants Anglais les avantages
qu'elle avait prodigués si généreusement en faveur de cette nation\,;
que la conduite prescrite à l'amiral Bing était la seule cause d'un
changement que le roi d'Espagne faisait à regret, et qu'ayant suivi son
inclination particulière en distinguant les Anglais des autres nations
par les grâces singulières qu'il leur avait faites, c'était aussi contre
son gré qu'il en suspendait les effets, même dans un temps où Sa Majesté
Catholique voulait, nonobstant les représentations du commerce de Cadix,
accorder la permission que les ministres d'Angleterre avaient instamment
sollicitée, pour le départ du vaisseau que la compagnie du Sud devait
envoyer aux Indes. Les Anglais en avaient obtenu la faculté parle traité
de paix conclu à Utrecht entre l'Espagne et l'Angleterre. Le roi
d'Espagne n'avait pas jusqu'à cette année refusé l'exécution de cette
condition. Il ne prétendait pas la refuser encore, mais seulement en
différer l'effet jusqu'à l'année suivante, et la raison du délai était
que le voyage serait inutile et infructueux, la contrebande ayant
introduit en Amérique tant de marchandises d'Europe, que le commerce de
Cadix jugeant de la perte qu'il y aurait pour les négociants d'envoyer
aux Indes de nouvelles marchandises avant que les précédentes fussent
vendues, avait obtenu sur ses remontrances que le départ des galions,
serait différé jusqu'à l'année suivante. Le roi d'Espagne avait par la
même raison remis aussi à l'autre année le départ du vaisseau Anglais,
et, pour dédommager les intéressés, il avait résolu de leur permettre
d'envoyer deux vaisseaux au lieu d'un seul. Enfin il était sur le point
de porter l'indulgence plus loin, même au préjudice du commerce de
Cadix, quand l'entrée de la flotté Anglaise changea ces dispositions.

Monteléon devait expliquer bien clairement aux négociants de Londres,
intéressés dans le commerce de la mer du Sud, les intentions favorables
du roi d'Espagne, et la raison qui les rendait inutiles. Il devait même
chercher dans leurs maisons ceux qui n'auraient pas la curiosité de lui
demander la cause d'un tel changement, et les en instruire. Albéroni se
promettait de leur part quelque mouvement, si ce n'était un soulèvement
général contre les ministres qui donnaient au roi d'Angleterre des
conseils si pernicieux aux avantages du commerce de la nation\,: soit
haine, soit défiance, il laissait peu de liberté à Monteléon sur
l'exécution des ordres qu'il lui prescrivait. Les exhortations
fréquentes de cet ambassadeur à la paix, ses représentations sur les
maux que la guerre entraînerait étaient mal interprétées. Albéroni les
regardait comme des preuves ou d'infidélité, ou tout au moins d'une
fidélité très équivoque, et disait que c'était mal connaître le roi
d'Espagne que de croire amollir ses résolutions par la terreur des
périls, dont on prétendait en vain l'effrayer. Beretti, sans être estimé
du cardinal, était bien plus de son goût. Il louait le zèle extrême de
cet ambassadeur pour le service du roi son maître, et lui accordait de
montrer au moins un bon coeur, persuadé cependant que si tes Hollandais
résistaient jusqu'alors aux instances de la France et de l'Angleterre,
on ne le devait pas attribuer aux négociations de Beretti, non plus
qu'au crédit de ses prétendus amis, mais seulement à la sagesse de la
république, trop prudente pour souscrire à des engagements dangereux,
surtout dans une conjoncture très critique.

L'inaction des Provinces-Unies était tout ce qu'Albéroni désirait de
leur part. Il avait espéré davantage des princes du nord, mais il
commençait à se détromper des différentes idées qu'il avait formées sur
les secours et sur les diversions du czar, du roi de Prusse et du roi de
Suède\,; car il avait porté ses vues sur les uns et sur les autres, et
désabusé de ces projets, il avouait qu'il n'entendait plus parler de ces
princes qu'avec dégoût. Il se flattait de réussir plus heureusement en
attaquant la France par elle-même\,; il entretenait dans le royaume des
intelligences secrètes qu'il croyait capables d'allumer le flambeau de
la guerre civile, et connaissant peu le crédit des conspirateurs, il
attendait les nouvelles du progrès de leurs complots avec la même
impatience que si leurs trames eussent dû faire triompher le roi
d'Espagne de tous ses ennemis. Cellamare avait ordre de dépêcher des
courriers pour instruire le roi son maître de tout ce qui regarderait
cette affaire capitale. La conjoncture paraissait favorable aux désirs
de ceux qui souhaitaient de voir régner la division en France\,; ils
comptaient beaucoup sur le mécontentement du parlement de Paris, sur les
vues qu'on lui attribuait de profiter d'un temps de faiblesse du
gouvernement pour étendre l'autorité de cette compagnie. Ses
entreprises, quand même elles ne réussiraient pas, seraient toujours
autant de piqûres à l'autorité de la régence, et les corps dont le
crédit {[}était{]} établi par une longue suite de temps, étaient,
suivant l'opinion d'Albéroni, un puissant correctif au gouvernement
despotique. Le temps lui paraissait un grand modérateur dans toutes les
affaires, et savoir le gagner était un grand art. Un aventurier qui se
faisait nommer le comte Marini, vint le trouver, envoyé, disait-il, par
un autre aventurier danois qu'on nommait, le comte Schleiber, trop connu
pour son honneur sous le règne du feu roi, Marini proposa, de concert
avec son ami, une ligue entre le roi d'Espagne et le roi de Prusse.

Albéroni, en garde contre l'industrie de ces sortes de gens, avertit
Cellamare que Marini partait pour Paris, et le pria d'éclaircir ce que
c'était que cet aventurier et quelle foi on pouvait donner à ses
paroles. Il est naturel à celui qui fait un grand usage d'espions de
croire qu'on lui rend la pareille, et que plusieurs inconnus qui lui
offrent ses services n'ont pour objet que de pénétrer ses secrets et
d'en informer ceux qui les emploient. Les principales vues d'Albéroni
étaient sur la succession du roi d'Espagne à la couronne de France\,; et
quoiqu'il fût de la prudence de cacher ces vues avec beaucoup de soin,
il ne put s'empêcher de dire un jour à un des ministres du roi de Sicile
que, si le cas arrivait, le parti du roi d'Espagne en France serait plus
fort que celui du régent.

La rupture entre les cours de Rome et de Madrid acheva d'éclater par
l'ordre que le nonce reçut de la part du roi d'Espagne, au commencement
de juillet, de sortir des États de Sa Majesté Catholique\,; et comme le
motif de cet ordre était principalement le refus des bulles de
l'archevêché de Séville pour le cardinal Albéroni, cette cause parut si
légère que bien des gens crurent la chose concertée entre les deux cours
uniquement pour cacher à l'empereur leur intelligence secrète. Mais ces
politiques, comme il arrive souvent, se trompaient dans leurs
raisonnements, et la rupture était sérieuse\,; le sort du pape était de
passer le cours de son pontificat brouillé avec les premières puissances
catholiques, la France, etc.

\hypertarget{chapitre-x.}{%
\chapter{CHAPITRE X.}\label{chapitre-x.}}

1718

~

{\textsc{Soupçons mal fondés d'intelligence du roi de Sicile avec le roi
d'Espagne.}} {\textsc{- Frayeurs du pape, qui le font éclater contre
l'Espagne et contre Albéroni, pour se réconcilier l'empereur avec un
masque d'hypocrisie.}} {\textsc{- Ambition d'Aubenton vers la pourpre
romaine.}} {\textsc{- Albéroni, de plus en plus irrité contre
Aldovrandi, est déclaré par le pape avoir encouru les censures.}}
{\textsc{- Rage, réponse, menaces d'Albéroni au pape.}} {\textsc{- Les
deux Albane, neveux du pape, opposés de parti.}} {\textsc{- Le cadet
avait douze mille livres de pension du feu roi.}} {\textsc{- Vanteries
d'Albéroni et menaces.}} {\textsc{- Secret de l'expédition poussé au
dernier point.}} {\textsc{- Vanité folle d'Albéroni.}} {\textsc{- Il
espère et travaille de plus en plus à brouiller la France.}} {\textsc{-
Le régent serre la mesure et se moque de Cellamare et de ses croupiers,
qui sont enfin détrompés.}} {\textsc{- Conduite du roi de Sicile avec
l'ambassadeur d'Espagne, à la nouvelle de la prise de Palerme.}}
{\textsc{- Cellamare fait le crédule avec Stanhope, pour éviter de
quitter Paris et d'y abandonner ses menées criminelles.}} {\textsc{- Ses
précautions.}} {\textsc{- Conduite du comte de Stanhope avec Provane.}}
{\textsc{- Situation du roi de Sicile.}} {\textsc{- Abandon plus
qu'aveugle de la France à l'Angleterre.}} {\textsc{- Rage des Anglais
contre Châteauneuf.}} {\textsc{- Pratiques, situation et conduite du roi
de Sicile sur la garantie.}} {\textsc{- Blâme fort public de la
politique du régent.}} {\textsc{- Il est informé des secrètes
machinations de Cellamare.}} {\textsc{- Triste état du duc de Savoie.}}
{\textsc{- Infatuation de Monteléon sur l'Angleterre.}} {\textsc{-
Albéroni fait secrètement des propositions à l'empereur, qui les
découvre à l'Angleterre et les refuse.}} {\textsc{- Le roi de Sicile et
Albéroni crus de concert, et crus de rien partout.}}

~

L'armée d'Espagne, débarquée en Sicile sous le commandement du marquis
de Lede, avait pris Palerme le 2 juillet. Maffeï, vice roi de l'île,
s'était retiré à Messine, et personne ne doutait que cette ville,
attaquée par les Espagnols, ne se rendît aussi facilement que Palerme.
On doutait encore si le roi de Sicile, averti depuis longtemps par
l'abbé, del Maro son ambassadeur à Madrid, des dispositions, de
l'Espagne, n'était pas secrètement de concert avec Sa Majesté
Catholique, et si ce ne serait pas en conséquence de cette intelligence
secrète que les troupes du Piémont avaient été augmentées depuis peu
jusqu'au nombre de quatorze mille hommes. De tels doutes augmentaient
plutôt que de calmer les agitations du pape. Les armes du roi d'Espagne
offensé paraissaient de nouveau comme aux portes de Rome, puisqu'il ne
savait pas encore quel progrès elles pourraient faire. Le duc de Savoie,
s'il était son allié, pouvait faciliter le succès\,; il ne pouvait les
empêcher s'il était ennemi. L'empereur voulait croire qu'il y avait
intelligence et liaison étroite entre le pape et le roi d'Espagne, et
que les Espagnols n'avaient rien entrepris que de concert avec Sa
Sainteté. La vengeance des Allemands, plus prochaine, plus facile et
plus dure que toute autre, lui paraissait aussi la plus à craindre\,;
elle crut par ces raisons que son intérêt principal et celui du
saint-siège était de tout employer pour en prévenir les effets. Il
fallait pour calmer le ressentiment vrai où feint que l'empereur
témoignait, que le pape fit voir évidemment qu'il n'avait pas la moindre
part à l'entreprise du roi d'Espagne\,; que jamais le projet ne lui en
avait été communiqué, et que même Sa Sainteté avait été abusée par les
mensonges d'Albéroni\,; qu'elle était irritée au point de rompre
ouvertement avec le roi d'Espagne. Elle lui écrivit donc un bref
fulminant, et pour justifier ses plaintes et sa conduite, en même temps
que ce bref fut imprimé\,; elle rendit publique une lettre que ce prince
lui avait écrite le 29 novembre de l'année précédente. Il promettait
expressément par cette lettre d'observer exactement la neutralité
d'Italie sans inquiéter les États que l'empereur y possédait, et sans y
porter la guerre, pendant que les Turcs continueraient de faire la
guerre en Hongrie. Sur une parole si précise, le pape avait exhorté et
pressé l'empereur de poursuivre les avantages que Dieu lui donnait sur
les infidèles\,; Sa Sainteté s'était positivement engagée à ce prince
qu'il ne serait troublé par aucune diversion\,; que, s'il se livrait
entièrement à la guerre du Seigneur, nulle autre n'interromprait le
cours de ses victoires. Elle justifiait la cour de Vienne des
infractions à la neutralité que les ministres d'Espagne lui imputaient.
Ces prétendus chefs de plaintes étaient, disait-elle, antérieurs à la
promesse solennelle que Sa Majesté Catholique avait faite, et le seul
incident à reprocher aux Allemands était l'enlèvement de Molinez arrêté
et conduit au château de Milan, retournant à Madrid de Rome où il avait
rempli pendant plusieurs années la place d'auditeur et de doyen de la
rote. Mais l'aventure d'un particulier, sujette à discussion, ne
dégageait pas le roi d'Espagne de la parole qu'il avait donnée et dont
le pape était le dépositaire. Sa Sainteté, persuadée qu'il était de son
honneur comme de son devoir d'en procurer l'effet, voulait que dans le
temps qu'elle traitait le plus durement le roi d'Espagne, ce prince lui
sût gré des ménagements qu'elle avait eus pour lui. Elle alléguait donc,
comme preuves de considération portée peut-être trop loin, l'inaction où
elle était demeurée tout l'hiver\,; le parti qu'elle avait pris, au lieu
d'instances vives et pressantes, au lieu d'user de menaces et de passer
aux effets, de se borner à des insinuations tendres et pathétiques, mais
inutiles, dont les réponses avaient été injures et nouvelles offenses\,;
qu'elle était donc forcée de publier ce bref terrible, comme la dernière
ressource et le dernier moyen qu'elle pouvait avoir encore pour vaincre
l'opiniâtreté du roi d'Espagne\,; arrêter dans son commencement une
guerre si fatale à la chrétienté, empêcher enfin le mauvais usage des
grâces que le saint-siège avait accordées à cette couronne, dont le
produit devait être employé, contre les infidèles, et par un abus
intolérable servait à faire une diversion utile et avantageuse, au
rétablissement de leurs affaires. On croyait encore à Rome que les mêmes
intérêts unissaient les cours de France et d'Espagne, et le pape
craignait que le régent ne prît vivement le parti du roi catholique.
Mais depuis, la régence les maximes étaient changées. Sa Sainteté
pouvait agir librement à l'égard de l'Espagne\,; la France ne songeait
pas à détourner ni même à retarder les coups qui menaçaient Madrid.
Toutefois le pape prit la précaution superflue d'avertir son nonce à
Paris, et de ses résolutions et de ses motifs. Le seul était
l'obligation et le désir de faire son devoir\,; car il importe bien
plus, disait Sa Sainteté, de ne pas tomber entre les mains du Dieu
vivant que de tomber entre les mains des hommes. Cette nécessité,
détachée de tout intérêt et de toute vue humaine, l'avait fait agir.
Nulle réflexion sur la cour de Vienne n'avait part à sa conduite. Elle
n'en était pas mieux traitée que celle d'Espagne. Elle recevait
également des injures de l'une et de l'autre. Mais dans le cas présent
la justice et la raison de se plaindre étaient du côté de l'empereur,
qui, se croyait trompé par la confiance qu'il avait prise en la parole
du roi d'Espagne, garantie par Sa Sainteté. Aldovrandi avait ordre de
s'expliquer ainsi à Madrid, au sujet des résolutions de son maître\,;
mais tout accès lui étant fermé, il fallut se contenter d'une longue
conférence qu'il eut avant son départ avec le P. Daubenton, confesseur
du roi d'Espagne. On sut que ce jésuite lui avait conseillé de marcher
lentement, de régler chacune de ses journées à quatre lieues, et de
s'arrêter à la frontière de France. Le reste demeura secret. Aubenton
avait de grandes vues. Son élévation dépendait de la cour de Rome\,; la
rupture avec celle d'Espagne renversait ses projets. Il voulut faire le
pacificateur. Un tel rôle déplut à Albéroni, personnellement offensé, et
autant irrité contre Aldovrandi que contre le pape. Il se plaignit du
nonce comme ayant manqué de confiance pour lui\,; et c'était à cette
défiance que ce ministre, disait Albéroni, devait attribuer son malheur
qu'il aurait évité par une meilleure conduite, s'il n'avait pas perdu la
tramontane.

Le pape offensait Albéroni en faisant déclarer qu'il avait encouru les
censures. Le cardinal voulut croire son honneur attaqué par une telle
déclaration. Il aurait désiré persuader le public que ce point était ce
qu'il avait de plus cher au monde\,; et, comme le croyant lui-même, il
dit hautement qu'il ne lui était plus permis de se taire\,; qu'il avait
gardé le silence tant que le pape, ajoutant foi aux calomnies des
ministres impériaux, avait seulement essayé de le faire mourir de
faim\,; que la même retenue devenait impossible à conserver, s'agissant
d'accusations énormes portées contre lui, effet ordinaire de la haine et
de l'artifice infâme et grossier des Allemands\,; que le motif des
censures si formidables de la cour de Rome était apparemment le profit
de quatre baïoques qu'il avait retiré de l'évêché de Tarragone\,; qu'il
ne connaissait pas d'autre prétexte pour appuyer un jugement si
rigoureux\,; qu'il était triste pour lui que le pape le réduisît à la
fâcheuse nécessité d'oublier qu'il était sa créature\,; mais peut-être
que cette extrémité ne serait pas moins désagréable pour Sa Sainteté\,;
que Leurs Majestés Catholiques soutiendraient leur engagement, et que de
sa part il ferait tout ce que les lois divines et humaines lui
suggéreraient\,; que, s'il secondait seulement le génie de certaines
gens, on verrait bientôt de si belles scènes, que le pape regretterait
d'y avoir donné lieu. Le cardinal Albane, neveu du pape, était dévoué à
l'empereur. Don Alexandre Albane, frère cadet du cardinal, qui n'était
pas encore honoré de la pourpre, avait pris une route contraire à celle
que suivait son aîné\,; et, soit par antipathie, soit par une politique
assez ordinaire dans les familles papales, il avait reçu du feu roi une
pension secrète de douze mille livres. Il continuait par les mêmes
motifs de se dire attaché à la France et à l'Espagne. Albéroni lui fit
part de ses, plaintes. Il affectait de ne pouvoir croire que le pape
voulut ajouter foi à la calomnie dont les Allemands prétendaient le
noircir dans l'esprit de Sa Sainteté\,; mais il protestait en même temps
que, si elle était assez faible pour se porter à quelque résolution
contraire à la dignité comme à la réputation d'un cardinal, il avait
reçu de Dieu assez de force comme assez de courage pour se défendre\,;
qu'on verrait de belles scènes, et qu'elle serait fâchée d'y avoir donné
lieu. Il fit prier don Alexandre de ne rien cacher au pape, même de lui
dire que, si les choses continuaient comme elles avaient commencé, le
marquis de Lede serait aux portes de Rome avant le mois d'octobre.
Albéroni louait la reine d'Espagne d'avoir dit que le saint-père abusait
de la bonté, de la piété et de la religion du roi catholique. Ce
ministre annonçait une division prochaine, qui ne serait pas honorable
pour le pape, parce qu'enfin Sa Majesté Catholique, se voyant forcée
d'exposer par un manifeste ce qu'elle avait souffert\,; rouvrirait des
plaies refermées, qu'il serait plus à propos pour Sa Sainteté de laisser
oublier\,; que le public disait déjà que le pape ne refusait les bulles
de Séville, que parce que le comte de Gallas avait menacé Sa Sainteté de
se retirer si elle les accordait, et annoncé qu'en ce cas le nonce
serait chassé de Vienne\,; mais Albéroni prétendait que l'Espagne
pouvait aussi menacer à plus juste titre. Il se plaisait à parler de la
flotte qu'il avait équipée et, mise en mer, des forces de cette
couronne, et de sa puissance qu'il se vantait d'avoir relevée. L'Europe
devait avoir de plus grands efforts et de plus grands succès l'année
suivante, et dès lors, il prenait les mesures nécessaires pour y
réussir. Des machines en l'air devaient produire des scènes curieuses,
et tel, qui se croyait alors obligé à des respects humains, jouerait un
autre jeu, s'il pénétrait dans l'avenir. C'était ainsi qu'Albéroni
s'applaudissait de ses projets et des ordres qu'il avait donnés pour
leur exécution, s'expliquant mystérieusement, même à ceux qui devaient
concourir au succès de ces grands desseins.

Le marquis de Lede, général de l'armée, ignorait en s'embarquant, quelle
en était la destination. Il devait, quand il serait à la hauteur de
l'île de Sardaigne, ouvrir un paquet écrit de la main d'Albéroni, signé
du roi d'Espagne. Il y trouverait seulement le lieu du rendez-vous de la
flotte indiqué aux îles de Lipari. En y arrivant, il ouvrirait une
seconde enveloppe, qui renfermait les ordres de Sa Majesté Catholique.
C'était ainsi que le cardinal prétendait conserver le secret, l'âne des
grandes entreprises, et pour y parvenir il se plaignait de se voir
obligé de faire en même temps les fonctions de ministre, de secrétaire
et d'écrivain, d'être réduit à ne sortir de son appartement que pour
aller en ceux de Sa Majesté Catholique et des princes, consolé cependant
dans cette vie pénible, par la satisfaction que le roi d'Espagne goûtait
du changement subit qu'il voyait dans sa monarchie. En cet état
florissant, le cardinal ne pouvait croire que l'amiral Bing, commandant
la flotte Anglaise, eût l'ordre ni la hardiesse d'en venir à des actes
d'hostilité. Il croyait voir, la crainte et l'agitation du gouvernement
d'Angleterre clairement marquées par l'arrivée du comte de Stanhope à
Paris, en intention de passer à Madrid. Il supposait que ce ministre ne
se serait pas engagé à faire le voyage d'Espagne, si le roi d'Angleterre
pensait à rompre avec le roi catholique. Toutefois Cellamare eut ordre
de persuader, s'il pouvait, au régent de suspendre tout engagement
jusqu'à ce que Son Altesse Royale eût vu l'effet que produirait à Madrid
l'éloquence du comte de Stanhope. De part et d'autre, on voulait gagner
du temps. Le ministre d'Espagne embrassait beaucoup d'affaires\,; il
était fertile en projets, se flattait aisément de les voir tous réussir.
Aucun cependant ne s'accomplissait. Cellamare, par ordre du roi son
maître, cultivait le ministre du czar à Paris. Jamais, disait-il, Sa
Majesté Catholique n'accepterait le traité qu'on lui proposait\,; elle
le regardait comme injuste, offensant son honneur. Elle était prête, au
contraire, à travailler avec le czar. Elle s'obligeait à mettre en mer
trente vaisseaux de guerre, en même temps qu'elle agirait par terre avec
une armée de trente ou quarante mille hommes. Une telle parole était
plus aisée à donner qu'à exécuter\,; mais Albéroni n'était point avare
de promesses qui ne lui coûtaient rien. Il fallait aussi {[}ajouter{]}
que, s'il ne pouvait y satisfaire, les mouvements qu'il comptait de
susciter en France le dédommageaient assez de ce qu'il perdait en
manquant de parole aux alliés de son maître. Il espérait alors beaucoup
des liaisons que Cellamare avait formées. Il fallait les conduire avec
prudence, ménager les intérêts, la considération, le crédit, le rang, la
fortune de ceux qui entraient dans ces intrigues, leur laisser le loisir
de les conduire sagement, et de profiter des conjonctures. Le temps
était donc nécessaire, et pour les alliances à contracter et pour les
trames secrètes dont Albéroni espérait encore plus que des alliances et
des secours des étrangers.

Le régent, méprisant les discours du public et les raisonnements sur
l'intérêt particulier qui portait Son Altesse Royale à rechercher avec
tant d'empressement l'alliance du roi d'Angleterre, pressait la
négociation, et quoiqu'elle fût près de sa conclusion, le temps était
nécessaire aussi pour lui donner sa perfection. Ainsi ce prince
dissimulait si bien l'état où elle était, que les ministres les plus
intéressés à le savoir l'ignoraient. Celui d'Espagne faisait des
représentations et des déclarations très inutiles\,; il ameutait
quelques ministres étrangers et faisait valoir à Madrid, comme fruits de
ses soins, quelques déclamations vaines des ministres du czar et du duc
de Holstein contre la quadruple alliance. Il ne leur coûtait rien de les
faire\,; elles ne faisaient aussi nulle impression. Le régent laissait
cependant à Cellamare le plaisir de croire que ses manèges et ses
représentations réussissaient\,; il l'assurait, de temps en temps, que
les bruits répandus sur la conclusion de l'alliance étaient faux, et
suivant le penchant qui conduit à croire ce qui flatte et ce qu'on
souhaite, Cellamare voulait se persuader que ces assurances qu'il
trouvait fondées en raison étaient vraies, parce qu'elles lui
paraissaient vraisemblables. Le parlement faisait alors de fréquentes
remontrances, souvent sans sujet, quelquefois avec raison. L'extérieur
suffisait pour donner des espérances à l'ambassadeur d'Espagne, et comme
le bruit se répandit bientôt que le procureur général appellerait comme
d'abus de tout ce que le pape pourrait faire au préjudice des libertés
de l'Église gallicane et contre les évêques opposés à la bulle
\emph{Unigenitus}, ce ministre espéra de voir aussi, à cette occasion,
des mouvements dans le royaume\,: car il comprenait qu'un tel dénouement
devenait enfin nécessaire pour arrêter cette fatale négociation qu'il ne
pouvait rompre, et que le roi d'Espagne son maître ne pouvait approuver.
Les avis que Cellamare recevait sans cesse, et de différents endroits,
l'emportaient enfin sur les assurances que le régent lui avait données.
Il commençait à croire, malgré ce que Son Altesse Royale lui avait dit
au contraire, que la proposition de la quadruple alliance avait été
portée au conseil de régence, qu'elle y avait été approuvée à la
pluralité des voix, nonobstant l'opposition {[}de{]} quelques ministres
bien intentionnés. Il n'osait cependant rien affirmer encore, parce que
le régent continuait de nier également aux autres ministres étrangers
qu'il y eût rien de conclu. Provane, ministre de Sicile, sur les
assurances du régent, doutait comme Cellamare\,; mais bientôt tous deux
furent éclaircis, l'un de manière à ne conserver ni doute ni
espérance\,; l'autre, voulant se flatter et se réserver un prétexte de
prolonger son séjour en France, trouva dans les discours qui lui furent
tenus les moyens qu'il cherchait de parvenir à son but.

Un courrier, dépêché par l'ambassadeur de France à Turin, apporta la
nouvelle du débarquement des troupes d'Espagne, descendues le 3 juillet
près de Palerme. Elles s'étaient emparées de la ville sans résistance.
Dans un événement que le roi de Sicile n'avait pas prévu, il fit arrêter
le marquis de Villamayor, ambassadeur d'Espagne, et, s'adressant au
régent et au roi d'Angleterre, il demanda l'effet de la garantie du
traité d'Utrecht, promise par la France et par l'Angleterre. Villamayor
donna parole de demeurer dans les États du roi de Sicile, jusqu'à ce que
les ministres piémontais qui étaient alors à Madrid sortissent
d'Espagne. Après cet engagement, il ne fut plus gardé. Provane jugea
sans peine que c'était demande et sollicitation inutile, que celle de la
garantie de la France et de l'Angleterre. Cellamare, au contraire,
voulait faire croire qu'il ajoutait foi aux promesses que lui fit le
comte de Stanhope, avant que de passer de Paris à Madrid. Elles
n'auraient pas abusé un ministre moins clairvoyant que lui\,; mais il y
a des conjonctures où on ne veut pas voir, et Cellamare, ménageant à
Paris des affaires secrètes où sa présence était nécessaire, voulut
prendre pour de assurances réelles et solides les vains discours de
Stanhope, croire ou faire semblant de croire, comme lui disait cet
Anglais, qu'il y avait dans le nouveau projet de traité des changements
tels, qu'ils étaient beaucoup plus conformes à ce que le roi d'Espagne
désirait, qu'aux espérances de la cour de Vienne. Stanhope n'expliqua ni
la qualité des engagements, ni celle des propositions avantageuses dont
il se disait chargé. Il ajouta seulement qu'il avait dépêché un courrier
à Vienne, et qu'il espérait, lorsqu'il serait à Madrid, surmonter les
grandes difficultés que les médiateurs avaient trouvées jusqu'alors de
la part de cette cour. Cellamare, recevant pour bon et valable tout ce
qu'il plut à Stanhope de lui dire, avertit cependant le roi son maître
qu'il y avait une alliance intime et particulière entre le régent et le
roi d'Angleterre, et, se défiant des sujets de querelle qu'on lui
susciterait en France, il pria instamment Beretti, de qui la prudence
lui était très suspecte, de ne lui adresser aucun paquet de Hollande
capable d'exciter des soupçons, ou de lui attirer la moindre affaire,
voulant en éviter avec une attention extrême, non seulement les causes,
mais même les prétextes. Il aurait été difficile alors de désabuser le
public de l'opinion généralement répandue d'une alliance secrète entre
le roi d'Espagne et le roi de Sicile. L'entreprise des Espagnols était
regardée comme un jeu joué entre ces deux princes, et quoique l'un agît
réellement en ennemi, pour dépouiller l'autre d'un royaume, dont il
était en possession, il semblait qu'il ne fût pas permis de douter de
l'intelligence qui était entre eux, pour donner une apparence de guerre,
capable de cacher leurs conventions secrètes. Stanhope, bien instruit de
la vérité, dit à Provane que, si le roi approuvait le projet de paix,
sitôt qu'il en ferait remettre la déclaration entre les mains de Stairs,
Provane en échange recevrait des mains {[}de{]} ce ministre un ordre du
roi d'Angleterre à l'amiral Bing de faire ce que le roi de Sicile lui
commanderait pour s'opposer aux Espagnols. Ces offres, loin de plaire à
Provane, zélé pour les intérêts de son maître, le firent gémir sur
l'étrange situation où se trouvait ce prince, forcé d'accepter un projet
qu'il ne pouvait goûter, ou de perdre la Sicile dont la perte devenait
encore plus malheureuse que n'en avait été l'acquisition. Le régent
ajouta aux discours de Stanhope, qu'il déclarerait incessamment au roi
d'Espagne que, s'il ne retirait ses troupes de la Sicile, la France ne
pouvait refuser l'effet de sa garantie. Stanhope partit pour Madrid,
portant à ceux qui étaient chargés des affaires de France en cette
cour-là les ordres que lui-même avait dictés. Ce n'était pas seulement
en Espagne que le ministère d'Angleterre les prescrivait, comme il n'a
que trop continué, et même depuis que l'intérêt particulier a changé. En
tout endroit de l'Europe où la France tenait un ministre, s'il voulait
plaire et conserver son poste, il fallait qu'il fût non seulement
subordonné, mais obéissant aux Anglais, et de cette obéissance qu'ils
appellent passive. Châteauneuf, ambassadeur en Hollande, leur était
insupportable parce que, ce joug lui étant nouveau, il semblait
quelquefois vouloir y résister. Les Anglais ne cessaient donc de
représenter que, tant que cet homme demeurerait à la Haye, il
embarrasserait la négociation. Ils l'accusèrent d'intelligence avec le
secrétaire de Savoie, avec le baron de Norwick du collège des nobles,
partisan d'Espagne, et avec beaucoup d'autres amis de cette couronne.
Ils prétendaient que tout ce qu'ils communiquaient de plus important et
de plus secret, était aussitôt révélé par l'ambassadeur de France.

On pressait vivement la conclusion de la triple alliance entre cette
couronne, l'empereur et l'Angleterre. Stairs, ardent à exécuter les
ordres qu'il recevait de Londres, était parvenu à régler les conditions
du traité au commencement du mois de juillet. S'il y restait encore
quelques difficultés de la part de l'empereur, elles devaient être
aplanies par Penterrieder, son envoyé à Londres, muni des pouvoirs
nécessaires pour signer au plus tôt un traité que ce prince regardait
comme avantageux pour lui et pour sa maison. L'avis de ses ministres
était conforme au sien, et, selon eux, cette alliance était l'unique
moyen d'assurer à leur maître la conservation des États qu'il possédait
en Italie\,; ils jugeaient en même temps qu'il était de l'intérêt de
l'empereur de s'opposer au succès des pratiques du duc de Savoie, qui
n'avait rien oublié pour engager le roi d'Espagne dans ses intérêts, et
ne désespérait pas encore d'y réussir, nonobstant la descente des
Espagnols en Sicile. En effet, jusqu'alors le ministre d'Espagne à
Vienne s'était intéressé en faveur de ce prince, et ne cessait d'appuyer
la proposition d'une alliance entré l'empereur, le roi d'Espagne et le
roi de Sicile\,; mais alors Sa Majesté Catholique se désistait de cette
proposition, et demandait qu'en l'abandonnant l'empereur consentît à
laisser à l'Espagne l'île de Sardaigne, offrant en échange de consentir
réciproquement que Sa Majesté Impériale reprît la partie du Milanais
qu'elle avait cédée au duc de Savoie, et que le Montferrat y fût encore
ajouté. Un Suisse, nommé Saint-Saphorin, homme plus intrigant qu'il
n'appartient à la franchise de sa nation, employé autrefois par le roi
Guillaume et toujours opposé aux intérêts de la France, était encore
employé par le roi Georges, et même avait gagné trop de confiance de la
part du régent. Cet homme, devenu négociateur, soutenait qu'il était de
l'intérêt de toutes les puissances de l'Europe d'abaisser celle du duc
de Savoie. Ce prince, étonné de la descente imprévue des Espagnols en
Sicile, suivie de la prise de Palerme, écrivit aussitôt au régent pour
lui demander, en exécution du traité d'Utrecht, les secours de troupes
que la France était obligée de fournir pour la garantie du repos de
l'Italie\,; le courrier, dépêché à Paris au comte de Provane, remit
aussi au comte de Stanhope, qui s'y trouvait encore alors, une lettre
pour le roi d'Angleterre, contenant les mêmes instances. Cellamare ne
manqua pas, de s'y opposer\,; mais le régent lui répondit que par le
traité d'Utrecht le roi était également garant et du repos de l'Italie
et de la réversion de la Sicile à la couronne d'Espagne\,; que Sa
Majesté, manquant à l'un de ses engagements, ne pourrait se croire
obligée à l'autre, stipulé par le même traité. Son Altesse Royale offrit
donc des secours à Provane\,; mais on jugeait par la manière dont ce
prince les offrait qu'il n'avait nulle intention d'exécuter ce qu'il
promettait\,; on sut même qu'il avait fait quelques railleries de l'état
où, se trouvait le duc de Savoie, et il revint dans le public qu'il
avait dit que le renard était tombé dans le piège, que le trompeur avait
été trompé, enfin plusieurs discours dont ceux qui les avaient entendus
n'avaient pas gardé le secret. La discrétion n'était pas plus grande
alors sur les affaires d'État, dont les particuliers n'ont pas, droit de
raisonner, encore moins de censurer les résolutions du gouvernement\,;
on condamnait librement et sans la moindre contrainte tant de traités
différents, tant d'engagements opposés les uns aux autres, tant de
liaisons avec les ennemis anciens et naturels de la France, prises
secrètement et sans la connaissance du conseil de régence. On ne blâmait
pas moins les dépenses immenses faites mal à propos pour s'assurer de la
foi légère et de la constance plus que douteuse de ces puissances, et
les raisonneurs concluaient qu'il était difficile de comprendre comment
et par quelle maxime on se séparait de l'Espagne dont l'alliance, loin
d'être à charge à la France, serait toujours très utile à ses amis, et
qu'on l'abandonnait dans la fausse vue d'acquérir chèrement des amis
très infidèles. Cellamare était préparé à faire cette réponse au régent,
s'il lui eût parlé, comme il s'y attendait, des bruits répandus alors
d'un parti considérable que le roi d'Espagne avait en France\,; mais ce
n'était pas par un aveu de l'ambassadeur d'Espagne que Son Altesse
Royale comptait de découvrir toutes les circonstances des trames
secrètes, dont elle savait déjà la plus grande partie. Le duc de Savoie,
s'adressant de tous côtés pour être secouru, ne trouva pas en Angleterre
plus de compassion de son état qu'il en avait trouvé en France. La
Pérouse, son envoyé à Londres, exposait le triste état de son maître. Il
demandait inutilement en conséquence du traité d'Utrecht, des secours
contre l'invasion que les Espagnols faisaient de la Sicile. Loin de
toucher et de persuader par ses représentations, l'opinion commune à
Londres, comme à Paris, était que le roi d'Espagne et le roi de Sicile
agissaient de concert\,; et sur ce fondement les ministres d'Angleterre
répondirent à La Pérouse que l'escadre Anglaise secourrait son maître au
moment qu'il aurait signé le traité d'alliance que le roi d'Angleterre
lui avait proposé. Monteléon persistait cependant à croire que le roi
d'Espagne n'avait rien à craindre de la part de l'Angleterre, et soit
persuasion, soit désir de flatter Albéroni et de lui plaire, il l'assura
que le comte de Stanhope, nouvellement parti pour Madrid, joignait à son
penchant pour l'Espagne une estime singulière pour ce cardinal, en sorte
que, possédant la confiance intime du roi d'Angleterre, son voyage à
Madrid ne pouvait produire que de bons effets. Albéroni ne donnait à qui
que ce soit sa confiance entière, et l'aurait encore moins donnée à
Monteléon qu'à tout autre ministre. Il se défiait généralement de tous
ceux que le roi d'Espagne employait dans les cours étrangères. Alors il
avait envoyé secrètement à Vienne un ecclésiastique, qu'il avait chargé
de proposer à l'empereur un accommodement particulier avec le roi
d'Espagne, sans intervention de médiateur. Les conditions étaient que la
Sardaigne serait laissée au roi d'Espagne\,; qu'en même temps l'empereur
lui donnerait l'investiture des duchés de Toscane et de Parme\,; que le
roi d'Espagne réciproquement mettrait l'empereur en possession de la
Sicile\,; et que, de plus, il l'aiderait à recouvrer la partie de l'État
de Milan, qu'il avait cédée au duc de Savoie. Enfin on procurerait de
concert la propriété du Montferrat au duc de Lorraine.

\hypertarget{chapitre-xi.}{%
\chapter{CHAPITRE XI.}\label{chapitre-xi.}}

1718

~

{\textsc{Belle et véritable maxime, et bien propre à Torcy.}} {\textsc{-
Les Anglais frémissent des succès des Espagnols en Sicile et veulent
détruire leur flotte.}} {\textsc{- Étranges et vains applaudissements et
projets d'Albéroni.}} {\textsc{- Son opiniâtreté.}} {\textsc{- Menace le
régent.}} {\textsc{- Ivresse d'Albéroni.}} {\textsc{- Il menace le pape
et les siens.}} {\textsc{- Son insolence sur les grands d'Espagne.}}
{\textsc{- Le pape désapprouve la clôture du tribunal de la nonciature
faite par Aldovrandi.}} {\textsc{- Exécrable caractère du nonce
Bentivoglio.}} {\textsc{- Sagesse d'Aldovrandi.}} {\textsc{-
Représentations d'Aubenton à ce nonce pour le pape.}} {\textsc{-
Audacieuse déclaration d'Albéroni à Nancré.}} {\textsc{- Le traité entre
la France, l'Angleterre et l'empereur, signé à Londres.}} {\textsc{-
Trêve ou paix conclue entre l'empereur et les Turcs.}} {\textsc{- Idées
du régent sur le nord.}} {\textsc{- Cellamare travaille à unir le czar
et le roi de Suède pour rétablir le roi Jacques.}} {\textsc{- Artifices
des Anglais pour alarmer tous les commerces par la jalousie des forces
maritimes des Espagnols.}} {\textsc{- Attention d'Albéroni à rassurer
là-dessus.}} {\textsc{- Inquiétude et projets d'Albéroni.}} {\textsc{-
Albéroni se déchaîne contre M. le duc d'Orléans.}} {\textsc{- Fautes en
Sicile.}} {\textsc{- Projets d'Albéroni.}} {\textsc{- Il se moque des
propositions faites à l'Espagne par le roi de Sicile.}} {\textsc{-
Albéroni pense à entretenir dix mille hommes de troupes étrangères en
Espagne\,; fait traiter par leurs Majestés Catholiques, comme leurs
ennemis personnels, tous ceux qui s'opposent à lui.}} {\textsc{- Inquiet
de la lenteur de l'expédition de Sicile, il introduit une négociation
d'accommodement avec Rome.}} {\textsc{- Son artifice.}} {\textsc{- Les
Espagnols dans la ville de Messine.}}

~

Ce siècle était celui des négociations, en même temps celui où régnait
entre les souverains une défiance réciproque, leurs ministres bannissant
la bonne foi et se croyant habiles autant qu'ils savaient le mieux
tromper. L'empereur, persuadé que nulle alliance n'était aussi solide
pour lui que celle d'Angleterre, ne perdit pas de temps à communiquer au
roi d'Angleterre les propositions secrètes d'Albéroni. La droiture et la
sincérité du ministre n'étaient pas mieux établies que celles du duc de
Savoie. Ainsi l'opinion commune à Londres comme à Vienne était que,
malgré les apparences, tous deux agissaient de concert, et que l'Espagne
n'envahissait la Sicile que du consentement secret du duc de Savoie,
quelque soin que prît ce prince de déguiser une convention cachée, et de
demander des garanties qu'il serait fâché d'obtenir. Sur ce fondement
l'empereur répondit aux propositions d'Albéroni qu'il en accepterait le
projet, lorsqu'il serait sûr du consentement et du concours des
médiateurs. Mais l'artifice d'un ministre tel qu'Albéroni, dont la bonne
foi était plus que douteuse, et suspectée également dans toutes les
cours, loin de suspendre, comme il l'espérait, la conclusion du traité
de la triple alliance, en pressa la signature\,: car il ne suffit pas
que la probité des princes soit connue et hors de doute, si la
réputation de ceux dont ils se servent dans leurs affaires les plus
importantes n'est aussi sans tache ni susceptible parleur conduite
passée d'accusation ni même de soupçon. Albéroni ne jouissait pas de
cette réputation si flatteuse et si nécessaire au succès des affaires
dont un ministre est chargé. La cour de Rome ne se plaignait pas moins
que le duc de Savoie de la fausseté des promesses et des assurances
qu'il avait faites et données à l'une et à l'autre de ces deux cours.

Leurs plaintes n'arrêtaient pas le progrès des Espagnols, et la Sicile
était soumise au roi d'Espagne à la fin de juillet. Cette conquête si
rapide et si facile déplaisait aux Anglais, à mesure du peu d'opposition
que les Espagnols trouvaient à s'emparer totalement de l'île. Les agents
d'Angleterre en différents lieux d'Italie représentaient qu'il était de
l'intérêt de cette couronne d'anéantir la flotte d'Espagne, sinon
qu'elle serait bientôt employée en faveur du prétendant\,; qu'on devait
se souvenir à Londres du projet formé en sa faveur peu de temps
auparavant avec les princes du nord et de l'arrêt du comte de
Gyllembourg, alors ambassadeur du roi de Suède\,; qu'on ne devait pas
non plus oublier que Monteléon était instruit de son dessein\,; que,
ruinant la flotte d'Espagne, chose facile, non seulement l'Angleterre
aurait la gloire et l'avantage de secourir le duc de Savoie, mais qu'il
serait impossible à l'Espagne de réparer la perte qu'elle aurait faite
et de ses vaisseaux et de son armée, au lieu que, laissant à cette
couronne la liberté entière de poursuivre ses desseins, elle joindrait
bientôt la conquête du royaume de Naples à celle de la Sicile. Les
ennemis de l'Espagne craignaient le génie de son premier ministre, et
n'oubliaient rien pour inspirer de tous côtés la crainte des projets et
des entreprises qu'il était capable de former et d'exécuter. Mais
pendant qu'ils travaillaient à décrier Albéroni, il s'applaudissait à
Madrid du succès étonnant des mesures prises et des ordres donnés pour
la conquête de la Sicile. Il admirait qu'une flotte de cinq cents
voiles, partie de Barcelone le 27 juin, eût débarqué heureusement dans
le port de Palerme, le 3 juillet, toutes les troupes dont elle était
chargée avec l'attirail nécessaire pour une descente. Cet heureux début
lui ouvrit de grandes vues pour l'avenir. Comme il fallait cependant
donner une couleur à cette entreprise et justifier une expédition faite
en pleine paix, au préjudice des traités, Albéroni supposa que le roi
d'Angleterre, médiateur de la triple alliance qui se négociait
actuellement, avait intention d'engager le duc de Savoie de livrer la
Sicile à l'archiduc, contre les dispositions du traité d'Utrecht,
portant expressément que cette île retournerait au pouvoir de l'Espagne
au défaut d'héritiers mâles du duc de Savoie à qui la Sicile était
cédée. Albéroni voulait persuader qu'une telle contravention aux traités
de paix avait forcé le roi d'Espagne à prévenir le coup en s'assurant
d'un royaume qui lui appartenait par toutes les raisons de droit divin
et humain.

Le projet d'Albéroni était d'entretenir en Sicile une armée de
trente-six mille hommes, nombre de troupes suffisant non seulement pour
conserver sa conquête, mais encore pour tenir en inquiétude les
Allemands dans le royaume de Naples et leur faire sentir les
incommodités d'un pareil voisinage. La conquête de la Sicile,
l'espérance de la conserver, de passer facilement à celle de Naples, et
l'idée de chasser ensuite les Allemands de toute l'Italie, devinrent
pour le roi d'Espagne de nouveaux motifs de rejeter absolument le traité
d'alliance proposé par le roi d'Angleterre et de s'irriter de la
facilité que le régent a voit eue d'acquiescer aux propositions de ce
prince, d'envoyer même Nancré à Madrid pour appuyer les instances que le
comte de Stanhope devait faire, et persuader à Sa Majesté Catholique d'y
consentir. Albéroni prétendit que, bien loin que tant de mouvements
dussent toucher Sa Majesté Catholique, ils faisaient voir, au contraire,
quelle était l'agitation des ministres du roi d'Angleterre, la crainte
qu'ils avaient des recherches d'un nouveau parlement qui s'élèverait
contre une conduite si contraire aux véritables intérêts de la nation,
enfin la partialité. déclarée du roi Georges pour l'empereur et sa
maison. «\,On ne comprend pas, disait Albéroni, comment le régent ne
connaît pas une vérité si évidente, comment il veut s'unir à un
ministère si incertain et avec une nation sur qui on ne peut pas
compter.\,» De ces réflexions Albéroni passait à une espèce de menace\,:
«\,Si, disait-il, Son Altesse Royale veut signer une ligue si
détestable, le roi d'Espagne fera les pas qu'il estimera convenables aux
intérêts du roi son neveu, aussi bien qu'à la conservation d'une
monarchie et d'une nation qu'il protégera et qu'il défendra jusqu'à la
dernière goutte de son sang. Sa Majesté Catholique pourra dire qu'elle a
satisfait à tous ses devoirs par les représentations qu'elle a faites
pour mettre le régent dans le chemin de la justice. Enfin
\emph{cvravimus Babyloneni}.\,» Albéroni ajoutait\,: «\,Dieu sait ma
peine à modérer la juste indignation du roi d'Espagne, quand il a su les
sollicitations du régent envers la Hollande\,; je suis las de parler
davantage de modération, Leurs Majestés Catholiques commencèrent à
s'ennuyer de cette chanson.\,» Cet échantillon des conférences de Nancré
avec Albéroni peint à peu près le fruit qu'il remporta de sa mission en
Espagne, où il avait été envoyé principalement pour appuyer et seconder
les instances de Stanhope. Albéroni disait que le régent aurait été
convaincu de la solidité des réponses du roi d'Espagne, s'il eût été
question de persuader \emph{l'entendement et non la volonté}.

Le cardinal, encore plus piqué du refus des bulles de Séville que des
négociations du régent avec le roi d'Angleterre, ne doutait pas que la
conquête de la Sicile ne lui donnât les moyens de se venger du pape
personnellement, aussi bien que des principaux personnages, de la cour
de Rome. Il menaçait déjà la maison Albane \emph{d'une estafilade que le
roi d'Espagne pouvait aisément lui donner}. Il voulut aussi avoir une
liste exacte des cardinaux et prélats romains possesseurs d'abbayes ou
de pensions ecclésiastiques dans la Sicile. Ébloui du désir de
vengeance, il bravait par avance les censures de Rome, et disait que,
puisque Sa Sainteté n'avait pas osé en lancer la moindre contre le
cardinal de Noailles, qui s'était fait chef d'une hérésie en France,
elle oserait encore moins faire un coup d'éclat contre le roi d'Espagne,
bien informé que l'acharnement de la cour de Rome contre lui était tel,
que Sa Majesté Catholique devait penser à la réprimer à quelque prix que
ce pût être. Elle se trompait, selon lui, si elle comptait sur
l'ancienne superstition espagnole. \emph{Altri tempi}, etc. Ces
superstitions étaient l'ouvrage des grands, persuadés qu'il était de
leur intérêt de les imprimer dans l'esprit des peuples\,; mais ces mêmes
grands étaient sans autorité, sans crédit, toujours dans la crainte et
le tremblement, enfin comptant pour beaucoup de vivre en repos. Albéroni
donc ajoutait que, le roi son maître ayant fait connaître qu'il n'était
pas un \emph{zéro, et} que ceux qui l'avaient méprisé auraient un jour à
s'en repentir, trouverait des amis\,; que plusieurs même
s'empresseraient d'être admis dans ce nombre. «\,Du temps, disait-il, de
la santé et de la patience\,!» Il savait que le pape avait désapprouvé
la demande que le nonce, Aldovrandi avait faite de fermer, sans ordre de
Sa Sainteté, le tribunal de la nonciature à Madrid, et véritablement le
ministre de Sa Sainteté faisait tort à la juridiction que le saint-siège
s'était attribuée et maintenait dans ce royaume. Ainsi le pape fit voir
par un bref postérieur que son intention avait été seulement de
suspendre les grâces et privilèges que ses prédécesseurs avaient
accordés aux rois d'Espagne. Le nonce Bentivoglio, averti de ce bref et
de ce qu'il contenait, jugea que la cour de France s'intéresserait peu à
l'embarras qu'il pourrait causer à celle d'Espagne, et de plus, que le
régent ne serait pas fâché de voir croître en même temps le nombre des
ennemis du pape et les oppositions que le roi d'Espagne trouverait à
l'exécution de ses projets. Le caractère de ce nonce impétueux, violent,
sans érudition, uniquement occupé que du désir effréné de parvenir au
cardinalat, se montrait, dans toute sa conduite, persuadé que le moyen
le plus sûr, le plus prompt, le plus aisé d'obtenir cette dignité était
d'irriter le pape et de mettre le feu dans l'Église de France\,; il
n'oubliait rien pour arriver à son but, etc.

Le nonce du pape à Madrid, plus sage que celui qui résidait en France,
avait aussi mieux connu de quelle importance il était pour le
saint-siège de ménager les grandes couronnes\,; il jugea donc qu'il
était essentiel pour le bien de l'Église de conserver une voie à
l'accommodement, lorsque le temps aurait un peu calmé l'aigreur de part
et d'autre. Aubenton, jésuite, confesseur du roi d'Espagne, ouvrit cette
voie, Il vint trouver Aldovrandi la veille de son départ de Madrid, et
le priant de ne le nommer jamais dans ses lettres, il le chargea bien
expressément de bien représenter au pape quel mal il ferait s'il fermait
la voie à tout accommodement\,; que déjà la cour d'Espagne se croyait
méprisée, et qu'elle s'irriterait au point de perdre le respect et
l'obéissance due au saint-siège, si Sa Sainteté n'y prenait garde et
n'adoucissait par sa prudence les différends survenus au sujet des
bulles de Séville\,; il représenta que l'intérêt d'un particulier tel
qu'Albéroni ne devait point causer de pareils désordres.

La cour d'Espagne était alors occupée d'affaires plus sensibles pour
elle que ne l'étaient celles de Rome. La mission de Nancré n'avait pas
eu tout, le succès que le régent s'en était promis, et le cardinal avait
déclaré à cet envoyé que le roi d'Espagne, informé de la résolution que
son Altesse Royale a voit prise de signer un traité d'alliance avec
l'empereur et le roi d'Angleterre, souhaitait qu'elle voulût abandonner
un tel projet ou tout au moins en suspendre l'exécution. En ce cas, Sa
Majesté Catholique s'engagerait à regarder les intérêts du régent comme
les siens propres. Au contraire, le ressentiment d'un refus serait tel
que ni le temps ni même les services ne le pourraient effacer, et qu'il
aurait en toute occasion le roi d'Espagne pour ennemi personnel. Nancré,
pressé par le cardinal d'envoyer un courrier à Paris porter une telle
déclaration, le refusa, et dit de plus que, quand même il se pourrait
charger d'en rendre compte, il serait inutile, parce que le traité
devait être déjà signé. Albéroni répliqua que, lorsque le roi d'Espagne
serait assuré de la signature, Nancré ne demeurerait pas encore un quart
d'heure à Madrid. Albéroni ne s'expliquait pas moins clairement aux
ministres d'Angleterre qu'il avait parlé à Nancré au sujet du traité
dont le roi d'Espagne rejetait toute proposition. Ainsi le colonel
Stanhope, ne pouvant douter de la résolution de Sa Majesté Catholique,
détournait le comte de Stanhope son cousin, ministre confident du roi
d'Angleterre, de faire le voyage de Madrid, prévoyant que la peine en
serait inutile, ainsi que les fréquentes déclarations du cardinal
réitérées à toute occasion ne permettaient pas d'en douter. En effet, le
traité était signé à Londres, et le roi d'Angleterre avait conseillé au
duc de Savoie d'y souscrire comme le meilleur parti qu'il pût prendre
pour résister à l'invasion des Espagnols.

La flotte Anglaise naviguait en même temps vers la Sicile\,; et déjà les
ministres d'Angleterre avaient déclaré à Monteléon que le roi leur
maître n'avait pu se dispenser d'envoyer ses vaisseaux pour maintenir la
neutralité d'Italie, et défendre, en conséquence des traités, les États
possédés par l'empereur\,; que cependant Sa Majesté Britannique
attendait encore quel serait le succès du voyage que le comte de
Stanhope ferait à Madrid, d'où dépendait la paix générale ou une
malheureuse rupture. Quoique le roi de Sicile n'eût de secours à espérer
que de la part de l'Angleterre, il hésitait cependant à l'accepter avec
la condition d'accéder au traité d'alliance, comme le demandait le roi
d'Angleterre. Stairs, son ambassadeur en France, offrait à Provane,
ministre de Savoie à Paris, de lui remettre l'ordre par écrit de Sa
Majesté Britannique, adressé à l'amiral Bing pour attaquer les Espagnols
sitôt que le duc de Savoie aurait accepté le projet de traité, et
Provane n'était pas autorisé à promettre que cette acceptation serait
faite. Il se bornait à demander au régent la garantie de la Sicile\,;
instances inutiles. Son Altesse Royale lui répondait que la France
n'avait point d'armée navale. Le mariage d'une des princesses ses filles
avec le prince de Piémont était alors une de ses vues, et c'était
vraisemblablement un moyen d'y réussir que de dégager le duc de Savoie
de la guerre de Sicile en persuadant au roi d'Espagne de consentir aux
propositions de Stanhope. Deux motifs pouvaient y porter Sa Majesté
Catholique. L'un était la difficulté de réduire les places de Sicile\,;
l'autre motif, la conclusion d'une trêve entre l'empereur et les Turcs,
dont la nouvelle était récemment arrivée.

Ces apparences de pacification et d'assurer là tranquillité générale de
l'Europe, n'empêchaient pas le régent de chercher encore d'autres moyens
d'en assurer le repos, et soit pour en être plus sûr, soit que le génie
dominant du siècle fût de négocier, Son Altesse Royale voulait que les
monarques du nord, particulièrement le czar, crussent que la conclusion
du traité proposé au roi d'Espagne ne l'empêcherait pas de s'unir avec
ces princes\,; même, s'il était nécessaire, qu'elle renouvellerait de
concert avec eux la guerre contre l'empereur\,; mais, soit vérité, soit
dessein d'amuser, les ministres de ces princes, principalement celui du
czar, ajoutèrent peu de foi à de tels discours. Ce dernier assura
Cellamare que le czar ne pouvant approuver les liaisons nouvelles de la
France avec l'Angleterre et la maison d'Autriche, voulait de concert
avec le roi de Suède, unir leurs intérêts communs à ceux du roi
d'Espagne. On attribuait à de mauvais conseils (Dubois) la confiance que
le régent avait prise aux promesses du roi d'Angleterre, et Cellamare,
persuadé de l'utilité dont une ligue des princes du nord pouvait être à
son maître, pressait le ministre du czar de le représenter à Son Altesse
Royale, et de l'engager, s'il était possible, à fomenter lés troubles
qu'on croyait prêts à s'élever en Écosse.

Le duc d'Ormond, nouvellement arrivé à Paris, où il se tenait caché,
prétendait qu'il y avait en Angleterre un parti pour le roi Jacques plus
ardent que jamais pour les intérêts de ce prince. L'argent pour le
soutenir et le fortifier était absolument nécessaire, et ne pouvant en
espérer de France, il s'était adressé à l'ambassadeur d'Espagne pour
obtenir l'assistance de Sa Majesté Catholique. Ce ministre ne doutait
pas de la bonne volonté de son maître, mais il connaissait l'état de
l'Espagne et son impuissance. Étant donc persuadé qu'elle ne pouvait
fournir les sommes nécessaires pour le succès d'une si grande
entreprise, son objet était de la faire goûter au czar, mécontent du roi
d'Angleterre, et de l'engager à s'unir avec le roi de Suède pour se
venger tous deux de concert des sujets qu'ils pouvaient avoir d'être
mécontents de la conduite de ce prince à leur égard. Le temps était
précieux, et Cellamare connaissant l'importance d'en ménager tous les
moments, n'en perdit aucun pour animer le ministre de Moscovie. Il alla
secrètement le trouver à la campagne où il était auprès de Paris, et
l'ayant informé des dispositions du roi d'Espagne, il le pressa de
dépêcher au plus tôt un courrier à Pétersbourg pour instruire le czar
des dispositions de Sa Majesté Catholique, et demander des instructions
sur une négociation dont il connaissait parfaitement toutes les
conséquences. Cellamare informa le roi de Suède par une voie détournée
des mêmes avis qu'il donnait au czar, et non content d'exciter les
puissances étrangères à traverser les desseins du régent, il cherchait
encore à détacher du service du roi des gens dont le nom, plutôt que le
mérite peu connu, pouvait faire plus d'impression dans les pays
étrangers qu'ils n'en faisaient en France.

Si la descente des Espagnols en Sicile, la conquête facile de Palerme et
celle de toute l'île qu'on regardait déjà comme assurée, avait surpris
toute l'Europe, on ne l'était pas moins d'avoir vu paraître, et comme
sortir du fond de la mer une flotte en ordre, armée par une couronne qui
ne s'était pas distinguée par ses armements de mer depuis le règne de
Philippe II. Cette nouvelle puissance maritime alarmait déjà les
Anglais. Ils croyaient aisément, et publiaient que la véritable vue du
conseil d'Espagne en relevant ses forces de mer, était de s'opposer
généralement à tout commerce que les nations étrangères pourraient faire
aux Indes occidentales. Il était facile qu'un tel soupçon fît en peu de
temps un grand progrès en Hollande et en Angleterre. Albéroni, prévoyant
l'effet que la jalousie du commerce pourrait causer dans l'un et l'autre
pays, écrivit par l'ordre du roi d'Espagne à son ambassadeur en Hollande
d'assurer non seulement les négociants Hollandais, mais encore les
Anglais qui se trouveraient dans ce pays, et généralement tout homme de
commerce, que jamais Sa Majesté Catholique n'altérerait les lois
établies, et ne manquerait aux traités. Ce ministre devait aussi leur
dire que le peu de forces que le roi son maître avait en mer était
seulement pour la sûreté de ses côtes dans la Méditerranée, aussi bien
que pour la défense et la conduite de ses galions\,; qu'à la vérité, Sa
Majesté Catholique avait lieu de se plaindre de la déclaration des
Anglais\,; mais un tel procédé de leur part n'avait pas empêché qu'elle
n'eût donné ordre de ne pas toucher aux effets qui appartiendraient aux
Anglais sur la flotte nouvellement arrivée à Cadix, l'intention de Sa
Majesté Catholique étant de faire remettre à chacun des intéressés ce
qui pouvait leur appartenir.

Le ministre d'Espagne n'était pas cependant sans inquiétude du succès
qu'aurait la descente des Espagnols en Sicile, et de la suite de leur
premier succès. Son projet n'était pas encore bien formé, et ses
résolutions incertaines dépendaient de l'événement. Albéroni voulait
croire que la Sicile serait soumise en peu de temps\,; il se proposait
de faire ensuite passer l'armée d'Espagne\,; mais il sentait, et
l'avouait même, que c'était uniquement aux officiers généraux qui
commandaient l'armée à délibérer et décider des résolutions qu'il
conviendrait de prendre. L'escadre Anglaise lui donnait de justes
inquiétudes\,; il savait qu'elle voguait vers le Levant, mais depuis
assez longtemps il ignorait sa route, et les premiers jours d'août, il
n'en savait de nouvelles que du 14 juillet, écrites de Malaga. Ce même
jour 14, le château de Palerme se rendit aux Espagnols. Le vice-roi de
Naples faisait quelques mouvements, comme ayant dessein d'envoyer en
Sicile un détachement des troupes de l'empereur pour fortifier la
garnison de Messine. Ce secours paraissait difficile, et l'opinion
publique était que les ministres allemands ne faisaient ces
démonstrations que pour satisfaire par des apparences les ministres de
Savoie, et d'ailleurs, le public était persuadé que, si les troupes
allemandes marchaient effectivement et secouraient Messine, ce ne serait
pas pour la rendre aux Piémontais. La défiance était généralement
répandue dans toutes les cours, et les sentiments du pape n'étaient pas
exempts de soupçon, en sorte que, quelques brouilleries qu'il y eût
actuellement entre la cour de Rome et celle de Madrid, l'opinion
publique était qu'il régnait secrètement une union intime entre Sa
Sainteté et le roi d'Espagne. Les troupes de ce prince, après une légère
résistance à Palerme, dont elles s'étaient emparées, avaient marché vers
Messine, et les galères du duc de Savoie s'étaient retirées à leur
approche.

Jusqu'alors l'entreprise de la Sicile réussissait comme le roi d'Espagne
et son ministre le pouvaient désirer, et ces succès heureux augmentant
la fierté du ministre, irrité du refus constant des bulles de Séville,
il se déchaîna sans mesure contre Sa Sainteté, et l'accusait de se
laisser conduire par les conseils du comte de Gallas, ambassadeur de
l'empereur auprès d'elle, qui, de son côté, prétendait que le pape était
secrètement uni avec le roi d'Espagne. Mais Albéroni s'élevait sans
ménagement contre la personne de M. le duc d'Orléans et l'empressement
qu'il avait fait paraître à signer le traité de la quadruple alliance.
«\,Ainsi, disait Albéroni, ce prince s'est déclaré à la face de tout
l'univers ami d'une puissance ennemie d'un roi son parent, et le temps
est venu où vraisemblablement il sera obligé à se porter contre ce même
roi à des actes d'hostilité. Le maréchal d'Huxelles, qui a consenti à
cette alliance pour n'avoir point de guerre, verra la France agir contre
le roi d'Espagne, qui, de son côté, sera ferme à continuer éternellement
la guerre plutôt que de consentir à l'infâme projet, et tant qu'il aura
de vie et de forces, il se vengera de ceux qui prétendent le forcer à
l'accepter. Si Stanhope veut parler du ton de législateur, il sera mal
reçu. Le passeport qu'il a demandé a été expédié, on entendra ses
propositions\,; mais il sera difficile de les écouter si elles ne sont
pas différentes en tout de la substance du projet. Stanhope,
ajoutait-il, sera surpris d'entendre, que le roi d'Espagne ne veut pas
qu'on parle présentement des États de Toscane et de Parme, se réservant
d'user de ses droits en temps et lieu.\,» Albéroni, s'expliquant
hautement contre le traité de la quadruple alliance, voulut en même
temps faire voir aux Anglais que, si le roi d'Espagne rejetait un pareil
projet, il n'en était pas moins prêt à donner à la nation Anglaise des
preuves de son affection pour elle\,; que c'était un témoignage bien
sensible de cette affection, que la modération dont Sa Majesté
Catholique donnait une preuve évidente en défendant à ses sujets
d'exercer aucun acte d'hostilité contre les négociants Anglais demeurant
dans ses États, quoiqu'on dût l'attendre comme une suite naturelle de la
rupture faite à contretemps par le commandant de la flotte Anglaise.

Albéroni, flatté des premiers succès de l'entreprise de Sicile, ne
laissait pas de remarquer les fautes que le marquis de Lede avait faites
dans cette expédition, et de prévoir les suites funestes qu'il y avait
lieu de craindre du flegme de ce général, et de sa lenteur à finir une
conquête aisée. Tout délai en cette occasion était d'autant plus à
craindre que l'escadre Anglaise faisait voile vers la Sicile. Il fallait
donc prévenir son arrivée, et sans perdre de temps faire marcher les
troupes vers Messine, dont il serait désormais difficile de s'emparer,
le coup de la prise de Palerme ayant mis en mouvement, suivant
l'expression du cardinal, toutes les puissances infernales, et les
mesures étant prises de tous côtés pour embarrasser l'Espagne. Il
reprochait encore au marquis de Lede, général de l'armée d'Espagne,
d'avoir laissé au comte Maffeï, vice-roi de l'île pour le duc de Savoie,
la liberté entière de se retirer à Syracuse, qu'on devait regarder non
seulement comme la meilleure forteresse du royaume, mais qu'on savait de
plus être en état de recevoir les secours d'hommes et de vivres
proportionnés au besoin qu'elle en, aurait. Il était encore de la
prudence de faire suivre Maffeï par un détachement de cavalerie\,; et
quoique fatiguée, ce n'était pas une raison pour l'exempter de marcher,
la conjoncture étant si importante qu'il n'était pas permis de ménager
les troupes, quand même il aurait été sûr qu'elles périraient dans la
marche. D. Jos. Patiño était alors intendant de l'armée. Albéroni
l'exhorta pour l'amour de Dieu, disait-il, à donner un peu plus de
chaleur au naturel froid de son ami le marquis de Lede. «\,S'il est bon,
disait le cardinal, d'épargner les troupes quand on le peut, il faut
aussi songer qu'elles sont faites pour fatiguer et pour crever quand il
convient\,; qu'à plus forte raison, on doit en user de même à l'égard
des bêtes.\,» La facilité de faire passer des troupes de Naples en
Sicile augmentait les difficultés que les Espagnols trouvaient à
s'emparer de Messine dont ils auraient pu se rendre maîtres sans peine,
si leur général, à qui Dieu pardonne son indolence, n'avait perdu le
temps à prendre Palerme, ville sans résistance. Albéroni comptait déjà
que la France, l'Angleterre, l'empereur et le duc de Savoie, s'uniraient
contre l'Espagne\,; le projet du cardinal était en ce cas de laisser
quinze mille hommes en Sicile, pour en faire la conquête entière\,; et
lorsqu'elle serait achevée, il prétendait transporter toutes ces troupes
en Espagne. Il soutenait que le duc de Savoie n'avait songé qu'à tromper
le roi d'Espagne, employant différentes voies pour l'amuser par de
vaines propositions de traité\,; qu'enfin Lascaris, le dernier des
ministres que ce prince avait employés, était venu, au moment que la
flotte partait, déclarer qu'il avait un pouvoir de son maître dans la
forme la plus solennelle, pour conclure avec le roi d'Espagne une ligue
offensive et défensive à des conditions véritablement à faire rire\,; ce
qu'on en sait est, que la première de ces conditions était deux millions
d'écus que le duc de Savoie demandait pour se mettre en campagne, et par
mois soixante mille écus de subside\,; la seconde, que le roi d'Espagne
fît passer en Italie douze mille hommes, pour les unir aux troupes de
Savoie et faire la guerre dans l'État de Milan. Mais Albéroni, persuadé
qu'on ne pouvait s'assurer sur la foi du duc de Savoie tant qu'il serait
maître de la Sicile, avait jugé nécessaire que le roi d'Espagne s'en
rendît maître soit pour la garder, soit pour la rendre au duc de Savoie
si Sa Majesté Catholique, faisant la guerre aux Allemands, ne pouvait
procurer à ce prince une récompense plus avantageuse de son alliance
avec l'Espagne.

Le cardinal, persuadé qu'il était de l'honneur et de l'intérêt de cette
couronne d'avoir toujours un corps de troupes en Espagne, prenait alors
des mesures pour maintenir sur pied huit ou dix mille hommes de troupes
étrangères. Ce fut à Cellamare qu'il s'adressa, pour savoir de lui
quelles mesures il jugerait nécessaires à prendre pour accomplir ce
dessein. Cette marque de confiance ne s'accordait guère avec le
traitement que le cardinal del Giudice, oncle de Cellamare, recevait
alors de la cour d'Espagne, tous les revenus des bénéfices qu'il
possédait en Sicile ayant été mis en séquestre. Il est vrai que les
revenus des bénéfices que d'autres cardinaux et prélats avaient dans le
même royaume eurent aussi le même sort, depuis la descente des Espagnols
en Sicile\,; mais le vrai motif était l'animosité particulière
d'Albéroni qui ne cessait d'aigrir Leurs Majestés Catholiques contre
Giudice, car il n'oubliait rien pour les engager à regarder et à traiter
comme leurs ennemis personnels ceux qui se déclaraient contre leur
premier, ministre. Il n'avait pas même ménagé le pape, désirant se
venger du refus constant qu'il lui faisait des bulles de Séville. Il
changea cependant de conduite, lorsque la lenteur de l'expédition de
Sicile lui donna lieu de craindre qu'après de beaux commencements, la
fin de l'entreprise ne répondît pas à ses espérances. Alors il jugea
nécessaire de ménager la cour de Rome, et de la prudence d'introduire,
une négociation pour un accommodement entre cette cour et celle
d'Espagne. Le cardinal Acquaviva eut ordre de le confier à D. Alexandre
Albane, second neveu du pape. Il fallait flatter ce jeune homme, neveu
chéri de Clément XI, en lui faisant entendre que le roi d'Espagne
n'ayant encore formé aucune prétention au préjudice de la cour de Rome,
tous différends entre les deux cours étaient faciles à terminer\,; que
D. Alexandre en aurait l'honneur, par conséquent avancerait sa
promotion, au cardinalat si son oncle, profitant d'une conjoncture
heureuse, l'envoyait nonce à Madrid. Mais pour y réussir sûrement, il
serait absolument nécessaire qu'il y vînt porteur des bulles de Séville,
préliminaire indispensable pour finir à son entière satisfaction toutes
les affaires qu'il trouverait à régler. Autrement Leurs Majestés
Catholiques deviendraient inexorables, et s'engageraient sans retour à
suivre les projets formés par le conseil de Castille, et par la junte
des théologiens et des canonistes. Albéroni, voulant mêler à cette
espèce de menace quelque espérance de toucher le pape, instruisit
Acquaviva de ce qu'il avait fait pour détromper Leurs Majestés
Catholiques de l'opinion où, elles étaient que, Sa Sainteté offrait même
d'envoyer un nouveau nonce, soit ordinaire, soit extraordinaire, comme
il plairait le plus à Leurs Majestés Catholiques. Albéroni,
s'applaudissant d'avoir eu le bonheur, grâce à Dieu, de leur persuader
que cette démarche du pape était fort honorable, concluait que Sa
Sainteté devait profiter d'une porte qui lui était ouverte pour sortir
d'un engagement qui durerait autant que sa vie, s'il négligeait ce moyen
facile de s'en débarrasser\,; que ce serait une satisfaction, pour un
ministre revêtu de la pourpre, d'avoir donné cette nouvelle preuve de
son respect et de son obéissance au pape et au saint-siège\,; mais que
Sa Sainteté devait aussi commencer par un acte de générosité tel que
serait l'expédition et l'envoi des bulles de Séville, grâce légère,
telle qu'on ne la pouvait refuser aux services importants d'un ministre
dont le travail assidu avait mis les finances du roi son maître en si
bon état que, non seulement il n'était rien dû à personne, mais qu'il
restait encore quelques sommes pour les dépenses journalières et
casuelles outre les consignations données sur les provinces pour le
payement des troupes, en sorte qu'il n'avait pas été détourné ni employé
un seul maravedis sur les fonds de l'année suivante.

Pendant que la cour de Rome cherchait les moyens d'apaiser celle
d'Espagne, et qu'il s'en fallait peu qu'Albéroni ne dictât les
conditions, dont le premier article était de lui accorder une grâce
contraire aux plus saintes règles, le pape n'en usait pas de même à
beaucoup près à l'égard des prélats qui tenaient le premier rang dans
l'Église de France, etc.

On apprit en France au commencement d'août que les Espagnols, continuant
leurs progrès en Sicile, étaient entrés sans résistance dans la ville de
Messine, aux acclamations unanimes du sénat et du peuple, les troupes
piémontaises s'étant retirées dans la citadelle. Mais en même temps on
apprit que la flotte Anglaise était à Naples, événements dignes
d'occuper l'attention des princes de l'Europe et de leurs ministres. Il
est par conséquent à propos de rappeler ce qui s'était passé depuis
l'année 1716.

\hypertarget{chapitre-xii.}{%
\chapter{CHAPITRE XII.}\label{chapitre-xii.}}

1718

~

{\textsc{Court exposé depuis 1716.}} {\textsc{- Négociation secrète de
Cellamare avec le duc d'Ormond caché dans Paris, où cet ambassadeur
continue soigneusement ses criminelles pratiques, que le régent n'ignore
pas.}} {\textsc{- Avis, vues et conduite de Cellamare.}} {\textsc{-
Fâcheux état du gouvernement en France.}} {\textsc{- Quadruple alliance
signée à Londres le 2 août, puis à Vienne et à la Haye.}} {\textsc{- Ses
prétextes et sa cause.}} {\textsc{- Dubois.}} {\textsc{- Morville en
Hollande très soumis aux Anglais.}} {\textsc{- Conduite de Beretti et de
Monteléon.}} {\textsc{- Plaintes réciproques des Espagnols et des
Anglais sur le commerce.}} {\textsc{- Violence du czar contre le
résident de Hollande.}} {\textsc{- Plaintes et défiances du roi de
Sicile.}} {\textsc{- Conduite de l'Angleterre à son égard, et de la
Hollande à l'égard du roi d'Espagne.}} {\textsc{- Projets de l'Espagne
avec la Suède contre l'Angleterre.}} {\textsc{- Mouvements partout
causés par l'expédition de Sicile.}} {\textsc{- Vues, artifices, peu de
ménagement de l'abbé Dubois pour M. le duc d'Orléans.}} {\textsc{-
Conduite et propos d'Albéroni.}} {\textsc{- Sa scélérate duplicité sur
la guerre, aux dépens du roi et de la reine d'Espagne.}} {\textsc{- Ses
artificieux discours au comte de Stanhope, qui n'en est pas un moment la
dupe.}} {\textsc{- Albéroni et Riperda en dispute sur un présent du roi
d'Angleterre au cardinal.}} {\textsc{- Embarras de Rome.}} {\textsc{- Le
pape et le roi d'Espagne fortement commis l'un contre l'autre.}}
{\textsc{- Poison très dangereux du cardinalat.}} {\textsc{- Lit de
justice des Tuileries qui rend au régent toute son autorité.}}
{\textsc{- Les Espagnols défaits\,; leur flotte détruite par Bing.}}
{\textsc{- Fausse joie de Stairs.}} {\textsc{- Sages et raisonnables
désirs.}} {\textsc{- Cellamare de plus en plus appliqué à plaire en
Espagne par ses criminelles menées à Paris.}} {\textsc{- Galions arrivés
à Cadix.}} {\textsc{- Demandes du roi d'Espagne impossibles.}}
{\textsc{- Le comte de Stanhope part de Madrid pour Londres, par
Paris.}} {\textsc{- Fin des nouvelles étrangères.}}

~

La république de Venise, alors attaquée par les Turcs, engage l'empereur
à la secourir en vertu des traités et de l'alliance qu'il avait
contractée avec elle\,; il déclara donc la guerre au Grand Seigneur, et
le roi d'Espagne uniquement par zèle pour la religion joignit sa flotte
à celle de la république, si à propos, que ce secours préserva Corfou de
l'extrême danger de tomber sous la puissance des infidèles. L'année
1717, le roi d'Espagne mit encore une flotte en mer. Elle paraissait
destinée à porter des secours aux Vénitiens, mais elle fut employée à
enlever la Sardaigne à l'empereur\,; le prétexte de cette invasion fut
que ce prince manquait à la parole qu'il avait donnée de retirer ses
troupes de la Catalogne et de l'Île de Majorque. L'entreprise faite en
Sicile en 1718 était la suite de l'invasion de la Sardaigne, et fondée
sur le même prétexte. Le comte de Koenigseck était alors à Paris
ambassadeur de l'empereur auprès du roi. On peut juger de l'attention
d'un ministre éclairé et vigilant, attentif à pénétrer quelle part la
France pouvait avoir à l'entreprise des Espagnols, aussi bien qu'à
découvrir les résolutions qu'elle prendrait pour ou contre le duc de
Savoie. Le bruit commun était que ce prince avait signé un traité
d'alliance offensive et défensive avec l'empereur\,; mais son
ambassadeur à Paris l'ignorait, et quoiqu'il ne pût douter que le régent
ne fût très disposé à cultiver une intelligence parfaite avec
l'empereur, Koenigseck, soupçonnant l'intention des ministres, était
scandalisé du peu de joie que la cour avait fait paraître à la nouvelle
de la conclusion de la paix entre l'empereur et le Turc. Le désir de cet
ambassadeur était alors d'obtenir comme récompense de ses services la
vice-royauté de Sicile, persuadé que la possession de cette île
retournerait immanquablement à l'empereur.

Les mouvements du parlement contre la banque de Law attiraient dans ces
conjonctures l'attention particulière des ministres étrangers résidents
à Paris. Celui d'Espagne continuait ses conférences secrètes avec le duc
d'Ormond, et ce dernier, suivant le génie ordinaire des bannis, espérait
toujours, et se promettait des révolutions sûres en Angleterre, si les
mécontents du gouvernement étaient soutenus\,; il demandait, pour les
secourir avec succès, douze vaisseaux, six mille hommes de débarquement,
quinze mille fusils, des armes pour mille dragons, et des munitions de
guerre\,; il ajoutait à ces demandes l'assurance d'une retraite en
quelque ville de Biscaye, et son projet était d'y faire passer le roi
Jacques pour le conduire ensuite comme en triomphe en Angleterre, où il
assurait que les deux tiers de la nation se déclareraient pour lui. Le
duc d'Ormond, caché aux environs de Paris et changeant souvent de
demeure, comptait d'attendre ainsi la réponse d'Espagne à ces mêmes
propositions, que le cardinal Acquaviva avait déjà communiquées au
cardinal Albéroni, et qui depuis avaient été portées à Madrid par un
capitaine de vaisseau Anglais nommé Camok, dévoué au roi Jacques.

L'objet d'exciter ou de fomenter des troubles en Angleterre n'était pas
le principal dont Cellamare fût alors occupé\,; il savait qu'Albéroni
donnait sa première attention à la suite des mouvements qu'il espérait
qu'on verrait incessamment éclore en France, article qui touchait le
plus sensiblement le roi et la reine d'Espagne et leur premier ministre.
C'était, par conséquent, l'affaire que Cellamare suivait avec le plus de
soin, et qu'il croyait traiter avec le plus de secret, quoique M. le duc
d'Orléans fût bien informé de ses démarches et des noms de ceux qui
croyaient faire ou avancer leur fortune en s'engageant imprudemment avec
le ministre d'une, cour étrangère. L'ambassadeur d'Espagne envoyait à
Madrid, sous le nom de Pattes, le rapport des conférences qu'il avait
avec eux, et par le récit favorable qu'il leur faisait des réponses de
Leurs Majestés Catholiques, il s'appliquait à fortifier de plus en plus
les engagements imprudents qu'ils avaient déjà pris. Cellamare
n'oubliait rien aussi pour faire entendre au roi son maître la nécessité
de les appuyer, si ce prince voulait maintenir leur bonne volonté et les
mettre en état d'agir avec succès. La France était alors dans une
profonde paix, et comme on ne voyait nulle apparence d'une guerre
prochaine, plusieurs officiers sans emploi désiraient de passer au
service d'Espagne. Cellamare, persuadé qu'il était de l'intérêt de son
maître d'avoir à son service non seulement des officiers, mais encore un
corps de troupes françaises, et sachant qu'Albéroni avait dessein de
lever jusqu'au nombre de huit mille étrangers, lui proposa de former un
corps de soldats qu'on lèverait aisément en France, et qu'on enrôlerait
dans les régiments wallons et irlandais que le roi d'Espagne avait
actuellement à son service. Il y avait en effet lieu de croire que
plusieurs officiers se trouvant sans emploi ne demanderaient pas mieux
que d'en obtenir en Espagne, et Cellamare en était persuadé par les
demandes fréquentes de ceux qui s'adressaient à lui pour être reçus dans
le service d'Espagne. Le chevalier Folard était du nombre\,; mais il
pouvait auparavant faire ses conditions et ne pas passer comme
aventurier.

L'ambassadeur connaissait ses talents et lui rendit justice, ajoutant
seulement qu'il battait beaucoup la campagne, et que par cette raison il
avait jugé à propos d'éluder sa proposition. On pouvait encore, suivant
l'avis de l'ambassadeur, former quelques nouveaux régiments français,
et, pour cet effet, recevoir sur la frontière de Catalogne, d'Aragon et
de Navarre, ceux qui se présenteraient pour s'enrôler sous des
commandants de leur nation. Outre les avantages du service, il s'en
trouverait encore d'autres par rapport à la politique. Cellamare ne
laissait pas d'être effrayé de la difficulté qu'il prévoyait à puiser
des eaux hors de leur source, et vaincre les obstacles que le
gouvernement de France apporterait à de telles levées. Comme on reçut
alors la nouvelle de l'entrée des troupes d'Espagne dans Messine, il
assura Albéroni que toute la nation française s'était réjouie de cet
événement, qu'on ne parlait à Paris que de la gloire du roi d'Espagne,
et qu'il serait à souhaiter que le régent eût les mêmes sentiments, au
moins intérieurement\,; mais Cellamare, persuadé que Son Altesse Royale
en était bien éloignée, ramassait avec soin tous les discours de la
ville, comptant faire sa cour en Espagne en rendant compte exact non
seulement de ce qui était, mais encore des faits qu'on supposait contre
le gouvernement du régent.

Les nouveautés introduites dans l'administration des finances,
l'établissement de la banque, les projets qu'on attribuait à Law, l'abus
que le régent avait fait de toutes ces nouveautés, l'opposition du
parlement, une espèce de guerre entre les arrêts du conseil et les
arrêts de cette compagnie pour les annuler, donnaient lieu d'ajouter foi
à toutes les funestes prédictions qui se débitaient d'une guerre
intestine et prochaine non seulement dans la capitale, mais encore dans
toutes les parties du royaume. Cellamare recueillait avec joie les faux
avis et les étudiait avec d'autant plus de soin qu'il croyait, en les
donnant à Albéroni, effacer l'impression que ce premier ministre
pourrait avoir prise contre le neveu du cardinal del Giudice, tel que
l'était Cellamare. Il grossissait donc tous les objets et croyait donner
une bonne nouvelle à Madrid en assurant que le régent faisait venir
autour de Paris plusieurs régiments\,; que l'ordre était donné aux
gardes ainsi qu'aux mousquetaires de se tenir prêts. Il espérait en même
temps que la république de Hollande refuserait d'entrer dans le traité
qui se négociait à Londres, pour former l'alliance dont il était
question depuis longtemps entre l'empereur, la France, l'Angleterre et
les États généraux\,; traité dans lequel on s'efforçait inutilement de
faire entrer le roi d'Espagne, et dont la négociation était le sujet de
l'envoi du sieur de Nancré à Madrid de la part de la France, et de celui
du comte de Stanhope, de la part de l'Angleterre.

Mais pendant que l'ambassadeur d'Espagne se flattait de tant de vaines
espérances, le traité de la quadruple alliance négocié à Londres fut
signé premièrement dans cette ville le 2 août, et ensuite à Vienne et à
la Haye, le roi d'Espagne ayant refusé d'y entrer, nonobstant les vives
instances qui lui en avaient été faites. Le prétexte de cette quadruple
alliance était premièrement de réparer les troubles apportés, soit à la
paix conclue à Bade en septembre 1714, soit à la neutralité d'Italie
établie par le traité d'Utrecht en 1713. Une paix solide, bien affermie
et soutenue par les principales puissances de l'Europe était le but que
celles qui contractaient semblaient se proposer, et pour y parvenir,
elles réglaient entre elles non seulement de quelle manière la France
accomplirait parfaitement la démolition du port et des fortifications de
Dunkerque promise par le traité d'Utrecht\,; comment elle détruirait le
canal de Mardick dont l'Angleterre regardait l'ouverture comme une
infraction faite à ce même traité. On disposait de plus de différents
États souverains situés en Italie\,; on donnait des successeurs aux
princes qui possédaient encore les mêmes États, lorsque ces possesseurs
actuels viendraient à mourir\,; en sorte que, suivant ces dispositions,
nul des changements qui renouvellent ordinairement les guerres ne
troublerait désormais le repos de l'Europe. Mais ce grand objet du bien
et de la tranquillité publique n'était pas le seul de tant de mesures
prises en apparence pour en assurer le repos\,: un intérêt particulier
et trop à découvert était le ressort de cette alliance.

Le régent, persuadé que, si malheureusement le roi encore enfant était
enlevé aux désirs comme aux voeux que ses sujets formaient pour sa
conservation, Son Altesse Royale aurait peine à faire valoir les
renonciations exigées du roi d'Espagne, elle avait jugé que le meilleur
moyen d'en assurer la validité était de se préparer des défenseurs tels
que le roi d'Angleterre et les États généraux pour soutenir la
disposition faite à Utrecht pour le bien de la paix, mais contre toutes
les lois et la constitution inviolable du royaume. Celles de la
Grande-Bretagne n'avaient pas été moins violées en faveur de la maison
de Hanovre, et le prince appelé en Angleterre au préjudice du roi
légitime n'avait pas moins à craindre une révolution qui le priverait
quelque jour, lui ou sa postérité, du trône qu'il avait usurpé. Ainsi,
l'intérêt réciproque unissant le roi d'Angleterre avec le régent, tous
deux consentirent sans peine à garantir, l'un le maintien des
renonciations du roi d'Espagne à la succession de France, l'autre
l'ordre de succession à la couronne établi nouvellement en Angleterre au
préjudice du véritable roi de la Grande-Bretagne et de ses héritiers
légitimes. On peut ajouter à ces grands intérêts l'ambition du
négociateur employé par M. le duc d'Orléans, qui de valet d'un docteur
de Sorbonne était parvenu, par ses intrigues et ses fourberies, à
devenir précepteur de ce prince, et que le caprice de la fortune ou
plutôt la juste colère de Dieu, éleva depuis à l'archevêché de Cambrai
et à la dignité de cardinal, enfin au poste de premier ministre, avec
une telle autorité que, lorsqu'il mourut au mois d'août 1723, Son
Altesse Royale avait lieu de craindre, le pouvoir excessif dont elle
voyait clairement qu'il était prêt d'abuser contre son maître et son
bienfaiteur.

Les États généraux des Provinces-Unies entrèrent sans peine dans les
vues de la France et de l'Angleterre, et les ministres Anglais en
Hollande parurent d'autant plus contents de Morville, nouvellement
arrivé à la Haye en qualité d'ambassadeur de France, qu'ils le
trouvèrent soumis à leurs conseils, pour ne pas dire à leurs ordres,
conduite très différente de celle de Châteauneuf son prédécesseur, dont
ils avaient souvent éprouvé la contrariété et qu'ils avaient enfin fait
révoquer. Beretti, ambassadeur d'Espagne, travaillait inutilement à
traverser les ministres de France et d'Angleterre. Ses instances, qu'il
exaltait à Madrid, étaient tournées en ridicule à la Haye et ne
persuadaient personne. Il interprétait à sa fantaisie les démarches les
plus indifférentes, et si chacune des Provinces-Unies, si les États
étaient assemblés, ou si chaque province délibérait séparément, Beretti
se persuadait, et voulait se persuader, que c'était pour l'intérêt du
roi son maître, et s'attribuait l'honneur et l'utilité prétendue des
résolutions prises sans qu'il y eût la moindre part. Pendant qu'il se
vantait des heureux effets de sa vigilance, de son industrie et du
crédit de ses amis eu Hollande, la signature du traité d'alliance
démentit les éloges qu'il donnait à tant de démarches qu'il supposait
avoir faites. Il est vrai que le traité ne fut pas si aisément signé,
nonobstant le désir unanime et l'intérêt qui pressait les parties
contractantes de le conclure au plus tôt\,; mais plus cette conclusion
était ardemment désirée, plus on voulait aussi prévoir et prévenir
toutes les difficultés capables d'ébranler une alliance qui devait être
le fondement solide de la paix générale de l'Europe. Comme il est plus
aisé de prévoir le mal que d'empêcher qu'il n'arrive, on voulut, avant
de conclure le traité, remédier, à chacun des inconvénients qui se
présentaient à la pensée. La multitude en était si grande, que le
résident de l'empereur à la cour d'Angleterre prétendit savoir que les
ministres du roi d'Angleterre avaient apposé vingt-quatre fois leurs
signatures et leurs cachets aux articles de ce traité, secrets et
séparés. Monteléon, sans témoigner d'inquiétude de cette alliance,
demanda qu'elle lui fût communiquée, et s'adressa pour cela à Craggs,
alors secrétaire d'État\,: il répondit à l'ambassadeur d'Espagne que,
s'il en voulait voir tous les articles, il ne lui en serait fait aucun
mystère\,; que, s'il voulait en informer le roi d'Espagne, le comte de
Stanhope, encore à Madrid, le communiquerait à Sa Majesté Catholique
sans la moindre réserve. Monteléon répondit que, n'ayant jamais eu de
curiosité de ce qui s'était traité et conclu, il rendrait simplement
compte au cardinal Albéroni de la réponse du secrétaire d'État
d'Angleterre.

Le traité de la quadruple alliance n'était pas le seul sujet d'aigreur
qu'il y eût alors entre l'Espagne et l'Angleterre\,: Les esprits
s'aliénèrent de part et d'autre à l'occasion des prérogatives que
l'Espagne avait accordées à l'Angleterre pour son commerce aux Indes.
Les Espagnols se plaignaient de l'abus que les Anglais faisaient des
conditions avantageuses que l'Angleterre avait exigées et obtenues par
le traité d'Utrecht\,; et réciproquement, on prétendait en Angleterre
que ces conditions n'étaient pas exécutées de la part de l'Espagne,
principalement en ce qui regardait le privilège de la traite des nègres,
en sorte que le préjudice, que le commerce des sujets de la
Grande-Bretagne en souffrait, aigrissait une nation également superbe et
avare, plus facile à blesser qu'il n'est facile de l'adoucir. Les
Hollandais eurent en même temps sujet de craindre un trait de la
vengeance du czar, aussi facile au moins que les Anglais à s'irriter, et
plus difficile à calmer. Le résident de Hollande auprès de lui avait dit
imprudemment, et même écrit, que le czarowitz était mort de mort
violente, et que le penchant à la révolte était général en Moscovie. Le
czar, offensé d'un pareil discours, avait fait arrêter ce résident sans
égard au droit des gens, et s'était emparé de tous ses papiers. Non
content d'une expédition si violente et si contraire à la sûreté dont un
ministre étranger doit jouir, ce prince demanda satisfaction à la
république de Hollande, déclarant qu'il ferait arrêter tous les
vaisseaux Hollandais allant dans les ports de Suède, et qu'il
retiendrait en prison le résident de la république, jusqu'à ce qu'il eût
nommé ceux dont il tenait de tels avis.

Quoique l'esprit de paix dût régner dans les principaux États de
l'Europe, après avoir essuyé de longues guerres, dont le temps et le
repos étaient les seuls moyens de réparer les dommages, la défiance
réciproque entre les princes était telle, qu'aucun d'eux ne s'assurait
sur la bonne foi de ceux même que l'intérêt commun et le désir de la
paix engageaient à se secourir. Ainsi le roi de Sicile se défiait et de
la France et de l'Angleterre, et différait d'accepter les assistances
qui lui étaient offertes de part et d'autre, s'il souscrivait au projet
que ces deux puissances lui proposait. Il ne voulait s'expliquer que
lorsqu'il serait établi dans la possession tranquille du royaume de
Sicile, et que l'Espagne aurait restitué la Sardaigne à l'empereur. En
vain l'Angleterre le menaçait de lui refuser tout secours s'il ne
s'expliquait. Il se plaignait également de la France et de l'Angleterre.
Ses ministres prétendaient que le régent manquait aux promesses qu'il
avait faites à leur maître, et Provane attribuait cette variation aux
vues secrètes que le régent conservait encore de marier une des
princesses ses filles au prince de Piémont. Toutefois, dans la suite de
la négociation, le roi d'Angleterre voulut que son ministre à Vienne
appuyât celle du marquis de Saint-Thomas auprès de l'empereur, à
condition que, si le roi d'Espagne rejetait le projet de paix, et qu'il
fût accepté par le duc de Savoie, ce prince aurait, en considération de
son acceptation, la Sardaigne qui lui serait cédée absolument sans la
condition de retour en faveur de l'Espagne, et de plus encore quelques
autres avantages que ses alliés lui procureraient. La république de
Hollande soumise aux décisions de l'Angleterre, et désirant néanmoins
pour son intérêt particulier, de conserver les bonnes grâces du roi
d'Espagne, amusait l'ambassadeur de ce prince, en l'assurant que toutes
les provinces étaient persuadées qu'il était de l'intérêt du public et
des particuliers de se conserver les bonnes grâces de Sa Majesté
Catholique, et que certainement ce serait suivant cette maxime que les
États généraux se conduiraient. Celle de Beretti était de faire sa cour
au premier ministre, et par conséquent de lui donner les nouvelles et
les assurances qui étaient le plus à son goût. Craignant cependant que
l'événement ne démentît ce qu'il avait écrit, il faisait observer que la
conduite de la république était amphibie, et que sa politique tendait à
ne pas déplaire au roi d'Espagne, en même temps qu'elle voulait éviter
avec beaucoup de soin de se rendre suspecte aux autres puissances.

Le roi d'Espagne comptait alors sur les projets de Charles XII, roi de
Suède, et sur les grands armements que ce héros du nord faisait pour les
exécuter. L'envoyé de Suède en Hollande assura Beretti que son maître
avait sur pied soixante-quinze mille hommes effectifs et vingt-deux
navires armés\,; mais l'argent lui manquait, et c'était le seul secours
qu'il eût à demander à ses alliés pour l'aider à faire la guerre au roi
d'Angleterre. Le roi d'Espagne, ayant les mêmes vues, promettait au roi
de Suède trente mille hommes et trente vaisseaux de guerre\,; et c'était
par une diversion si puissante que Sa Majesté Catholique pouvait avec
raison se flatter de renverser et d'anéantir les projets de la quadruple
alliance, surtout s'il était possible d'engager le czar et le roi de
Prusse à s'unir avec le roi de Suède pour exécuter de concert de si
grands projets. Ils causaient peu d'inquiétude en Angleterre. Le roi de
Sicile continuait ses instances à cette cour pour en obtenir des
secours. Elle pressait, de son côté, le régent de faire cause commune
avec elle pour sauver la Sicile et la garantir de l'invasion totale de
la part des Espagnols. Stairs, ministre d'Angleterre, appuyé par les
lettres de l'abbé Dubois, prêt à partir de Londres pour retourner en
France, agissait fortement, et ne désespérait pas d'obtenir, au moins
comme préliminaire, que Son Altesse Royale fit mettre au moins pour
quelque temps à la Bastille le, duc d'Ormond, qui pour lors était à
Paris.

Les deux ambassadeurs d'Espagne, l'un à Londres, l'autre à la Haye,
pensaient bien différemment sur l'état où les affaires se trouvaient
alors. Le premier déplaisait et s'était rendu suspect au premier
ministre du roi son maître en représentant ce qu'il voyait des forces de
l'Angleterre et des intentions de son roi et de ses ministres. Beretti
ne déplaisait pas moins par l'exagération continuelle de son crédit en
Hollande et des services importants selon lui qu'il y rendait au roi son
maître. Monteléon pressait Albéroni de, terminer le plus tôt qu'il
serait possible l'affaire de Sicile. Il ne cessait de représenter
combien les moments étaient chers et les conséquences fâcheuses de
laisser traîner cette expédition. Le duc de Savoie sollicitait vivement
des secours de la part de l'empereur, et demandait au roi d'Angleterre
d'ordonner à l'amiral Bing de passer incessamment à Naples avec
l'escadre Anglaise qu'il commandait. Il n'y avait pas lieu de douter que
ce prince n'obtînt des demandes si conformes aux sentiments comme à
l'inclination de la cour de Vienne et de celle d'Angleterre. L'unique
moyen d'en empêcher l'effet était que le roi d'Espagne souscrivît au
traité de la quadruple alliance. Monteléon l'avait toujours conseillé et
désiré, et ses instances réitérées le rendaient odieux à Albéroni, dont
il était obligé de combattre les vues et les raisonnements,
principalement pendant le séjour que le comte de Stanhope faisait encore
à Madrid, et l'événement de la négociation étant regardé comme une
décision certaine ou de l'affermissement de la paix, ou d'une rupture
ouverte entre l'Espagne et l'Angleterre. L'envoyé de Savoie à Londres,
pressant vivement les ministres d'Angleterre de garantir les États
possédés par le roi son maître, obtint enfin l'assurance du secours que
l'amiral Bing lui donnerait. Il était parti du port Mahon le 22 juillet
pour se rendre à Naples, déclarant que, s'il rencontrait la flotte
d'Espagne, il ne pourrait pas se résoudre à demeurer simple spectateur
des entreprises des Espagnols, par conséquent faire une mauvaise figure
à la tête d'une flotte Anglaise.

L'abbé Dubois, partant de Londres pour retourner en France, n'oublia
rien pour persuader le ministre de Savoie de ce qu'il avait fait et
voulu faire pour le service de ce prince, et les protestations de son
zèle allèrent au point de contredire à Londres ce que M. le duc
d'Orléans avait dit à Paris, en sorte que l'envoyé de Savoie en conclut
qu'il fallait qu'il y eût nécessairement un mensonge, soit de la part de
Son Altesse Royale qu'on ne devait pas en soupçonner, soit de la part de
son agent en Angleterre. Le même accident arrivait souvent dans un temps
où les traités fréquents qu'on était curieux de négocier se
contredisaient assez ordinairement, et que des gens peu instruits des
affaires politiques désiraient pour leur intérêt personnel d'être
employés à les administrer.

L'incertitude des événements de Sicile et du succès qu'aurait
l'entreprise des Espagnols suspendait toute décision de la négociation
du comte de Stanhope à Madrid. L'intention d'Albéroni était de la
prolonger et de la régler suivant les nouvelles qu'il recevrait
d'Italie, persuadé que d'ailleurs on ne pouvait être trop en garde
contre les artifices de la cour de Vienne, dont toute la conduite,
disait-il, était un tissu de momeries, et dans l'opinion qu'il n'y avait
à la cour d'Espagne que des stupides et des insensés. Peut-être ne
pensait-il pas mieux de ceux qui se mêlaient en France des affaires les
plus importantes\,; car en parlant du maréchal d'Huxelles, il disait
«\,que ce pauvre vieux maréchal avançait comme un trait de politique
profonde que, la supériorité de l'empereur étant bien connue, il fallait
travailler à l'augmenter.\,» Raisonnement et conséquence qu'il était
assez difficile de comprendre. Un ministre éclairé et pénétrant, tel que
l'était Stanhope, comprit aisément et dès les premières conférences
qu'il eut avec Albéroni, que, malgré les protestations de ce cardinal de
son adversion pour la guerre et du désir d'établir une paix solide, on
ne devait cependant attendre de sa part aucune facilité pour un
accommodement. Albéroni, rejetant sur son maître tout ce qu'il y avait
d'odieux dans le désir de la guerre, protestait qu'il n'en était pas
l'auteur, et que, s'il en était le maître, la paix régnerait bientôt
dans toute l'Europe, qu'il ne désirait pour le roi d'Espagne aucune
augmentation d'États en Italie parce que, gouvernant bien son royaume
renfermé dans son continent, et possédant les Indes, il serait beaucoup
plus puissant qu'en dispersant ses forces. Oran, suivant la pensée
d'Albéroni, valait mieux que l'Italie. Leurs Majestés Catholiques
avaient cependant pris à coeur, les affaires d'Italie, et ne
souffriraient pas que l'empereur se rendît maître d'une si belle partie
de l'Europe. À ces vues politiques, le cardinal ajoutait que la paix et
l'amitié des puissances voisines était ce qui convenait le mieux à ses
intérêts particuliers et personnels. Sans cette union, il était
impossible de soutenir la forme de gouvernement qu'il avait établie en
Espagne, et qui ne subsisterait pas toujours quand il aurait abandonné
la pénible administration des affaires\,; mais la paix, l'amitié des
voisins convenaient à l'Espagne, et il n'importait pas moins aux autres
puissances d'empêcher que l'empereur s'agrandît en Italie\,; et c'était
pour elles une fausse politique que celle de s'opposer à un monarque
qui, loin d'agir par un motif d'ambition, employait contre ses propres
intérêts les forces de son royaume pour établir et maintenir un juste
équilibre en Europe. Stanhope et Nancré vécurent dans une grande
intelligence pendant que tous deux demeurèrent à Madrid, et se
communiquèrent réciproquement le peu de succès de leur négociation.

Quelque temps auparavant le roi d'Angleterre avait fait remettre au
baron de Riperda, ambassadeur de Hollande, une somme de quatorze mille
pistoles pour les donner au cardinal Albéroni de la part de Sa Majesté
Britannique, et jamais Albéroni n'en avait entendu parler. Il envoya
chercher Riperda pour approfondir cette affaire, dont on ignore quel a
été l'éclaircissement. Si le cardinal reçut cette somme, elle fut mal
employée\,; car il témoigna toujours la même opposition à la quadruple
alliance, aussi peu goûtée dans les cours qui n'y furent pas invitées
qu'elle l'avait été à la cour d'Espagne. Celle de Rome crut avoir lieu
de craindre l'association des deux premiers princes de l'Europe avec les
principales puissances protestantes, et, voyant la guerre à ses portes,
elle ne savait à qui recourir, ni de quel côté elle attendrait du
secours contre les événements qui intéresseraient infailliblement les
États de l'Église.

Le roi d'Espagne, mécontent du pape, et qu'Albéroni ne cessait d'animer
contre Sa Sainteté, avait ordonné aux réguliers ses sujets, étant à
Rome, d'en sortir, et de retourner, en leur pays. Sa Sainteté leur
avait, au contraire, défendu de se retirer, et fait la même défense à
tout Espagnol, sous peine d'excommunication et autres peines
spirituelles. On devait s'attendre que le roi d'Espagne défendrait
réciproquement à ses sujets d'obéir aux ordres du pape, et {[}que{]},
par conséquent, les deux cours, loin de se concilier, s'aigriraient
chaque jour de plus en plus. Sa Sainteté n'espérait guère de meilleures
dispositions de la part de la France, malgré le grand nombre de
partisans que Rome avait dans le clergé du royaume, et leur empressement
à rechercher et à pratiquer tous les moyens de lui plaire, aux dépens
même de la paix et de l'union de l'Église\,; ils croyaient s'avancer,
obtenir des grâces particulières, parvenir à ces dignités supérieures,
si capables d'éblouir et d'aveugler les ecclésiastiques\,; dignités qui
ne dépendent que du pape, et que les rois, contre leur propre intérêt,
ont admises et honorées en leurs cours. Ces vues éloignées et
différentes, suivant le rang de ceux dont elles faisaient l'objet, les
animaient également à chercher et employer les moyens de plaire à
Rome\,; les uns comme zélés défenseurs des maximes et de l'autorité du
saint-siège\,; d'autres, d'un plus bas étage, comme espions, et capables
de donner, soit au nonce, soit aux autres agents, des avis importants de
ce qu'il se passait en France, et des résolutions que le pape devait
prendre pour maintenir ses droits et son autorité. Ils y avait longtemps
qu'ils pressaient le pape de, etc.

Dans ces circonstances, le roi tint son lit de justice. Il n'y fut pas
question des affaires de Rome, mais des prétentions des princes
légitimés, et de leurs contestations avec les princes du sang.
L'opposition du parlement à la création d'un garde des sceaux ne fut pas
écoutée\,; il fallut obéir et enregistrer les lettres. L'autorité du
régent, attaquée par le parlement, parut par le succès qu'il avait eu au
lit de justice, et les étrangers le considérèrent comme un premier fruit
des traités que ce prince avait signés dernièrement.

La résistance du roi d'Espagne à souscrire à ces mêmes traités fit
échouer son entreprise en Sicile, et de plus, elle lui coûta la perte de
sa flotte. Elle était partie du Phare de Messine le 9 août, à quatre
heures du matin, pendant que l'armée espagnole continuait de bombarder
la citadelle de Messine. Cette flotte, fuyant celle d'Angleterre
commandée par l'amiral Bing, faisait voile vers Catane. Le lendemain 10
août, les vaisseaux Anglais arrivèrent à deux heures après midi dans le
Phare, et, le vent manquant à la flotte d'Espagne, ils l'atteignirent à
douze lieues de Syracuse, vers le cap Passaro. Les meilleurs vaisseaux
espagnols très maltraités, étaient encore poursuivis par Bing le 11 août
à midi, et six ou sept navires Anglais, demeurés en arrière pour
attaquer l'arrière-garde espagnole, avaient déjà coulé bas quatre
navires, cinq autres étaient sautés en l'air à la vue de Syracuse, et
l'amiral Bing avait envoyé dire à Maffeï, vice-roi de l'île, que le
reste de la flotte était réduit à ne pouvoir ni fuir ni se défendre. La
nouvelle de la défaite de la flotte d'Espagne ne causa nulle peine au
régent\,; au contraire, l'union était si bien cimentée entre Son Altesse
Royale et le roi d'Angleterre que l'un et l'autre réciproquement se
regardaient comme intéressés dans la même cause.

Stairs se réjouissait de la faiblesse du parti opposé au régent, de
l'union du gouvernement, et de penser que Son Altesse Royale ne serait
plus exposée à l'infinité d'inconvénients et de dangers intestins dont
elle était sans cesse environnée\,; enfin que ses amis au dehors
pourraient se reposer sur lui et compter sur sa conservation. Peut-être
Stairs écrivait et disait ce qu'il ne pensait pas, et souhaitait, au
contraire, de voir le feu de la division embraser tout le royaume\,;
mais il était loin d'avoir cette satisfaction. L'esprit de paix régnait
en France, celui de sédition en était banni, et ceux qui connaissaient
le bonheur d'y voir la tranquillité maintenue désiraient seulement que
Dieu voulût donner à la régence l'esprit de conseil, et de profiter des
avantages que la France et l'Espagne trouveraient à bien vivre ensemble
dans une parfaite intelligence. C'était ainsi que s'expliquait
l'ambassadeur d'Espagne à Paris\,; mais secrètement il agissait
différemment. Appliqué à l'exécution ponctuelle des commissions secrètes
qu'il recevait, il assurait Albéroni de ses soins à bien instruire ceux
qu'il nommait les artisans, comment et quand ils devaient faire leurs
travaux. Il tâchait, disait-il, de les tenir contents et disposés à
servir de bon coeur. Il gardait entre ses mains les matériaux qu'il
recevait du cardinal, et s'en servirait seulement dans les temps
convenables. Lorsqu'il serait nécessaire d'envoyer de nouveaux modèles,
il ne le ferait pas par la voie ordinaire, parce qu'elle était
évidemment pernicieuse.

Les mémoires secrets et nécessaires pour achever le récit de ce qui
s'est passé de particulier dans le reste de l'année 1718 manquent depuis
la fin du mois d'août\,; on sait seulement par les écrits publics que le
comte de Stanhope, après avoir espéré un heureux succès de sa
commission, cessa de se flatter lorsque les nouvelles arrivèrent à
Madrid, où il était, de la destruction de la flotte espagnole par les
Anglais dans les mers de Sicile, et de l'arrivée des galions à Cadix.
Albéroni avait demandé pour conditions de l'accession du roi d'Espagne
au traité de la quadruple alliance, que la propriété des îles de
Sardaigne et de Sicile fût laissée et cédée au roi catholique moyennant
un équivalent pour la Sicile que l'empereur donnerait au duc de Savoie
dans le Milanais\,; que, de plus, Sa Majesté Catholique eût à satisfaire
les princes d'Italie sur toutes leurs prétentions\,;

À rappeler les troupes qu'elle faisait alors marcher en Italie\,;

Fixer le nombre de celles qu'il y maintiendrait à l'avenir\,;

S'engager à ne se pas mêler de la succession de la Toscane\,;

Renoncer à toute prétention sur les fiefs de l'empire.

La flotte d'Angleterre venait de causer trop de dommages à l'Espagne
pour la laisser tranquillement séjourner dans la Méditerranée. Albéroni
exigeait donc que le roi d'Angleterre eût à la rappeler incessamment.

Ces demandes soutenues avec opiniâtreté et si contraires aux
instructions données au comte de Stanhope, aussi bien qu'aux pouvoirs
qu'il avait reçus du roi son maître, l'obligèrent à partir d'une cour où
désormais il ne pouvait que perdre son temps. Il prit donc congé du roi
et de la reine d'Espagne, et retournant en France le 26 août\,; il
trouva que le traité de la quadruple alliance entre la France,
l'empereur, l'Angleterre et la Hollande, avait été signé le 22 du même
mois de la même année 1718.

\hypertarget{chapitre-xiii.}{%
\chapter{CHAPITRE XIII.}\label{chapitre-xiii.}}

1718

~

{\textsc{J'ai pris tout ce qui est d'affaires étrangères de ce que M. de
Torcy m'a communiqué.}} {\textsc{- Matériaux indiqués sur la suite de
l'affaire de la constitution, très curieux par eux-mêmes et par leur
exacte vérité.}} {\textsc{- Religion sur la vérité des choses que je
rapporte.}} {\textsc{- Réflexions sur ce qui vient d'être rapporté des
affaires étrangères.}} {\textsc{- Albéroni et Dubois.}} {\textsc{- État
de la France et de l'Espagne avant et après les traités d'Utrecht.}}
{\textsc{- Fortune d'Albéroni.}} {\textsc{- Caractère du roi et de la
reine d'Espagne.}} {\textsc{- Gouvernement d'Albéroni.}} {\textsc{-
Court pinceau de M. le duc d'Orléans et de l'abbé Dubois, des degrés de
sa fortune.}} {\textsc{- Perspective de l'extinction de la maison
d'Autriche, nouveau motif à la France de conserver la paix et d'en
profiter.}} {\textsc{- Considération sur l'Angleterre, son intérêt et
ses objets à l'égard de la France, et de la France au sien.}} {\textsc{-
Folle ambition de l'abbé Dubois de se faire cardinal, dès ses premiers
commencements.}} {\textsc{- Artifices de Dubois pour se rendre seul
maître du secret de la négociation d'Angleterre, et son perfide manège à
ne la traiter que pour son intérêt personnel, aux dépens de tout
autre.}} {\textsc{- Dubois vendu à l'Angleterre et à l'empereur pour une
pension secrète de quarante mille livres sterling et un chapeau, aux
dépens comme éternels de la France et de l'Espagne.}} {\textsc{-
Avantages que l'Angleterre en tire pour sa marine et son commerce, et le
roi d'Angleterre pour s'assurer de ses parlements.}}

~

On a vu en plusieurs endroits de ces Mémoires que j'y ai toujours parlé
sur les affaires étrangères d'après Torcy. Il les avait administrées
avec son père et son beau-père, puis seul après eux jusqu'à la mort du
roi\,: ensuite il en avait conservé le fil par le secret de la poste
dont il était demeuré directeur, puis devenu surintendant. Quelque part
qu'il plût au régent de m'y donner dans son cabinet depuis que le
conseil de régence n'était plus devenu qu'une forme à qui tout était
dérobé en ce genre jusqu'à conclusion résolue, ma mémoire n'aurait pu
m'en fournir la suite et les dates parmi tant de faits croisés, avec
l'exactitude et la précision nécessaire si je n'avais eu d'autre
secours. Torcy s'était fait à mesure un extrait de toutes les lettres
qu'il continua jusqu'à la fin d'août 1718, et c'est un dommage
irréparable, et que je lui ai bien reproché depuis, de ne l'avoir pas
continué tant qu'il a eu les postes, que nous verrons que le cardinal
Dubois lui arracha en 1721. On y verrait jusque-là dans ces trois années
bien des choses curieuses qui demeureront ensevelies, et tout le manège
et l'intrigue de la chute d'Albéroni et du double mariage d'Espagne.
Torcy m'a prêté ses extraits\,; c'est d'où j'ai puisé le détail du récit
que j'ai donné depuis la mort du roi, de la suite et du détail des
affaires étrangères. Je les ai abrégées et n'ai rapporté que le
nécessaire. Mais ce qui s'est passé en 1718 m'a paru si curieux et si
important que j'ai cru devoir non pas abréger ni extraire, mais
m'astreindre à copier fidèlement tout et n'en pas omettre un mot\,; j'ai
seulement laissé tout ce qui regarde la constitution, comme j'avais fait
dans les extraits que j'ai abrégés sur les années précédentes, parce que
je me suis fait une règle ainsi que je l'ai dit plusieurs fois, de ne
point traiter cette matière\,; mais j'ai conservé la copie exacte et
entière de tous les extraits des lettres que M. de Torcy m'a prêtés et
qu'il a faits, dans lesquels on pourra justifier tout ce que je rapporte
des affaires étrangères, et voir, de plus, ce qui regarde la suite de
l'affaire de la constitution, de laquelle je n'ai rien dit, et où on
verra des horreurs à faire dresser les cheveux à la tète de la part du
nonce Bentivoglio, des cardinaux de Rohan et de Bissy, et des principaux
athlètes de cette déplorable bulle, de tout ordre et de toute espèce,
avec une suite, une exactitude, une précision qui ôtent tout moyen de
s'inscrire en faux contre la moindre circonstance de tant de faits
secrets et profonds et presque tous plus scélérats et plus abominables
les uns que les autres, et le parfait contradictoire en plein en
droiture, candeur, douceur, vérité, et trop de patience et de mesure
dans le cardinal de Noailles et les principaux qui ont figuré de ce côté
avec lui et sous lui.

Quoique la netteté, le coulant, la noblesse et la correction du style
que j'ai copié, fasse par son agrément et sa douceur sauter aux yeux sa
différence d'avec le mien, je n'ai pas voulu toutefois laisser ignorer
au lecteur, si jamais ces Mémoires en trouvent, ce qui n'est pas de moi,
par le mépris que j'ai pour les plagiaires, et lui donner en même temps
la confiance la plus entière dans ce que je rapporte des affaires
étrangères, en lui expliquant d'où je l'ai pris pour suivre fidèlement
la règle que je me suis imposée, de ne rien exposer dans ces Mémoires
qui n'ait passé par mes mains ou sous mes yeux, ou qui ne soit tiré des
sources les plus certaines que je nomme en exprimant de quelle manière
je l'y ai puisé. Reste maintenant, avant que de reprendre le fil des
événements de cette année 1718, à faire quelques courtes réflexions sur
ce qu'on vient de voir des affaires étrangères. Ce n'est pas que
j'ignore le peu de place et la rareté dont les réflexions doivent
occuper qui fait et qui lit des histoires, et plus encore des Mémoires,
parce qu'on veut suivre les événements, et que la curiosité ne soit pas
interrompue pour ne voir que des raisonnements souvent communs,
insipides et pédants, et ce que celui qui écrit veut donner à penser de
son esprit et de son jugement. Ce n'est point aussi ce qui me conduit à
donner ici quelques réflexions, mais l'importance de la matière et les
suites funestes de l'enchaînement qu'elles ont formé, sous lesquelles la
France gémira peut-être des siècles.

J'ai souvent ouï dire au P. de La Tour, général de l'Oratoire, qui était
un homme de beaucoup de sens, d'esprit et de savoir, et d'une grande
conduite et piété, qu'il fallait que les hommes fussent bien peu de
chose devant Dieu, à considérer, dans la plupart des empereurs romains,
quels maîtres il avait donnés à l'univers alors connu, et en comparaison
desquels les plus puissants monarques de ces derniers siècles n'égalent
pas en puissance et en étendue de gouvernement les premiers officiers
que ces empereurs employaient sous eux au gouvernement de l'empire. Si,
de ces monarques universels, on descend à ceux qui leur ont succédé dans
la suite des siècles et dans les diverses divisions qu'a successivement
formées la chute de l'empire romain, on y retrouvera en petit la même
réflexion à faire, et on s'étonnera de qui les divers royaumes sont
devenus la proie et le jouet sous les rois particuliers. Je ne sais si
c'est que le spectacle frappe plus que la lecture, mais rien ne m'a fait
tant d'impression que ce qui vient d'être exposé sur les affaires
étrangères. On y voit les deux plus puissantes monarchies gouvernées,
par deux princes entièrement différents, dont le très différent
caractère s'aperçait pleinement en tout avec une supériorité d'esprit
transcendante, et très pénétrante dans l'un des deux, également conduits
comme deux enfants par deux hommes de la lie du peuple, qui font
tranquillement et sans obstacle chacun leur maître et la monarchie qu'il
domine, l'esclave et le jouet de leur ambition particulière contre les
intérêts les plus évidents des deux princes et des deux monarchies. Deux
hommes sans la moindre expérience, sans quoi que ce soit de
recommandable, sans le plus léger agrément personnel, sans autre appui
chacun que de soi, qui ne daignent ou ne peuvent cacher leur intérêt et
leur ambition à leur maître, ni leur fougue et leurs fureurs, et qui
presque dès le premier degré ne ménagent personne, et ne montrent que de
la terreur. Un court détail trouvera son application importante.

Il faut premièrement se rappeler ce qui s'est passé dans la guerre qui a
suivi l'avènement de Philippe V à la couronne d'Espagne, les funestes
revers qui ont ébranlé les trônes du grand-père et du petit-fils, les
circonstances affreuses et déplorables où ils se sont trouvés de ne
pouvoir ni soutenir la guerre davantage ni obtenir la paix\,; l'un prêt
à passer la Loire pour se retirer vers la Guyenne et le Languedoc,
l'autre à s'embarquer avec sa famille pour les Indes\,; l'énormité et la
mauvaise foi des propositions faites à Torcy dans la Haye, et à nos
plénipotentiaires à Gertruydemberg\,; enfin les miracles de Londres, qui
tirèrent ces deux monarques des abîmes par la paix d'Utrecht, et
finalement par celles de Rastadt et de Bade. C'est ce qui se voit dans
ces Mémoires pour les événements et pour les pourparlers de paix et les
traités, par les copies des Pièces originales que Torcy, par qui tout a
passé, m'a prêtées, et dont j'ai parlé plus d'une fois\,; on les
trouvera dans les Pièces. D'une situation si forcée et si cruelle, des
conditions affreuses ardemment désirées pour en sortir du temps du
voyage de Torcy à la Haye, et de la négociation de Gertruydemberg à
l'état où la paix d'Utrecht et sa suite de Rastadt et de Bade ont laissé
la France et l'Espagne, la disproportion est telle que de là mort à la
vie. Tout conspirait donc à persuader la jouissance d'un si grand bien,
et si peu espérable\,; d'en profiter pour la longue réparation des deux
royaumes, que de si grands et si longs revers avaient mis aux abois, et
se garantir cependant avec sagesse de tout ce qui pouvait troubler cette
heureuse tranquillité, et exposer l'épuisement où on était encore à de
nouveaux hasards. La droite raison, le simple sens commun démontrent que
ce but était ce qui devait faire l'entière et la continuelle application
du gouvernement de la France et de l'Espagne. Celle-ci à la vérité
n'était pas comme la France en paix avec toute l'Europe.

L'empereur seul, séparé à son égard de toutes les autres puissances,
n'avait consenti qu'à une longue trêve, mais aussi bien cimentée qu'une
paix, et pour les conditions et pour les garanties. L'Espagne en
jouissait paisiblement, en attendant que les temps et les conjonctures
devinssent assez favorables pour convertir cette trêve en une paix. Le
roi d'Espagne ne pensait qu'à en jouir cependant, et à réparer son
royaume et ses forces. Il y était également convié par le dedans qui en
avait grand besoin, et par le dehors où il n'aurait pu compter que sur
la France, qui sentait ses besoins et qui voulait conserver la paix\,;
qui de plus avait perdu Louis XIV\,; qui était ainsi tombée dans une
minorité\,; enfin qui, au lieu d'un grand roi, aïeul paternel de
Philippe V, était gouvernée par un régent, que M\textsuperscript{me} des
Ursins avait, comme on l'a vu, brouillé avec lui jusqu'à un degré peu
commun entre princes, et sur lequel il n'était rien moins qu'apparent
qu'il pût compter. C'est dans cette situation qu'Albéroni parvint à être
le maître absolu de l'Espagne, par les prompts degrés qu'on a vu que la
fortune lui dressa. Le néant de son extraction, ses premiers
commencements auprès du duc de Vendôme, ses moeurs, sa vie, son
caractère, la disgrâce de ce prétendu héros qui le conduisit à sa suite
en Espagne, le fatal hasard du second mariage de Philippe V à la fille
de son maître, la chute de la princesse des Ursins, l'usage qu'il sut
faire d'être sujet et après ministre de Parme en Espagne, et de l'exacte
clôture où la politique de M\textsuperscript{me} des Ursins avait su
enfermer et accoutumer Philippe V, en sorte qu'il n'eût qu'à continuer
ce qu'il trouvait en usage, et qui ne lui était pas moins nécessaire
qu'il avait été utile à celle qui l'avait établie\,: Gibraltar, demeuré
aux Anglais pour n'avoir jamais voulu laisser approcher Louville, arrivé
à Madrid de la part du régent, comme on l'a vu ici en son temps, est un
fatal monument de cette exacte et jalouse clôture\,; tout cela a été
raconté en son temps avec exactitude, en sorte qu'il n'y a qu'à s'en
souvenir ou le repasser dans ces Mémoires sans en rien retoucher ici.

Albéroni trouve un roi solitaire, enfermé, livré par son tempérament au
besoin d'une épouse, dévot et dévoré de scrupules, peu mémoratif des
grands principes de la religion et abandonné à son écorce, timide,
opiniâtre, quoique doux et facile à conduire, sans imagination,
paresseux d'esprit, accoutumé à s'abandonner à la conduite d'un autre,
commode au dernier point pour la certitude de ne parler à personne ni de
se laisser approcher, ni encore moins parler par personne et pour la
sécurité de ne songer jamais à autre femme qu'à la sienne, glorieux
pourtant, haut et touché de conquérir et d'être compté en Europe, et, ce
qui est incompréhensible, sans penser avec de la valeur à sortir de
Madrid, et content de la vie du monde la plus triste, la plus la même
tous les jours, sans penser jamais à la varier ni à donner le moindre
amusement à son humeur mélancolique que dès battues, et tête à tête avec
la reine en chemin, et dans la feuillée destiné à tirer sur les bêtes
qu'on y faisait passer\,; une reine pleine d'esprit, de grâces\,; de
hauteur, d'ambition, de volonté de gouverner et de dominer sans partage,
à qui rien ne coûta pour s'y porter et s'y maintenir\,; hardie,
entreprenante, jalouse, inquiète, ayant toujours en perspective le
triste état des reines veuves d'Espagne, pour l'éviter à quelque, prix
que ce pût être, et voulant pour cela à quelque prix que ce fût aussi,
former à un de ses fils un État souverain, et à plus d'un dans la
suite\,; haïssant les Espagnols à visage découvert, abhorrée d'eux de
même, et n'ayant de ressource que dans les Italiens qu'elle avança tant
qu'elle put\,; de conseil et de confiance qu'au sujet et au ministre de
Parme qui l'était allé chercher et était venu avec elle\,; d'ailleurs
ignorant toutes choses, élevée dans un grenier du palais de Parme par
une mère austère, qui ne lui donna connaissance de rien, et ne la laissa
voir ni approcher de personne, et passée de là sans milieu dans la
\emph{spelonque} du roi d'Espagne où elle demeura tant qu'elle vécut,
sans communication avec qui que ce pût être\,; réduite ainsi à ne voir
que par les yeux d'Albéroni, le seul à qui elle fût accoutumée par le
temps du voyage, le seul à qui elle crût pouvoir se confier par sa
qualité de sujet et de ministre de Parme en Espagne, le seul dont elle
voulût se servir pour gouverner le roi et la monarchie, parce que,
n'ayant point d'État, il ne pourrait se passer d'elle, ni jamais à son
avis lui manquer ni lui porter ombrage. Tel fut le champ offert et
présenté à Albéroni pour travailler à sa fortune sans émule et sans
contradicteur. Telle fut la source de sa sécurité à tout entreprendre au
dedans et au dehors, à s'enrichir dans les ténèbres d'une administration
difficile à découvrir, impossible à révéler, à se rendre redoutable,
sans nulle sorte d'égard pour ne trouver aucun obstacle à commettre sans
ménagement le roi et la reine d'Espagne pour son cardinalat avec les
plus grands et les plus scandaleux éclats, et depuis pour l'archevêché
de Séville, qui fut le commencement de son déclin, enfin à engager, une
guerre folle contre l'empereur malgré toute l'Europe et abandonné de
toute l'Europe\,; et l'empereur, au contraire, puissamment secouru et
aidé vigoureusement par la France, l'Angleterre et la Hollande. De là
les efforts prodigieux pour soutenir une guerre si follement entreprise,
pour se rendre nécessaire et se maintenir dans le souverain pouvoir et
dans les moyens de s'enrichir, et de pêcher en eau trouble dans les
marchés, les fournitures, les entreprises de toutes les sortes dont il
disposait seul\,; de là cette opiniâtreté funeste à rejeter tout
accommodement que l'Espagne n'eût osé espérer, et qui établissait un
fils de la reine dès lors en Italie avec promesse et toute apparence de
le voir bientôt en possession des États de Parme et de Toscane par les
offices de l'Angleterre sur l'empereur, laquelle voulait éviter une
guerre qui la privait du commerce de l'Espagne et des Indes.

Ces efforts qui achevèrent d'épuiser inutilement l'Espagne, anéantirent
sa marine qui venait de se relever, d'où cette couronne souffrit après,
par un enchaînement de circonstances, un préjudice accablant dans les
Indes, dont il est bien à craindre qu'elle ne puisse jamais se relever.
C'est ce qu'opéra le tout-puissant règne de ce premier ministre en
Espagne, quoique fort court, qui après avoir insulté toute l'Espagne,
traité Rome indignement, offensé toutes les puissances de l'Europe et
très dangereusement le régent de France en particulier, contre lequel il
voulut soulever tout le royaume, chassé enfin honteusement d'Espagne,
s'en trouva quitte après quelques mois d'embarras\,; et à l'abri de sa
pourpre et de ses immenses richesses qu'il s'était bien gardé de placer
en Espagne, figura à Rome dans les premiers emplois, et s'y moqua
pleinement de la colère de toute l'Europe qu'il avait excitée contre
lui, et méprisa impudemment celle de ses maîtres, qui de la plus vile
poussière l'avaient élevé jusqu'au point de ne pouvoir lui nuire ni se
venger de lui. Cette leçon toutefois, quelque forte qu'elle fût, ni la
connaissance qu'eut le roi d'Espagne de tous les criminels et fous
déportements d'Albéroni, après qu'il l'eut chassé, et que les langues
furent déliées, ne fut pas capable de le dégoûter de l'abandon à un
seul. La paresse et l'habitude furent plus fortes\,; on vit encore en
Espagne quelque chose, sinon de plus violent, au moins de plus ridicule
dans le règne du Hollandais qui succéda à la toute-puissance d'Albéroni,
et qui, chassé à son tour, en fut combler la mesure chez les corsaires
de Barbarie, où, faute d'autre retraite, il alla finir ses jours\,; mais
rien ne put déprendre Philippe V du faux et ruineux repos d'un premier
ministre, dont il n'a pu se passer jusqu'à sa mort, au grand malheur de
sa réputation et de sa monarchie.

La France ne fut pas plus heureuse, et ce qui est incompréhensible, sous
un prince à qui rien ne manqua pour le plus excellent gouvernement,
connaissances de toutes les sortes, connaissance des hommes, expérience
personnelle et longue tandis qu'il ne fut que particulier\,; traverses
les moins communes, réflexions sur le gouvernement des différents pays,
et surtout sur le nôtre\,; mémoire qui n'oubliait et qui ne confondait
jamais\,; lumières infinies\,; nulle passion incorporelle, et les autres
sans aucune prise sur son secret, ni sur son administration\,;
discernement exquis, défiance extrême, facilité surprenante de travail,
compréhension vive, une éloquence naturelle et noble, avec une justesse
et une facilité incomparable de parler en tout genre\,; infiniment
d'esprit, et je l'ai dit ailleurs, un sens si droit et si juste, qu'il
ne {[}se{]} serait jamais trompé si en chaque affaire et en chaque chose
il avait suivi la première lumière et la première appréhension de son
esprit. Personne n'a jamais eu tant ni une si longue expérience que lui
et l'abbé Dubois\,; personne aussi ne l'a-t-il jamais si bien connu\,;
et quand je me rappelle ce qu'il m'en a dit dans tous les temps de ma
vie et dans le moment même qu'il le déclara premier ministre, et encore
depuis, il m'est impossible de comprendre ce qu'il en a fait, et
l'abandon total où il s'est mis de lui. On en verra encore d'étranges
traits dans la suite. Il est inutile de reprendre ici ce qu'on a vu dans
ces Mémoires de l'infime bassesse, des serviles et abjects
commencements, de l'esprit, des moeurs, du caractère de l'abbé Dubois,
des divers degrés qui le tirèrent de la boue, et de sa vie jusqu'à la
régence de M. le duc d'Orléans. On l'a même conduit plus loin\,: on a
exposé son profond projet d'arriver à tout par Stanhope et par
l'Angleterre\,; le commencement de son exécution par son adresse et ses
manèges à infatuer le régent du besoin réciproque que le roi
d'Angleterre et lui auraient l'un de l'autre\,; enfin ces Mémoires l'ont
conduit à Hanovre et à Londres, et c'est ce fil qu'il ne faut pas perdre
de vue depuis son commencement. Voilà donc M. le duc d'Orléans
totalement livré à un homme de néant, qu'il connaissait pleinement pour
un cerveau brûlé, étroit, fougueux outre mesure\,; pour un fripon livré
à tout mensonge et à tout intérêt, à qui homme vivant ne s'était jamais
fié, perdu de débauches, d'honneur, de réputation sur tous chapitres,
dont les discours et les manières n'avaient rien que de rebutant, et qui
sentait le faux en tout et partout à pleine bouche, un homme enfin qui
n'eut jamais rien de sacré\,; à qui a connu l'un et l'autre, cette
fascination ne peut paraître qu'un prodige du premier ordre, augmenté
encore par les avertissements de toutes parts.

La France n'avait besoin que d'un gouvernement sage au dedans pour en
réparer les vastes ruines, et au dehors pour conserver la paix\,; son
épuisement et la minorité, qui est toujours un état de faiblesse, le
demandaient. Il n'était pas temps de songer à revenir sur les cessions
que les traités de Londres et d'Utrecht avaient exigées, et nulle
puissance n'avait à former de prétentions contre elle. Outre la
nécessité de profiter de la paix pour la réparation des finances et de
la dépopulation du royaume, une perspective éloignée y engageait
d'autant plus qu'on devait être instruit par la faute de la guerre
terminée par la paix de Ryswick, uniquement due à l'ambition personnelle
de Louvois, qui l'avait allumée, comme il a été remarqué dans ces
Mémoires. On aurait dû prévoir alors l'importance de se tenir en force,
de profiter de l'ouverture de la succession d'Espagne, que la santé
menaçante de Charles II faisait regarder comme peu éloignée, et en
attendant ne pas alarmer l'Europe par l'ambition de faire les armes à la
main un électeur de Cologne et rétablir un roi d'Angleterre, et
s'affaiblir par une longue guerre, dont deux ans de paix entre le traité
de Ryswick et la mort de Charles II n'avaient pas eu le temps de
remettre la France, ni de refroidir cette formidable alliance de toute
l'Europe contre elle, qui se rejoignit comme d'elle-même après la mort
de Charles II. L'empereur se trouvait le dernier mâle de la maison
d'Autriche avec peu ou point d'espérance de postérité\,; son âge et sa
santé pouvaient faire espérer une longue vie. Mais il n'en est pas des
États comme des hommes\,; quelque longue que pût être la vie de
l'empereur il {[}était{]} toujours certain que la France le survivrait.
Comme elle n'avait point de prétentions à former à sa mort sur l'empire,
ni sur pas un de ses États, elle n'avait pas à craindre la même jalousie
qui lui avait attiré toute l'Europe sur les bras à l'ouverture de la
succession d'Espagne. Il était néanmoins de son plus pressant intérêt
d'empêcher que des cendres de la maison d'Autriche il n'en naquît une
autre aussi puissante, aussi ennemie, aussi dangereuse, qu'elle avait
éprouvé celle-là depuis Maximilien et les rois catholiques, et, pour
l'empêcher, profiter des occasions d'alliance d'une part, et se mettre
intérieurement en état de l'autre de soutenir utilement des alliés pour
diviser cette puissance, en morcelant les nombreux États de la maison
d'Autriche.

Il n'est pas besoin d'un grand fonds de politique pour comprendre
l'intérêt en ce cas-là tout opposé de l'Angleterre. Sa position la rend
inaccessible à l'invasion étrangère quand elle-même n'y donne pas les
mains. Elle est riche et puissante par son étendue, et beaucoup plus par
son commerce\,; mais elle ne peut figurer par elle-même que sur mer et
par la mer. Sa jalousie contre la France est connue depuis qu'elle en a
possédé plus de la moitié, et qu'elle n'y a plus rien. Par terre elle ne
peut donc rien, et sa ressource ne peut être que dans l'alliance d'une
grande puissance jalouse aussi de la France, et terrienne, qui ait en
hommes et en pays de quoi lui faire la guerre, et qui manquant d'argent,
et n'en pouvant tirer que de l'Angleterre, ait tout le reste. C'est ce
que l'Angleterre a trouvé dans la maison d'Autriche, dont toutes deux
ont si bien su profiter\,; et c'est pour cela même qu'il n'était pas
difficile de prévoir l'intérêt pressant de l'Angleterre, de voir
renaître des cendres de la maison d'Autriche, le cas arrivant, une autre
puissance non moins grande ni moins redoutable dont elle pût faire le
même usage contre la France qu'elle avait fait de la maison d'Autriche.
Ce n'est pas qu'en attendant il ne fût à propos de bien vivre avec
l'Angleterre comme avec tout le reste de l'Europe, mais toutefois sans y
compter jamais, et beaucoup moins se livrer à elle et se mettre dans, sa
dépendance\,; mais se conduire avec elle honnêtement, sans bassesse, et
intérieurement la considérer toujours comme une ennemie naturelle qui ne
se cachait pas depuis longues années de vouloir détruire notre commerce,
et de s'opposer avec audace et acharnement à tout ce que la France a de
temps en temps essayé de faire sur ses propres côtes en faveur de sa
marine, dont tout ce qui s'est sans cesse passé à l'égard de Dunkerque
est un bel exemple et une grande leçon, tandis qu'à nos portes ils font,
à Jersey et à Guernesey, tous les ports, les fortifications et les
magasins qu'il leur plaît, et cela de l'aveu du cardinal Fleury qui leur
permit d'en prendre tous les matériaux en France, plus proche de ces
dangereuses îles que l'Angleterre\,; complaisance qui ne se peut
imaginer. Il fallait donc dans un royaume flanqué des deux mers, et qui
borde la Manche si près, et vis-à-vis de l'Angleterre, et un royaume si
propre au plus florissant commerce par la position et par l'abondance de
ses productions de toutes espèces nécessaires à la vie, porter toute son
application à relever la marine et à se mettre peu à peu en état de se
faire considérer à la mer, et non l'abandonner à l'Angleterre, et la
mettre ainsi en état de porter l'alarme à son gré tout le long de nos
côtes, et le joug anglais, à menacer et envahir nos colonies. Il fallait
exciter l'Espagne au même soin et au même empressement d'avoir une bonne
marine, et se mettre conjointement en état de ne plus recevoir la loi de
l'Angleterre sur la mer dans le commerce, ni à l'égard des colonies
françaises et des États espagnols, delà les mers, et pour cela favoriser
sous main toute invasion, tout trouble domestique en Angleterre le plus
qu'il serait possible, et il n'y avait lors qu'à le vouloir, ce que le
ministère d'Angleterre sentait parfaitement. C'était là le vrai, le
grand, le solide intérêt de la France malheureusement ce n'était pas
celui de l'abbé Dubois. Le sien était tout contraire, et c'est celui-là
qui a prévalu.

On a vu en son temps dans ces Mémoires qu'après que le chevalier de
Lorraine et le marquis d'Effiat se furent servis de lui pour faire
consentir son maître à son mariage avec la dernière fille du roi et de
M\textsuperscript{me} de Montespan, l'ambition lui fit tourner la tête
au point de se flatter qu'il méritait les plus grandes récompenses, et
que, peu content d'une bonne abbaye qu'il eut sur-le-champ, il demanda
et il obtint une audience du roi dans laquelle il eut l'audace de lui
demander sa nomination au cardinalat, dont le roi fut si surpris et si
indigné qu'il lui tourna le dos sans lui répondre, et ne l'a jamais pu
souffrir depuis. Si dès lors il osa penser au chapeau, il n'est pas
surprenant qu'il y ait visé du moment qu'il a vu jour à s'introduire
dans les affaires par l'Angleterre, et qu'il n'y ait tout sacrifié pour
y parvenir, comme il est aussi très apparent qu'il n'a imaginé les
moyens de s'introduire dans les affaires par l'Angleterre, que pour y
trouver ceux qu'il espérait le pouvoir conduire à ce but si
anciennement, quoique si follement désiré.

Possesseur de l'esprit de son maître, il le fut jusqu'à ne lui en
laisser pas la liberté et à l'entraîner par un ascendant
incompréhensible à son avis, à son sentiment, et pour tout dire à sa
volonté, souvent tous contraires par le bon esprit et le grand sens, la
justesse et la perspicacité de ce prince. Il devint ainsi seul maître de
toute la machine des affaires étrangères, dont le maréchal d'Huxelles
n'eut plus dès lors qu'une vaine écorce, le conseil des affaires
étrangères encore moins, et les serviteurs les plus confidents du régent
quelques légères participations rares par morceaux et par simples
récits, courts, destitués de tout raisonnement, encore plus de
consultation la plus légère. Dubois donc n'eut plus d'entraves, et sut
profiter de sa liberté pour en user dans son entier, et se délivrer de
tout instrument qui l'eût pu contraindre, il voulut aller à Hanovre,
puis à Londres, et n'avoir avec son maître qu'une correspondance
immédiate, pour sevrer Huxelles son conseil et tout autre de toute
connaissance de sa négociation, dont il ne leur laissa voir que les
dehors, et il choisit pour la remise de ses lettres au régent et du
régent à lui un homme dont il était sûr, qui espérait tout par lui,
qu'il trompa quand il n'en eut plus que faire, selon sa coutume, et
qu'il fit enfin chasser, parce que cet homme s'avisa de se plaindre de
lui. C'était Nocé, dont j'ai parlé quelquefois, et dont j'ai fait
connaître le caractère, pour qui M. le duc d'Orléans avait de tout temps
de l'amitié et de la familiarité, mais qu'il connaissait assez pour se
contenter de lui faire du bien, et de l'amusement de sa conversation et
de ses fougues souvent justes et plaisantes, car il avait beaucoup
d'esprit et de singularité, mais pour se garder de l'employer dans
aucune sorte d'affaire. C'est ce que l'abbé Dubois cherchait\,; il y
trouvait de plus un homme fort accoutumé au prince, et en état de lui
rendre fidèlement compte de la mine, de l'air et du visage du régent,
quand il lui rendait ses lettres, et qu'il recevait de sa main celles
qu'il devait envoyer en réponse. Ces réponses, excepté pour l'écorce ou
pour les choses que l'un et l'autre ne se souciaient pas de cacher,
comme il s'en trouve toujours dans le cours d'une négociation longue,
étaient toujours de la main de M. le duc d'Orléans. Il avait la vue fort
basse\,; elle peinait surtout en écrivant, et regardait son papier de si
près que le bout de sa plume s'engageait toujours dans sa perruque aussi
n'écrivait-{[}il{]} jamais de sa main que dans la nécessité et le plus
courtement qu'il lui était possible. C'était encore un artifice de
l'abbé Dubois, et pour n'admettre personne entre lui et son maître dans
le secret de sa négociation, et pour profiter de cette difficulté
d'écrire qui jointe de la paresse en ce genre, et à cet ascendant que le
prince avait laissé prendre à l'abbé Dubois sur lui, opérait une
contradiction légère et un raisonnement étranglé quand il arrivait que
le régent n'était pas de son avis, et qui par l'opiniâtreté, la fougue
et l'ascendant de Dubois finissait toujours par se rendre à ce qu'il
voulait.

Dans cette position, l'infidèle ministre ne pensa plus qu'à profiter de
la conjoncture, faire en effet tout ce qui conviendrait à l'Angleterre,
le faire de manière qu'à lui seul elle en eût toute l'obligation, lui
bien faire sentir ses forces auprès de son maître, et faire son marché
aux dépens du régent et du royaume. Il n'ignorait pas que le commerce
était la partie la plus sensible à l'Angleterre\,; il ne pouvait ignorer
sa jalousie du nôtre. Il l'avait déjà bien servie en persuadant au
régent de laisser tomber la marine pour ôter toute jalousie au roi
Georges, dans ce beau système tant répété du besoin réciproque qu'ils
avaient de l'union la, plus intime, de concert avec Canillac séduit par
les hommages de Stairs, et par le duc de Noailles que cela soulageait
dans sa finance et qui fit toujours bassement sa cour à Dubois. Je ne
fais que remettre ces choses qui se trouvent expliquées en leur temps.
Il fallait continuer cet important service, mais ce n'était pas tout\,;
il fallait l'étendre jusque sur l'Espagne, si la folie de son premier
ministre se roidissait jusqu'au bout à ne vouloir point de paix, ou à
prétendre de l'empereur des conditions qu'il ne voudrait jamais passer,
ce qui était la même chose. Rien de si essentiel à l'Angleterre pour se
saisir de tout commercé et pour se fonder solidement dans les Indes\,;
et c'était de l'abbé Dubois uniquement que l'Angleterre dépendait pour
arriver à un si grand but, tel qu'elle n'aurait jamais osé l'espérer.
Dubois n'oublia rien aussi pour en bien persuader Georges et ses
ministres, qui en sentirent enfin la vérité. Dubois ainsi les amena à
son point, et ce point était double, de l'argent et le chapeau. Le
premier n'était pas difficile, on donne volontiers un écu pour avoir un
million\,; mais, l'autre n'était pas en la puissance immédiate des
ministres d'Angleterre\,; aussi les laissa-t-il longtemps dans la
détresse de deviner par où le prendre, quoiqu'il se montrât en prise. Il
voulait échauffer la volonté par le besoin, afin de ne trouver plus de
difficulté dès qu'il jugerait qu'il pourrait s'expliquer. Le roi
d'Angleterre était bien plus occupé de ses établissements d'Allemagne
que des intérêts de la couronne à laquelle il était parvenu. Brême et
Verden à attacher à ses États personnels par les lois et les formes de
l'empire, était son objet principal. L'empereur, fort occupé de la paix
du nord dont il voulait être le dictateur, se sentait des entraves qui
l'empêchaient de donner cette investiture à Georges qui soupirait après
et qui faisait tout pour l'empereur dans la négociation de sa paix avec
l'Espagne, avec peu de retenue de montrer toute sa partialité. Moins
l'empereur était prêt à satisfaire Georges sur un point si désiré, plus
il le caressait d'ailleurs dans le besoin qu'il en avait contre
l'Espagne, pour se maintenir dans toutes ses possessions d'Italie. Il
avait entièrement gagné les ministres hanovriens de Georges, par des
bienfaits et par des espérances dont il pouvait disposer à leur égard
dans l'empire. Il s'était acquis aussi les ministres Anglais qui
sentaient le goût et l'intérêt de leur maître. Dans cette situation
réciproque, le roi d'Angleterre et ses ministres pouvaient compter
d'obtenir de l'empereur tout ce qui ne lui coûtait rien, et l'empereur
lui-même désirait ces occasions faciles de s'attacher l'Angleterre de
plus en plus\,; il pouvait tout à Rome, et on a vu dans l'extrait des
lettres sur les affaires étrangères de cette année jusqu'à quel point
Rome et le pape tremblaient devant lui, et jusqu'à quel point encore il
savait profiter et abuser de cette frayeur démesurée. Demander et
obtenir était pour lui même chose\,; il avait réduit le pape à craindre
qu'il ne dédaignât et qu'il ne renvoyât même les chapeaux qu'il lui
avait accordés.

L'abbé Dubois, parfaitement au fait de l'intérieur de toutes ces cours,
voulait obliger Georges et ses ministres d'employer l'autorité de
l'empereur à lui obtenir un chapeau. Dans la passion ardente de l'avoir,
il ne lui parut pas suffisant d'y disposer efficacement les Anglais par
ses complaisances qui ne tendaient qu'à ce but, s'il ne se rendait
encore assez agréable à l'empereur dans le cours de la négociation, non
seulement pour éviter un obstacle personnel à la demande des Anglais en
sa faveur, mais encore pour se rendre ce prince assez favorable, pour
être bien aise de faire ce plaisir à Georges et à ses ministres, et
s'acquérir à si bon marché celui qui disposait de la France et qui
d'avance lui aurait montré de la bonne volonté dans la négociation.
C'est ce qui y fit toute l'application de l'abbé Dubois, ce qui la
tourna toute au gré des Anglais et à celui de l'empereur, aux dépens de
la France et de l'Espagne, et ce qui lui valut une pension secrète de
l'Angleterre, de quarante mille livres sterling, qui est une somme
prodigieuse, mais légère pour disposer de la France, et, comme on verra
bientôt, ce chapeau si passionnément désiré, que, pressé par Georges et
par ses ministres, et par les bons offices de Penterrieder, témoin des
facilités de Dubois pour l'empereur dans la négociation, ce prince lui
fit donner peu après par son autorité sur le pape. Le sceau de cette
grande affaire fut l'engagement de faire déclarer la France contre
l'Espagne, non seulement par des subsides et par souffrir que la flotte
Anglaise, non contente de secourir la Sicile, poursuivît et détruisît
l'espagnole qui avait tant coûté, mais encore de faire porter les armes
françaises dans le Guipuscoa, moins pour y faire les faciles conquêtes
qu'elles y firent et qu'on ne pouvait se proposer de conserver, que pour
anéantir à forfait la marine d'Espagne en brûlant ses vaisseaux dans ses
ports et ses chantiers, ses amas et ses magasins au port du Passage,
comme nous le verrons, pour donner champ libre à la marine d'Angleterre,
la délivrer de la jalousie de celle d'Espagne, lui assurer l'empire de
toutes les mers, et lui faciliter celui des Indes en y détruisant celui
de l'Espagne.

Qui ne croirait que l'Angleterre ne dût être satisfaite d'un marché
avantageux pour elle jusqu'au prodige, et si promptement exécuté, comme
on le verra bientôt en son lieu\,? Mais le ministère Anglais l'ayant si
belle, était trop habile pour en demeurer là\,; il n'avait pas donné une
pension si immense au maître des démarches de la France, pour n'en pas
tirer un parti proportionné, tant que durerait la toute puissance du
ministre de France qui la recevait. Nous verrons bientôt qu'ils en
tirèrent la complaisance non seulement de souffrir tranquillement que
les escadres Anglaises assiégeassent celles d'Espagne dans les ports
espagnols des Indes, un an durant et plus, les y fissent périr, y
empêchassent tout secours et fissent cependant tout le commerce des
Indes par contrebande\,; mais encore de tirer de la France tous les
subsides suffisants à l'armement et à l'entretien des escadres
Anglaises, tant qu'il leur plut de maintenir ce blocus qui se fit tout
entier à nos dépens en toutes les sortes je dis en toutes les sortes
pour la réputation, parce que de la France à l'Espagne rien ne pouvait
avoir moins de prétexte ni être plus odieux, et à la fin de plus
difficile à cacher, puisque l'intérêt des Anglais à tenir toujours
brouillées les deux branches royales de la maison de France, n'avait
garde d'être de moitié du secret que le régent du moins aurait voulu
garder et qu'il crut vainement exiger d'eux\,; et parce que rien n'était
plus ruineux à l'Espagne et à la France que de livrer les mers, tout le
commerce et le nouveau monde aux Anglais. Cette ruine ne sera pas sitôt
réparée\,; les Espagnols sont encore aujourd'hui aux prises avec les
Anglais pour le commerce des Indes, et par l'affaiblissement que leur a
causé l'abbé Dubois, ils ont vainement acheté quelques intervalles de
paix par les plus avantageuses concessions de commerce et
d'établissements aux Anglais, qui ne s'en sont fait que des degrés et
des titres pour en obtenir davantage, et qui enfin, les armes à la main,
se servent de tout ce qu'ils ont acquis sur le commerce et sur les
établissements\,; pour s'y accroître de plus en plus, et devenir enfin
les seuls maîtres de toutes les mers et de tout le commerce, et dominer
l'Espagne dans les Indes, tandis que sa faible marine n'a pu se relever
de tant de pertes et que la nôtre est enfin anéantie\,; l'un et l'autre
par l'intérêt et le fait de Dubois.

C'étaient sans doute de grands coups, incomparables pour la grandeur
solide de l'Angleterre aux dépens de toutes les nations de l'Europe, de
celles surtout dont elle avait le plus à craindre et le plus de
jalousie, la française et l'espagnole, avec l'avantage encore de les
brouiller et de les diviser. Mais le grappin une fois attaché sur celui
qui peut tout, qui attend un chapeau pour lequel il brûle de désir
depuis tant d'années, et qui a tous les ans quarante mille livres
sterling à recevoir, dont il n'ose rien montrer, et dont il redoute au
contraire jusqu'au soupçon, qui craint, par conséquent, des
retardements, et plus encore une soustraction dont il n'oserait ouvrir
la bouche, il n'est rien qu'on ne puisse obtenir. Georges et ses
ministres, peu satisfaits de tout ce qu'ils tiraient de la France, et
incapables de se dire\,: \emph{C'est assez}, voulurent se donner les
moyens de se rendre pour longues années les maîtres de leurs parlements.
La liste civile et ce qu'ils savaient prendre d'ailleurs leur servait à
gagner des élections dans les provinces et des voix dans le parlement\,;
mais elle ne suffisait pas pour s'en rendre maîtres par le très grand
nombre, et leurs manèges dans le parlement y trouvaient souvent des
résistances importunes et même quelquefois de fâcheuses oppositions,
dont l'expérience les rendait retenus à entreprendre. Ils se servirent
donc du bénéfice du temps, et se firent donner par la France de
monstrueux subsides, et en outre des sommes prodigieuses où tout notre
argent alla\,; et c'est de cette source que la cour d'Angleterre a tiré
les trésors qui lui ont servi, et lui servent peut-être encore, tant
l'amas en a été grand, à faire élire qui elle a voulu dans les
provinces, et faire voter à son gré dans les divers parlements avec
cette supériorité presque totale de voix qui anéantit enfin la liberté
de la nation, et rend le roi despotique sous le masque de quelques
mesures et de quelques formes, et la politique de ne tenir pas ferme sur
tout ce qui ne l'intéresse pas précisément.

\hypertarget{chapitre-xiv.}{%
\chapter{CHAPITRE XIV.}\label{chapitre-xiv.}}

1718

~

{\textsc{Gouvernement de M. le Duc, mené par M\textsuperscript{me} de
Prie, à qui l'Angleterre donne la pension de quarante mille livres
sterling du feu cardinal Dubois.}} {\textsc{- Époque et cause de la
résolution de renvoyer l'infante et de marier brusquement le roi.}}
{\textsc{- Gouvernement du cardinal Fleury.}} {\textsc{- Chaînes dont
Fleury se laisse lier par l'Angleterre.}} {\textsc{- Fleury sans la
moindre teinture des affaires, lorsqu'il en saisit le timon.}}
{\textsc{- Aventure dite d'Issy.}} {\textsc{- Fleury parfaitement
désintéressé sur l'argent et les biens.}} {\textsc{- Lui et moi nous
nous parlons librement de toutes les affaires.}} {\textsc{- Avarice
sordide de Fleury, non pour soi, mais pour le roi, l'État et les
particuliers.}} {\textsc{- Fleury met sa personne en la place de
l'importance de celle qu'il occupe, et en devient cruellement la dupe.}}
{\textsc{- Walpole, ambassadeur d'Angleterre, l'ensorcelle.}} {\textsc{-
Trois objets des Anglais.}} {\textsc{- Avarice du cardinal ne veut point
de marine, et, à d'autres égards, encore pernicieuse à l'État.}}
{\textsc{- Il est personnellement éloigné de l'Espagne, et la reine
d'Espagne et lui brouillés sans retour jusqu'au scandale.}} {\textsc{-
Premiers ministres funestes aux États qu'ils gouvernent.}} {\textsc{-
L'Angleterre ennemie de la France, à force titres anciens et nouveaux.}}
{\textsc{- Intérêt de la France à l'égard de l'Angleterre.}} {\textsc{-
Perte radicale de la marine, etc., de France et d'Espagne\,; l'empire de
la mer et tout le commerce passé à l'Angleterre, fruits du gouvernement
des premiers ministres de France et d'Espagne, avec bien d'autres
maux.}} {\textsc{- Comparaison du gouvernement des premiers ministres de
France et d'Espagne, et de leur conseil, avec celui des conseils de
Vienne, Londres, Turin, et de leurs fruits.}} {\textsc{- Sarcasme qui
fit enfin dédommager le chapitre de Denain des dommages qu'il a
soufferts du combat de Denain.}}

~

Dubois mort ne laissa de regrets qu'à l'Angleterre. Les subsides établis
continuèrent les quatre mois que M. le duc d'Orléans survécut. M. le
Duc, bombardé en sa place par Fleury, ancien évêque de Fréjus, et
précepteur du roi, qui compta faire de ce prince plus que borné un
fantôme de premier ministre, et devenir lui-même le maître de l'État\,;
M. le Duc, dis-je, fut un homme fait exprès pour la fortune de
l'Angleterre, possédé aveuglément qu'il était par la marquise de Prie.
Avec de la beauté, l'air et la taille de nymphe, beaucoup d'esprit, et
pour son âge et son état de la lecture, et des connaissances, c'était un
prodige de l'excès des plus funestes passions\,: ambition, avarice,
haine, vengeance, domination sans ménagement, sans mesure, et depuis que
M. le Duc fut le maître, sans vouloir souffrir la moindre contradiction,
ce qui rendit son règne un règne de sang et de confusion. Les Anglais,
bien au fait de notre intérieur, se hâtèrent de la gagner, et moyennant
la même pension qu'avait d'eux le cardinal Dubois, tout fut bientôt
conclu. Ils ne perdirent donc rien en perdant le cardinal Dubois, tant
que dura le ministère de M. le Duc qui, mené par cette Médée, marcha
totalement sur les traces de Dubois, par rapport à l'Angleterre. Le
bonheur de cette couronne fut tel que bientôt après M. le Duc crut avoir
grand besoin, d'elle. Le roi tomba malade, et quoique le mal ne fût pas
menaçant et qu'il finit en peu de jours, M. le Duc en fut tellement
effrayé qu'il se releva une nuit tout nu, en robe de chambre, et monta
dans la dernière antichambre du roi de l'appartement bas de feu
Monseigneur, où M. le duc d'Orléans était mort, et que M. le Duc avait
eu ensuite. Il était seul une bougie à la main. Il trouva Maréchal qui
passait cette nuit-là dans cette antichambre, qui me le conta peu de
jours après, et qui, étonné de cette apparition, alla à lui et lui
demanda ce qu'il venait faire. Il trouva un homme égaré, hors de soi,
qui ne put se rassurer sur ce que Maréchal lui dit de la maladie, et à
qui enfin d'effroi et de plénitude, il échappa\,: «\,Que deviendrais-je,
répondant entre haut et bas à son bonnet de nuit\,; je n'y serai pas
repris s'il en réchappe\,; il faut le marier.\,» Maréchal avec qui il
était seul à l'écart ne fit pas semblant de l'entendre\,; il tâcha de
lui remettre l'esprit, et le renvoya se coucher. Ce fut l'époque du
renvoi de l'infante. M. le Duc en a voit indignement usé avec le fils de
feu M. le duc d'Orléans, qui l'avait comblé de considération et de
grâces, et y avait eu beau jeu et à bon marché avec {[}ce{]} prince. Il
redoutait comme la mort de se voir soumis à lui\,; et, pour l'éviter, il
voulut mettre le roi en état d'avoir promptement des enfants. Ainsi,
faisant à l'Espagne une aussi cruelle injure, que la tromperie jusqu'au
moment et la manière de l'exécution rendirent encore plus sensible, il
compta bien sur une haine irréconciliable, et se jeta de plus en plus à
l'Angleterre.

Son règne trop violent pour durer, se termina comme on sait par n'avoir
pu se résoudre à se séparer de M\textsuperscript{me} de Prie, ni elle à
laisser gouverner Fleury qui se lassa d'avoir compté vainement d'en
avoir la réalité, et d'en laisser à M. le Duc la figure et, l'apparence.
Ce prince succéda à M. le duc d'Orléans à l'instant de sa mort, le 23
décembre 1723, et finit le lundi de la Pentecôte 1726, par l'ordre que
lui porta le duc de Charost, capitaine des gardes du corps, un moment
après que le roi fut parti de Versailles pour aller à Rambouillet, de se
retirer sur-le-champ à Chantilly, où il alla à l'heure même accompagné
par un lieutenant des gardes du corps.

Le cardinal Fleury, qui ne l'était pas encore, mais qui le devint six
semaines ou deux mois après, prit donc le jour même les rênes du
gouvernement, et ne les a quittées avec la vie que tout à la fin de
janvier 1743. Jamais roi de France, non pas même Louis XIV, n'a régné,
d'une manière si absolue, si sûre, si éloignée de toute contradiction,
et n'a embrassé si pleinement et si despotiquement toutes les
différentes parties du gouvernement, de l'État et de la cour, jusqu'aux
plus grandes bagatelles. Le feu roi éprouva souvent des embarras par la
guerre domestiqué de ses ministres, et quelquefois par les
représentations de ses généraux d'armée et de quelques grands distingués
de sa cour. Fleury les tint tous à la même mesure sans consultation,
sans voix de représentation, sans oser hasarder nul débat entre eux. Il
ne les faisait que pour recevoir et exécuter ses ordres\,; sans la plus
légère réplique, pour les exécuter très ponctuellement et lui en rendre
simplement compte sans s'échapper une ligne au delà, et sans que pas un
d'eux ni des seigneurs de la cour, des dames ni des valets qui
approchaient le plus du roi, osassent proférer une seule parole à ce
prince de quoi que ce, soit, qui ne fût bagatelle entièrement
indifférente. Comment il gouverna, c'est ce qui dépasse de loin le temps
que ces Mémoires doivent embrasser. Je dirai seulement ici ce qui fait
la suite nécessaire de cette digression.

Il trouva le gouvernement entièrement monté au ton de l'Angleterre, et
un ambassadeur de cette couronne bien plus mesuré, mais aussi bien plus
habile que n'avait été Stairs, auquel il avait succédé. C'était Horace
Walpole, frère de Robert, qui gouvernait alors principalement en
Angleterre. La partie n'était pas égale entre eux. Horace, nourri dans
les affaires comme le sont tous les Anglais, mais de plus, frère et ami
de celui qui les conduisait toutes, qui les consultait avec lui de
longue main, et qui le dirigeait de Londres, étaient l'un et l'autre
deux génies très distingués. Je dirai seulement qu'il avait passé sa vie
d'abord dans l'intimité, après à se pousser et à faire sa cour à tout le
monde, puis dans les ruelles, les parties, les bonnes compagnies, loin
de toute étude, de toute affaire, de toute espèce d'application\,; enfin
évêque, de la manière qu'on l'a vu dans ces Mémoires, et depuis qu'il le
fut confiné quelquefois dans un trou solitaire, tel qu'est Fréjus, mais
la plupart du temps dans les bonnes villes et les meilleures maisons de
la Provence et du Languedoc avec la bonne compagnie, dont il se fit
toujours désirer. Il n'avait donc pas la plus légère notion d'affaires,
lorsqu'il prit tout à coup le timon de toutes. Il avait alors
soixante-douze ou soixante-treize ans, et de ce moment, il en fut
toujours moins occupé, quoiqu'il en disposât seul et uniquement de
toutes, que de se maintenir dans cette autorité, et de la porter au
comble où, dix-huit ans durant, on l'a vue sans le plus petit nuage. Le
léger travail de M. le Duc avec le roi lorsqu'il était premier ministre,
où Fleury s'était introduit en tiers d'abord, n'avait pu lui donner la
moindre teinture d'affaires. Il ne s'y agissait que des grâces à
distribuer, en présenter la liste toute faite, en dire deux mots fort
courts, car M. le Duc n'avait pas le don de la parole, et faire mettre
le bon du roi au bas de la feuille. Cela donnait lieu seulement à Fleury
de dire quelque chose surs les sujets et de l'emporter quelquefois aussi
quand il s'agissait de bénéfices.

M. le Duc, peut-être mieux M\textsuperscript{me} de Prie, qui le
gouvernait et qui était elle-même conduite par les Pâris, s'ennuya de ce
témoin unique de ce travail, et pour s'en défaire pratiqua un jour,
qu'au moment que M. le Duc allait arriver pour le travail, et que le
cardinal était déjà entré, le roi prit son chapeau, et sans rien dire au
cardinal s'en alla chez la reine qu'il trouva dans son cabinet, qui
l'attendait avec M. le Duc. Le cardinal demeura seul plus d'une heure
dans le cabinet du roi à se morfondre. Voyant le temps du travail bien
dépassé il s'en alla chez lui, envoya chercher son carrosse et s'en alla
coucher à Issy au séminaire de Saint-Sulpice, où il s'était fait une
retraite pour s'y reposer quelquefois. En attendant son carrosse il
écrivit au roi en homme piqué, et très résolu de partir sans le voir
pour s'en aller pour toujours dans ses abbayes. Il l'envoya à Nyert,
premier valet de chambre en quartier. Quelque temps après le roi revint
chez lui et Nyert lui donna la lettre. Les larmes, car, il était bien
jeune, le gagnèrent en la lisant, il se crut perdu n'ayant plus son
précepteur, et s'alla cacher sur sa chaise percée. Le duc de Mortemart,
premier gentilhomme de la chambre en année, arriva là-dessus. Nyert lui
conta ce qui était arrivé du travail, de la lettre, des larmes, et de la
fuite sur la chaise percée. Le duc de Mortemart y entra et le trouva
dans la plus grande désolation. Il eut peine à tirer de lui ce qui
l'affligeait de la sorte. Dès qu'il le sut, il représenta au roi qu'il
était bien bon de pleurer pour cela, puisqu'il était le maître
d'ordonner à M. le Duc d'envoyer de la part de Sa Majesté chercher
Fleury, qui sûrement ne demanderait pas mieux, et dans l'extrême
embarras où il vit le roi là-dessus, il s'offrit d'en aller porter
sur-le-champ l'ordre à M. le Duc. Le roi délivré sur l'exécution
l'accepta, et le duc de Mortemart alla tout aussitôt chez M. le Duc qui
se trouva fort étourdi, et qui après une courte dispute obéit à l'ordre
du roi. Comme la chose était arrivée avant le soir sur la fin de
l'après-dînée elle fit grand bruit et force dupes, car on ne douta pas
que Fleury ne fût perdu et chassé sans retour, qui n'eût été cardinal ni
premier ministre de sa vie, si M. le Duc l'eût fait paqueter sur le
chemin d'Issy et fait gagner pays toute la nuit. Le roi aurait bien
pleuré, mais la chose serait demeurée faite\,; M. de Mortemart n'aurait
pas porté l'ordre à temps. Après cet éclat il fallait que l'un chassât
l'autre. L'un était prince du sang, premier ministre et sur les lieux,
tandis que l'autre, sans nul appui courait la poste, ou pour le moins
les champs vers un exil. Qui que ce soit n'eût osé faire tête à M. le
Duc, ni peut-être voulu quand on l'aurait pu, et l'un demeurait perdu et
l'autre pour toujours le maître. Voici pourquoi je raconte ici cette
anecdote, qui outrepasse le temps que ces Mémoires doivent embrasser.
Walpole, averti de tout à temps, le fut de cette aventure\,; il
ménageait Fleury comme un homme qui pointait, et que l'amitié de mie
pouvait conduire loin. Il alla sur-le-champ à Issy, et par cette
démarche se dévoua personnellement le cardinal à un point qui est
inexprimable, et dont je ne puis douter comme on va le voir.

Fleury était incapable non seulement d'accepter des présents et des
pensions étrangères, mais hors de toute mesure qu'on osât lui en
présenter. Ce ne fut donc pas cette voie qui le gagna, c'est peu dire,
qui le livra à l'Angleterre, et encore sans penser à elle ni à l'intérêt
de cette couronne, et c'est ce qu'il faut maintenant expliquer. Pour le
bien faire il faut dire ici que je fus toujours en usage que lui et moi
nous nous parlions de tout. Il trouva toujours très bon que je lui
demandasse à quoi il en était avec telle ou telle puissance\,; il m'y
répondait toujours franchement et avec détail. Très ordinairement aussi
il m'en parlait le premier, si bien même qu'allant chez lui pour lui
parler de choses qui me regardaient, et craignant d'y être interrompu,
faute de temps, par l'heure pour lui d'aller chez le roi, ou par quelque
autre nécessité semblable, je lui fermais souvent la bouche sur les
affaires, en lui disant que j'étais là pour les miennes, que je
craignais de manquer de temps, et qu'après que je lui aurais expliqué ce
qui m'amenait, je serais ravi d'apprendre ensuite ce qu'il voudrait bien
me dire\,; et en effet, quand j'avais achevé, il revenait à me parler
d'affaires d'État, quelquefois de cour, mais jamais qu'en récit, en
raisonnements de sa part et de la mienne, sans rien qui approchât de la
consultation. Cela suffit ici\,; on pourra voir dans la suite ce qui
m'avait mis et établi dans cette stérile confiance. J'ajouterai
seulement que jamais en aucun temps ni moment son cabinet ne me fut
fermé, et qu'à moins de cause majeure et rare c'était toujours moi qui
le quittais\,; qu'il ne me montra jamais qu'il trouvât que c'était assez
demeurer avec lui, et que souvent il me retenait, me demandait pourquoi
je m'en allais, causait en me suivant à la porte, et assez souvent
encore quelque peu debout devant la porte avant de l'ouvrir.

Ce ministre tourna une vertu en défaut que je lui ai souvent reproché.
La vie pauvre qu'il avait menée jusqu'à son épiscopat, car il avait
d'ailleurs très peu de bénéfices, celle surtout qu'il avait menée dans
sa jeunesse dans les collèges et les séminaires, l'avait accoutumé à une
vie dure, à se passer de tout, et à une grande épargne\,; mais cette
habitude n'avait point dégénéré en, lui comme en presque tous ceux qui
sortent d'une longue pauvreté, surtout destituée de naissance, en soif
d'argent, de biens, de bénéfices, d'entasser et d'accumuler des revenus,
ou en avarice crasse et sordide. C'était l'homme du monde qui se
souciait le moins d'avoir, et qui, maître de se procurer tout ce qu'il
aurait voulu, s'est le moins donné, comme il y a paru dans tout le cours
de son long et toujours tout-puissant ministère. Mais avec ce
désintéressement personnel et cette simplicité même portée trop loin, de
table, de maison, de meubles et d'équipages, et libéral du sien aux
pauvres, à sa famille, même à quelques amis, sans faire pour soi le
moindre cas de l'argent, il l'estima trop en lui-même, et non content
d'une sage et discrète économie, choqué à l'excès des profusions des
ministères qui avaient précédé le sien, il tomba dans une avarice pour
l'État et pour les particuliers, dont les suites ont été très funestes.
Quelque curieux et important que cela soit, ce n'est pas ici le lieu de
traiter cette matière, qui peut-être se pourra retrouver ailleurs. Il
suffit de dire ici qu'il excellait aux ménages de collège et de
séminaire, et qu'on pardonne ce mot bas, au ménage des bouts de
chandelle, parce qu'à la lettre il a fait pratiquer ce dernier, dont le
roi pourtant se lassa, dans ses cabinets, et dont un malheureux valet se
rompit le cou sur un degré du grand commun. Un autre défaut encore trop
commun à ceux qui occupent de grandes places, et qui a mené le cardinal
Fleury bien loin, sans s'en être pu corriger par les fatales
expériences, c'est qu'il prenait aisément les hommages, les avances, les
louanges, les fausses protestations des étrangers et des souverains,
pour réels et pour estime, de sa personne, pour confiance en lui, même
pour amitié véritable, sans songer qu'il ne les devait qu'à l'importance
de sa place et au besoin qu'ils avaient de lui, ou {[}au{]} désir de le
gagner et de le tromper, comme il l'a été de presque toutes les
puissances de l'Europe l'une après l'autre.

Pensant et agissant de la sorte, Walpole, qui en savait bien plus que
lui, se le dévoua et au gouvernement d'Angleterre. Il joignit à ses
adorations, à ses hommages, à son air de respect, d'attachement et
d'admiration personnels, ceux de son frère qui gouvernait l'Angleterre,
et tous deux parvinrent à le persuader qu'ils ne se gouvernaient que par
ses conseils. Leur grand objet était triple, et ils le remplirent
triplement et complètement\,: empêcher que la France ne relevât sa
marine et leur donnât d'inquiétude sur Dunkerque, etc., et se conserver
par là l'empire de la mer et du commerce, en sapant doucement ce qui
nous en restait\,; tenir la France et l'Espagne en jalousie et mal
ensemble, tant par celle de toute l'Europe de l'union des deux branches
royales, et de ses suites, que pour saper aussi le commerce d'Espagne de
plus en plus, et à continuer à s'établir à ses dépens et à sa ruine dans
les Indes\,; enfin par rapport à Hanovre et autres États du roi Georges
en Allemagne, se rendre considérables à l'empereur par disposer à son
égard de la France\,: tous ces trois points furent aisés à Walpole.
Indépendamment de ses manèges auprès du cardinal, l'avarice de celui-ci
l'empêcha non seulement de vouloir rien écouter sur le rétablissement de
la marine\,; mais elle le poussa à tous les ménages qui en achevèrent la
destruction. Pour le commerce, la crainte de blesser les Anglais qu'il
croyait gouverner faisait avorter les mesures et les propositions les
plus sages, et lui fermait les oreilles aux plaintes les plus criantes,
dont j'ai vu sans cesse Fagon désolé, qui était un conseiller d'État
très distingué, mon ami, qui avait deux fois refusé la place de
contrôleur général, qui avait grande autorité dans les finances et qui
était à la tête du commerce, par qui j'en ai su des détails infinis.

L'article de l'Espagne ne fut pas plus difficile. Comme je ne dis que ce
que je sais, et, que j'avoue sans honte, et pour l'amour de la vérité ce
que j'ignore, je suivrai ici la même route. Dès l'entrée du cardinal
dans les affaires, il s'éleva des nuages entre l'Espagne et lui
personnellement, dont j'ai toujours ignoré la cause, quoique j'aie tâché
de la découvrir. Ces nuages allèrent toujours croissant, et mirent enfin
un mur de séparation personnelle entre la reine d'Espagne et lui, qui
monta jusqu'à l'aversion des deux côtés, et réciproquement peu ménagés
jusqu'à l'indécence. J'ai toujours cru que le renvoi de l'infante en
était la source, qui en effet n'eût pu se faire sans lui, quoique M. le
Duc eût enfin fait sa paix apparente par l'abbé de Montgon, qu'il envoya
en Espagne, exprès sous une autre couleur. Mais ces choses, qui ne sont
pas de l'espace de ces Mémoires, nous mèneraient ici trop loin. On peut
juger que Walpole, trouvant de telles dispositions, à l'égard de
l'Espagne, n'eut pas de plus grand soin que de jeter de l'huile sur ce
feu\,; et il eut la joie sous tout ce ministère de voir la France et
l'Espagne intérieurement dans le plus funeste éloignement, quoi que
l'Espagne pût quelquefois faire, et qu'osassent doucement hasarder le
peu de gens qui, pouvant quelquefois dire quelque mot au cardinal,
pensaient que le plus essentiel intérêt de la France, comme le plus
véritable, était l'union intime avec l'Espagne, comme il m'est souvent
et toujours inutilement arrivé. Ces deux points gagnés, le dernier
n'était pas difficile, et les Anglais parvinrent aisément à lui
persuader que ce n'était que par eux qu'il pouvait amener l'empereur aux
choses qui conviendraient à la France, tellement, qu'enivré de leur
encens et de leur discours, il se conduisit entièrement à leur gré sur
toutes choses, jusqu'à ce qu'après plusieurs années ils le méprisèrent,
parce qu'ils n'en avaient plus besoin, et qu'ils avaient formé aux
dépens de la France des alliances qui leur convenaient davantage. Ils
passèrent donc pour flatter les Anglais et leurs nouveaux confédérés
jusqu'à montrer en plein parlement les lettres qu'ils avaient gardées de
lui, et en faire des dérisions publiques. Souvent j'avais hasardé de lui
parler de marine, de commerce et de cet abandon aux Anglais, nos plus
ardents et invétérés ennemis\,; car les torys qui nous avaient sauvés
sous la reine Anne, étaient en butte aux whigs depuis sa mort et
anéantis, et l'abbé Dubois, secondé de Canillac et du duc de Noailles,
les avait fait abandonner publiquement et sacrifier par M. le duc
d'Orléans. C'étaient donc ceux qui avaient appelé le roi Guillaume et la
ligue protestante, c'est-à-dire les plus envenimés ennemis de la France,
qui régnaient en Angleterre, et qui depuis la mort du feu roi
gouvernaient la France à leur plaisir. Quand je pressais le cardinal
Fleury. «\,Vous n'y êtes pas, me répondait-il avec un sourire de
complaisance. Horace Walpole est mon ami personnel. Il est le seul qui
ait osé me venir voir à Issy, lorsque j'y étais prêt à partir me retirer
dans mes abbayes. Il a toute confiance en moi. Croiriez-vous qu'il me
montre les lettres qu'il reçoit d'Angleterre, et toutes celles qu'il y
écrit, que je les corrige, et que souvent je les dicte. Je sais bien ce
que je fais. Son frère a la même confiance. Il faut laisser dire que je
m'abandonne à eux, et moi je vous dis que je les gouverne, et que je
fais de l'Angleterre tout ce que je veux.\,» Jamais il n'a pu se mettre
dans l'esprit qu'un ministre d'Angleterre ne risquait rien de l'aller
voir à Issy. S'il était chassé, c'était un coup d'épée dans l'eau, qui
ne mettait Walpole en nulle crise de M. le Duc, sous la coupe duquel il
ne pouvait être en aucune sorte\,; et si le cardinal était rappelé,
comme il arriva, c'était s'être fait un mérite auprès de lui sans le
moindre risque et à très grand marché. Il put aussi peu se déprendre de
l'opinion qu'il gouvernait les Walpole, qu'après l'éclat dont je viens
de parler, qui le mit au désespoir d'une telle duperie, mais dont il se
garda bien de se plaindre à moi ni à personne, et moi aussi de lui en
parler depuis.

De tout ce récit abrégé de la fortune de l'Angleterre par l'abbé Dubois,
puis par M\textsuperscript{me} de Prie sous M. le Duc, enfin du temps du
cardinal Fleury en France et de ce qui s'est passé en Espagne sous
Albéroni et ses successeurs, tous gens, et en France et en Espagne, qui,
par le néant de leur naissance et par leur isolement personnel,
n'étaient pas pour prendre grand intérêt à l'État qu'ils ont gouverné,
ni pour être touchés d'aucun autre que du leur propre sans le plus léger
balancement ni remords, on voit de quel funeste poison est un premier
ministre à un royaume. Soit par intérêt, soit par aveuglement, quel
qu'il soit, il tend avant tout et aux dépens de tout à conserver,
affermir, augmenter sa puissance\,; par conséquent son intérêt, ne peut
être celui de l'État qu'autant qu'il peut concourir ou compatir avec le
sien particulier. Il ne peut donc chercher qu'à circonvenir son maître,
à fermer tout accès à lui, pour être le seul qui lui parle et qui soit
uniquement le maître de donner aux choses et aux personnes le ton et la
couleur qui lui convient, et pour cela se rendre terrible et funeste à
quiconque oserait dire au roi le moindre mot qui ne fût pas de la plus
indifférente bagatelle. Cet intérêt de parler seul et d'être écouté seul
lui est si cher et si principal, qu'il n'est rien qu'il n'entreprenne et
qu'il n'exécute pour s'affranchir là-dessus de toute inquiétude.
L'artifice et la violence ne lui coûtent rien pour perdre quiconque lui
peut causer la moindre jalousie sur un point délicat, et pour donner une
si terrible leçon là-dessus, que nul sans exception ni distinction n'ose
s'y commettre. Par même raison, moins il est supérieur en capacité et en
expérience, moins veut-il s'exposer à consulter, à se laisser
représenter, à choisir sous lui de bons ministres, soit pour le dedans,
soit pour le dehors. Il sent qu'ayant un intérêt autre que celui de
l'État, il réfuterait mal les objections qu'ils pourraient lui faire,
parce que son opposition à s'y rendre viendrait de cet intérêt personnel
qu'il veut cacher\,; c'est par cette raison et par celle de craindre
d'être jamais pénétré qu'il ne veut choisir que des gens bornés et sans
expérience\,; qu'il écarte tout mérite avec le plus grand soin\,; qu'il
redoute les personnes d'esprit, les gens capables et d'expérience\,;
d'où il résulte qu'un gouvernement de premier ministre ne peut être que
pernicieux. Je ne fais ici qu'écorcher la matière que j'aurai lieu
ailleurs d'étendre davantage\,; venons au point qui m'a engagé à cette
digression\,; il est bien court, bien fatal. Le voici\,:

L'expérience de plusieurs siècles doit avoir appris ce qu'est
l'Angleterre à la France\,; ennemie de prétentions à nos ports et à nos
provinces, ennemie d'empire de la mer, ennemie de voisinage, ennemie de
commerce, ennemie de colonies, ennemie de forme de gouvernement\,; et
cette mesure comblée par l'inimitié de la religion, par les tentatives
d'avoir voulu rétablir la maison Stuart sur le trône malgré la nation,
ce qu'elle a de commun avec le reste de l'Europe, ce qui l'a unie avec
les autres puissances contre la nôtre, et qui en maintient l'union\,; la
jalousie extrême de voir l'Espagne dans la maison de France, et la
terreur que toute l'Europe conçoit de ce que pourrait l'union des deux
branches royales pour leur commune grandeur, si elles avaient être
guidées par la sagesse de l'esprit, qui a sans cesse présidé aux
conseils des deux branches couronnées de la maison d'Autriche en
Allemagne et en Espagne, et qui les a portées à un tel degré de grandeur
et de puissance malgré la vaste séparation de leurs États, inconvénient
qui l'a sans cesse embarrassée, et qui ne se trouve point entre la
France et l'Espagne dont les terres et les mers sont contiguës. La même
expérience apprend aussi que la France a toujours eu tout à craindre de
l'Angleterre tant qu'elle a été paisible au dedans\,; que la France
même, sans s'en mêler, a tiré les plus grands avantages des longues et
cruelles divisions de la Rose blanche et de la Rose rouge, et depuis,
des secousses par intervalles que l'autorité et les passions de Henri
VIII y ont, causées\,; enfin des longs troubles qui y ont porté Cromwell
à la suprême puissance. Marie a peu régné, et dans l'embarras de
rétablir la religion catholique après le court règne de son frère
mineur. Élisabeth, cette reine si fameuse, était personnellement amie de
Henri IV, et d'ailleurs, elle ne laissait pas de se trouver embarrassée
de l'Écosse, de l'Irlande même, et de son sexe encore avec des sujets
qui la pressaient de se marier, n'osant les refuser, et ne voulant
pourtant partager son trône avec personne. La faiblesse de Jacques Ier,
sa maladie d'être auteur et d'exceller en savoir, sa passion pour la
chasse, son dégoût pour les affaires, empêchèrent de son temps
l'Angleterre d'être redoutable. Son petit-fils, rétabli après de si
étranges révolutions, était ami personnel du feu roi, et eut pourtant la
main forcée par son parlement pour lui déclarer la guerre, et eut
beaucoup de mouvements domestiques à essuyer. Du court règne de Jacques
II, ce n'est pas la peine d'en parler. La France a cruellement senti
tout le règne de Guillaume\,; et, si les fins de celui de la reine Anne
l'en ont consolée, ce n'a pas été sans le payer chèrement par Dunkerque,
et toutes les entraves de cette côte mise à découvert. On voit de plus
quel fut l'esprit des Anglais à son égard après la paix, et en haine de
là paix. Il n'y a qu'à lire ce que Torcy en rapporte et qu'on trouvera
ici dans les Pièces.

Il est donc clair que l'intérêt sensible de la France, est autant
qu'elle le peut sagement, d'exciter et d'entretenir les troubles
domestiques parmi une nation qui y est elle-même si portée. C'est ce que
le feu roi projetait, et que la mort l'empêcha d'exécuter. Tout était
prêt. Il n'y avait qu'à suivre, lorsque l'intérêt de l'abbé Dubois
l'empêcha par Canillac et par le duc de Noailles. Il n'y a qu'à lire ce
qui est rapporté dans ces Mémoires, d'après Torcy, sur les affaires
étrangères pour voir que l'Angleterre fût continuellement agitée dans
l'intérieur, qu'elle avait tout à craindre de l'entreprise, d'une
révolution, à laquelle la position de la France à son égard pouvait
donner le plus grand branle\,; que l'Angleterre avait infiniment plus
besoin de la France que la France de l'Angleterre\,; que cette dernière
le sentait parfaitement, et payait de l'audace de Stairs et, de
l'artifice de ceux qu'ils avaient gagnés auprès du régent, et que depuis
que l'abbé Dubois eut pris le grand vol dès son premier passage en
Angleterre, cette dernière couronne n'eut plus, non seulement rien à
craindre de la France, mais lui commanda despotiquement par l'intérêt de
l'abbé Dubois, par celui de M\textsuperscript{me} de Prie ensuite, enfin
par l'avarice si mal entendue du cardinal de Fleury pour la marine, et
sur le reste par l'ensorcellement que Horace Walpole eut l'art de lui
jeter. Dans tous ces temps, on a pu troubler l'Angleterre par le
prétendant, comme on peut en tirer les preuves des extraits des lettres
faits par Torcy et depuis la régence encore. En aucun temps on en a
jamais fait que de misérables et très rares semblants. L'affaire infâme
de Nonancourt déshonorera toujours le temps où elle arriva\,; et
l'entreprise échouée du prince de Galles, en 1746, est une chose qui ne
peut avoir de nom.

Ce qui résulte de tout ce qu'on vient de voir, c'est que la marine de
France se trouve radicalement détruite, son commerce par conséquent,
tous les magasins épuisés, les constructions impossibles\,; qu'elle ne
peut hasarder de vaisseaux à la mer qu'ils ne soient pourchassés en
quelque endroit que ce soit, de toute la vaste étendue des mers de l'un
et de l'autre monde\,; que ses ports et ses côtes sont exactement
bloqués, ses meilleures colonies enlevées, ce qui lui en reste très
menacé et à la discrétion des Anglais, quand il leur plaira d'en prendre
sérieusement la peine. Nul contrepoids à la puissance maritime de
l'Angleterre, qui couvre toutes les mers de ses navires. La Hollande,
qui en gémit intérieurement, n'ose pas même le montrer. L'Espagne ne
pourra de longtemps se relever de la fatale assistance que nous avons
prêtée à l'Angleterre de ruiner sa marine et d'estropier son commerce et
ses établissements des Indes\,; et il faudrait à la France trente ans de
paix et du plus sage gouvernement pour remonter sa marine au point que
Colbert et Seignelay l'ont laissée. C'est, avec bien d'autres maux, ce
que la France doit aux premiers ministres qui l'ont gouvernée depuis la
mort du feu roi. Ainsi l'Angleterre triomphe de notre ineptie. Tandis
qu'elle étourdit le monde de ce grand mot de contre-poids et d'équilibre
de puissance en Europe, elle a usurpé le plein empire de toutes les mers
et de tout commerce. L'abondance des richesses qu'elle en retire la met
en état d'exécuter tout ce qui lui convient, et de payer la reine de
Hongrie, la Hollande, le roi de Sardaigne contre la France, de faire
renaître une seconde maison d'Autriche des cendres de la première, et de
faire à la France la plus cruelle guerre, en laquelle le cardinal Fleury
s'est imbécilement laissé engager par l'intérêt d'un très simple
particulier (Belle-Ile), qu'il haïssait, et dont il se défiait, sans que
contre tant de puissances ennemies on puisse encore apercevoir une fin
possible, ni à quel prix la France pourra obtenir la paix, après des
victoires et des conquêtes qui ne l'en éloignent guère moins que n'ont
fait les tristes et profondes pertes qu'elle a faites en Allemagne et en
Italie\footnote{Saint-Simon a dû écrire cette partie de ses Mémoires
  vers 1746, d'après les événements auxquels il fait allusion. Les
  pertes essuyées en Allemagne sont du commencement de la guerre de la
  succession d'Autriche\,; mais les désastres d'Italie ne datent que de
  1746. Quant aux victoires et conquêtes, dont parle Saint-Simon, elles
  avaient pour théâtre la Belgique, dont beaucoup de places furent
  prises par les Français, à la suite des batailles de Fontenoy (1745)
  et de Raucoux (1746).}.

Comparons maintenant le gouvernement de nos ennemis avec le nôtre, et
tâchons de voir enfin la source déplorable de nos malheurs. La France et
l'Espagne, gouvernées par des gens de robe et de peu, ensuite par des
premiers ministres encore moindres\,; les uns et les autres en garde
continuelle contre la naissance, l'esprit, le mérite, l'expérience,
uniquement occupés à les écarter, et de leur cabinet à gouverner ceux
qu'ils employaient au dehors, et à commander les armées. Je n'en dis pas
davantage, et je renvoie sur cette importante matière à ce qui s'en
trouve ici sur le règne du feu roi, et à ce qui vient d'être courtement
dit des premiers ministres, qui depuis sa mort ont gouverné la France et
l'Espagne. Les cours de Turin, de Londres et de Vienne ont le bonheur de
détester de tout temps cette sorte de gouvernement\,; les premiers
ministres y sont inconnus depuis des siècles, et la robe y est avec
l'honneur qu'elle mérite dans les fonctions qui lui sont propres\,; mais
la nécessité de porter un rabat pour être capable de toutes les parties
civiles, politiques, militaires du gouvernement, privativement à toute
autre condition et profession, est une gangrène dont ces cours n'ont
jamais été susceptibles, et dont notre fatal exemple les saura de plus
en plus préserver.

Ces puissances n'emploient dans leurs conseils que, des gens de qualité,
et le plus qu'il se peut distinguée, persuadées qu'elles sont que la
noblesse des sentiments et l'attachement à la prospérité de l'État
auquel ils tiennent par leur naissance, leurs terres, leurs alliances,
leur état en tout genre, est un gage certain de leur conduite qui les
éloigne de l'indifférence pour le général, et de l'ardeur pour la
fortune prompte et particulière, des nuisibles efforts de rapide
élévation dont l'honneur et la position des personnes de qualité les
préserve. On s'y garde bien des choix au hasard, surtout de confier les
plus importants ministères à qui n'en a aucune notion. Ces cours qui
n'ont jamais été tachées de la pernicieuse persuasion que leur pouvoir
et leur prospérité consiste à faire que tout soit peuple, et peuple
ignorant et sans émulation, sont au contraire appliquées à essayer des
sujets pour les divers ministères de toutes les parties du gouvernement,
à les employer par degrés dans le civil et le politique, comme dans le
militaire, à laisser promptement tomber les ineptes, à pousser les
autres, suivant leurs talents, à ne laisser pas languir ceux qui
montrent valoir dans la lenteur des degrés et des grades\,; et par cette
conduite elles ont toujours à choisir pour le grand en tout genre. Avant
les malheurs de Lintz, de Prague, etc., que serait devenue la reine de
Hongrie, réduite à quitter Vienne, si son conseil ou plutôt ses conseils
avaient été uniquement composés de quatre ou cinq ministres de l'espèce
du nôtre\,? Les siens, attachés de père en fils à sa maison par leurs
alliances, par leurs terres, par leur état qui se perdait avec le sien,
tous généraux d'armée ou expérimentés en maniement d'affaires, tous en
dignités et en considération par leur naissance, se sont surpassés en
efforts pour la soutenir, et de la situation la plus désespérée l'ont
ramenée à celle où on la voit aujourd'hui par leur science politique et
militaire, et par l'autorité de leur naissance, de leurs alliances, de
leur crédit dans les provinces héréditaires et dans le reste de
l'Allemagne. Je n'irai pas plus loin dans une matière également
importante et inutile. Théorie, comparaison, expérience, tout en montre
l'importance\,; et le pli fatal que la France a pris là-dessus,
l'inutilité d'espérer un changement si salutaire. Le fil des choses m'a
naturellement emporté à cette digression, et la douleur de la situation
présente de la France à n'en pas taire les causes. À mon âge et dans
l'état où est ma famille, on peut juger que les vérités que j'explique
ne sont mêlées d'aucun intérêt. Je serais bien à plaindre, si c'était
par regret d'être demeuré oisif depuis la mort de M. le duc d'Orléans.
J'ai appris dans les affaires que s'en mêler n'est beau et agréable
qu'au dehors, et de plus, si j'y étais resté, à quelles conditions\,? et
il serait temps de m'en retirer à présent où je n'aurais plus qu'à
envisager le compte que j'aurais à en rendre à celui qui domine le temps
et l'éternité, et qu'il demandera bien plus rigoureusement aux grands
effectifs et aux puissants de ce monde, qu'à ceux qui se sont mêlés de
peu ou de rien.

Avant de prendre sérieusement la suite de ces Mémoires où cette
digression l'a interrompue, je ne veux pas oublier une bagatelle, parce
qu'elle caractérise M. le duc d'Orléans, et qu'elle m'a échappé et
m'échapperait encore si je ne la saisissais dans cet intervalle de
choses, au moment qu'elle me revient dans l'esprit. La dernière année de
la vie du feu roi, le chapitre de Denain députa deux de ses chanoinesses
pour venir représenter ici les dommages et la ruine que leurs biens et
leur maison avait souffert du combat qui s'était donné chez elles, et
dont la victoire fut le commencement de la résurrection de la France. Je
les avais souvent vues dans les tribunes à la messe du roi, et su qui
elles étaient et pourquoi venues. M\textsuperscript{me} de Dangeau les
protégea, mais le roi mourut sans qu'on eût songé à elles. La régence
formée, elles s'adressèrent aux maréchaux de Villeroy et de Villars, et
au duc de Noailles, parce que leur demande allait aux finances à cause
de la guerre. Elles frappèrent encore à d'autres portes inutilement plus
d'un an, et souvent, à ce qu'elles m'ont dit depuis, très mal reçues et
éconduites. Lassées d'un séjour si long, si infructueux et si coûteux
pour l'état où elles étaient, et voulant apparemment ne laisser rien
qu'elles n'eussent tenté, elles vinrent me parler. L'une s'appelait
M\textsuperscript{me} de Vignacourt, l'autre M\textsuperscript{me}
d'Haudion. Je les reçus avec l'ouverture qu'on doit à des personnes
pressées et malheureuses, et avec la politesse et les égards que leur
naissance et leur état demandait. Elles en furent assez surprises pour
que je le pusse remarquer\,; c'est qu'elles n'y avaient pas été
accoutumées, à ce qu'elles me dirent depuis, par ceux à qui elles
s'étaient auparavant adressées, et j'en fus d'autant plus étonné du duc
de Noailles particulièrement, qu'encore que sa naissance n'ait pas
besoin d'appuis, il montre le cas qu'il fait de la bricole un peu
fâcheuse de l'alliance de Vignacourt par le portrait en pied qu'il a
chez lui, en grand honneur et montre, d'un des deux grands maîtres de
Malte du nom de Vignacourt, qui étaient oncles de Française de
Vignacourt qui, faute de bien apparemment, épousa Antoine Boyer, dont
elle eut Louise Boyer, mère du cardinal, du bailli, et du maréchal de
Noailles, et de la marquise de Lavardin, femme d'une rare vertu et d'un
singulier mérite, qui a été l'unique mais forte mésalliance des aînés de
Noailles de père en fils. Elle était sueur de la vieille Tambonneau,
dont j'ai parlé ici en son temps, et de M\textsuperscript{me} de Ligny
dont le mari était aussi fort peu de chose, et qui fut mère de la
princesse de Fürstemberg, dont j'ai parlé aussi. Pour revenir aux
chanoinesses, je m'instruisis de leur affaire\,; j'en rendis compte à M.
le duc d'Orléans, et lui représentai la justice de leur demande, le
mérite de son origine, qui avait commencé le salut de l'État chancelant,
l'indécence d'une si longue poursuite et la réputation bonne ou mauvaise
qui en résultait dans le pays étranger. J'ajoutai ce qu'il y avait à
dire sur la considération du chapitre et du besoin pressant de ces
filles de qualité, surtout des deux députées qui se consommaient en
frais à Paris. Tout cela fut bien reçu, bien écouté\,; mais je fus six
mois à poursuivre cette affaire.

Ces chanoinesses, qui n'espéraient plus rien que de mon côté, et que je
consolais de mon mieux, que j'avais accoutumées à venir dîner assez
souvent chez moi, me témoignèrent de plus en plus de l'ouverture, et
finalement m'avouèrent qu'on les allait mettre hors de leur logis, sans
savoir que devenir. J'allai le lendemain exprès de bonne heure chez
M\textsuperscript{me} la duchesse d'Orléans, que je voyais de règle une
fois ou deux la semaine seule ou tout au plus. M\textsuperscript{me}
Sforze et quelquefois M. le comte de Toulouse en tiers. Je trouvai M..
le duc d'Orléans seul avec elle, à l'entrée de son petit jardin en
dehors, où ils étaient assis auprès du fond de l'appartement\,; je m'y
assis avec eux, et la conversation dura assez longtemps. Comme je voulus
m'en aller, je priai M. le duc d'Orléans de me donner deux écus, avec un
sérieux qui augmenta la surprise de la demande. Après m'être bien laissé
faire des questions sur cette plaisanterie, moi toujours insistant que
ce n'en était point une, que très véritablement je lui demandais deux
écus et que je ne croyais pas qu'il voulût me les refuser\,; à la fin je
lui dis l'état où ces deux chanoinesses étaient réduites par la longueur
de leur séjour à Paris et la lenteur sans fin de leur rendre justice\,;
que de moi elles ne prendraient pas de l'argent, que de lui elles n'en
feraient pas difficulté\,; que les deux écus que je lui demandais
étaient pour les leur donner de sa part, afin qu'elles eussent au moins
pour quelques jours à dîner de quelque gargote. Tous deux se mirent à
rire, et moi de moraliser sur une situation si extrême pour ne vouloir
pas décider et finir. Je m'en allai avec promesse plus satisfaisante que
je n'en avais encore pu tirer\,; j'eus soin d'en presser l'effet. Au
bout d'un mois j'eus l'expédition de ce que le chapitre demandait, une
gratification honnête aux deux chanoinesses, pour les sortir de Paris et
les reconduire chez elles, et leur fis faire leur payement. Je n'ai
jamais vu deux filles si aises ni plus reconnaissantes\,; je leur contai
ce sarcasme des deux écus qui avaient enfin terminé leur affaire, dont
elles rirent de bon coeur. J'eus de grands remerciements de l'abbesse et
du chapitre, et tous les ans une lettre de souvenir des deux
chanoinesses tant qu'elles ont vécu. Revenons maintenant à des choses
plus sérieuses.

\hypertarget{chapitre-xv.}{%
\chapter{CHAPITRE XV.}\label{chapitre-xv.}}

1718

~

{\textsc{Mouvements audacieux du parlement contre l'édit des monnaies.}}
{\textsc{- Le parlement rend un arrêt contre l'édit des monnaies, lequel
est cassé le même jour par le conseil de régence.}} {\textsc{- Prétextes
du parlement, qui fait au roi de fortes remontrances.}} {\textsc{-
Conseils de régence là-dessus.}} {\textsc{- Ferme et majestueuse réponse
au parlement en public, qui fait de nouvelles remontrances.}} {\textsc{-
Le don gratuit accordé à l'ordinaire, par acclamation, aux états de
Bretagne.}} {\textsc{- Leurs exilés renvoyés.}} {\textsc{- Question
d'apanages jugée en leur faveur au conseil de régence.}} {\textsc{-
Absences singulières.}} {\textsc{- Cinq mille livres de menus plaisirs
par mois, faisant en tout dix mille livres, rendues au roi.}} {\textsc{-
Manèges du parlement pour brouiller, imités en Bretagne.}} {\textsc{-
Saint-Nectaire, maréchal de camp, fait seul lieutenant général longtemps
après avoir quitté le service.}} {\textsc{- Son caractère.}} {\textsc{-
M\textsuperscript{me} d'Orléans fait profession à Chelles fort
simplement.}} {\textsc{- Arrêt étrange du parlement en tous ses chefs.}}
{\textsc{- Le parlement de Paris a la Bretagne en cadence.}} {\textsc{-
Le syndic des états est exilé.}} {\textsc{- Audacieuse visite de la
duchesse du Maine au régent.}} {\textsc{- Fureur et menées du duc et de
la duchesse du Maine et du maréchal de Villeroy.}} {\textsc{- Commission
étrange sur les finances donnée aux gens du roi par le parlement.}}
{\textsc{- Bruits de lit de justice\,; sur quoi fondés.}} {\textsc{-
Mémoires de la dernière régence fort à la mode, tournent les têtes.}}
{\textsc{- Misère et léthargie du régent.}} {\textsc{- L'abbé Dubois,
Argenson, Law et M. le Duc, de concert, chacun pour leur intérêt,
ouvrent les yeux au régent et le tirent de sa léthargie.}} {\textsc{- M.
le duc d'Orléans me force à lui parler sur le parlement.}} {\textsc{-
Duc de La Force presse contre le parlement par Law, espère par là
d'entrer au conseil de régence.}} {\textsc{- Mesures du parlement pour
faire prendre et pendre Law secrètement, en trois heures de temps.}}
{\textsc{- Le régent envoie le duc de La Force et Fagon conférer avec
moi et Law.}} {\textsc{- Frayeur extrême et raisonnable de Law.}}
{\textsc{- Je lui conseille de se retirer au Palais-Royal, et
pourquoi.}} {\textsc{- Il s'y retire le jour même.}} {\textsc{- Je
propose un lit de justice aux Tuileries, et pourquoi là.}} {\textsc{-
Plan pris dans cette conférence.}} {\textsc{- Abbé Dubois vacillant et
tout changé.}}

~

Il y avait déjà du temps qu'on se plaignait dans les fermes générales de
beaucoup de faux sauniers\,; les précautions y furent peu utiles\,; on
vit de ces gens-là paraître en troupes et armés. Ce désordre ne fit que
s'augmenter. Il y eut un vrai combat dans la forêt de Chantilly entre
eux, des archers et des Suisses postés des garnisons voisines sur leur
marche qu'on avait éventée, et les faux sauniers furent battus, leur sel
pris et leurs prisonniers branchés, mais beaucoup de Suisses et
d'archers tués. Les exécutions ne firent qu'en accroître le nombre, les
aguerrir, les discipliner\,; en sorte que, ne faisant d'ailleurs de mal
à personne, ils étaient favorisés et avertis partout. La chose alla si
loin que des personnes principales furent plus que soupçonnées de les
soutenir et de les encourager, pour s'en faire des troupes dans le
besoin. Le comté d'Eu en fourmillait et en répandait un grand nombre.

Le parlement, avec les secours qu'il se promettait de M. et de
M\textsuperscript{me} du Maine, de ce qui s'appelait la noblesse, des
maréchaux Villeroy, de Tessé, d'Huxelles, du dépit et des respects du
duc de Noailles, et de ce qui se brassait en Bretagne, n'était occupé
qu'à faire contre au régent, à établir son autorité sur les ruines de la
sienne, à l'ombre de sa faiblesse et de la trahison d'Effiat, de Besons
et de ceux qui avaient sa confiance sur les choses qui regardaient le
parlement. Dans cette vue et de faire les pères du peuple, comme
l'affectent tous ceux qui pour leurs intérêts particuliers veulent
brouiller et troubler l'État, {[}ils{]} mandèrent Trudaine, prévôt des
marchands et conseiller d'État, à leur venir rendre compte de l'état des
rentes de l'hôtel de ville, lequel prétendit qu'elles n'avaient jamais
été si bien payées, et qu'il n'y avait aucun lieu de s'en plaindre. De
là, ils s'en prirent à un édit rendu depuis peu sur la monnaie. Il fut
proposé d'envoyer les gens du roi représenter au régent qu'il était très
préjudiciable au royaume\,; mais, pour avoir l'air plus mesuré, ils
députèrent des commissaires à l'examen de l'édit. La cour prétendait,
qu'ayant été enregistré à la cour des monnaies, le parlement n'avait pas
droit de s'en mêler. Dans une nouvelle assemblée du parlement, il suivit
les errements qu'il avait pris dans la dernière régence et qui eurent de
si grandes suites. Il résolut de demander à la chambre des comptes, à la
cour des aides et à celle des monnaies, leur adjonction au parlement sur
cette affaire pour des remontrances communes, et manda les six corps des
marchands, et six banquiers principaux pour leur faire représenter le
préjudice que ce nouvel édit apportait à leurs intérêts et en général au
commerce. J'abrège et abrégerai tous ces manèges, parce que si je
voulais entrer dans tous ceux qui furent pratiqués au parlement et dans
les intérêts et les intrigues de tant de conducteurs de toutes ces
pratiques, il faudrait en écrire un volume à part, et qui serait fort
gros.

Les six banquiers et les députés des six corps des marchands comparurent
à la grand'chambre, qui leur demanda des mémoires. Ils répondirent que
l'affaire était assez importante pour en communiquer encore entre eux,
et qu'ils les apporteraient le lendemain. Les six banquiers particuliers
et affidés avaient lés leurs tout prêts qu'ils présentèrent\,; mais il
leur fut répondu d'attendre au lendemain à les fournir avec les
marchands. Ce lendemain qui fut le mercredi 15 juin, les uns et les
autres apportèrent leurs mémoires, mais la lecture en fut remise au
vendredi suivant, pour en conférer avec les autres cours, si elles se
joignaient au parlement. La chambre des comptes avait répondu qu'elle ne
pouvait rien sans avoir assemblé les deux semestres, et avoir su si ces
démarches seraient agréables au régent\,; la cour des aides, qu'elle
avait été assemblée tout le matin sans avoir pu prendre de résolution\,;
que ce serait pour le vendredi, et qu'elle enverrait en attendant à M.
le duc d'Orléans\,; celle des monnaies, qu'elle avait reçu une lettre de
cachet pour ne se point trouver au parlement. Le vendredi 17, le
parlement s'assembla le matin et l'après-dînée, puis députa au régent
pour lui demander la suspension de l'édit du changement des monnaies,
qu'on y fasse les changements dont le parlement sera d'avis, et qu'il
lui soit envoyé ensuite pour y être enregistré. La cour des aides
s'excusa de la jonction, et n'y voulut pas entendre\,; la chambre des
comptes l'imita incontinent après, dont le parlement fut fort fâché. Il
le fut aussi de ce que les six corps des marchands ne se plaignirent
point de l'édit. Il n'eut donc que les six banquiers pratiqués, qui se
plaignirent du ton qui leur fut inspiré. Le lendemain samedi, le
parlement s'assembla encore le matin et l'après-dînée. Il envoya les
gens du roi dire au régent qu'il ne se séparerait point qu'il n'eût en
sa réponse. Elle fut que Son Altesse Royale était fort lasse des
tracasseries du parlement\,; il pouvait employer un autre terme plus
juste\,; qu'il avait ordonné à toutes les troupes de la maison du roi
qui sont à Paris et autour, de se tenir prêtes à marcher, et qu'il
fallait que le roi fût obéi. L'ordre en effet en fut donné, et de se
pourvoir de poudre et de balles. Le lendemain dimanche, le premier
président, accompagné de tous les présidents à mortier et de plusieurs
conseillers, fut au Palais-Royal. Il était l'homme de M. et de
M\textsuperscript{me} du Maine, et le moteur des troubles\,; mais il y
voulait aussi pêcher, se tenir bien avec le régent, pour en tirer et se
rendre nécessaire, conserver en même temps crédit sur sa compagnie pour
la faire agir à son gré. Son discours commença donc par force louanges
et flatteries pour préparer à trois belles demandes qu'il fit\,:
première, que l'édit des monnaies fût envoyé au parlement pour
l'examiner, y faire les changements qu'il croirait y devoir apporter et
après l'enregistrer\,; seconde, que le roi eût égard à leurs
remontrances dans une affaire de cette conséquence, et que le parlement
croit fort préjudiciable à l'État\,; troisième, qu'on suspendit à la
monnaie le travail qu'on y faisait pour la conversion des espèces. Le
régent répondit à la première, que l'édit avait été enregistré à la cour
des monnaies, qui est cour supérieure, conséquemment suffisante pour cet
enregistrement\,; qu'il n'y avait qu'un seul exemple de règlement pour
les monnaies porté au parlement\,; qu'il n'y avait envoyé celui-ci que
par pure (il pouvait ajouter très sotte et dangereuse) complaisance pour
ses faux et traîtres confidents, valets du parlement, tels que les
maréchaux de Villeroy, d'Huxelles, et de Besons, Canillac, Effiat et
Noailles\,: à la seconde, que l'affaire avait été bien examinée et les
inconvénients pesés\,; qu'il était du bien du service du roi que l'édit
eût son entier effet\,: à la troisième, qu'on continuerait à travailler
à la conversion des espèces à la monnaie, et qu'il fallait que le roi
fût obéi.

Le lendemain lundi, le parlement s'assembla et rendit un arrêt contre
l'édit des monnaies. Le conseil de régence, qui se tint l'après-dînée du
même jour, cassa l'arrêt du parlement. Il fut défendu d'imprimer et
d'afficher ce bel arrêt du parlement, et on répandit des soldats du
régiment des gardes dans les marchés pour empêcher que la nouvelle
monnaie y fût refusée. Le parlement saisit une occasion spéciale, en ce
{[}que{]} les louis valant trente livres étaient pris à trente-six
livrés, et les écus de cent sous à six livres par, cet édit qui faisait
de plus passer des billets d'État, avec une certaine proportion d'argent
nouvellement refondu et fabriqué, quand la refonte aurait de quoi en
fournir à mesure. Cela soulageait le roi d'autant de papier, et il
gagnait gros à la refonte. Mais le particulier perdait à cette rehausse
qui excédait de beaucoup la valeur intrinsèque, et qui donnait lieu à
tout renchérir. Ainsi le parlement, pour se faire valoir, et ses moteurs
pour troubler, avaient beau jeu à prendre le masque de l'intérêt public,
et à tâcher d'ôter cette ressource aux finances qui n'en trouvaient
point d'autre. Aussi n'en manquèrent-ils pas l'occasion. On surprit la
nuit un conseiller au parlement, nommé la Ville-aux-Clercs, qui, à
cheval par les rues, arrachait et déchirait les affiches de l'arrêt du
conseil de régence, qui cassait l'arrêt du parlement rendu contre l'édit
des monnaies. Il fut conduit en prison. Le dimanche 26 juin, les six
corps des marchands vinrent déclarer au régent qu'ils ne se plaignaient
point de l'édit des monnaies, mais qu'ils le suppliaient seulement,
lorsqu'il jugerait à propos de diminuer les monnaies, que cela se fit
peu à peu. Le lundi 27 juin, le premier président à la tête de tous les,
présidents à mortier, et d'une quarantaine de conseillers, alla aux
Tuileries, où il lut au roi, en présence du régent, les remontrances
fort ampoulées du parlement. Le garde des sceaux lui dit que dans
quelques jours le roi leur ferait répondre. Cela se passa le matin à
l'issue du conseil de régence, qui se rassembla encore l'après-dînée
là-dessus. Il y en eut un autre extraordinaire le jeudi 30 au matin\,;
le garde des sceaux y lut un résumé plus de lui que des précédents
conseils sur cette affaire. Je m'y tins en tout fort réservé et fort
concis. J'étais en garde contre l'opinion que M. le duc d'Orléans avait
prise, que je haïssais le parlement depuis le bonnet. J'étais piqué de
la façon dont il s'était conduit dans cette affaire. Je l'étais de sa
mollesse à son propre égard, et de l'autorité du roi dans les diverses
échappées du parlement à ces égards, et je lui avais bien déclaré que
jamais je ne lui ouvrirais la bouche sur cette matière. Je tins parole
avec la plus ferme exactitude, et je ne voulus dire au conseil que ce
que je ne pouvais m'empêcher d'opiner, mais dans le plus simple et court
laconique, et peu fâché, car il faut l'avouer, de l'embarras du régent
avec le parlement. Au sortir de ce conseil, la chambre des comptes, et
après elle la cour des aides, vinrent faire leurs remontrances au roi,
mais fort mesurées, sur le même édit.

Le samedi 2 juillet, la même députation du parlement vint aux Tuileries
recevoir la réponse du roi\,; le garde des sceaux la fit en sa présence,
et de tout ce qui voulut s'y trouver. Le régent et tous les princes du
sang y étaient, les bâtards aussi. Argenson, si souvent malmené, et même
fortement attaqué par cette compagnie étant lieutenant de police, lui
fit bien sentir sa supériorité sur elle, et les bornes de l'autorité que
le roi lui donnait de juger les procès des particuliers sans qu'elle pût
s'ingérer de se mêler d'affaires d'État. Il finit par leur dire qu'il ne
serait rien changé à l'édit des monnaies, et qu'il aurait son effet tout
entier sans aucun changement. Ces, messieurs du parlement ne
s'attendaient pas à une réponse si ferme, et se retirèrent fort
mortifiés.

Pendant cette contestation les états de Bretagne, dès le premier ou le
second jour qu'ils furent assemblés, accordèrent le don gratuit par
acclamation à l'ordinaire\footnote{On a mis sur la marge du manuscrit la
  note suivante\,: «\,Il n'y eut point d'acclamation\,; on prit un
  \emph{mezzo-termine}, qui subsiste encore aujourd'hui.\,» Cette note
  est de la même main, qui avait ajouté les deux notes que nous avons
  reproduites précédemment.}. Cela se fit plus par le clergé et le tiers
état, que par la noblesse, laquelle insista fort à demander le rappel de
ses commissaires exilés, et qui envoya un courrier pour le demander au
régent. Outre le point d'honneur, l'attachement à se servir d'eux pour
l'examen des comptes de Montaran, leur receveur général, frère du
capitaine aux gardes, était leur principal objet. Les gens du roi
vinrent le mardi matin 11 juillet, demander au régent la permission que
le parlement fît au roi des remontrances sur sa réponse aux premières.
Cette demande forma unie nouvelle agitation. Le régent mené par ses
perfides confidents, l'accorda à la fin, mais avec différentes remises.
Le premier président, assez peu accompagné de députés du parlement, les
fit par un écrit qu'il présenta au roi le mardi matin 26 juillet, en
présence du régent, du garde des sceaux et de beaucoup de monde en
public, et quelques jours après les sieurs du Guesclair, de Bonamour et
de Noyan, demeurés à Paris par ordre du roi, eurent liberté de retourner
chez eux en Bretagne, mais avec défense d'aller aux états. Rochefort et
Lambilly, l'un président à mortier, l'autre conseiller au parlement de
Rennes, eurent aussi permission de retourner chez eux.

Il s'était présenté une question à juger sur les apanages, qui
intéressait Madame et M. le duc d'Orléans, et qui fut jugée en leur
faveur le samedi 30 juillet, au conseil de régence. Il n'y vint pas,
parce qu'il s'agissait de son intérêt, ni M. du Maine non plus, ce qui
parut très singulier de celui-ci. M. le Duc y présida\,; l'affaire fut
fort balancée. M. de Troyes et le marquis d'Effiat s'en abstinrent,
parce que les conseillers d'État qui avaient examiné l'affaire dans un
bureau exprès vinrent à ce conseil pour y opiner, lesquels, suivant leur
moderne prétention, et la faiblesse du régent, n'y cédaient qu'aux ducs
et aux officiers de la couronne.

Parmi tous ces mouvements du parlement et ceux de Bretagne, M. le duc
d'Orléans rétablit au roi devenu plus grand les cinq mille livres par
mois, qui lui avaient été retranchées depuis quelque temps, en sorte
qu'il eut comme auparavant dix mille livres par mois pour ses menus
plaisirs et aumônes, à quoi le bas étage de son service, qui en tirait
par-ci par-là, fut fort sensible.

Trudaine, conseiller d'État et prévôt des marchands, alla mandé chez le
premier président le jeudi 4 août, pour y rendre compte de l'état de
l'hôtel de ville aux commissaires du parlement, qui y étaient assemblés.
Échoués sur l'affaire des monnaies, ils cherchèrent à ressasser les
rentes pour s'attacher les rentiers et s'en servir s'ils pouvaient,
comme ils firent dans la dernière minorité, à commencer des troubles et
à usurper l'autorité. La Bretagne de concert marchait du même pied et
préparait de nouvelles brouilleries.

Ce fut dans ces circonstances que l'abbé Dubois revint de Londres après
y avoir achevé ce qu'on a ci-devant vu sur les affaires étrangères. En
même temps, Saint-Nectaire, maréchal de camp, qui avait quitté le
service quelques campagnes avant la fin de la dernière guerre, fut fait
seul lieutenant général. C'était un très bon officier général et de
beaucoup d'esprit et d'intrigue, qui faisait fort sa cour à qui pouvait
l'avancer, et qui avec tous les autres avait un air de philosophe et de
censeur. Il avait toujours été fort du grand monde et de la meilleure
compagnie. Ceux qu'il fréquentait le plus étaient La Feuillade, M. de
Liancourt, les ducs de La Rochefoucauld et de Villeroy. Mais à la fin
ils l'avaient démêlé et écarté. C'était un homme à qui personne, avec
raison, ne voulait se fier. Cette promotion, d'abord secrète, ne réussit
pas dans le monde lorsqu'elle y fut sue. Mais Saint-Nectaire n'en était
plus à son approbation, et comme que ce pût être voulait cheminer, M. le
duc d'Orléans n'alla point à la procession de l'Assomption, comme il
l'avait fait l'année précédente. Il consentit enfin à la profession de
M\textsuperscript{lle} sa fille. Le cardinal reçut ses voeux en l'abbaye
de Chelles dans la fin d'août. Madame, ni M. {[}le duc{]}, ni
M\textsuperscript{me} la duchesse d'Orléans n'y furent, ni aucun prince
ni princesse du sang. Il n'y eut même que très peu de personnes du
Palais-Royal qui s'y trouvèrent et quelques autres dames.
M\textsuperscript{me} la duchesse d'Orléans alla passer quelque temps à
Saint-Cloud, où Madame demeurait six mois tous les étés.

Le parlement s'assembla le 11 et le 12 août, et rendit enfin tout son
venin par l'arrêt célèbre dont voici le prononcé\,: «\,La cour ordonne
que les ordonnances et édits, portant création d'offices de finances et
lettres patentes concernant la banque registrées en la cour, seront
exécutés. Ce faisant, que la Banque demeurera réduite aux termes et aux
opérations portées par les lettres patentes des 2 et 20 mai 1716\,; et
en conséquence, fait défenses de garder ni de retenir directement ni
indirectement aucuns deniers royaux de la caisse de la Banque, ni d'en
faire aucun usage ni emploi pour le compte de la Banque et au profit de
ceux qui la tiennent, sous les peines portées par lés ordonnances\,;
ordonne que les deniers royaux seront remis et portés directement à tous
les officiers comptables, pour être par eux employés au fait de leurs
charges, et que tous les officiers et autres maniant les finances
demeureront garants et responsables en leurs propres et privés noms,
chacun à leur égard, de tous les deniers qui leur seront remis et portés
par la voie de la Banque\,; fait défenses en outre à tous étrangers,
même naturalisés, de s'immiscer directement ni indirectement, et de
participer sous des noms interposés au maniement ou dans
l'administration des deniers royaux, sous les peines portées par les
ordonnances et les déclarations enregistrées en la cour. Enjoint au
procureur général du roi, etc.\,»

On peut juger du bruit que fit cet arrêt\,; ce n'était rien moins
qu'ôter de pleine et seule autorité du parlement toute administration
des finances, les mettre sous la coupe de cette compagnie, rendre
comptables à son gré tous ceux que le régent y employait et lui-même,
interdire personnellement Law, et le mettre à la discrétion du parlement
qui aurait été sûrement plus qu'indiscrète. Après ce coup d'essai, il
n'y avait plus qu'un pas à faire pour que le parlement devînt en effet,
comme de prétention folle, le tuteur du roi et le maître du royaume, et
le régent plus en sa tutelle que le roi, et peut-être aussi exposé que
le roi Charles Ier d'Angleterre. Messieurs du parlement ne s'y prenaient
pas plus faiblement que le parlement d'Angleterre fit au commencement\,;
et quoique simple cour de justice, bornée dans un ressort comme les
autres cours du royaume à juger les procès entre particuliers, à force
de vent et de jouer sur le mot de parlement, ils ne se croyaient pas
moins que le parlement d'Angleterre, qui est l'assemblée législative et
représentante de toute la nation\footnote{Voyez la note I à la fin du
  volume.}.

Le prévôt des marchands fut mandé le 17 au parlement, où il fut traité
doucement\,; la compagnie, contente de sa vigueur, voulait régner, mais
capter les corps. Elle s'assembla presque continuellement pour délibérer
des moyens de se faire obéir et d'aller toujours en avant\,; les états
de Bretagne marchèrent en cadence et devinrent très audacieux\,;
Coetlogon-Mejusseaume fut exilé par une lettre de cachet\,: il était
syndic des états.

Dans tout ce bruit, M\textsuperscript{me} la duchesse du Maine eut
l'audace de s'aller plaindre fort hautement à M. le duc d'Orléans, de ce
qu'elle apprenait qu'il lui imputait beaucoup de choses. Par ce qui
éclata incontinent après, on peut juger de sa justification, que son
timide et dangereux époux n'osa hasarder lui-même. Le jugement du
conseil de régence, qui ôta aux bâtards la succession à la couronne, que
M. du Maine avait arrachée au feu roi, que toutes leurs menées n'avaient
pu empêcher, avait outré, à n'en jamais revenir, le mari et la femme,
qui ne songea plus qu'à exécuter ce qu'on a vu qu'elle avait dit à
Sceaux aux ducs de La Force et d'Aumont\,: \emph{Qu'elle mettrait tout
le royaume en feu et en combustion pour ne pas perdre cette
prérogative}. Les adoucissements énormes que M. le duc d'Orléans y mit
après l'arrêt, de son autorité absolue et pleine puissance, comme s'il
eût été roi, et dans le moment même, ne leur avaient paru qu'une marque
de sa faiblesse et une preuve de sa crainte, conséquemment une raison de
plus d'en profiter. Ils s'estimaient en trop beau chemin pour ne pas
pousser leur pointe. Tout riait à leurs projets cette partie de la
noblesse séduite, la Bretagne, le parlement de Paris, au point où ils le
voulaient contre le régent\,; l'Espagne, où ils disposaient
d'Albéroni\,; la révolte de tous les esprits contre la quadruple
alliance et contre l'administration des finances\,; le crédit que
donnait au renouvellement des infâmes bruits, l'affectation fastueuse et
maligne des plus folles précautions du maréchal de Villeroy sur le
manger et le linge du roi. Il ne s'agissait que d'endormir, en attendant
les moyens très prochains d'une exécution si flatteuse à la vengeance et
à l'ambition. Ce fut aussi à répandre ces mortifères pavots, très
nécessaires pour gagner un temps si cher et non encore tout à fait
imminent, que le rang, le sexe, l'esprit, l'éloquence, l'adresse,
l'audace de la duchesse du Maine lui parurent devoir être employés. Elle
sortit du cabinet, du régent, contente de leur effet, et le laissa plus
content encore de lui avoir persuadé de l'être.

Le parlement, assemblé le matin du 22 août, ordonna aux gens du roi de
savoir «\,ce que sont devenus les billets d'État qui ont passé à la
chambre de justice\,; ceux qui ont été donnés pour les loteries qui se
font tous les mois\,; ceux qui ont été donnés pour le Mississipi ou la
compagnie d'Occident\,; enfin ceux qui ont été portés à la monnaie
depuis le changement des espèces.\,» Les gens du roi allèrent au sortir
du palais dire au régent de quoi ils étaient chargés. Il leur répondit
froidement qu'ils n'avaient qu'à exécuter leur commission\,; ils
voulurent lui demander quelque instruction là-dessus. Le régent pour
toute réponse leur tourna le dos et s'en alla dans ses cabinets, dont
ils demeurèrent assez étourdis. Racontons maintenant comment le régent
remit le frein à ces chevaux qui avaient si bien pris le mors aux dents,
et qui se préparaient hautement à exciter les plus grands désordres. Le
détail en est curieux.

Aussitôt après la commission donnée par le parlement aux gens du roi,
dont on vient de parler, le bruit commença à se répandre d'un prochain
lit de justice. Ce n'était pas que le régent y eût encore pensé\,; il
n'était fondé que sur les monstrueuses entreprises du parlement dont
l'une n'attendait pas l'autre sur l'autorité royale, sur la nécessité
que les uns voyaient du seul moyen de les réprimer, sur la crainte qu'en
avaient les autres\,; mais ce qui était le grand ressort de tant
d'audace était l'opinion juste et générale qui avait prévalu de la
faiblesse du régent fondée sur toute sa conduite, surtout à l'égard de
ce qui se passait depuis longtemps à Paris et en Bretagne. Cela donnait
aux factieux la confiance de regarder un lit de justice comme une
entreprise à laquelle le régent n'oserait jamais se commettre, au point
où il avait laissé monter les liaisons et les entreprises. La lecture
des Mémoires du cardinal de Retz, de Joly, de M\textsuperscript{me} de
Motteville, avaient tourné toutes les têtes. Ces livres étaient devenus
si à la mode, qu'il n'y avait homme ni femme de tous états qui ne les
eût continuellement entre les mains. L'ambition, le désir de la
nouveauté, l'adresse des entrepreneurs qui leur donnait cette vogue,
faisait espérer à la plupart le plaisir et l'honneur de figurer et
d'arriver\,; et persuadait qu'on ne manquait non plus de personnages que
dans la dernière minorité. On croyait trouver le cardinal Mazarin dans
Law, étranger comme lui, et la Fronde dans le parti du duc et de la
duchesse du Maine\,; la faiblesse de M. le duc d'Orléans était comparée
à celle de la reine mère, avec la différence de plus de la qualité de
mère d'avec celle de cousin germain du grand-père du roi.

Les intérêts divers et la division des ministres et de leurs conseils
paraissaient les mêmes que sous Louis XIV enfant. Le maréchal de
Villeroy se donnait pour un duc de Beaufort, avec l'avantage de plus de
sa place auprès du roi, et de son crédit dans le parlement, sur qui on
ne comptait guère moins que sur celui de la dernière minorité. On
imaginait plusieurs Broussel, et on était assuré d'un premier président
tout à la dévotion de la Fronde moderne. La paix au dehors, dont l'autre
minorité ne jouissait pas, donnait un autre avantage à des gens qui
comptaient d'opposer au régent le roi d'Espagne, irrité contre lui en
bien des façons, avec les droits de sa naissance. Les manèges de la
Ligue contre Henri III n'étaient pas oubliés. M. du Maine, à la valeur
près, était un duc de Guise, et M\textsuperscript{me} sa femme une
duchesse de Montpensier. Pour en dire la vérité, tout tendait à
l'extrême, et il était plus que temps que le régent se réveillât d'un
assoupissement qui le rendait méprisable, et qui enhardissait ses
ennemis et ceux de l'État à tout oser et à tout entreprendre. Cette
léthargie du régent jetait ses serviteurs dans l'abattement et dans
l'impossibilité de tout bien. Elle l'avait conduit enfin sur le bord du
précipice, et le royaume qu'il gouvernait, à la veille de la plus grande
confusion.

Le régent, sans avoir eu l'horrible vice ni les mignons de Henri III,
avait encore plus que lui affiché la débauche journalière, l'indécence
et l'impiété, et, comme Henri III, était trahi dans le plus intérieur de
son conseil et de son domestique. Comme à Henri III, cette trahison lui
plaisait, parce qu'elle allait à le porter à ne rien faire, tantôt par
crainte, tantôt par intérêt, tantôt par mépris, tantôt par politique.
Cet engourdissement lui était agréable, parce qu'il se trouvait conforme
à son humeur et à son goût, et qu'il en regardait les conseillers comme
des gens sages, modérés, éclairés, que l'intérêt particulier
n'offusquait point, et qui voyaient nettement les choses telles qu'elles
étaient, tandis qu'il se trouvait importuné des avis qui allaient à lui
découvrir la véritable situation des choses, et qui lui en proposaient
les remèdes. Il regardait ceux-ci comme des gens vifs, qui précipitaient
tout, qui grossissaient tout, qui voulaient tirer sur le temps pour
satisfaire leur ambition, leurs aversions, leurs passions différentes.
Il se tenait en garde contre eux, il s'applaudissait de n'être pas leur
dupe. Tantôt il se moquait d'eux, souvent il leur laissait croire qu'il
goûtait leurs raisons, qu'il allait agir et sortir de sa léthargie. Il
les amusait ainsi, tirait de long, et s'en divertissait après avec les
autres. Quelquefois il leur répondait sèchement, et quand ils le
pressaient trop, il leur laissait voir des soupçons.

Il y avait longtemps que je m'étais aperçu de la façon d'être là-dessus
de M. le duc d'Orléans. Je l'avais averti, comme on l'a vu, des premiers
mouvements du parlement, des bâtards, et de ce qui avait usurpé le nom
de la noblesse. J'avais redoublé, sitôt que j'en avais vu la cadence et
l'harmonie. Je lui en avais fait sentir tous les desseins, les suites,
combien il était aisé d'y remédier dans ces commencements, et difficile
après, surtout pour un homme de son humeur et de son caractère. Mais je
n'étais pas l'homme qu'il lui fallait là-dessus. J'étais bien le plus
ancien, le plus attaché, le plus libre avec lui de tous ses
serviteurs\,; je lui en avais donné les preuves les plus fortes, dans
tous les divers temps les plus critiques de sa vie et de son abandon
universel\,; il s'était toujours bien trouvé des conseils que je lui
avais donnés dans ces fâcheux temps\,; il était accoutumé d'avoir en moi
une confiance entière\,; mais quelque opinion qu'il eût de moi et de ma
vérité et probité, dont il a souvent rendu de grands témoignages, il
était en garde contre ce qu'il appelait ma vivacité, contre l'amour que
j'avais pour ma dignité si attaquée par les usurpations des bâtards, les
entreprises du parlement, et les modernes imaginations de cette
prétendue noblesse. Dès que je m'aperçus de ses soupçons, je les lui
dis, et j'ajoutai que, content d'avoir fait mon devoir comme citoyen et
comme son serviteur, je ne lui en parlerais pas davantage. Je lui tins
parole\,; il y avait plus d'un an que je ne lui en avais ouvert la
bouche de moi-même. Si quelquefois on lui en parlait devant moi, sans
que je pusse garder un total silence, qui eût été pris en pique et en
bouderie, je disais nonchalamment et faiblement quelque mot qui
signifiait le moins qu'il m'était possible, et qui allait à faire tomber
le propos.

Le retour d'Angleterre de l'abbé Dubois, dont la fortune ne
s'accommodait pas de la diminution de son maître, la frayeur que Law eut
raison de prendre que le parlement ne lui mît la main sur le collet, et
de se voir abandonné, la crainte pour sa place que conçut le garde des
sceaux, si haï du parlement pendant qu'il eut la police, firent une
réunion, à laquelle Law attira M. le Duc, si grandement intéressé dans
le système, lequel se proposa de saisir la conjoncture de culbuter le
duc du Maine, satisfaire sa haine et occuper sa place auprès du roi. Ce
concert de différents intérêts, qui aboutissaient au même point, forma
un effort qui entraîna le régent, et qui lui fit voir tout d'un coup son
danger et son unique remède, et le persuada qu'il n'y avait plus un
moment à perdre. Dubois et Law l'investirent contre ceux dont il n avait
que trop goûté et suivi les dangereux avis, et tout fut si promptement
résolu, que personne n'en eut aucun soupçon. C'est ce qu'il s'agit
maintenant d'exposer.

Dans ces circonstances que j'ignorais, travaillant à mon ordinaire une
après-dînée avec M. le duc d'Orléans, je fus surpris qu'interrompant ce
sur quoi nous en étions, il me parla avec amertume des entreprises du
parlement. J'en usai dans ma réponse avec ma froideur et mon air de
négligence accoutumé sur cette matière, et continuai tout de suite où
j'en étais. Il m'arrêta\,; me dit qu'il voyait bien que je ne voulais
pas lui répondre sur le parlement. Je lui avouai, qu'il était vrai, et
qu'il y avait longtemps qu'il pouvait s'en être aperçu. Pressé enfin, et
pressé outre mesure, je lui dis froidement qu'il pouvait se souvenir de
ce que je lui avais dit et conseillé avant et depuis sa régence sur le
parlement\,; que d'autres conseils, ou traîtres, ou pour le moins
intéressés à se faire valoir et à s'agrandir, en balançant le parlement
et lui, l'un par l'autre, avaient prévalu sur les miens\,; que, de plus,
il s'était laissé persuader que l'affaire du bonnet et ses suites ne me
laissaient pas la liberté de penser de sang-froid sur le parlement ni
sur les bâtards, tellement que cela m'avait fermé la bouche comme je
l'en avais averti, et au point que j'aurais beaucoup de peine à la
rouvrir sur cette matière\,; que néanmoins je voyais s'avancer à grands
pas l'accomplissement de la prophétie que je lui avais faite\,; que de
maître qu'il avait été longtemps de réprimer et de contenir le parlement
d'un seul froncement de sourcil, sa molle débonnaireté lui en avait tant
laissé faire, et de plus en plus entreprendre, qu'elle l'avait conduit
par degrés à ce détroit auquel il se trouvait maintenant, de se laisser
ôter toute l'autorité de sa régence, et peut-être encore de courir le
risque d'être obligé de rendre compte de l'usage qu'il en avait fait,
ou, de la revendiquer par des coups forcés, mais si violents qu'ils ne
seraient pas trop sûrs, et en même temps fort difficiles\,; que plus il
tarderait et pis ce serait\,; que, c'était donc à lui premièrement à se
bien sonder lui-même, y bien penser, ne se point flatter ni sur la chose
ni sur ce que lui-même se pouvait promettre de lui-même, et se
déterminer d'un côté ou d'un autre, et si tant était qu'il prît le parti
de vouloir ravoir son autorité, ne se pas livrer légèrement à le prendre
pour, une fois pris, ne pas tomber dans la faiblesse infiniment plus
grande et plus dangereuse, qui serait de commencer et ne pas achever, et
se livrer par là au dernier mépris, et conséquemment dans l'abîme. Un
discours si fort et si rare depuis longtemps dans ma bouche, arraché par
lui malgré moi, et prononcé avec une ferme et lente froideur, et comme
indifférente au parti qu'il voudrait prendre, lui fit sentir combien peu
je le croyais capable du bon, et de le soutenir jusqu'au bout, et
combien aussi je me mettais peu en peine de l'y induire. Il en fut
intérieurement piqué, et comme il était tenu à la suite de l'impression
que Dubois, Law et Argenson lui avaient faite et que j'ignorais
parfaitement, il opéra un effet merveilleux.

Le duc de La Force, lié à Law, poussait contre le parlement. Outre les
raisons générales, il espérait entrer par cette porte dans le conseil de
régence. Il me vint trouver pour l'y aider, et me dit que le régent lui
avait promis de l'y faire entrer tout à fait. On a vu d'ailleurs que je
n'avais pas approuvé qu'il fût entré dans le conseil des finances,
encore moins le personnage qu'il y avait fait, de sorte que je m'étais
fort refroidi avec lui. Il avait excité Law et d'Argenson, à qui il
avait fait peur, que son peu d'union avec Law, si vivement attaqué par
le parlement, ne donnât des soupçons au régent contre lui, s'il le
trouvait mou là-dessus. Il parlait à des gens qui avaient pour le moins
autant d'envie que lui pour leurs intérêts personnels de pousser le
régent, mais qui ne le lui disaient pas, et encore moins leurs démarches
là-dessus, que je sus par Law, presque aussitôt que le régent m'eût
parlé, comme je viens de le raconter. L'arrêt du parlement que j'ai
transcrit n'avait point été publié. Il transpira, il fut suivi de cette
commission de recherche par les gens du roi, et ce fut le coup qui
précipita les choses, et qui acheva de déterminer le régent. On sut que
le parlement, en défiance du procureur général, avait nommé d'autres
commissaires en son lieu, pour informer d'office\,; qu'on y
instrumentait très secrètement\,; qu'il y avait déjà beaucoup de témoins
ouïs de la sorte\,: que tout s'y mettait très sourdement en état
d'envoyer un matin quérir Law par des huissiers, ayant en main décret de
prise de corps, après ajournement personnel soufflé, et le faire pendre
en trois heures de temps, dans l'enclos du palais.

Sur ces avis, qui suivirent de près la publication de l'arrêt susdit, le
duc de La Force, et Fagon, conseiller d'État, dont j'ai parlé plus d'une
fois, allèrent le vendredi mâtin 19 août trouver le régent, et le
pressèrent tant qu'il leur ordonna de se trouver tous deux, dans la
journée, chez moi avec Law, pour aviser ensemble à ce qu'il fallait
faire. Ils y vinrent en effet, et ce fut le premier avertissement, que
j'eus que M. le duc d'Orléans commençait à sentir son mal et à consentir
à faire quelque chose. En cette conférence chez moi, je vis la fermeté
jusqu'alors grande de Law ébranlée jusqu'aux larmes, qui lui
échappèrent. Nos raisonnements ne nous satisfirent point d'abord, parce
qu'il était question de force, et que nous ne comptions pas sur celle du
régent. Le sauf-conduit dont Law s'était muni n'eût pas arrêté le
parlement un moment. De casser ses arrêts, point d'enregistrement à en
espérer\,; de lui signifier ces cassations, faiblesse que le parlement
mépriserait et qui l'encouragerait à aller plus avant. Embarras donc de
tous côtés. Law, plus mort que vif, ne savait que dire, beaucoup moins
que devenir. Son état pressant nous parut le plus pressé à assurer. S'il
eût été pris, son affaire aurait été faite avant que les voies de
négociation qui auraient été les premières suggérées et suivies par le
goût et la faiblesse du régent eussent fait place aux autres, sûrement,
avant qu'on eût eu loisir de se résoudre à mieux et d'enfoncer le palais
avec le régiment des gardes, moyen critique en telle cause, et toujours
fâcheux au dernier point, même en réussissant\,; épouvantable si, au
lieu de Law, on n'eût trouvé que le cadavre avec sa corde. Je conseillai
donc à Law de se retirer dès lors même dans la chambre de Nancré au
Palais-Royal, qui était fort son ami et actuellement en Espagne, et je
lui rendis la vie par ce conseil que le duc de La Force et Fagon
approuvèrent et que Law exécuta au sortir de chez moi. Il y avait bien
moyen de le mettre en sûreté en le faisant loger à la Banque\,; mais je
crus que la retraite au Palais-Royal ayant plus d'éclat frapperait et
engagerait le régent davantage et nous fournirait un véhicule assuré et
nécessaire par la facilité que Law aurait de lui parler à toute heure et
de le presser.

Cela conclu, le lit de justice fut par moi proposé et embrassé par les
trois autres comme le seul moyen qui restait de faire enregistrer la
cassation des arrêts du parlement. Mais, tandis que les raisonnements se
poussaient, je les arrêtai tout court par une réflexion qui me vint dans
l'esprit\,; je leur représentai que le duc du Maine, moteur si principal
des entreprises du parlement, et le maréchal de Villeroy d'autant plus
lié avec lui là-dessus qu'il s'en cachait plus soigneusement, ne
voudraient jamais d'un lit de justice si contraire à leurs vues, à leurs
menées, à leurs projets\,; que pour le rompre ils allégueraient la
chaleur qui en effet était extrême, la crainte de la foule, de la
fatigue, du mauvais air\,; qu'ils prendraient le ton, pathétique sur la
santé du roi très propre à embarrasser le régent\,; que, s'il persistait
à le vouloir, ils protesteraient contre ce qui en pouvait arriver au
roi, déclareraient peut-être que, pour n'y point participer, ils ne l'y
accompagneraient pas\,; que le roi, préparé par eux, s'effaroucherait
peut-être et ne voudrait pas aller au parlement sans eux\,; alors tout
tomberait, et l'impuissance du régent si nettement manifestée pouvait
conduire bien loin et bien rapidement\,; que, si le lit de justice
n'était que disputé, ces deux hommes auraient encore à faire débiter et
répandre à la suite de toutes lés artificieuses précautions nouvellement
prises pour la conservation du roi avec une affection si marquée,
qu'entre le roi et Law le régent balançait d'autant moins qu'un lit de
justice dans une saison si dangereuse était un moyen simple et doux à
tenter, qui avait flatté le régent et qui lui en pouvait épargner de
plus difficiles. Ces réflexions arrêtèrent tout court, mais j'en montrai
aussitôt après le remède, par la proposition que je fis de tenir le lit
de justice aux Tuileries. Par cet expédient, nulle nécessité d'avertir
personne que le matin même qu'il se tiendrait, et par ce secret chacun
hors de mesure et de garde nul prétexte par rapport au roi, et toute
liberté, soit par rapport au peuple, soit par rapport à la force dont on
pourrait avoir besoin, laquelle serait plus crainte et plus sûre, sans
sortir de chez le roi qu'au palais. Ce fut à quoi nous nous arrêtâmes,
et Law parti, je dictai un mémoire à Fagon de tout ce que j'estimais
nécessaire tant pour conduire ce dessein avec secret, que pour en
assurer l'exécution, et en prévenir tous les obstacles. Sur les neuf
heures du soir nous eûmes fait\,; je lui conseillai de le porter à
l'abbé Dubois, revenu d'Angleterre avec un crédit nouveau sur l'esprit
de son maître. J'avais su par Law, avant cette conférence, ce que j'ai
expliqué ci-dessus des sentiments de cet abbé et du garde des sceaux, et
de leur résolution de presser le récent de se tirer de page. Dans la
visite que Dubois me rendit le surlendemain de son arrivée, où il me
rendit poliment compte de sa négociation en homme qui ne demande pas
mieux pour s'attirer des applaudissements, nous traitâmes après la
matière du parlement. Il m'y avait paru dans de bons sentiments. C'était
un personnage duquel on ne pouvait espérer de se passer dans sa
situation présente auprès du régent, et nous comptions de nous en servir
pour achever de déterminer son maître. Tel fut le plan du vendredi 19
août, qui fut le premier jour que j'entendis pour la première fois
parler sérieusement que le régent, enfin alarmé, voulait faire quelque
chose pour se tirer des pattes de la cabale et de celles du parlement.
Il faut remarquer que depuis le 12 août, jour de son arrêt célèbre, nous
étions bien avertis de ce qui se brassait pour aller vigoureusement en
avant, et de sa résolution de commettre pour l'information susdite de ce
qu'étaient devenus les différents billets d'État, quoiqu'elle né fût
consommée et annoncée au régent par les gens du roi que le 22 août,
trois jours après la conférence dont je viens de parler, tenue chez moi
le vendredi 19 août, qui dura toute l'après-dînée jusqu'à neuf heures du
soir.

Le lendemain samedi 20 août, sur la fin de la matinée, M. le duc
d'Orléans me manda de me trouver chez lui sur les quatre heures de
l'après-dînée du même jour. Un peu après, Fagon me vint dire qu'il avait
trouvé l'abbé Dubois tout vacillant, et à propos de rien \emph{tout
d'Aguesseau}, dont il était auparavant ennemi\,; qu'il lui avait parlé
du parlement en modérateur, et tenu de mauvais propos d'Argenson, qui
était pourtant son ami particulier. Cela me donna fort à penser d'un
cerveau étroit, qui tremble sur le point d'une exécution nécessaire,
d'un homme jaloux de ce que son maître avait, sans lui en parler, envoyé
le duc de La Force, Fagon et Law conférer chez moi\,; enfin qu'ambitieux
sans mesure, fier de la conclusion de son traité de Londres, il voulût
en tirer le fruit, imaginait peut-être de faire tomber les cris
universellement émus contre ce traité et contre lui, en se mettant entre
le régent et le parlement, comme un homme tout neuf\,; se faire honneur
d'une sorte de misérable conciliation, dont le régent serait la dupe,
flatter le parlement et le parti janséniste (car pour se faire entendre
il faut adopter les termes), en ramenant de Fresnes le chancelier. Ce
n'était pas pour avancer notre dessein, ni pour tirer le régent de page.
Fagon et le duc de La Force qui survint en parurent inquiets, quoique
contents de la situation d'esprit en laquelle ils venaient de laisser le
régent, à qui ils avaient rendu compte de ce qui s'était passé chez moi
la veille. Ils le furent beaucoup davantage de ce que je leur appris que
j'étais mandé au Palais-Royal pour l'après-dînée, dont le régent avec
ses demi-confidences accoutumées leur avait fait le secret. Fagon, en
habile homme, s'était bien gardé de confier notre mémoire à l'abbé
Dubois\,; sur la lecture qu'il lui en fit, il le laissa dans le goût
d'en faire un autre. L'abbé le lui avait apporté le matin. Il était plus
détaillé, mais il contenait des parties beaucoup moins fermes. Je ne
m'arrête point à ces mémoires\,; le récit de l'événement fera voir à
quoi ils aboutirent.

\hypertarget{chapitre-xvi.}{%
\chapter{CHAPITRE XVI.}\label{chapitre-xvi.}}

1718

~

{\textsc{Le régent m'envoie chercher.}} {\textsc{- Conférence avec lui
tête à tête, où j'insiste à n'attaquer que le parlement, et point à la
fois le duc du Maine, ni le premier président, comme M. le Duc le
veut.}} {\textsc{- Marché de M. le Duc, moyennant une nouvelle pension
de cent cinquante mille livres.}} {\textsc{- Conférence entre M. le duc
d'Orléans, le garde des sceaux, La Vrillière, l'abbé Dubois et moi, à
l'issue de la mienne tête à tête.}} {\textsc{- M. le Duc survient\,; M.
le duc d'Orléans le va entretenir, et nous nous promenons dans la
galerie.}} {\textsc{- Propos entre M. le duc d'Orléans, M. le Duc et
moi, seuls, devant et après la conférence recommencée avec lui.}}
{\textsc{- Je vais chez Fontanieu, garde-meuble de la couronne, pour la
construction très secrète du matériel du lit de justice.}} {\textsc{-
Contre-temps que j'y essuie.}} {\textsc{- Effroi de Fontanieu, qui fait
après merveilles.}} {\textsc{- M. le Duc m'écrit, me demande un
entretien dans la matinée chez lui ou chez moi, à mon choix.}}
{\textsc{- Je vais sur-le-champ à l'hôtel de Condé.}} {\textsc{- Long
entretien entre M. le Duc et moi.}} {\textsc{- Ses raisons d'ôter à M.
du Maine l'éducation du roi.}} {\textsc{- Les miennes pour ne le pas
faire alors.}} {\textsc{- M. le Duc me propose le dépouillement de M. du
Maine.}} {\textsc{- Je m'y oppose de toutes mes forces\,; mais je
voulais pis à la mort du roi.}} {\textsc{- Mes raisons.}} {\textsc{-
Dissertation entre M. le Duc et moi sur le comte de Toulouse.}}
{\textsc{- M. le Duc propose la réduction des bâtards, si l'on veut, à
leur rang de pairs parmi les pairs.}} {\textsc{- M. le Duc veut avoir
l'éducation du roi, sans faire semblant de s'en soucier.}} {\textsc{-
Raisons que je lui objecte.}} {\textsc{- Discussion entre M. le Duc et
moi, sur l'absence de M. le comte de Charolais.}} {\textsc{- M. le Duc
me sonde sur la régence, en cas que M. le duc d'Orléans vînt à manquer,
et sur les idées de M\textsuperscript{me} la duchesse d'Orléans
là-dessus pour faire M. son fils régent, et le comte de Toulouse
lieutenant général du royaume.}} {\textsc{- Je rassure M. le Duc sur ce
qu'en ce cas la régence lui appartient.}} {\textsc{- Conclusion de la
conversation.}} {\textsc{- M. le Duc déclare que son attachement au
régent dépend de l'éducation.}} {\textsc{- Je donne chez moi à Fontanieu
un nouvel éclaircissement sur la mécanique dont il était chargé.}}

~

Je me rendis sur les quatre heures au Palais-Royal\,; un moment après,
La Vrillière y vint, qui me soulagea de la compagnie de Grancey et de
Broglio, deux des roués, que j'avais trouvés dans le grand cabinet au
frais, familièrement, sans perruques. Nous ne fûmes pas longtemps sans
être avertis d'entrer dans la galerie neuve, peinte par Coypel, où nous
trouvâmes quantité de cartes et de plans des Pyrénées, qu'Asfeld
montrait au régent et au maréchal de Villeroy, M. le duc d'Orléans me
reçut avec une ouverture et des caresses qui sentaient le besoin. Un
moment après, il me dit bas qu'il avait fort à m'entretenir avant que
nous fussions assemblés, mais qu'il fallait laisser sortir le maréchal
c'était le premier mot que j'entendais d'assemblée\,; je ne savais donc
avec qui\,; La Vrillière me demanda si j'avais affaire au régent. Je lui
dis que oui. Il me répondit qu'il était mandé à quatre heures. «\,Et moi
aussi,\,» répartis-je. Le maréchal me prit après en particulier, avec
ses bavarderies et ses protestations accoutumées sur les précautions
qu'il venait de prendre sur la personne du roi, avec une sorte d'éclat
plat et malin, et sur les avis anonymes qui lui pleuvaient, et dont M.
du Maine et lui étaient peut-être les auteurs. Enfin il s'en alla avec
la compagnie. Alors M. le duc d'Orléans se mit à respirer, et me mena
dans les cabinets derrière le grand salon sur la rue de Richelieu.

En y entrant, il me prit par le bras, et me dit qu'il était à la crise
de sa régence, et qu'il s'agissait de tout pour lui en cette occasion.
Je répondis que je ne le voyais que trop\,; que le tout ne dépendait que
de lui dans une conjoncture si critique. Nous étions à peine assis que
l'abbé Dubois entra, qui lui parla par énigmes sur le parlement. Il me
parut qu'il y était question de menées, de découvertes, du duc de
Noailles, et du président. Le régent reçut assez mal l'abbé Dubois, en
homme pressé de s'en défaire, le renvoya, défendit qu'on l'interrompît,
excepté pour l'avertir de l'arrivée du garde des sceaux\,; et encore à
travers la porte qu'il alla fermer au verrou. Alors je lui dis qu'avant
d'entrer en matière, j'avais à l'avertir de ce que Fagon avait remarqué
le matin en l'abbé Dubois, sur le chancelier et le garde des sceaux\,;
et que Dubois avait marché comme sur des oeufs à l'égard du parlement.
J'y ajoutai mes réflexions. Le régent me répondit que cela se rapportait
à ce que lui-même avait aperçu de l'abbé, qui ne lui avait loué que le
chancelier, qu'il avait tant haï auparavant, fort mal parlé du garde des
sceaux, et du parlement, en effet, comme en marchant sur des oeufs. Mes
réflexions lui parurent fondées\,: c'étaient les mêmes que je viens
d'expliquer. Je l'exhortai à la défiance sur cet article d'un homme si
promptement changé, et sans cause apparente. Il m'assura que Dubois ne
le trahirait pas\,; mais il convint aussi que la sonde à la main sur les
matières présentes était le meilleur parti. Après ce court préambule,
nous entrâmes en matière. Il me dit qu'il était résolu à frapper un
grand coup sur le parlement\,; qu'il approuvait beaucoup le lit de
justice aux Tuileries, par les raisons qui me l'avaient fait proposer là
plutôt qu'au palais\,; qu'il était assuré de M. le Duc, moyennant une
nouvelle pension de cent cinquante mille livres, comme chef du conseil
de régence, et qu'il avait aussi de ce matin la parole de M. de Conti\,;
que M. le Duc voulait que l'éducation du roi fût ôtée au duc du Maine,
chose qui était aussi de son intérêt à lui, parce que le roi avançait en
âge et en connaissance\,; qu'il lui était important d'ôter de là son
ennemi\,; qu'ainsi il avait envie de tenir le lit de justice, s'il le
pouvait, dès le mardi suivant, et là d'ôter l'éducation au duc du Maine.

Je l'interrompis, et lui dis nettement que ce n'était point là mon avis.
«\,Eh\,! pourquoi n'est-ce pas votre avis, m'interrompant à son tour.
--- Parce, lui dis-je, que c'est trop entreprendre à la fois. Quelle est
maintenant votre affaire urgente avant toute autre, et qui ne souffre
point de délais\,? C'est celle du parlement\,: voilà le grand point\,;
contentez-vous-en. Frappant dessus un grand coup, et le sachant soutenir
après, vous regagnez en un instant toute votre autorité, après quoi vous
aurez tout le temps de penser au duc du Maine. Ne le confondez point
avec le parlement\,; ne l'identifiez point avec lui\,: par leur disgrâce
commune, vous les joignez d'intérêt. Il sera et se professera le martyr
du parlement\,; conséquemment du public dans l'esprit qu'ils ont su y
répandre. Voyez donc auparavant ce que le public fera et pensera de
l'éclat que vous allez faire contre le parlement. Vous n'avez pas voulu
abattre M. du Maine, lorsque vous le pouviez et le deviez, lorsque le
public et le parlement s'y attendaient et le désiraient ouvertement\,;
vous avez laissé pratiquer l'un et l'autre au duc du Maine à son aise,
et vous le voulez ôter à contre-temps. D'ailleurs, espérez-vous que cet
affront ne vous conduise pas plus loin\,? Mais de plus, M. le Duc
veut-il l'éducation ou se contente-t-il de l'ôter à M. du Maine\,? ---
Il ne s'en soucie pas, me répondit le régent. --- À la bonne heure, lui
dis-je\,; mais tâchez donc de lui faire entendre raison sur le moment
présent qui vous engage à un trop fort mouvement. Pensez encore,
monsieur, ajoutai-je, que quand je m'oppose à l'abaissement de M. du
Maine, je combats mon intérêt le plus cher\,: de l'éducation au rang il
n'y a pas loin\,: vous connaissez sur ce point l'ardeur de mes désirs,
et que d'ailleurs je hais parfaitement M. du Maine, qui nous a, par
noirceur profonde et pourpensée, induits forcément au bonnet, et, de
dessein prémédité, nous a coûté tout ce qui s'en est suivi\,; mais le
bien de l'État et le vôtre m'est plus cher que mon rang et ma vengeance,
et je vous conjure d'y bien faire toutes vos réflexions.\,»

Le régent fut surpris autant peut-être de ma force sur moi-même que de
celle de mes raisons. Il m'embrassa, me céda tout court\,; me dit que je
lui parlais en ami, non en duc et pair. J'en pris occasion de quelques
légers reproches de ses soupçons à cet égard. Nous convînmes donc de
laisser le duc du Maine pour une autre fois non compliquée. M. le duc
d'Orléans revint au parlement et me proposa de chasser le premier
président. Je m'y opposai de même, et lui dis que cet homme tenait trop
au duc du Maine pour frapper sur lui en laissant l'autre entier\,; que
rien n'était plus dangereux que d'offenser à demi un homme aussi
puissamment établi et aussi méchant que le duc du plaine\,; qu'il
fallait attendre pour l'un comme pour l'autre\,; qu'en cela encore je
lui parlais en ami, contre moi-même, puisque mon plaisir le plus
sensible serait de perdre un scélérat, auteur et instrument de toutes
les horreurs qui nous étaient arrivées\,; qu'il fallait, au contraire,
le caresser en apparence et faire accroire, malgré lui, au parlement
qu'il avait été dans la bouteille, pour achever de le perdre dans sa
compagnie et achever après de le déshonorer par faire publier tout
l'argent qu'il a eu depuis la régence et ses infamies avec Bourvalais\,;
qu'éreinté de la sorte, on s'en déferait après bien aisément, quand il
serait temps de tomber sur le duc du Maine. Le régent me loua et me
remercia encore, et convint que j'avais raison. Il me dit qu'il était
résolu de suivre le mémoire que j'avais dicté à Fagon et point celui de
l'abbé Dubois. Celui-ci voulait différer le lit de justice jusqu'après
la Saint-Martin, se contenter maintenant de casser les arrêts du
parlement, et attendre aux vacances à exiler plusieurs membres mutins de
cette compagnie. Et moi, au contraire, je voulais précipiter les
coups\,; tant sur le général que sur les particuliers. Après avoir bien
discuté tous les inconvénients et leurs remèdes, nous vînmes à la
mécanique. Je la lui expliquai telle que je l'imaginais, et je me
chargeai, à la prière du régent, de la machine matérielle du lit de
justice, par Fontanieu, garde meuble de la couronne, à l'insu de tout le
monde, et particulièrement du duc d'Aumont, son supérieur comme premier
gentilhomme de la chambre en année, et valet à gage de M. du Maine et du
premier président.

Il y avait déjà longtemps que le barde des sceaux était annoncé. Tout
ceci concerté, le régent passa dans le salon qui joignait les cabinets
où nous étions, et de la porte appela le garde des sceaux, La Vrillière
et l'abbé Dubois, qui attendaient dans le salon à l'autre bout, où ils
étaient seuls. C'était le lieu où M. le duc d'Orléans travaillait l'été.
Il était le dos à la muraille du cabinet de devant, assis au milieu de
la longueur d'un grand bureau en travers devant lui il prit sa place
ordinaire, moi à côté de lui, le garde des sceaux et l'abbé Dubois
vis-à-vis, la largeur du bureau entre eux et nous, La Vrillière au bout
le plus proche de moi. Après une assez courte conversation sur la
matière, le garde des sceaux lut le projet d'un arrêt du conseil de
régence et de lettres patentes, tel que ces pièces furent imprimées
après, en cassation des arrêts du parlement, etc., où nous ne fîmes que
quelques légers changements. L'abbé Dubois contredit tout, au point que,
pour l'adresse, je le crus animé de l'esprit double et parlementaire du
chancelier. Nous disputâmes tous et tout d'une voix contre lui. Il en
fut enfin embarrassé, mais non pas jusqu'à changer rien de sa
surprenante contradiction. Comme la lecture venait de finir, M. le Duc
fut annoncé. M. le duc d'Orléans prit, sa perruque et l'alla voir dans
le cabinet de devant. Le garde des sceaux nous proposa de nous promener
cependant dans la galerie. Nous y fîmes deux ou trois tours pendant
lesquels la dispute ne cessa point entre Argenson et Dubois. La
Vrillière et moi en haussions les épaules et soutenions le garde des
sceaux. La Vrillière cependant me montra un projet de déclaration de
suppression de charges nouvelles du parlement, qui me parut très bon.

Peu après j'entendis ouvrir la porte du salon qui donne dans ce grand
cabinet, où Son Altesse Royale était allée trouver M. le Duc\,;
j'avançai devant les autres, et vis, le régent et M. le Duc derrière
lui\,; j'allai à eux, et comme j'étais au fait de leur intelligence, je
demandai en riant à M. le duc d'Orléans ce qu'il voulait faire de M. le
Duc, et pourquoi l'amener ainsi dans son intérieur pour nous
embarrasser. «\,Vous l'y voyez, me répondit-il, en prenant M. le Duc par
le bras, et vous l'y verrez encore bien davantage.\,» Alors les
regardant tous deux, je leur témoignai ma joie de leur union, et
j'ajoutai que c'était leur véritable intérêt, et non pas de se joindre à
la bâtardise. «\,Oh\,! pour celui-ci, dit le régent à M. le Duc, en me
prenant par les épaules, vous pouvez parler en toute confiance, car
c'est bien l'homme du monde qui aime le mieux les légitimes et leur
union, et qui hait le plus cordialement les bâtards.\,» Je souris, et
répondis une confirmation nette et ferme\,; M. le Duc, des respects à
Son. Altesse Royale, et des honnêtetés à moi. Nous nous approchâmes du
bureau. Les autres cependant, restés dans le bout le plus proche de la
galerie, me parurent fort étonnés de ce qu'ils voyaient lorsque je me
retournai vers eux\,; ils s'approchèrent, et en même temps nous reprîmes
nos places au bureau. M. le Duc se mit entre M. le duc d'Orléans et moi.
Son Altesse Royale, après un petit mot très léger sur M. le Duc, pria le
garde des sceaux de recommencer sa lecture\,; elle se fit presque de
suite avec très peu d'interruption. M. le Duc l'approuva fort et m'en
parlait bas de fois à autre. Quand elle fut achevée, M. le duc d'Orléans
se leva, appela M. le Duc, le mena à l'autre bout du salon, et m'y
appela un moment après. Là, il me dit qu'ils allaient raisonner sur la
mécanique, que la plus pressée de toutes ses différentes parties était
celle du lit de justice, et qu'il me priait de m'en aller sur-le-champ
chez Fontanieu pour cela. En les quittant, j'élevai la voix et dis à Son
Altesse Royale que La Vrillière m'avait montré dans la galerie un projet
de déclaration fort bon à voir.

Comme je fus à la galerie des hommes illustres, je m'entendis appeler\,;
c'était l'abbé Dubois. Il ne me fit point de question, ni moi à lui\,;
mais nous avions envie de savoir tous deux pourquoi chacun de nous
sortait, et nous ne nous le dîmes point. Comme j'allais monter en
carrosse, un laquais de Law, en embuscade me dit que son maître me
priait instamment d'entrer dans sa chambre qui était tout contre\,:
c'était le logement de Nancré. Je l'y trouvai seul avec sa femme, qui
sortit aussitôt\,; je lui dis que tout allait bien, et que M. le Duc
avait été avec nous et était demeuré chez Son Altesse Royale\,; je
savais par elle que c'était Law qui avait été l'instrument de leur
union. J'ajoutai que j'étais pressé pour une commission nécessaire à ce
dont il s'agissait\,; qu'il en saurait davantage par Son Altesse Royale
ou par moi dès que je le pourrais. Il me parut respirer\,; je m'en allai
delà chez Fontanieu à la place de Vendôme.

On a vu au temps de la chambre de justice dont les taxes furent portées
au conseil de la régence, que Fontanieu en fut quitte à bon marché par
le service que je lui fis. Il avait marié sa fille à Castelmoron, fils
d'une soeur de M. de Lauzun qui m'en avait instamment prié. M. et
M\textsuperscript{me} de Lauzun avaient lors, une affaire pour
l'acquisition, par une sorte de retrait lignager\footnote{Le retrait
  lignager était le droit qu'avait un parent de la ligne, par où était
  venu un héritage, de reprendre le bien, lorsqu'il avait été aliéné.},
de la terre de Randan, du feu duc de Foix, laquelle devait demeurer à
M\textsuperscript{me} de Lauzun après son mari. Cela se décidait devant
des avocats commis, et Fontanieu conduisait toute cette affaire. On me
dit chez lui qu'il y était allé, et c'était au fond du Marais que ces
avocats s'assemblaient. Le portier me vit si fâché de l'aller chercher
là, qu'il me dit que, si je voulais voir M\textsuperscript{me} de
Fontanieu, il irait voir si son maître n'était point encore dans le
voisinage où il était allé d'abord, pour de là aller au Marais. J'allai
donc voir M\textsuperscript{me} de Fontanieu qui était souvent à l'hôtel
de Lauzun et que je trouvai seule. J'eus donc le passe-temps de
l'entretenir, avec tout ce que j'avais dans la tête, de cette, affaire
de M\textsuperscript{me} de Lauzun\,; ce fut mon prétexte d'avoir à
parler à Fontanieu d'un incident pressé qui y, était survenu. Fontanieu,
qu'on trouva encore au voisinage, arriva bientôt\,; ce fut un autre
embarras que de me dépêtrer de leurs instances à tous les deux de
traiter là cette affaire sans me donner la peine de descendre chez
Fontanieu, et comme la femme en était informée autant que le mari, je
vis le moment que je ne m'en tirerais pas. J'emmenai pourtant à la fin
Fontanieu chez lui, à force de compliments à la femme de ne la vouloir
pas importuner de la discussion de cette affaire de Randan.

Quand nous fûmes, Fontanieu et moi, en bas de son cabinet, je demeurai
quelques moments à lui parler de cela pour laisser retirer les valets
qui nous avaient ouvert les portes. Puis, à son grand étonnement,
j'allai dehors voir s'ils étaient sortis, et je fermai bien les portes.
Je dis après à Fontanieu qu'il n'était pas question de l'affaire de
M\textsuperscript{me} de Lauzun, mais d'une autre toute différente, qui
demandait toute son industrie et un secret à toute épreuve, que M. le
duc d'Orléans me chargeait de lui communiquer\,: mais qu'avant de
m'expliquer, il fallait savoir si Son Altesse Royale pouvait compter
entièrement sur lui. C'est une chose étrange que l'impression des plus
hautes sottises, dont la noirceur est répandue avec art. Le premier
mouvement de Fontanieu fut de trembler réellement de tout son corps et
de devenir plus blanc que son linge. Il balbutia à peine quelques mots,
qu'il était à Son Altesse Royale tant que son devoir le lui permettrait.
Je souris en le regardant fixement, et ce sourire l'avertit apparemment
qu'il me devait excuses de n'être pas en pleine assurance quand une
affaire passait par moi, car il m'en fit tout de suite, et avec
l'embarras d'un homme qui sent bien que la première vue lui a offusqué
la seconde, et qui, plein de cette première vue, n'ose rien montrer et
laisse tout voir. Je le rassurai de mon mieux, lui dis que j'avais
répondu de lui à M. le duc d'Orléans, et après, qu'il s'agissait d'un
lit de justice pour la construction duquel et sa position nous avions
besoin de lui. À peine m'en fus-je expliqué, que le pauvre homme se prit
à respirer tout haut, comme qui sort d'une oppression étouffante, et
qu'on lui eût ôté une pierre de taille de dessus l'estomac, et cela à
quatre ou cinq reprises tout de suite, en me demandant autant de fois si
ce n'était que cela qu'on lui voulait. Il promit tout dans la joie d'en
être quitte à si bon marché, et dans la vérité, il tint bien tout ce
qu'il promit, et pour le secret et pour l'ouvrage, il n'avait jamais vu
de lit de justice et n'en avait pas la moindre notion. Je me mis à son
bureau et lui en dessinai la séance. Je lui en dictai les explications à
côté parce que je ne voulus pas qu'elles fussent de ma main. Je
raisonnai plus d'une heure avec lui\,; je lui dérangeai ses meubles pour
lui mieux inculquer l'ordre de la séance et ce qu'il avait à faire faire
en conséquence avec assez de justesse pour n'avoir qu'à être transporté
et dressé tout prêt aux Tuileries en fort peu de moments. Quand je crus
m'être suffisamment expliqué, et lui avoir bien tout compris, je m'en
retournai au Palais-Royal comme par un souvenir, étant déjà dans les
rues, pour tromper mes gens. Un garçon rouge m'attendait au haut du
degré, et d'Ibagnet, concierge du Palais-Royal, à l'entrée de
l'appartement de M. le duc d'Orléans, avec ordre de me prier de lui
écrire. C'était l'heure sacrée des roués et du souper, contre laquelle
point d'affaire qui ne se brisât. Je lui écrivis donc dans son cabinet
d'hiver ce que je venais de faire, non sans indignation qu'il n'eût pu
différer ses plaisirs pour une chose de cette importance. Je fus réduit
encore à prier d'Ibagnet de prendre garde à ne lui donner mon billet que
quand il serait en état de le lire et de le brûler après. Je m'en fus de
là chez Fagon, que je ne trouvai pas, et après chez moi, où il était
venu. Bientôt après M. de La Force y arriva aux nouvelles, dont il fut
fort satisfait.

Le lendemain dimanche 21, sortant de mon lit à sept heures et demie, on
m'annonça un valet de chambre de M. le Duc, qui avait une lettre de lui
à me rendre en main propre, qui était déjà venu plus matin, et qui était
allé ouïr la messe aux Jacobins en attendant mon réveil. Je n'étais lors
ni n'avais jamais été en aucun commerce direct ni indirect avec lui.
J'en avais eu très peu lors, de son affaire contre les bâtards, mais
comme nous n'en avions pu tirer aucun parti pour la nôtre, j'avais perdu
de vue tous ces princes jusqu'à la messéance. Je passai dans mon cabinet
avec ce valet de chambre, et j'y lus la lettre que M. le Duc m'écrivait
de sa main, que voici\,:

«\,Je crois, monsieur, qu'il est absolument nécessaire que j'aie une
conversation avec vous sur l'affaire que vous savez\,; je crois aussi
que le plus tôt sera le mieux. Ainsi je voudrais bien, si cela se peut,
que ce fût demain dimanche, dans la matinée\,; voyez à quelle heure vous
voulez venir chez moi ou que j'aille chez vous\,; choisissez celui que
vous croirez qui marquera le moins, parce qu'il est inutile de donner à
penser au public. J'attendrai demain matin votre réponse, et vous prie
en attendant de compter sur mon amitié en me continuant la vôtre.

«\,\emph{Signé\,:} H. de Bourbon.\,»

Je rêvai quelques moments après l'avoir lue, et je me déterminai à voir
M. le Duc, que je ne pouvais éconduire, après quelques questions au
valet de chambre sur l'heure et le monde de son lever, à en tenter le
hasard plutôt que celui de le faire remarquer à ma porte par le
président Portail, qui en logeait vis-à-vis, et qui pouvait être chez
lui un dimanche matin. Je ne voulus point écrire, et je me contentai de
charger le valet de chambre de lui dire que je serais chez lui à l'issue
de son lever. Je n'étais pas achevé d'habiller que Fagon vint savoir des
nouvelles de la veille. Il en fut ravi, et encore plus du message de M.
le Duc par l'espérance que lui donnait cette suite pour un homme de
plus, et de ce poids par sa naissance, à soutenir M. le duc d'Orléans.
Je renvoyai Fagon promptement, et me rendis à l'hôtel de Condé, où je
trouvai M. le Duc qui achevait de s'habiller, et qui n'avait
heureusement que ses gens autour de lui, comme son valet de chambre me
l'avait fait espérer sur ce qu'il se devait lever ce jour-là plus tôt
que son ordinaire. Il me reçut en homme sage pour son âge, poliment,
mais sans empressement. Il me dit même que c'était une nouveauté que de
me voir. Je répondis que les conseils ayant presque toujours été le
matin, et lui peu à Paris les autres jours, je profitais avec plaisir du
changement de leur heure pour avoir l'honneur de le voir. Il fut achevé
d'habiller aussitôt, me pria de passer dans son cabinet, en ferma la
porte, me présenta un fauteuil, en prit un autre pareil, et nous nous
assîmes de la sorte vis à vis l'un de l'autre\,; il commença par des
excuses d'en avoir usé avec moi avec liberté, et après quelques
compliments il entra en matière.

Il me dit qu'il avait cru nécessaire de ne perdre point de temps à
m'entretenir sur l'affaire de la veille aussi nécessaire que pressante,
et que d'abord il me voulait demander avec confiance si je ne pensais
pas, comme lui le croyait, que ce n'était rien faire de frapper sur le
parlement, si du même coup on ne frappait pas sur son principal moteur,
et si M. le duc d'Orléans n'en jugeait pas de même. À ce que le régent
m'avait dit la veille, je m'étais bien douté du dessein de M. le Duc sur
moi\,; mais sans lui paraître stupide, je ne fus pas fâché de lui faire
nommer le premier le duc du Maine. J'en vins à bout par quelques souris
en balbutiant, et puis je lui demandai comment il l'entendait de frapper
sur M. du Maine. «\,En lui ôtant l'éducation,\,» me dit-il. Je répondis
que l'éducation se pouvait ôter indépendamment d'un lit de justice, et
les deux choses se faire à deux fois. Il repartit que M. le duc
d'Orléans était persuadé que cet emploi ayant été conféré ou confirmé au
duc du Maine dans un lit de justice, il ne se pouvait ôter que dans un
autre lit de justice. Je contestai un peu, mais il trancha court en me
disant que telle était l'opinion du régent, et l'opinion arrêtée, qu'il
le lui avait dit ainsi, sur quoi il était question de se servir de
l'occasion naturelle de celui qu'on allait tenir, d'autant qu'elle ne
reviendrait pas sitôt, et qu'il voulait savoir ce que je pensais
là-dessus.

Je battis un peu la campagne\,; mais je fus incontinent ramené par des
politesses de M. le Duc sur la confiance, et par une prière précise
d'examiner présentement avec lui, s'il n'était pas bon d'ôter le roi
d'entre les mains de M. du Maine par rapport à l'État et à l'intérêt
même de M. le duc d'Orléans, et supposé que cela fût, s'il ne valait pas
mieux le faire plus tôt que plus tard, et ne se pas commettre aux
irrésolutions du régent, au prétexte de la nécessité d'un autre lit de
justice, aux longueurs de le déterminer. Il fallut donc entrer tout de
bon en lice. J'avoue que plus j'avais réfléchi à ce qui regardait le duc
du Maine, et moins je croyais de sagesse à l'entreprendre. J'étais en
garde infiniment contre mon inclination là-dessus, et peut-être que la
rigueur que je m'y tenais m'en grossissait les inconvénients. J'avais
horreur de tremper dans les suites funestes à l'État d'une chose quoique
juste en elle-même par des intérêts particuliers, et plus cet intérêt
m'était cher et sensible, plus aussi je m'en détournais avec force pour
ne rien faire qu'en homme de bien. Je ne m'amusai donc plus au verbiage,
pressé comme je l'étais. Je répondis nettement à M. le Duc que les deux
points qu'il me proposait à discuter étaient infiniment différents\,;
qu'aucun esprit impartial et raisonnable ne pouvait nier qu'il ne fût
expédient à l'État, au roi, au régent, d'ôter l'éducation à M. du Maine,
mais que j'estimais qu'il n'y en avait aucun aussi qui n'en considérât
la démarche comme infiniment dangereuse. De là je lui détaillai avec
beaucoup d'étendue ce que je n'en avais dit qu'en raccourci à M. le duc
d'Orléans, parce qu'il s'était rendu d'abord, et que je voyais bien que
celui-ci n'était pas pour en faire de même. Je lui fis sentir de quel
prix l'éducation du roi était à M. du Maine, conséquemment quel coup
pour lui que de vouloir y toucher\,; quelle puissance il avait en
gouvernements et en charges pour la disputer, du moins pour brouiller
l'État\,; quelle force lui pouvait être ajoutée par le parlement frappé
du même coup pour leurs intrigues communes et leurs menées\,; quelle
autorité la réputation encore plus que les établissements du comte de
Toulouse apporterait à ce parti\,; que rien n'était plus à craindre,
conséquemment plus à éviter qu'une guerre civile, dont le chemin le plus
prompt serait d'attaquer M. du Maine.

M. le Duc m'écouta fort attentivement, et me répondit que pour lui il
croyait que l'attaquer était le seul remède contre la guerre civile. Je
le priai de m'expliquer cette proposition si contradictoire à la mienne,
et de me dire auparavant avec franchise ce qu'il pensait de la guerre
civile dans la situation où le royaume se trouvait\,; il m'avoua que ce
serait sa perte. Mais plein de son idée, il revint à ce que je lui avais
avoué qu'il était utile d'ôter le roi des mains de M. du Maine\,; que
cela posé, il fallait voir s'il y avait espérance certaine de le faire
dans un autre temps, et de le faire alors avec moins de danger\,; que
plus on laisserait le duc du Maine auprès du roi, plus le roi
s'accoutumerait à lui, et qu'on trouverait dans le roi un obstacle, qui
par son âge n'existait pas encore\,; que plus M. du Maine avait gagné de
terrain depuis la régence par la seule considération de l'éducation qui
le faisait regarder comme le maître de l'État à la majorité, plus il en
gagnerait de nouveau à mesure que le roi avancerait en âge, plus il
serait difficile et dangereux de l'attaquer\,; que son frère sûrement ne
remuerait point par probité et par nature\,; qu'à la vérité la
complication du parlement était une chose fâcheuse, mais que c'était un
mauvais pas à sauter\,; qu'il me parlerait sur M. le duc d'Orléans, non
comme à son ami intime, mais comme à un fort honnête homme et à un homme
sûr, en qui il savait qu'on pouvait se fier de tout\,; que, s'il était
persuadé d'obtenir une autre fois de lui l'éloignement de M. du Maine
d'auprès du roi, il n'insisterait pas à le vouloir à cette heure\,; mais
que je savais moi-même ce qui en était, et me priait de lui dire si,
cette occasion passée, il y devait compter\,; qu'il avait {[}eu{]} sa
parole de le faire à la mort du roi, puis le lendemain de la première
séance au parlement, enfin lors du procès des princes du sang\,; que
tant de manquements de parole et à une parole si précise et si souvent
réitérée non vaguement, mais pour des temps préfix, lui ôtaient
l'espérance, s'il laissait échapper l'occasion qui se présentait, et que
de là venait ce que je pouvais prendre pour opiniâtreté\,; et qui
pourtant n'était que nécessité véritable\,; que le régent était perdu si
M. du Maine demeurait auprès du roi jusqu'à la majorité\,; que les
princes du sang et lui nommément ne l'étaient pas moins\,; que cette
vérité ne pouvait pas être révoquée, en doute\,; qu'il y avait donc de
la folie à s'y commettre et à ne pas profiter de l'expérience et de
l'occasion\,; et qu'on se sentait assez de l'affermissement de M. du
Maine, pour ne le laisser pas affermir davantage.

Cela dit plus diffusément que je ne le rapporte, M. le Duc me pria de
lui répondre précisément. Je ne pus disconvenir des vérités qu'il avait
avancées. «\,Mais, lui dis-je, monsieur, cela empêche-t-il une guerre
civile\,? Tout cela montre bien l'énormité de la faute d'avoir laissé
subsister les bâtards à la mort du roi, et encore un peu depuis. Chacun
comptait sur leur chute et la souhaitait\,; mais à présent que les
choses ont changé de face par l'habitude et encore plus par le titre qui
leur semble donné, par le jugement intervenu entre les princes du sang
et eux, on est où on en était, et ce qui était sage à faire à la mort du
roi, et tôt après encore ou dans le jugement des princes du sang et
d'eux, ne nous précipitera-t-il pas dans des troubles en le faisant
présentement\,? Vous dites que la nature et la probité de M. le comte de
Toulouse l'empêchera de remuer\,: c'est une prophétie. Est-il apparent
qu'il ne s'intéresse pas en la chute de son frère\,; qu'il ne la regarde
pas comme sienne par nature, par intérêt, par honneur, par réputation,
qui à son égard mettra sa probité à couvert\,? Mais il y a plus,
monsieur\,; espérez-vous en demeurer là, et concevez-vous comme possible
de laisser l'artillerie et tout ce qui en dépend, les Suisses et les
autres troupes que M. du Maine commande avec la Guyenne et le Languedoc,
ces grandes et remuantes provinces dans la position où elles sont par
rapport à l'Espagne, entre les mains d'un homme aussi cruellement
offensé, à qui vous ravissez par la soustraction de l'éducation sa
sûreté et sa considération présente, et ses vastes vues pour l'avenir\,?
--- Hé bien, monsieur, interrompit M. le Duc, il n'y a qu'à le
dépouiller. --- Mais y pensez-vous, monsieur\,? lui dis-je. Voilà comme
de l'un on s'engage à l'autre. Il faut au moins un crime pour
dépouiller\,; et ce crime, où le prendre\,? Ce serait pour l'unir encore
plus avec le parlement, en alléguant pour crime ses menées, ses manèges
et ses intelligences avec cette compagnie. Et dans le temps présent
oserez-vous lui en faire un capital de ses liaisons avec l'Espagne,
supposé qu'on eût de quoi les prouver\,? L'un passera pour une
protection généreuse du bien public, l'autre pour un péché personnel
contre le régent, qui n'a rien de commun avec le roi et l'État. Que
deviendrez-vous donc si, après l'éducation ôtée, vous êtes réduit à en
demeurer là\,? Voilà pourquoi je les voulais culbuter dès la mort du
roi, et pour les dépouiller, leur faire justement alors un crime de
lèse-majesté d'avoir attenté à la couronne par s'en être fait déclarer
capables, leur faire grâce de la vie, de la liberté, des biens, de leur
dignité de duc et pair au rang de leur ancienneté du temps qu'ils l'ont
obtenue, et les priver de tout le reste\,; à cela personne qui n'eût
applaudi alors, personne qui n'eût trouvé le traitement doux, personne
qui n'eût vu avec joie la sagesse d'un frein qui empêcherait à jamais
qui que ce soit de lever les yeux jusqu'au trône. Le comte de Toulouse
lui-même, après avoir rendu ses sentiments publics là-dessus dans le
temps, eût été bien embarrassé d'agir contre, et voilà le cas où sa
probité et sa nature aurait pu suivre librement son penchant\,; mais
d'avoir, trois ans durant, accoutumé le monde à les confondre avec les
princes du sang, après avoir reculé au delà de l'injustice et de
l'indécence à juger entre les princes du sang et eux, après avoir par ce
jugement même confirmé, canonisé leur état, leurs rangs, tout ce qu'ils
sont et ont, excepté l'habilité à succéder à la couronne, et qui pis
est, laissé entrevoir que cette habilité de succéder à la couronne n'est
que faiblement retranchée et pour un temps très indifférent, puisque par
le même arrêt on leur laisse les rangs et les honneurs qui n'ont jamais
eu et ne peuvent jamais avoir que cette habilité pour base et pour
principe, et qui sont inouïs pour tout ce qui n'est pas né prince du
sang\,; puisqu'on leur laisse encore par l'éducation un moyen clair et
certain de revenir à cette habilité dans quatre ans, puisqu'on fortifie
ainsi l'habitude publique de les identifier avec les princes du sang par
un extérieur entièrement semblable, quel moyen de pouvoir revenir à leur
faire un crime de cet attentat à la couronne et un crime digne du
dépouillement\,? Or le dépouillement sans crime est une tyrannie qui
attaque chacun, parce que tout homme revêtu craint le même sort quand il
en voit l'exemple, et s'irrite d'un si dangereux déploiement de
l'autorité. Ne les dépouillez pas, ils auront lieu de craindre de
l'être, ils auront raison de remuer pour leur propre sûreté\,; sans
compter la vengeance, la rage, les fureurs de M\textsuperscript{me} du
Maine qui n'a pas craint ni feint de dire du vivant du roi, que, quand
on avait le rang, les honneurs, l'habilité à la couronne qu'avait
obtenus M. du Maine, il fallait renverser l'État plutôt que s'en laisser
dépouiller. Après cela, monsieur, continuai-je avec moins de chaleur
mais avec autant de force, vous devez croire que je suis vivement
pénétré de ces raisons et du bien de l'État pour persévérer dans l'avis
dont je suis, qu'il ne faut pas toucher à M. du Maine. Vous me faites
l'honneur de me parler avec confiance, je vous en dois au moins une
pareille\,; comptez que je sens très bien que le rang des bâtards est
inaltérable tant que l'éducation demeure à M. du Maine, et qu'en la lui
ôtant ce rang ne peut subsister. Pour cela il ne faut point de crime, il
ne faut que juger un procès intenté par notre requête, présentée en
corps au roi et au régent lors de votre procès. Il ne serait donc pas
sage de ne le pas faire en ôtant l'éducation, et ce serait les laisser
trop grands et trop respectables par leur extérieur\,; or, je veux bien
vous avouer que ma passion la plus vive et la plus chère est celle de ma
dignité et de mon rang, ma fortune ne va que bien loin après, et je la
sacrifierais et présente et future avec transport de joie pour quelque
rétablissement de ma dignité. Rien ne l'a tant et si profondément avilie
que les bâtards, rien ne me toucherait tant que de les précéder. Je le
leur ai, dit en face, et à M\textsuperscript{me} d'Orléans et à ses
frères, non pas une fois, mais plusieurs fois, et du vivant du feu roi,
et depuis\,; personne ne nous a tant procuré d'horreurs que M. du Maine
par l'affaire du bonnet\,; il n'y a donc personne dont j'aie un plus vif
désir de me venger que de lui\,; quand donc j'étouffe tous ces
sentiments pour le soutenir, il faut que le bien de l'État me paroisse
bien évident et bien fort, et je ne sais point pour moi d'argument plus
démonstratif à vous faire.\,»

M. le Duc, qui m'avait écouté avec une extrême attention, en fut
effectivement frappé et demeura quelques moments en silence\,; puis d'un
ton doux et ferme, que je crains infiniment en affaires, parce qu'il
marque que le parti est pris, et qu'il ne dépend d'aucun obstacle,
lorsqu'il suit tous ceux qu'on a montrés, me dit\,: «\,Monsieur, je
conçois très bien toutes les difficultés que vous faites, et je conviens
qu'elles sont grandes\,; mais il y en a deux autres qui me semblent à
moi incomparablement plus grandes de l'autre côté\,: l'une, que M. le
duc d'Orléans et moi sommes perdus à la majorité, si l'éducation demeure
à M. du Maine jusqu'alors\,; l'autre, qu'elle lui demeurera
certainement, si à l'occasion présente elle ne lui est ôtée. Ajustez
cela tout comme il vous plaira, mais voilà le fait\,: car de me fier à
ce que M. le duc d'Orléans me promettra, c'est un panneau où je ne
donnerai plus, et de me jouer à être perdu dans quatre ans, c'est ce que
je ne ferai jamais. --- Mais la guerre civile, lui repartis je. --- La
guerre civile, me répliqua-t-il, voici ce que j'en crois\,: M. du Maine
sera sage ou ne le sera pas. De cela on s'en apercevra bientôt en le
suivant de près. S'il est sage, comme je le crois, point de troubles.
S'il ne l'est pas, plus de difficulté à le dépouiller. --- Mais son
frère, interrompis-je, dont le gouvernement est demi-soulevé, s'il s'y
jette\,? --- Non, me dit-il, il est trop honnête homme, il n'en fera
rien. Mais il le faudra observer et l'empêcher d'y aller. --- En
l'arrêtant donc\,? ajoutai-je. --- Bien entendu, me dit-il, et alors il
n'y a pas d'autre moyen, et il le méritera, car il faut commencer par le
lui défendre. --- Mais, monsieur, lui dis-je, sentez-vous où cela vous
conduit\,? À pousser dans la révolte forcée et dans le précipice
d'autrui un homme adoré et adorable par son équité, sa vertu, son amour
pour l'État, son éloignement des folles vues de son frère, dans le
soutien duquel il se perdra par honneur, comme vous avez vu qu'il s'est
donné tout entier à leur procès contre vous, bien qu'il en sentît tout
le faible, et qu'il en eût toujours désapprouvé l'engagement. Je vous
avoue que l'estime que j'ai conçue pour lui depuis la mort du roi est
telle qu'elle a gagné mon affection, et ce dont je m'émerveille, qu'elle
a eu la force d'émousser l'ardeur de mon rang à son égard. Vous, qui
êtes son neveu, et dont il a pris soin à votre première entrée dans le
monde, n'êtes-vous point touché de sa considération\,? --- Moi, me
dit-il, j'aime M. le comte de Toulouse de tout mon cœur, je donnerais
toutes choses pour le sauver de là. Mais quand c'est nécessité, et qu'il
y va de ma perte et de troubler l'État\ldots. Car enfin, monsieur, me
laisserai-je écraser dans quatre ans\,; et en verrai-je quatre ans
durant la perspective tranquillement\,? Mettez-vous en ma place\,:
troubles pour troubles, il y en aura moins à présent qu'en différant,
parce qu'ils croîtront toujours en considération et en cabales, et
peut-être, comme je le crois, n'y en aura-t-il point du tout à cette
heure. Eh bien\,! que, pensez-vous de tout ceci, et à quoi vous
arrêtez-vous\,?» Je voulus lui donner le temps de la réflexion par une
parenthèse, et à moi qui le voyais hors d'espérance de démordre. Je
voulus aussi le sonder sur ce qui nous regardait. Je lui dis que je
pensais qu'il avait fait une grande faute lors de son affaire avec les
bâtards, de n'avoir point voulu nous mettre à la suite des princes du
sang\,; que quelque différence qu'il y eût d'eux à nous, un tel
accompagnement eût bien embarrassé le régent, et l'eût forcé à remettre
les bâtards en leur rang de pairie\,; que par cela seul ils étaient
perdus, et qu'alors la disposition publique du monde, et celle du
parlement en particulier, était d'y applaudir\,; mais qu'il avait pris
une fausse idée, que nous savions bien, et que nous n'ignorions pas qui
nous avait perdus, qui est de mettre un rang intermédiaire entre les
princes du sang et nous\,; que cette faute était grossière, en ce que
jamais nous ne pouvions nous égaler aux princes du sang, au lieu que
tout rang intermédiaire se parangonnait à eux\footnote{Se comparait à
  eux.}, comme ils l'avaient vu arriver par degrés, presque en tout, de
MM. de Vendôme, et en tout sans exception, des bâtards et batardeaux du
feu roi, même depuis leur habilité à la couronne retranchée. Il en
convint très franchement, et il ajouta qu'il était prêt de réparer cette
faute\,; que son amitié pour le comte de Toulouse duquel je lui parlais
tout à l'heure, en avait été un peu cause, mais qu'il consentirait à
présent à leur réduction entière à leur rang de pairie. Il me dit, de
plus, qu'il ne me ferait point de finesse, qu'il en avait parlé au
régent sans s'en soucier, mais comme d'une facilité\,; et que pour la
lui donner tout entière, il avait proposé trois parties différentes\,:
1° ôter l'éducation\,; 2° le rang intermédiaire\,; 3° réduction à celui
de l'ancienneté de la pairie, et tout autre rang retranché\,; que M. le
duc d'Orléans lui avait demandé des projets d'édits et de déclaration,
qu'il les avait fait dresser et les lui avait remis. Il faut ici dire la
vérité\,: l'humanité se fit sentir à moi tout entière et sentir assez
pour me faire peur. Je repris néanmoins mes forces, et après quelques
courts propos là-dessus, je lui demandai comment il l'entendait pour
l'éducation\,: «\,La demander, me répondit-il avec vivacité. ---
J'entends bien, lui repartis je, mais vous souciez-vous de l'avoir\,?
--- Moi, non, me dit-il, vous jugez bien qu'à mon âge, je n'ai pas envie
de me faire prisonnier\,; mais je ne vois point d'autre moyen de l'ôter
à M. du Maine que de me la donner. --- Pardonnez-moi, lui répondis-je,
n'y mettre personne, car cela ne sert à rien. Y laisser le maréchal de
Villeroy\,; sans supérieur, qu'il faut bien y laisser, quoi qu'il fasse
avec tous les bruits anciens et nouveaux. --- Fort bien, me dit-il, mais
ôterez-vous l'éducation à M. du Maine si personne ne la demande\,? et il
n'y a que moi à la demander. --- Mais, lui dis-je, la demander et la
vouloir ce sont deux choses. Ne la pouvez-vous pas demander pour faire
qu'on l'ôte à M. du Maine, et convenir avec M. le duc d'Orléans que
personne ne l'aura\,? Il me semble même que Son Altesse Royale me dit
hier que vous ne vous en souciez pas, et à mon avis ce serait bien le
mieux. --- Il est vrai, me répondit-il, que je ne m'en soucie point du
tout, et que je l'aimerais autant ainsi\,; mais, il ne me convient pas
de la demander et de ne la pas avoir. Il faut que je la demande, et par
conséquent que je l'aie.\,» J'avais senti tout l'inconvénient d'agrandir
un prince du sang, et le second homme de l'État de l'éducation du roi,
c'est ce qui m'avait porté à cette tentative. Comme je vis mon homme si
indifférent, et pourtant si résolu à l'avoir, j'essayai un autre tour
pour l'en déprendre. «\,Monsieur, lui dis-je, cette conversation demande
toute confiance. Vous m'avez parlé librement sur M. le duc d'Orléans, la
nécessité me force à en user, de même. Vous ne le connaissez pas, quand
vous voulez l'éducation du roi. Rien de meilleur pour M. du Maine et
pour sa poltronnerie naturelle\,; car par là il loge chez le roi, ne le
quitte point, et se trouve à couvert de tout. En second lieu, pour
soutenir son état monstrueux, qui ne peut subsister que par faveur
insigne et manèges continuels. Mais vous, qu'en avez-vous besoin\,? vous
êtes le second homme de l'État. Cet emploi ne peut donc vous agrandir ni
vous servir de bouclier dont vous n'avez que faire. Il peut seulement
vous brouiller avec M. le duc d'Orléans, qui, puisqu'il faut le dire,
est de tous les hommes le plus défiant et le plus aisé à prendre des
impressions fâcheuses, qu'on sera toute la journée attentif à lui
présenter sur vous\,; et vous, monsieur, vous vous piquerez du défaut de
confiance, d'attention, de considération. Vous ne manquerez non plus de
gens pour vous mettre ces idées-là dans la tête et pour vous y confirmer
que Son Altesse Royale en manquera de sa part, et vous voilà brouillés.
Vous vous raccommoderez peut-être\,; mais ces brouilleries et ces
raccommodements ne laisseront que de l'extérieur\,: votre solide et
vraie grandeur consiste dans une vraie et solide union avec le régent.
L'union ou le défaut d'union avec lui sera votre salut ou votre perte,
autant que gens comme vous peuvent se perdre. Il faut entre vous deux
une union sans taches, sans rides, sans fautes, et qui ne s'alarme pas
aisément. Sans l'éducation, nulle occasion à l'entamer, avec l'éducation
cent mille. Il en naîtra partout, et vous le connaîtrez trop tard.\,»
J'eus beau dire, M. le Duc s'en tint à son peu de goût pour l'avoir, à
son point d'honneur de l'obtenir dès qu'il la demandait, et à la
nécessité de la demander sans qu'il fût possible de le déranger de pas
un de ces trois points qu'il s'était bien mis dans la tête. Comme je l'y
vis inflexible, je voulus du moins ranger une très fâcheuse épine ou
m'en servir pour revenir à mon but de sauver M. du Maine, par tous les
inconvénients que je craignais de l'attaquer\,; je dis à M. le Duc qu'il
fallait donc pousser la confiance à bout, et qu'il me pardonnât un
détail de sa famille où j'allais nécessairement entrer. Après cette
préface, qui fut reçue avec toute la politesse d'un homme qui veut
plaire et gagner, je lui dis\,: «\, Monsieur, puisque vous me le
permettez, expliquez-vous donc en deux mots sur M. votre frère.

«\,A la conduite qu'il tient par ses voyages, sa marche incertaine, et
par les bruits qui se répandent, où en sommes-nous à cet égard\,? ---
Monsieur, me répondit M. le Duc, je n'en sais rien moi-même. Mon frère
est un étourdi et un enfant qui prend son parti, l'exécute, puis le
mande voilà ce que c'est. --- Et moi, monsieur, lui répondis-je, je
trouve que ne savoir où vous en êtes, c'est en savoir beaucoup, car je
n'aurai jamais assez mauvaise opinion de M. le comte de Charolais pour
le croire capable de prendre un si grand parti sans vous et sans
M\textsuperscript{me} la Duchesse\,; elle est la mère commune. Tous,
quoique fort jeune, vous avez plusieurs années plus que lui, et par
toutes sortes de règles, vous lui devez tenir lieu de père\,:
éclaircissez-moi ce point, car il est capital.\,» À cela, pour réponse,
M. le Duc prend sur sa table une lettre de ce prince qui lui marquait,
en quatre lignes, sa route pour Gênes, et c'était tout. Il me la lut,
puis me pressa de la lire moi-même, protestant qu'il n'en savait pas
davantage. Néanmoins, pressé par moi, il lui échappa que son frère
n'avait aucun établissement, et que, s'il en trouvait un en Espagne,
comme on le débitait, il ne trouverait point qu'un cadet\,; sans bien et
sans établissement, fît mal de le prendre. «\,Fort bien, monsieur, lui
répartis-je vivement\,; ce cadet a soixante mille livres de pension,
n'est-ce rien à son âge pour vivre dans l'hôtel de Condé et à Chantilly
avec vous, où il est décemment et avec tous les plaisirs, sans
dépense\,? Mais quand il sera vice-roi de Catalogne, le voilà au roi
d'Espagne. Comment vous plaît-il après cela que M. le duc d'Orléans se
fie à vous\,? Vous aurez alors jambe deçà, jambe delà\,; vous serez, ou
tout au moins vous passerez, à très juste titre, pour le bureau
d'adresse de tout homme considérable qui, sans se montrer, voudra
traiter avec l'Espagne\,; non seulement vous, mais vos domestiques
principaux, et à votre insu, si l'on veut\,; et avec une telle épine, et
si prégnante\footnote{On a déjà vu ce mot employé par Saint-Simon dans
  le sens de piquant.} pour M. le duc d'Orléans, vous voulez qu'il vous
sacrifie les bâtards pour se lier intimement avec vous. Monsieur,
pensez-y bien, ajoutai-je, je vous prends à mon tour par vos propres
paroles sur M. du Maine. Le feriez-vous à la place de M. le duc
d'Orléans, et vous rendriez-vous, de gaieté de coeur, les bâtards
irréconciliables pour ne pouvoir jamais compter sur les princes du
sang\,? Monsieur, encore une fois, pensez-y bien, ajoutai-je d'un ton
ferme\,: à tout le moins si faut-il l'un ou l'autre, et non pas se
mettre follement, comme l'on dit, le cul entre deux selles, à terre.\,»

M. le Duc le sentit bien, et revint à me jeter tous les doutes qu'il put
sur ces établissements\,: moi, toujours à lui demander s'il en voulait
répondre\,; enfin je lui déclarai qu'il fallait de la netteté en de
telles affaires, et savoir qui on aurait pour ami ou pour ennemi.
Là-dessus, il me dit qu'avec un établissement son frère reviendrait.
«\,Hé bien\,! repris-je, voilà donc l'enclouure, et je n'avais pas tort
de vous presser\,; mais au moins ne faut-il pas demander l'impossible.
Où sont les établissements présents pour M. de Charolais\,?» M. le Duc
se mit à déplorer les survivances et les brevets de retenue qui,
véritablement, ne le pouvaient être assez\,; mais ce n'en était pas là
le temps. Je proposai l'engagement du premier gouvernement, et enfin de
donner une récompense de l'Ile-de-France au duc d'Estrées, lequel ne
valait ni l'un ni l'autre, et de donner ce gouvernement à M. de
Charolais. M. le Duc n'y eut pas de goût. Alors je lui citai le Poitou,
donné à M. le prince de Conti, et que M. de Charolais et lui étaient,
deux cadets tout pareils. Cela arrêta un moment M. le Duc\,; il me
proposa le mariage de M\textsuperscript{lle} de Valois, que son frère
avait toujours désiré.

Comme je traitais alors très secrètement celui du prince de Piémont avec
elle, qui dépendait de convenances d'échange d'États sur l'échange de la
Sicile, et qui pouvait traîner en longueur, je m'étais bien gardé de
rien dire qui fît naître cette ouverture\,; mais il fallut répondre. Je
dis donc assez crûment qu'ils étaient tous deux de bonne maison et bien
sortables, mais que ce serait la faim qui épouserait la soif. M. le Duc
l'avoua, et ajouta qu'en ce cas c'était au régent à pourvoir sa fille
convenablement à un mari qui n'aurait rien de lui-même. Je repartis que
l'état du royaume ne permettait pas de faire un mariage à ses dépens. M.
le Duc en voulut disconvenir en faveur des princes du sang. «\,Tant
d'égards pour eux qu'il vous plaira, monsieur, lui répondis-je\,; mais
approfondissez et voyez qui s'accommodera en France, en l'état où on
est, de contribuer aux mariages de princes du sang qui n'ont rien, et
qui, à l'essor qu'ils ont pris, ne vivront pas avec quatre millions pour
eux deux.\,» Il contesta sur la nécessité de quatre millions au moins,
mais il n'insista plus tant sur savoir où les prendre. Je me crus bien
alors, mais ce bien ne dura que pendant quelques verbiages sur les
dépenses des princes du sang d'autrefois, et de ceux d'aujourd'hui ou
que nous avons vus.

Après cela M. le Duc tourna court, et me dit que M. du Maine fournissait
à tout, si M. le duc d'Orléans le voulait, même à M. de Chartres, qui
n'était revêtu de quoi que ce soit\,; qu'il lui pouvait donner les
Suisses et l'un des deux gouvernements, et l'autre à son frère.
«\,J'entends bien, repartis-je, mais un gouvernement, est-ce de quoi se
marier\,? --- Mais au moins, répondit-il, c'est de quoi vivre et revenir
ici. Après cela on a du temps pour voir au mariage. --- Monsieur, lui
dis-je, vous voyez quel train nous allons de l'éducation au
dépouillement, et il est vrai qu'il n'est pas sage de faire l'un sans
l'autre. Mais faites-vous attention que l'artillerie est office de la
couronne, et ne se peut ôter que par voie juridique et criminelle\,? ---
Qu'est-ce que cela\,? répliqua-t-il vivement\,; l'artillerie n'est rien,
il n'y a qu'à la lui laisser jusqu'à ce qu'il donne lieu à en user
autrement, avoir attention qu'il né s'y passe rien, à en disperser les
troupes avec d'autres dont on soit sûr. Et les carabiniers\,?
ajouta-t-il. --- Voici, répartis-je, une belle distribution. Mais si
elle avait lieu, je tiendrais dangereux de renvoyer les carabiniers dans
leurs régiments\,; non que cette invention de les avoir mis en corps ne
soit pernicieuse aux corps, et très mauvaise au service, mais il ne faut
pas jeter des créatures de M. du Maine dans tous les régiments de
cavalerie\,; ainsi j'aimerais mieux par cette seule raison, les laisser
comme ils sont, et les donner à M. le prince de Conti pour qu'il eût
aussi quelque chose, et qu'il ne criât pas si fort de n'avoir rien.\,»
M. le Duc l'approuva en souriant, comme comptant peu son beau-frère, et
me demanda si je ne parlerais pas à M. le duc d'Orléans ce jour-là même,
parce qu'il s'agissait du surlendemain mardi\,; je lui répondis que je
ferais ce qu'il m'ordonnerait, mais qu'il fallait auparavant savoir que
lui dire et comment lui dire, et pour cela résumer notre conversation
pour convenir de nos faits\,; que je le suppliais de se souvenir de
toutes les grandes et fortes raisons que je lui avais alléguées pour ne
rien faire présentement contre M. du Maine\,; que quelque intérêt que je
trouvasse à le voir attaquer, je ne pouvais promettre ni de changer
d'avis sur ce que je venais d'entendre, ni porter Son Altesse Royale à
l'attaquer tant que je ne semis pas persuadé\,; que, du reste, il
n'avait qu'à voir quel usage il voulait que je fisse de cette
conversation, et qu'il serait fidèlement obéi. Il prit cette occasion de
me dire que j'en usais si franchement avec lui, qu'il me voulait parler
d'une chose sur laquelle il espérait que je voudrais bien lui répondre
de même.

Il me dit donc qu'il voudrait bien savoir ce que je pensais sur la
régence, non qu'il y, eût aucune apparence de mauvaise santé dans M. le
duc d'Orléans, mais qu'enfin on promenait son imagination sur des choses
plus éloignées, à la vie que ce prince menait, trop capable de le tuer,
ce qu'il regarderait comme le plus grand malheur qui pût arriver à
l'État et à lui-même. Je lui répondis que je n'userais d'aucun détour,
pourvu qu'il me promît un secret inviolable\,; et après qu'il m'en eût
donné sa parole, je lui dis qu'il y avait une loi pour l'âge de la
majorité très singulière, mais qui avait été reconnue si sage, par les
inconvénients plus grands auxquels elle remédiait que ceux dont elle est
susceptible, que la solennité avec laquelle un des plus sages de nos
rois l'avait faite et l'heureuse expérience l'avait tournée en loi
fondamentale de l'État, dont il n'était plus permis d'appeler, et qui
depuis Charles IX avait encore été interprétée d'une année de moins.
Mais que pour les régences n'y en ayant aucune, il fallait suivre la loi
commune du plus proche du sang, dont l'âge n'eût plus besoin de tuteur
pour lui-même\,; conséquemment qu'il n'y avait que lui par qui, en cas
de malheur\,; la régence pût être exercée. «\,Vous me soulagez
infiniment, me répondit M. le Duc, d'un air ouvert et de joie, car je ne
vous dissimulerai pas que je sais qu'on pense à M. le duc de Chartres\,;
que M\textsuperscript{me} la duchesse d'Orléans a cela dans la tête,
qu'elle y travaille, qu'il y a cabale toute formée pour cela, et qu'on
m'avait assuré que vous étiez à la tête.\,» Je souris et voulus
parler\,; mais il continua avec précipitation\,: «\, J'en étais fort
fâché, dit-il, non que je sois en peine de mon droit, mais il y a de
certaines gens qu'on est toujours fâché de trouver en son chemin, et je
n'étais pas surpris de vous, parce que je sais combien vous êtes des
amis de M\textsuperscript{me} la duchesse d'Orléans. Je vous voyais
outre cela en grande liaison avec M. le comte de Toulouse\,; vous parlez
toujours tous deux au conseil, quelquefois en particulier, devant ou
après, et on parle aussi en ce cas de faire le comte de Toulouse
lieutenant général du royaume, et M\textsuperscript{me} la duchesse
d'Orléans tutrice de son fils. J'ai cru que vous étiez par elle réuni
aux bâtards, et fort avant dans toutes ces vues. Toute notre
conversation m'a montré avec un grand plaisir que vous ne tenez point
aux bâtards\,; et cela m'a encouragé à vous parler du reste dont j'ai
une extrême joie de m'être expliqué librement avec vous.\,»

Je souris encore\,: «\,Monsieur, interrompis-je enfin, expliquez-vous
davantage, on m'aura donné à vous comme une manière d'ennemi\,; vous
voyez ce qui en est, et de quelle façon j'ai l'honneur de vous parler.
Mais il faut en deux mots\,; que vous sachiez que j'ai eu un procès
contre feu M\textsuperscript{me} de Lussan qui était une grande
friponne, et qu'il fallut démasquer. Je le fis après toutes les mesures
possibles de respect que M. le Prince reçut à merveille, et ne s'en mêla
point. M\textsuperscript{me} la Princesse, M. votre père et
M\textsuperscript{me} la Duchesse ne voulurent point m'entendre, ni me
voir, ni écouter personne\,; rien ne conduit plus loin que le respect
méprisé, et il est vrai que je ne me contraignis guère. Je n'ai jamais
vu feu M. le Duc depuis chez lui, et point ou fort peu depuis sa mort
M\textsuperscript{me} la Duchesse. Voilà le fait, monsieur, qui m'a
brouillé avec l'hôtel de Condé, et qui y aura fait trouver tout le monde
enclin à vous mal persuader de moi\,; mais défiez-vous de ce qui vous
sera dit, et croyez les faits.\,» Là-dessus, politesses infinies de M.
le Duc, désirs de mériter mon amitié, excuses de la liberté qu'il avait
prise, joie pourtant de tout ce qui en résultait, en un mot rien de plus
liant et de moins prince. J'y répondis avec tout le respect que je
devais, et puis lui dis\,: «\,Voyez-vous, monsieur, il y a déjà quelque
temps que je suis dans le monde, je sais aimer avec attachement, mais
nul attachement ne m'a encore fait faire d'injustice ni de folie à mon
su. Je tâcherai de m'en garder encore, et pour vous tout dire en un mot,
je tiens que ce serait l'un et l'autre que de donner ma voix à M. le duc
de Chartres pour la régence, qui dans le malheur possible que nous,
espérons qui n'arrivera pas, n'est due qu'à vous seul\,: voilà pour le
fond. Pour le goût, j'aime M. le comte de Toulouse, vous l'avez bien vu
en cette conversation. Je l'aime par une estime singulière. Ma séance au
conseil auprès de lui a formé ces liens\,; nous nous y parlons des
choses du conseil, et rarement d'autres. Je ne le vois point chez lui
que par nécessité qui n'arrive pas souvent, et cette nécessité me
déplaît à cause du cérémonial auquel je ne puis me ployer. Je lui
souhaite toutes sortes d'avantages\,; mais quelque mérite que je lui
sente avec goût, il est bâtard, monsieur, il est injurieusement
au-dessus de moi, jamais je ne consentirai à faire un bâtard lieutenant
général du royaume, beaucoup moins au préjudice des princes du sang.
Voilà mes sentiments, comptez-y. N'en parlez jamais, je vous en conjure
encore, parce que je ne veux pas me brouiller avec M\textsuperscript{me}
la duchesse d'Orléans, pour un futur contingent qui n'arrivera,
j'espère, jamais. Je ne puis douter de son entêtement là-dessus. J'y ai
répondu obliquement et me suis ainsi tiré d'affaire, vous ne voudriez
pas m'en faire avec elle.\,» Là-dessus nouvelles protestations du
secret, nouvelles honnêtetés, et je coupai la parenthèse, de laquelle
néanmoins je ne fus point du tout fâché, par supplier M. le Duc que nous
convinssions enfin de quelque chose pour ne pas demeurer inutilement
ensemble, et donner lieu à la curiosité de ceux qui peut-être
l'attendaient déjà.

Il me dit que toute la présomption de sa part n'allait qu'à ôter M. du
Maine d'auprès du roi, à me prier de voir M. le duc d'Orléans ce matin
même pour lui en parler de mon mieux, et que, pour ce faire, il
consentait à celui des trois édits, dont il avait porté les projets au
récent, qu'il voudrait préférer. Ce peu de paroles ne fut pas si court
que dans ce narré il n'y eut beaucoup de choses rebattues, après
lesquelles M. le Duc me déclara nettement que de cela dépendait son
attachement à M. le duc d'Orléans, ou de ne faire pas un pas ni pour ni
contre lui. Contre, parce qu'il en était incapable\,; pour, parce qu'il
le deviendrait par ce dernier manquement à tant de paroles données, à
l'accomplissement desquelles l'intérêt personnel du régent n'était pas
moins formel que le sien. J'avais bien ouï, par-ci par-là, divers propos
dans la conversation qui semblaient dire la même chose, mais celui-ci
fut si clair, qu'il n'y eut pas moyen de ne le pas entendre. C'est ce
qui me fit proposer à. M. le Duc d'aller ce même matin au Palais-Royal,
afin que le régent ne pût douter de toute, la force de sa volonté
déterminée\,; mais d'y aller après moi parce que je voulais me donner le
temps de préparer M. le duc d'Orléans, et d'essayer s'il n'aurait pas
plus d'autorité sur M. le Duc que mes raisons ne m'en avaient, donné. Je
promis donc d'être à onze heures et demie au Palais-Royal, et lui me dit
qu'il s'y trouverait à midi et demi. En le quittant je lui dis que je
n'oublierais rien de toutes les raisons qu'il m'avait alléguées, que je
n'en diminuerais la force en quoi que ce fût, que j'appuierais sur la
détermination en laquelle il me paraissait\,; mais que je ne m'engageais
à rien de plus, que je demeurais dans la liberté des sentiments où il
m'avait vu du danger de toucher alors à M. du Maine, que j'examinerais
fidèlement les deux avis, qu'après ce serait entre eux deux à se
déterminer. M. le Duc fut content de cette franchise, et nous nous
séparâmes avec toute la politesse qu'il y put mettre, jusqu'à me
demander mon amitié à plusieurs reprises avec toutes les manières d'un
particulier qui la désire, et du ton et du style des princes du sang
d'autrefois. Je payai de respects et de toute l'ouverture que ce procédé
demandait. Il voulut me conduire, même après que j'eus passé exprès
devant lui la porte de son cabinet pour l'en empêcher, et j'eus peine à
l'arrêter dans sa chambre où heureusement il n'y avait presque personne.

Je vins chez moi, et allai à la messe aux Jacobins, où j'entrais de mon
jardin. Ce ne fut pas sans distraction. Mais Dieu me fit la grâce de l'y
prier, de bon coeur et d'un coeur droit, de me conduire pour sa gloire
et pour le bien de l'État sans intérêt particulier. Je dirai même que je
reçus celle d'intéresser des gens de bien dans cette affaire sans la
leur désigner ni qu'ils pussent former aucune idée, pour m'obtenir
droiture et lumière et force dans l'une et l'autre contre mon
penchant\,; et, pour le dire une fois pour toutes, je fus exaucé dans ce
bon désir, et je n'eus rien à me reprocher dans toute la suite de cette
affaire où je suivis toujours les vues du bien de l'État\,; sans me
détourner ni à droite ni à gauche.

Fontanieu m'attendait chez moi au retour de la messe. Il fallut essuyer
ses questions sur sa mécanique, et y répondre comme si je n'eusse eu que
cela dans l'esprit. J'arrangeai ma chambre en lit de justice avec des
nappes, je lui fis entendre plusieurs choses locales du cérémonial qu'il
n'avait pas comprises, et qu'il était essentiel de ne pas omettre. Je
lui avais dit de voir le régent ce matin-là\,; mais il le fallait
éclaircir auparavant, et il reçut ses ordres l'après-dînée.

\hypertarget{chapitre-xvii.}{%
\chapter{CHAPITRE XVII.}\label{chapitre-xvii.}}

1718

~

{\textsc{Contre-temps au Palais-Royal.}} {\textsc{- Je rends compte au
régent de ma longue conversation avec M. le Duc.}} {\textsc{- Reproches
de ma part\,; aveux de la sienne.}} {\textsc{- Lit de justice différé de
trois jours.}} {\textsc{- Le régent tourne la conversation sur le
parlement\,; convient de ses fautes, que je lui reproche fortement\,;
avoue qu'il a été assiégé, et sa faiblesse.}} {\textsc{- Soupçons sur la
tenue du lit de justice.}} {\textsc{- Contre-temps, qui me fait manquer
un rendez-vous aux Tuileries avec M. le Duc.}} {\textsc{- Ducs de La
Force et de Guiche singulièrement dans la régence.}} {\textsc{- M. le
duc d'Orléans me rend sa conversation avec M. le Duc, qui veut
l'éducation du roi et un établissement pour M. le comte de Charolais.}}
{\textsc{- Découverte d'assemblées secrètes chez le maréchal de
Villeroy.}} {\textsc{- Je renoue, pour le soir, le rendez-vous des
Tuileries.}} {\textsc{- Dissertation entre M. le Duc et moi sur M. le
comte de Charolais, sur l'éducation du roi qu'il veut ôter sur-le-champ
au duc du Maine, et l'avoir.}} {\textsc{- Point d'Espagne sur M. de
Charolais.}} {\textsc{- M. le Duc me charge obstinément de la plus forte
déclaration, de sa part, au régent sur l'éducation.}} {\textsc{- M. le
Duc convient avec moi de la réduction des bâtards en leur rang de
pairie, au prochain lit de justice.}} {\textsc{- Nous nous donnons le
même rendez-vous pour le lendemain.}}

~

J'arrivai au Palais-Royal à onze heures et demie, et comme les
contre-temps sont toujours de toutes les grandes affaires, je trouvai M.
le duc d'Orléans enfermé avec le maréchal d'Huxelles et les cardinaux de
Rohan et de Bissy qui lui lisaient chacun une grande paperasse de sa
façon, ou soi-disant, sous le spécieux nom de ramener le cardinal de
Noailles à leur volonté. J'attendis, en bonne compagnie, dans le grand
cabinet devant le salon où se faisait cette lecture et où nous étions la
veille, et j'étais sur les épines\,; mais j'y fus bien davantage lorsque
je vis M. le Duc y entrer à midi et demi à la montre. Il ne voulut pas
faire avertir M. le duc d'Orléans, néanmoins au bout d'un quart d'heure
il y consentit. J'enrageais de le voir parler devant moi\,: il ne resta
qu'un demi-quart d'heure, et dit en sortant que M. le duc d'Orléans lui
avait dit qu'il en avait encore pour plus d'une heure avec les
cardinaux\,; sur quoi il avait pris son parti de s'en aller pour revenir
avant le conseil. J'oublie que j'étais convenu de le voir le soir au
Tuileries, dans l'allée d'en bas de la grande terrasse, si je le jugeais
à propos par ma conversation avec M. le duc d'Orléans, et que je le lui
dirais au conseil en tournant autour de lui. Nous ne nous donnâmes
presque aucun signe de vie lui et moi au Palais-Royal, et je fus soulagé
de le voir partir sans qu'il eût eu loisir d'enfoncer la matière.

Cependant, je jugeai que je retomberais dans le même inconvénient que je
venais de craindre, si je ne forçais le cabinet. Je m'y résolus donc
après avoir dit que je m'en allais aussi, et que ce n'était que pour
prendre l'ordre d'une autre heure, parce que la fin de la matinée des
dimanches était une des miennes, depuis que l'après-dînée, qui l'était,
était remplie par le conseil qui se tenait auparavant le matin. J'usai
donc de la liberté d'interrompre Son Altesse Royale, mais au lieu
d'entrer j'aimai mieux l'envoyer supplier, par le premier valet de
chambre, de me venir dire un mot pressé. Il parut aussitôt\,; je le pris
dans la fenêtre, et lui dis que, tandis qu'il s'amusait entre ces deux
cardinaux qui lui faisaient perdre un temps infiniment pressé et
précieux pour un accommodement qu'ils ne voulaient point faire, j'avais
à lui rendre un compte fort long, et avant qu'il vît M. le Duc qui
allait, revenir d'une grande et très importante conversation que j'avais
eue avec lui ce matin même sur un billet que j'en avais reçu. Il me
répondit qu'il s'en doutait bien, parce que M. le Duc lui venait de dire
qu'il m'avait écrit et vu, que c'était pour gagner le temps de me voir
qu'il s'en était défait sur le compte de l'affaire des cardinaux qui en
effet devait durer encore plus d'une heure, mais qu'il me priait de
rester et qu'il allait les renvoyer. Il rentra, leur dit qu'il était
las, que cette affaire s'entendrait mieux en deux fois qu'en une, et en
moins d'un demi-quart d'heure ils sortirent avec leur portefeuille sous
le bras. J'entrai en leur place, et portes fermées nous demeurâmes à
nous promener dans la galerie, M. le duc d'Orléans et moi, jusqu'à trois
heures après midi, c'est-à-dire plus de deux bonnes heures.

Quelque longue qu'eût été ma conversation avec M. le Duc, je la rendis
tout entière à M. le duc d'Orléans sans en oublier rien, et chemin
faisant j'y ajoutai mes réflexions. Il fut surpris de la force de mes
raisons pour ne pas tomber sur M. du Maine, et fort effarouché de la
ténacité de M. le Duc sur ce point. Il me dit qu'il était vrai, qu'il
lui avait demandé les trois projets d'édits différents, et qu'il les lui
avait donnés, sans se soucier duquel ni l'un ni l'autre\footnote{Cette
  locution équivaut à \emph{sans se soucier de l'un plus que de
  l'autre}.}, mais pour voir simplement lequel conviendrait mieux pour
assurer seulement l'éloignement du duc du Maine. Alors je sentis qu'il
s'y était engagé tout de nouveau. Il n'osa me l'avouer, mais il
n'échappa pas à mon reproche. «\,Hé bien\,! monsieur, lui dis-je trop
brusquement, vous voilà dans le bourbier que je vous ai prédit tant de
fois\,; voue n'avez pas voulu culbuter les bâtards quand les princesse
du sang, le parlement, le public entier n'avaient qu'un cri pour le
faire, et que tout le monde s'y attendait. Que vous dis-je alors, et que
ne vous ai-je pas souvent répété depuis, qu'il vous arriverait tôt ou
tard d'y être forcé par les princes du sang dans des temps où cela ne
conviendrait plus, et que ce serait un faire le faut à toutes risques\,?
Par quel bout sortirez-vous donc d'ici\,? Croyez-moi, continuai-je, mal
pour mal, celui-ci est si dangereux, et vous avez si souvent et si
gratuitement manqué de parole sur ce chapitre, que, si vous pouvez
encore échapper, n'oubliez rien pour le faire. M. le Duc vous dit tout à
la fois qu'il ne se soucie, pas de l'éducation du roi, mais qu'il la
veut dès qu'il la demande, et qu'on ne la peut ôter à M. du Maine que
parce qu'il la demandera. Sentez-vous bien, monsieur, toute la force de
cette phrase si simple en apparence\,? C'est le second homme de l'État
qui ne veut faire semblant que de sa haine en apparence, et veut se
fortifier de l'éducation sans vous montrer rien qui vous donne de
l'ombrage. Après, quand il l'aura, ce sera à vous à compter avec lui,
parce que vous ne lui ôterez pas l'éducation comme à M. du Maine, et
comprenez ce que c'est pour un régent qu'avoir à compter avec quelqu'un,
et encore d'avoir à y compter par son propre fait. Encore un coup, voilà
ce que c'est que n'avoir pas renversé les bâtards à la mort du roi.
Alors plus de surintendant de l'éducation du roi, et M. le Duc hors de
portée par son âge de la demander, trop content d'ailleurs d'une telle
déconfiture\,; le maréchal de Villeroy, gouverneur en seul, et vous
maître d'un tel particulier, si grand qu'il soit et de l'éducation par
conséquent\,; quelle différence\,!»

Le régent gémit, convint et me demanda ce que je pensais qu'il y eût à
faire. Je répondis que je venais de le lui dire\,; que je ne servais
point M. le Duc à plats couverts, qu'en le quittant je lui avais promis
de rendre à Son Altesse Royale toute notre conversation et toutes ses
raisons dans toute leur force, mais que je m'étais expressément réservé
la liberté de faire valoir aussi les miennes dans toute la leur. Je dis
ensuite au régent que, pour éviter d'ôter M. du Maine si à contre-temps,
je ne voyais de fourchette à la descente que M. de Charolais\,; qu'il
fallait insister sur son retour, que ce retour était très peu,
praticable, à la manière de penser de l'hôtel de Condé, par le défaut
d'établissements présents, puisque le gouvernement de l'Ile-de-France ne
leur convenait pas, et par la difficulté de doter suffisamment
M\textsuperscript{lle} de Valois\,; qu'il n'y avait qu'à tenir ferme sur
ce point\,; qu'il ne pouvait pas n'être pas trouvé essentiel par
eux-mêmes, puisqu'il s'agissait de savoir si on pouvait compter sur les
princes du sang en sacrifiant le duc du Maine, et qu'il était évident
qu'on ne pouvait y compter tant que M. de Charolais serait hors de
France, et en état de prendre en Espagne l'établissement de Catalogne
dont on parlait.

M. le duc d'Orléans goûta avec avidité cet expédient, si fort né de la
matière même que je ne croyais pas qu'il fallût le lui suggérer. Il
donnait à croire que le lit de justice était pour le surlendemain, au
pis aller dans quatre jours, terme trop étranglé pour qu'ils pussent
prendre un parti sur ce retour, ou que, le prenant M. de Charolais pût
être arrivé, et l'occasion passée, on avait du temps devant soi, car
l'affaire du parlement était si instante que M. le Duc lui-même ne
pouvait pas proposer de différer le lit de justice. Le régent m'assura
qu'il tiendrait ferme là-dessus avec M. le Duc\,; ajouta qu'il serait
très à propos que je le visse le soir aux Tuileries pour voir quel effet
Son Altesse Royale aurait fait sur lui, à qui j'en rendrais compte le
lendemain.

Ensuite il me dit qu'il doutait que le lit de justice pût être pour le
surlendemain mardi, parce que le garde des sceaux doutait lui-même
d'être prêt pour tout ce qu'il y aurait à faire. Ce délai me déplut\,;
je craignis qu'il ne fût un prélude de délai plus long et puis de
changement. Je lui demandai à quand donc il prétendait remettre, que ces
coups résolus, puis manqués se savaient toujours et faisaient des effets
épouvantables. «\,A vendredi, me dit-il, car mercredi et jeudi sont
fêtes, et on ne le peut plus tôt. --- À la bonne heure, répartis-je,
pourvu qu'à tout rompre ce soit vendredi.\,» Et je l'y vis bien
déterminé. Je lui rendis compte après plus en détail que par mon billet
de la veille de ce que j'avais fait avec Fontanieu, et puis il me parla
du parlement avec amertume.

«\,Vous n'avez, monsieur, lui répondis-je, que ce que vous avez bien
voulu avoir. Si dès l'abord, indépendamment même des autres fautes à cet
égard, vous aviez jugé notre bonnet, et si vous ne nous aviez pas
sacrifiés au parlement pour l'honneur de ses bonnes grâces, et avec nous
votre parole, votre honneur et votre autorité, l'arrêt de la régence,
vous lui eussiez montré que vous êtes régent, au lieu que vous lui avez
appris à le vouloir être, et votre faiblesse le lui a fait espérer. ---
Cela est vrai, me repartit-il vivement, mais en ce temps-là j'étais
environné de gens qui se relayaient les uns les autres pour le parlement
contre vous autres et qui ne me laissaient pas respirer. --- Oui, lui
dis-je, et qui, pour l'intérêt particulier, vous éloignaient de vos
vrais serviteurs, de moi, par exemple, pour qui tout cela se faisait, et
qui vous disaient sans cesse que je n'étais que duc et pair\,; vous le
voyez, et si je n'avais pas raison pour lors, et si maintenant je vous
parle en duc et pair quand le bien de l'État et le vôtre me semblent
opposés à mon intérêt de dignité\,; je vous somme de me dire si jamais
je vous ai parlé qu'en serviteur, indépendamment d'être duc et pair. ---
Oh\,! quelquefois,\,» me dit-il en homme moins persuadé que peiné d'être
acculé. Je ne voulus pas le battre à terre. «\,Monsieur, lui dis-je,
allez, vous me rendez plus de justice, mais au moins pour cette fois
vous voyez si je songe au bonnet, tandis que vous êtes piqué contre le
parlement, et si je ne soutiens pas les bâtards de toutes mes forces.
Pesez cette conduite avec mon goût, que je n'ai jamais caché, mais aussi
n'oubliez pas jusqu'à quel point vous vous êtes aliéné les ducs et de
quelle conséquence et en même temps de quelle facilité il est de les
regagner si le pied vous glisse avec M. le Duc sur M. du Maine\,; car si
vous faites la faute de lui ôter l'éducation, tablez que de lui ôter son
rang avec ne vous l'éloignera pas plus que le seul dépouillement de
l'éducation, son rempart présent et ses vastes espérances, et que cela
nous est si capital que vous vous en raccommoderez avec nous. --- Pour
cela, me dit-il, il n'y aura pas grand inconvénient\,; mais c'est qu'il
faut éviter d'ôter l'éducation à cette heure. Il est de mon intérêt de
le faire une autre fois, et alors comme alors, mais aujourd'hui il n'est
pas de saison et vous avez la plus grande raison du monde. Ce M. le Duc
me fait peur, il en veut trop et trop fermement. --- Mais comment
l'entendez-vous\,? lui répartis-je\,; ne me dîtes-vous pas hier que M.
le Duc vous avait assuré qu'il ne se souciait point de l'éducation et
qu'il ne l'aurait pas\,? --- Je l'entends, me répondit-il, qu'il me le
dit, mais vous voyez comme il a son dit et son dédit. Il ne s'en soucie
pas, mais c'est à condition qu'il l'aura et ce n'est pas mon compte. ---
Monsieur, lui dis-je d'un ton ferme, ce ne l'est point du tout, mais
mettez-le-vous donc si bien dans la tête qu'il ne l'ait pas, car je vous
déclare que s'il l'a fait, comme vous êtes, vous vous en défierez, lui
s'en apercevra, d'honnêtes gens se fourreront entre vous deux pour vous
éloigner l'un de l'autre, et puis ce sera le diable entre vous deux, qui
influera sur l'État, sur le présent, sur l'avenir\,; vous ne sauriez
trop y penser, et par rapport à sa qualité de premier des princes du
sang en âge et par rapport à l'opiniâtreté de ses volontés. Avec ces
réflexions je vous quitte pour m'en aller dîner. --- Voici mon gourmand,
me dit-il, de belles réflexions et le dîner au bout\,! --- Oui, dis-je,
en riant aussi, le dîner et non pas tant le souper\,; mais, puisqu'il
vous plaît de ne point dîner, ruminez bien tout ceci en attendant M. le
Duc, qui ne tardera guère, et préparez-vous bien à l'assaut.\,»

En effet je m'en allai dîner, et non sans cause, car je n'en pouvais
plus. Comme il était fort tard il fallut, au sortir de table, aller au
conseil. Il ne commença qu'à prés de cinq heures\,; l'entretien de M. le
Duc avec M. le duc d'Orléans en fut cause. Je tournai autour de M. le
Duc et lui dis bas que j'irais. C'était le mot convenu pour les
Tuileries. Rentrant chez moi, je trouvai Fagon\,; nous dissertâmes notre
lit de justice. Il me jeta des soupçons sur le garde des sceaux dont les
propos lui faisaient autant de peine que le délai. Il me conta de plus
qu'il avait passé presque toute la matinée avec lui et d'autres du
conseil des finances à des futilités, au lieu de la donner à la
préparation de ce qu'il avait à faire pour le lit de justice. M. de La
Force survint qui fortifia ces soupçons. Cependant le jour tombait et
mon rendez-vous pressait. Je priai Fagon de me mener dans son carrosse à
la porte des Tuileries, au bout du pont Royal, et donnai au mien et à
mes gens rendez-vous à l'autre bout du pont. J'eus toutes les peines du
monde à finir la conversation. Enfin nous nous embarquâmes Fagon et moi.

Comme nous étions encore sous ma porte\,: «\,Arrête, arrête\,!» C'était
l'abbé Dubois. Force fut de reculer et de descendre. Je lui dis que nous
avions bien affaire pour quelque chose qui regardait
M\textsuperscript{me} de Lauzun, dont Fagon se voulait bien mêler. Cela
devint ma défaite ordinaire, parce que je me souvenais de m'en être
servi chez Fontanieu. Fagon croyait que j'allais simplement raisonner
avec M. le Duc pour fortifier le régent contre le parlement et sur le
lit de justice. Mais ce commerce de M. le Duc eût davantage surpris et
aiguisé la curiosité de l'abbé Dubois, grand fureteur. Je n'eus donc
garde de lui en rien dire. Mal m'en prit en un sens, qui fut que je ne
pus jamais me défaire de lui à temps. Enfin pourtant je le renvoyai, et
montai devant lui dans le carrosse de Fagon, comme j'avais fait la
première fois devant M. de La Force.

Je descendis aux Tuileries, et Fagon les traversa pour ne rien montrer à
ses gens. Je courus toute l'allée du rendez-vous marqué. Je regardais
les gens sous le nez. Je parcourus trois fois l'allée et même le bout du
jardin. Ne trouvant rien, je sortis pour chercher parmi les carrosses si
celui de M. le Duc y était. Je trouve mes laquais qui crient et me font
faire place. Je les aurais battus de bon coeur. Je leur demandai
doucement pourtant ce qu'ils faisaient là, et leur dis de m'aller
attendre où je leur avais marqué. Je rentrai honteux dans le jardin, et
de tout ce manège je ne gagnai que de la sueur.

Remontons maintenant pour un moment à la première origine de cette
affaire, c'est-à-dire à la cause principale qui la mit en mouvement.
J'ai dit que ce fut l'intérêt particulier de Law, d'Argenson, de l'abbé
Dubois. Mais ce fut celui du duc de La Force, pour être du conseil de
régence, qui excita Law qui s'endormait, et, par lui, M. le Duc et
l'abbé Dubois, ami de Law, et enfin Argenson, par M. de La Force d'une
part, et par l'abbé Dubois de l'autre. Tant il est vrai que, dans les
affaires qui semblent parler et presser d'elles-mêmes, et en général
toutes les grandes affaires, si on les recherche bien, il se trouvera
que rien n'est plus léger que leur première cause, et toujours un
intérêt très incapable, ce semble, de causer de tels effets.

Le régent, avec sa facilité et sa timidité ordinaires, se défiait du
conseil de régence sur le parlement, et ne pouvait s'en passer dans
cette lutte avec cette compagnie, où il s'agissait de casser en forme
ses arrêts, comme il était parvenu à s'en passer en presque toutes les
affaires. M. de La Force, pour se rendre nécessaire, lui avait grossi
les objets de cette timidité à cet égard, et tiré en conséquence fort
facilement promesse de lui d'être appelé au conseil de régence lorsqu'il
s'y agirait des matières du parlement, et après lui avait laissé espérer
qu'entré une fois en ce conseil il y demeurerait toujours. Telle était
la cause de la chaleur du duc de La Force contre le parlement, et de
celle que, par lui et par les bricoles que je viens d'expliquer, il
avait tâché, d'inspirer au régent.

Ce prince, souvent trop lent, quelquefois aussi trop peu, voulut que dès
le dimanche où nous sommes encore, et dont je n'ai pas voulu interrompre
les récits importants pour cet épisode, voulut, dis-je, qu'on parlât au
conseil de régence de casser les arrêts du parlement. Il m'en parla le
matin après que je lui eus rendu compte de ma visite à l'hôtel de Condé.
Je lui représentai l'inconvénient d'annoncer sitôt la cassation de ces
arrêts, puisqu'il me disait que le lit de justice était remis au
vendredi suivant. Il l'avait dans la tête, de manière à y souffrir aussi
peu de réplique qu'il en était capable, s'appuyant là-dessus de l'avis
du garde des sceaux. Ce fut aussi l'une des choses qui jointe au délai
du lit de justice, me fit plus craindre quelque dessous de cartes, car
je ne voyais pas à quoi cette précipitation était bonne, sinon à
divulguer un parti pris, à en laisser entrevoir le moment, conséquemment
à le faire échouer, avec quatre jours devant soi à donner lieu d'y
travailler.

Il n'y eut pas moyen de l'empêcher. M. de La Force, qui n'était pas
moins sur les épaules du régent que sur les miennes, le sut de lui, et
me pria de faire en sorte qu'il fût mandé. C'était là mon moindre soin,
mais il y remédia par les siens, et il arracha du régent l'ordre de
venir au conseil de régence, avec quelques paperasses de finances pour
couvrir la chose, bien qu'il eût été éconduit d'y rapporter dès l'entrée
du garde des sceaux dans les finances. Chacun, avant de prendre séance,
se regarda quand on l'y vit arriver\,; et le maréchal de Villeroy, grand
formaliste, ne fut pas content de ce rapport à son insu, comme chef du
conseil des finances. Ce rapport de balle achevé en peu de mots, le duc
de La Force resta en place, et le régent proposa de délibérer sur les
arrêts du parlement. Le garde des sceaux les lut et les paraphrasa
légèrement, puis conclut à les casser. Il n'y eut qu'une voix là-dessus.
Ainsi les mémoires de M. de La Force demeurèrent dans sa poche. Ensuite
M. le duc d'Orléans dit qu'il fallait dresser l'arrêt pour cette
cassation, mais que, cette affaire n'étant pas encore prête, il la
croyait assez importante pour voir cet arrêt de cassation, dans un autre
conseil avant de le publier, et qu'on s'assemblerait pour cela dans deux
ou trois jours, quand le garde des sceaux l'aurait dressé. Dès le soir
même il fut public que les arrêts du parlement seraient cassés. On s'y
attendait tellement qu'on était surpris de ce qu'ils ne l'étaient pas
encore, et Dieu voulut qu'on ne pénétrât pas plus avant.

Question fut après pour M. de La Force de demeurer dans le conseil de
régence, et d'y assister le lendemain lundi. M. le duc d'Orléans ne s'en
souciait guère, et la cassation des arrêts du parlement avait si
légèrement passé qu'il n'était point tenu d'en récompenser M. de La
Force. Celui-ci le sentit bien et, vint me crier à l'aide avec une
importunité étrange. J'avais bien d'autres choses dans la tête. Je ne me
souciais du tout point de faire entrer M. de La Force dans la régence.
Je sentais bien que, s'il y entrait, on ne manquerait pas de me
l'attribuer. Il s'était mis dans une situation à rendre ce service pis
que ridicule. Il l'était de plus d'augmenter le conseil, déjà
absurdement nombreux. M. le duc d'Orléans le voyait bien\,; je ne
voulais pourtant pas tromper le duc de La Force.

Dans cet embarras insupportable avec de plus grands, j'allai le lundi
matin 22 août à onze heures et demie au Palais-Royal, sous prétexte que
je n'avais pas achevé ma besogne ordinaire de la veille. Je commençai
par dire au régent qu'il n'avait pas eu grand'peine à faire passer la
cassation des arrêts du parlement, et que les munitions de M. de La
Force s'étaient trouvées heureusement inutiles. Le régent sentit ce mot
et me dit que, pour qu'il ne parût pas qu'il l'eût fait venir exprès, il
lui avait fait rapporter une bagatelle de finance. «\,Oui, dis-je, mais
si bagatelle que personne n'a compris pourquoi il était venu la
rapporter, ni pourquoi, après l'avoir rapportée, il était demeuré au
conseil. Mais qu'en faites-vous aujourd'hui\,? --- Il a bien envie
d'entrer en la régence, me répondit-il en souriant et comme cherchant
mon suffrage. --- Je le sais bien, répartis-je, mais nous sommes
beaucoup. --- Vraiment, oui, me dit-il, et beaucoup trop.\,» Je me tus
pour ne faire ni bien ni mal, content d'avoir mis le doigt sur la
lettre, pour le pouvoir dire au duc de La Force. Un moment après M. le
duc d'Orléans ajouta comme par réflexion\,: «\,Mais ce n'est qu'un de
plus. --- Oui, dis-je, mais le duc de Guiche, vice-président de la
guerre, comme l'autre l'est des finances, et colonel des gardes de
plus\,; comment le laisser en arrière\,? --- Ma foi, vous avez raison\,;
dit le régent\,; allons, je n'y mettrai pas M. de La Force.\,»

Je l'avais dit exprès, et puis le remords de conscience me prit d'avoir
ainsi exclus un homme qui s'était fié à moi. Après quelque débat en
moi-même, je dis au régent, comme fruit de mon silence\,: «\,Mais si
vous le lui aviez promis. --- Il en est bien quelque chose, me
répondit-il. --- Voyez donc, répartis-je\,; car pour moi, je me contente
de vous représenter et de vous faire souvenir d'un homme qu'oublier en
ce cas-là, ce serait une injure. --- Vous me faites plaisir, me dit-il,
cela ne se peut pas l'un sans l'autre.\,» Et après un peu de silence\,:
«\,Mais au bout du compte, continua-t-il, pour ce qu'on y fait, et au
nombre qu'il y a deux de plus ou de moins, n'y font pas grand'chose. ---
Eh bien\,! le voulez-vous, lui dis-je\,? --- Ma foi, j'en ai envie, me
dit-il. --- Si cela est, répondis-je, n'en faites donc pas à deux fois
pour le faire au moins de bonne grâce. Le duc de Guiche est là dedans\,:
voulez-vous que je l'appelle\,! --- Je le veux bien,\,» dit-il aussitôt.

J'ouvris la porte, et j'appelai le duc de Guiche assez haut, parce qu'il
était assis assez loin avec M. Le Blanc. Pendant qu'il venait, M. le duc
d'Orléans s'avança assez près de moi, et puis au duc de Guiche. Je
fermai la porte, et me tins à quelque distance d'eux. La chose était
simple, et devint pourtant une scène dont je fus seul témoin.

M. le duc d'Orléans, je l'entendis, pria le duc de Guiche de vouloir
bien être de la régence, lui demanda si cela ne l'incommoderait point,
lui dit que l'assiduité n'était que de deux fois la semaine, et encore
que ce ne serait pour lui qu'autant qu'il voudrait\,; que cela ne le
contraindrait point pour sa maison de Puteaux\,; qu'il vît franchement
si cela lui convenait, qu'il ne lui demandait cela qu'autant que la
chose ne l'embarrasserait pas et ne le détournerait point du conseil de
la guerre. À toutes ces supplications si étrangement placées, le duc de
Guiche éperdu, non de la grâce, mais de la manière, se submergeait en
bredouillages et en plongeons jusqu'à terre. Je ne vis jamais tant de
compliments d'une part ni de révérences de l'autre. À la fin M. le duc
d'Orléans révérencia aussi, et tous deux, à bout de dire, se
complimentaient de gestes à fournir une scène au théâtre\,; enfin, las
de rire à part moi, et impatienté à l'excès, je les séparai par
complimenter le duc de Guiche.

En sortant, il me serra la main, et pour le dire tout de suite, il
m'attendit jusqu'à ce que je sortisse, et cela ne fut pas court. Il me
dit qu'il voyait bien à qui il avait l'obligation d'entrer au conseil de
régence. Il le dit à sa famille et à ses amis, et il était vrai que,
sans moi, M. le duc d'Orléans n'y songeait pas, mais ce que le duc de
Guiche ne fit pas si bien, c'est qu'il fit presque des excuses d'avoir
accepté. Au moins ses propos furent ainsi traduits dans le monde, et n'y
firent pas un bon effet. Il était vrai qu'il n'y pensait point, et qu'il
en fut prié comme d'une grâce, mais il n'en fallait pas rendre compte au
public.

On goûta peu cette nouvelle multiplication. Le duc de La Force s'était
décrié\,; le duc de Guiche ne passait pas pour augmenter beaucoup les
lumières du conseil. Ceux qui {[}en{]} étaient {[}du conseil{]} étaient
fâchés de devenir presque un bataillon, et ceux qui n'en étaient pas,
étaient à chercher l'occasion qui était nulle, et en trouvaient encore
plus ridicule cette augmentation à propos de rien. J'eus l'endosse de
tous les deux. Mais il m'en plut incontinent une autre qui fit
disparaître celle-là.

Le duc de Guiche sorti, je demandai à M. le duc d'Orléans à quoi il en
était avec M. le Duc, et lui dis comme je l'avais manqué aux Tuileries.
Il me répondit en s'arrêtant et se tournant vers moi, car nous marchions
vers la grande galerie, qu'il n'avait jamais vu un homme si têtu, et que
cet homme lui faisait peur. «\,Mais enfin\,? lui dis-je. --- Mais enfin,
me répondit-il, il veut l'éducation du roi, et n'en veut point démordre.
--- Et son frère\,? interrompis-je. --- Et son frère, me répondit-il,
c'est toujours la même chanson. Mais il s'est coupé à force de dire, et
je vois bien qu'ils s'entendent tous comme larrons en foire, car tantôt
il dit, comme à vous, que c'est un enfant et un étourdi, qui fait tout à
sa tête sans consulter, et dont il ne peut répondre, et quand je l'ai
pressé sur l'établissement, et si en ce cas-là il reviendrait et si on y
pourrait compter, il lui est échappé qu'il en répondrait alors, et s'en
faisait fort et son affaire. Je lui ai serré le bouton et fait remarquer
la différence de ce qu'il me disait. Cela l'a embarrassé\,; mais il n'en
a pas tenu moins ferme, et je n'en suis pas plus avancé. ---
C'est-à-dire, repris-je, que vous ne savez par là que ce dont vous ne
pouviez douter, qu'ils sont de concert, et que M. le Duc est maître de
son frère\,; mais, c'est-à-dire aussi que c'est le fer chaud du pont
Neuf, à ce que je vois, et que pour avoir M. le Duc il faut deux
choses\,: lui donner l'éducation du roi, et un établissement à son
frère. Comment ferez-vous pour tout cela, monsieur, et par où en
sortirez-vous\,? L'éducation est encore pis que l'établissement, et si
l'établissement, je ne le vois pas. --- Tout cela ne m'embarrasse pas,
me dit le régent. D'établissement, je n'en sais point faire quand il
n'en vaque pas, et la réponse est sans réplique. Je ne crains point
l'établissement d'Espagne\,; Albéroni y regardera à deux fois à se
mettre un prince du sang sur le corps, lequel n'a rien, et qui voudra
autorité et biens, et au bout du compte, ils prendront garde aussi qu'un
peu vaut mieux ici que plus et beaucoup là-bas, et l'espérance ici avec
les difficultés de l'autre côté les retiendra, et nous donnera du temps.
Pour l'éducation, je n'en ferai rien, et j'ai un homme bien à moi à
cette heure, qui ôtera à M. le Duc cette fantaisie de la tête, car il le
gouverne, et je le dois voir tantôt. --- Mais, monsieur, lui dis-je, qui
est cet homme\,? --- C'est La Faye, me répondit-il, qui est son
secrétaire, qu'il consulte, et croit surtout, et entre nous, je lui
graisse la patte. --- À la bonne heure, lui dis-je, faites tout comme il
vous plaira, pourvu que vous sauviez l'éducation.\,»

Là-dessus, nous nous mîmes à rebattre cette matière, puis celle du
parlement\,; et revenant à M. le Duc, je lui fis sentir la différence
d'un mariage où il aurait tout à faire, et encore à essuyer les
aventures domestiques, d'avec celui du prince de Piémont, oncle du roi.
Il le comprit très bien, et conclut par se très bien affermir dans le
parti de ne céder point à M. le Duc. Il me dit là-dessus qu'il lui avait
très bien expliqué que la pension de cent cinquante mille livres qu'il
venait de lui accorder, comme chef du conseil, n'avait jamais été donnée
en cette qualité à son bisaïeul dans, la dernière minorité, mais bien
comme premier prince du sang, qui était la même pension qu'en la même
qualité avait encore M. le duc de Chartres\,; que M. le Duc lui avait
encore demandé l'effet rétroactif depuis la régence\,; et qu'il l'avait
accordé à condition qu'on le payerait comme on pourrait de ces arrérages
supposés. Il ajouta qu'avec tout cet argent il fallait bien que M. le
Duc se contentât et entendît raison\,; que je ferais bien de tâcher à
renouer le rendez-vous des Tuileries, pour voir l'effet de leur
conversation\,; et nous convînmes que je lui en rendrais compte le
lendemain matin par la porte de derrière, pour ne point donner de
soupçon, parce que je n'avais pas accoutumé de le voir ainsi tous les
jours. Il faut se souvenir que ceci se passa le lundi matin 22 août.

En, rentrant chez moi, je mandai à M. de La Force de se trouver au
conseil de régence de l'après-dînée, dont il était désormais. Il vint
aussitôt chez moi. Je n'ai point vu d'homme plus aise. Je m'en défis
aussitôt que je pus. Cette entrée au conseil produisit une découverte.
M. de La Force le voulut aller dire au maréchal de Villeroy, et alla
l'après-dînée chez lui avant l'heure du conseil. Il y voulut entrer par
le grand cabinet où on allait le tenir. Le maréchal de Tallard, qui lui
en vit prendre le chemin lui demanda où il allait, et lui dit que,
s'étant trouvé tête à tête avec le maréchal de Villeroy, il s'était
endormi\,; sur quoi, il était venu `dans ce cabinet attendre. M. de La
Force, qui craignait les secouades du maréchal, s'y achemina toujours
pour s'y faire écrire\,; en entrant il trouva Falconnet, médecin de
Lyon, qui était toujours chez lui, qui lui demanda où il allait. Il le
lui dit, et ce que lui avait dit aussi le maréchal de Tallard. Le
bonhomme, qui n'y entendait pas finesse, lui répondit\,: «\,Ses gens le
disent\,; qu'il dort, mais, comme j'étais avec lui, M. le duc du Maine
est entré, un instant après M. le maréchal de Villars, et aussitôt on a
fermé la porte, et il y a déjà du temps.\,»

Dès que je fus arrivé, ce fut la première chose que me dit le duc de La
Force. Un peu après nous vîmes venir le maréchal de Villars, par la
porte ordinaire, qui avait fait le tour\,; puis, à distance raisonnable,
M. du Maine par la porte de chez le roi\,; enfin le maréchal de Villeroy
après lui. Cette manière d'entrer me frappa, et me fit presser M. de La
Force de le dire à M. le duc d'Orléans dès qu'il arriverait\,; il le
fit. Moi, cependant, je fus pris par M. le Duc, qui me dit qu'il m'avait
cherché aux Tuileries. Je le priai de s'y trouver le soir, et que je n'y
manquerais pas\,; que j'y avais été la veille trop tard, et que je lui
dirais pourquoi. Je coupai court ainsi, et me séparai de lui en hâte de
peur d'être remarqué, ce qu'on craint toujours quand on sent qu'il y a
de quoi. Après le conseil, M. le duc d'Orléans pria fort à propos les
princes, qui toutes les semaines allaient chasser chez eux, de ne
s'absenter point à cause de l'examen de l'arrêt du conseil en cassation
de ceux du parlement, et indiqua un conseil extraordinaire de régence
pour le jeudi suivant après dîner, qu'il colora même de l'expédition de
quelques affaires du conseil qui finissait, et qu'il laissa exprès en
arrière. On ne peut croire combien ce conseil indiqué au jeudi après
dîner servit à couvrir le projet.

Rentré chez moi, je ne songeai qu'à compasser mon heure des Tuileries
pour ne pas manquer M. le Duc une seconde fois. Je priai Louville de m'y
conduire pour dépayser mes gens qui ne m'avaient jamais vu aller aux
promenades publiques. Louville traversa le jardin, et je trouvai M. le
Duc au second tour de l'allée du rendez-vous. Je lui fis d'abord mes
excuses de la veille, et lui dis ce qui me l'avait fait manquer. Après
je lui demandai à quoi il en était avec Son Altesse Royale. Il me dit
qu'il avait peine à se résoudre. Je lui répondis que je ne m'en étonnais
pas, que l'article de M. son frère était une grande enclouure, et que
c'était à lui à l'ôter. Il se récria comme il avait accoutumé de faire
là-dessus, me fit le récit, tel qu'il lui plut, de sa sortie de France,
et en conclut ce qu'il voulut. Je repris son narré, et lui fis remarquer
que ce qu'il me faisait l'honneur de me dire était vrai sans doute,
puisqu'il me le donnait pour tel\,; mais qu'il fallait pourtant qu'il
m'avouât que c'était une de ces vérités qui ne sont pas vraisemblables,
qu'un prince de cet âge fît une première sortie, et pour pays étranger
si éloigné, sans en rien dire à M\textsuperscript{me} sa mère ni à lui,
et que, faisant cette équipée, il trouvât d'anciens domestiques de la
maison pour le suivre sans en avertir, un gentilhomme entre autres, dont
il me faisait l'éloge\,; que, de plus, cette sortie était arrivée lors
du plus opiniâtre déni de justice et de jugement de leur procès avec les
bâtards\,; que je le suppliais de bien remarquer combien cette
circonstance était aggravante.

Je vis sourire M. le Duc, autant que l'obscurité me le put permettre, et
non seulement il se démêla mal de la réponse, mais je sentis qu'il ne
cherchait pas trop à bien sortir de l'embarras de mon argument. Il sauta
à me dire que le tout dépendait de M. le duc d'Orléans\,; qu'un
établissement trancherait tout, et s'échauffant de raisonnement
là-dessus, il passa jusqu'à me répondre du retour de son frère, pourvu
qu'il fût seulement bien assuré d'un grand gouvernement il me l'avait
déjà dit à l'hôtel de Condé. J'insistai sur sa caution, et quand je
l'eus bien prise, je souris à mon tour, et lui prouvai par son dire
qu'il sentait donc bien qu'il était maître du retour de son frère, de
quelque manière qu'il se fût éloigné de lui. Cette conséquence
l'embarrassa davantage\,; il allégua des distinctions comme il put, mais
toujours buté à un établissement sûr, et donnant pour expédient le
dépouillement de M. du Maine.

Là-dessus longs propos, la plupart tenus de part et d'autre dès l'hôtel
de Condé. J'insistai principalement sur deux points, le danger des
mouvements dans l'état et la considération du comte de Toulouse\,; mais
rien n'y fit. Je trouvai un homme fermé à ne pas manquer une occasion,
peut-être unique, d'aller à son but et à ne se plus fier aux paroles du
régent. Il me le répéta vingt fois, convenant que ce qui regardait le
duc du Maine eût été mieux à remettre, mais protestant qu'il ne serait
plus assez sot pour s'y exposer. Il ajouta que de cette affaire M. le
duc d'Orléans saurait à quoi s'en tenir avec lui\,; qu'il était vrai que
Son Altesse Royale n'avait guère affaire de lui\,; mais que, comme que
ce fût, de l'éducation dans le vendredi suivant dépendait son
attachement sans réserve ou son éloignement pareil. Je répondis que le
régent et le second homme de l'État avaient besoin l'un de l'autre, l'un
à la vérité bien plus et l'autre beaucoup moins, mais toujours un besoin
réciproque d'union, de satisfaction, qui influait sur l'État\,; que
l'intérêt de tous les deux était d'ôter au duc du Maine l'éducation du
roi par toutes les raisons déjà tant répétées\,; conséquemment que je
croyais aussi qu'il devait s'en reposer sur Son Altesse Royale, et ne la
pas réduire à l'impossible sur M. de Charolais, au danger de la guerre
civile pour le temps mal choisi. «\, Voyez-vous, monsieur, reprit M. le
Duc avec vivacité, tout ceci n'est qu'un cercle. La guerre civile, je
vous l'ai déjà dit, elle n'est pas à craindre\,; et danger pour danger,
elle la serait moins à cette heure qu'en différant, parce que plus les
bâtards iront en avant, plus ils fortifieront leur parti. Il faudra bien
finir par ôter l'éducation à M. du Maine de votre aveu et de celui de M.
le duc d'Orléans, qui sans cela est le premier perdu\,; or, s'il se veut
bien perdre en différant toujours, tantôt pour une raison, tantôt pour
une autre, comme il fait malgré tant de paroles données depuis la mort
du roi, je ne veux pas me perdre, moi\,; et la guerre civile, soit pour
me conserver contre les bâtards, soit contre eux, en les ayant laissés
trop croître, sera cent fois pis qu'à présent\,: de plus c'est que je
n'en crois point. Le comte de Toulouse est trop sage, et son frère trop
timide. Cette raison, ne la rebattons donc plus. {[}Pour{]} mon frère,
que M. le duc d'Orléans s'engage, et qu'il s'en fie à moi. Le lit de
justice tenu, il aura le temps d'arranger ce qu'il faut à mon frère, qui
reviendra du moment que l'arrangement sera prêt. --- Mais, monsieur, lui
dis-je, faut-il trahir un secret\,? Vous êtes assez honnête homme pour
pouvoir vous tout confier\,; mais gardez-vous d'en laisser rien voir à
M. le duc d'Orléans\,; car c'est de lui que je le tiens, et je crois
nécessaire de vous en informer pour vous montrer que nous en savons plus
que vous ne pensez sur M. votre frère. --- Qu'y a-t-il donc\,?» me
répondit-il avec émotion et avec toute assurance de garder le secret.

Je ne m'en souciais guère\,; mais il était à propos de le lui beaucoup
demander, pour lui faire une impression plus forte. Je lui dis donc que
nous ne pouvions pas douter, par des lettres interceptées, et ce que je
ne lui dis pas par des lettres d'Albéroni au duc de Parme, que, parmi
les remises qui se faisaient d'Espagne en Italie pour le projet qui est
sur le tapis, il y en eût dix mille pistoles pour un seul particulier.
Je dis particulier, et lui spécifiai bien, comme il était vrai, que ce
n'était ni potentat, ni fournisseur, ni banquier, d'où la conclusion
était aisée à tirer que cette gratification si forte ne pouvait regarder
un particulier moindre que M. le comte de Charolais.

Là-dessus M. le Duc me témoigna le plaisir que je lui faisais de cette
confiance, et me fit le détail de la suite légère de M. son frère, telle
qu'il ne se pourrait passer pour quoi que ce fût de tant soit peu
important et encore pour des choses pécuniaires du sieur de Billy, cet
ancien gentilhomme de leur maison, qu'il m'avait tant vanté. Il ajouta
que Billy était entièrement incapable d'entrer en rien ni de savoir quoi
que ce fût, sans lui en rendre compte, et puis me protesta non seulement
avec serment, mais avec un air de vérité et de sincérité qui me
convainquit, qu'il n'en avait pas la moindre notion, ni même aucune que
son frère fût en commerce avec le cardinal Albéroni ni avec personne en
Espagne. Cela me soulagea fort à savoir, et je ne le lui dissimulai pas.
Il me parla encore de M\textsuperscript{lle} de Valois, et sur cela je
battis la campagne tant que je pus à cause du prince de Piémont. M. le
Duc ne m'en pressa pas tant qu'il avait fait à l'hôtel de Condé, soit
qu'il eût réfléchi sur la difficulté d'une dot pour deux, ou que, tout
occupé de son affaire, il se passât volontiers à un gouvernement pour M.
son frère.

Il me pressa ensuite de voir M. le duc d'Orléans le lendemain matin chez
lequel il devait aller ce même lendemain l'après-dînée, de me mettre en
sa place sur le peu de réalité de ses paroles, et sur le danger qu'il y
aurait en attendant\,; puis me répéta avec feu que, {[}de{]} ce qui se
passerait le vendredi prochain, et non un jour plus tard, dépendrait
aussi de son dévouement ardent et entier pour M. le duc d'Orléans, ou de
ne vouloir pas aller pour son service d'où nous étions au grand rond des
Tuileries, au bord presque duquel nous nous entretenions pour pouvoir
voir dans l'obscurité autour de nous. Il ne se contenta pas de me
répéter la même déclaration\,; mais il me pria de la faire de sa part au
régent, et d'y ajouter que, s'il n'avait l'éducation le vendredi
suivant, il lui en resterait un ressentiment dans le coeur, dont il
sentait bien qu'il ne serait pas maître, et qui lui durerait toute sa
vie.

Je me débattis encore là-dessus tant que je pus\,; mais enfin il me
força par me dire que, puisqu'il trouvait fort bon que j'appuyasse mes
raisons, il avait droit aussi d'exiger de moi que je ne cachasse rien à
M. le duc d'Orléans de ce qu'il désirait qui passât à lui par moi de sa
part. À bout donc sur ce beau message je crus, à voir une détermination
si forte, qu'à tout hasard je devais l'entretenir dans la bonne humeur
où je l'avais laissé sur nôtre rang à l'égard des bâtards. Je finis la
conversation par là, et il me promit de lui-même, sans que je l'en
priasse, de dire le lendemain à M. le duc d'Orléans que, toute réflexion
faite, leur réduction à leur rang de pairie parmi les pairs était ce qui
lui paraissait le meilleur à suivre des trois projets de déclarations ou
d'édits qu'il lui avait présentés. Je sentis bien qu'en effet je l'en
avais persuadé dès l'hôtel de Condé\,; mais je ne sentis pas moins qu'il
voulait me plaire et me toucher par un endroit aussi sensible pour
émousser mes raisons de ne pas toucher au duc du Maine.

Nous nous séparâmes avec un rendez-vous à la même heure et au même lieu
pour le lendemain, afin de nous dire l'un à l'autre ce qui se serait
passé avec M. le duc d'Orléans\,; et M. le Duc, en me quittant, me fit
excuses de toutes les peines qu'il me donnait, et les compliments de la
plus grande politesse, à quoi je répondis par tous les respects dus. Je
lui fis excuse de ne l'accompagner pas dans le jardin\,; il prit par une
allée, moi par une autre\,; et, pour cette fois, je trouvai mes gens où
je leur avais dis, et je m'en retournai chez moi.

\hypertarget{chapitre-xviii.}{%
\chapter{CHAPITRE XVIII.}\label{chapitre-xviii.}}

1718

~

{\textsc{Je rends compte au régent de ma conversation avec M. le Duc.}}
{\textsc{- Hoquet du régent sur l'élévation des sièges hauts comme à la
grand'chambre, qui m'inquiète sur sa volonté d'un lit de justice.}}
{\textsc{- Récit d'une conversation du régent avec le comte de Toulouse,
bien considérable.}} {\textsc{- Probité du comte, scélératesse de son
frère.}} {\textsc{- Misère et frayeur du maréchal de Villeroy.}}
{\textsc{- Nécessité de n'y pas toucher.}} {\textsc{- Je tâche de
fortifier le régent à ne pas toucher à M. du Maine.}} {\textsc{- Propos
sur le rang avec Son Altesse Royale.}} {\textsc{- Mes réflexions sur le
rang.}} {\textsc{- Conférence chez le duc de La Force.}} {\textsc{- Sage
prévoyance de Fagon et de l'abbé Dubois.}} {\textsc{- Inquiétude de
Fontanieu pour le secret.}} {\textsc{- Il remédie aux sièges hauts.}}
{\textsc{- Entretien entre M. le Duc et moi dans le jardin des
Tuileries, qui veut l'éducation plus fermement que jamais.}} {\textsc{-
Je lui fais une proposition pour la différer, qu'il refuse.}} {\textsc{-
Sur quoi je le presse avec la dernière force.}} {\textsc{- Outre
l'honneur, suites funestes des manquements de parole.}} {\textsc{-
Disposition de M\textsuperscript{me} la Duchesse sur ses frères toute
différente de M\textsuperscript{me} la duchesse d'Orléans.}} {\textsc{-
Prince de Conti à compter pour rien.}} {\textsc{- J'essaye à déranger
l'opiniâtreté de M. le Duc sur avoir actuellement l'éducation, par les
réflexions sur l'embarras de la mécanique.}} {\textsc{- Je presse
vivement M. le Duc.}} {\textsc{- Il demeure inébranlable.}} {\textsc{-
Ses raisons.}} {\textsc{- Je fais expliquer M. le Duc sur la réduction
des bâtards au rang de leur pairie.}} {\textsc{- Il y consent.}}
{\textsc{- Je ne m'en contente pas.}} {\textsc{- Je veux qu'il en fasse
son affaire, comme de l'éducation même, et je le pousse fortement.}}
{\textsc{- Trahison des Lassai.}} {\textsc{- M. le Duc désire que je
voie les trois divers projets d'édits, qu'il avait donnés au régent.}}
{\textsc{- Millain\,; quel.}} {\textsc{- Je déclare à M. le Duc que je
sais du régent que la réduction du rang des bâtards est en ses mains, et
que le régent la trouve juste.}} {\textsc{- Je presse fortement M. le
Duc.}} {\textsc{- M. le Duc me donne sa parole de la réduction des
bâtards au rang de leur pairie.}} {\textsc{- Je propose à M. le Duc de
conserver le rang sans changement au comte de Toulouse par un
rétablissement uniquement personnel.}} {\textsc{- Mes raisons.}}
{\textsc{- M. le Duc consent à ma proposition en faveur du comte de
Toulouse, et d'en faire dresser la déclaration.}} {\textsc{- Je la veux
faire aussi, et pourquoi.}} {\textsc{- Raisonnement encore sur la
mécanique.}} {\textsc{- Renouvellement de la parole de M. le Duc de la
réduction susdite des bâtards.}} {\textsc{- Dernier effort de ma part
pour le détourner de l'éducation et de toucher au duc du Maine.}}

~

Le lendemain mardi 23 août, je fus entre neuf et dix dit matin chez M.
le duc d'Orléans, par la porte de derrière, introduit par d'Ibagnet, qui
m'attendait. Il le fut avertir dans son grand cabinet, et le trouva déjà
à la messe, au retour de laquelle Son Altesse Royale fit fermer ses
portes et me vint trouver. Nous nous promenâmes dans sa grande galerie,
où je lui rendis compte de ce qui s'était passé entre M. le Duc et moi
la veille dans le jardin des Tuileries. Il approuva fort la confidence
que je lui avais faite des dix mille pistoles, et je remarquai que M. le
duc d'Orléans fut très soulagé de ce qu'il y avait lieu de croire que
cette somme n'était pas pour M. le comte de Charolais et que ce prince
n'avait point encore de commerce en Espagne.

Nous rebattîmes la plupart des choses principales en question, et il me
parut qu'il regardait son mariage avec sa fille comme assez praticable.
Je lui remontrai là-dessus toute la différence de celui du prince de
Piémont pour la réputation de sa régence, pour se faire une nouvelle et
plus prochaine alliance avec un prince tel que le roi de Sicile, et si
bienséante par rapport à leurs qualités de grand-père et d'oncle du roi,
de père et de frère d'une princesse qui lui avait rendu un si grand
service par le mariage de M\textsuperscript{me} la duchesse de Berry.
J'ajoutai la considération qu'il devait à M\textsuperscript{me} la
duchesse d'Orléans pour qui le coup de poignard serait doublement
affreux de sceller la perte de ses frères par le mariage de sa fille
avec le fils d'une soeur qu'elle haïssait à mort, et le frère de celui
qui culbutait le sien et qui profitait de sa plus chère dépouille. Enfin
je n'omis rien de tout ce que je crus de plus propre à donner des forces
à M. le duc d'Orléans pour combattre les raisons de M. le Duc. Mais, je
sentis que deux choses lui faisaient une impression forte. Ce que je
viens de rapporter sur M. le comte de Charolais et l'Espagne, et la dure
protestation de M. le Duc, qu'il fallut bien lui rapporter dans toute sa
force. Je ne lui dissimulai pas non plus que le nombre accumulé de ses
manquements de parole à M. le Duc sur l'éducation faisait toute sa
roideur à la vouloir à cette heure. Le régent les contesta, dit qu'il ne
disait pas vrai, puis laissa voir, ce dont je me doutais bien, qu'il n'y
avait rien à rabattre des justes plaintes de M. le Duc à cet égard.

Ensuite, passant au mécanique, car cette conversation fut très
sautillante, je lui dis, et je ne sais pas trop comment je m'en avisai,
que les sièges hauts du lit de justice n'auraient qu'une marche, par la
difficulté de les élever davantage\,; mais que je croyais que cela
suffisait pour marquer seulement des hauts et des bas sièges. Là-dessus
il s'éleva, me dit que cela ne pouvait passer de la sorte, que les hauts
sièges de la grand'chambre avaient cinq degrés. J'eus beau lui
représenter la difficulté mécanique, et lui dire enfin que puisque moi,
à son avis si pair, j'en étais convenu, il pouvait bien le trouver bon.
Point du tout. Le voilà à entrer dans tous les expédients de cet ouvrage
sans en trouver pas un, et pour fin à me charger de voir Fontanieu pour
remédier en toutes sortes à cet inconvénient. Cela pensa me désespérer,
car jamais, pour le trancher court, M. le duc d'Orléans n'eut de
dignité\,; et ne s'en soucia pour soi-même ni pour les autres. Pour lui,
un peu plus ou moins d'élévation aux hauts sièges ne faisait rien à un
régent du royaume qui, au lit de justice, n'a que la première place sur
le banc des laïques, sans distance ni différence quelconque d'avec
eux\,; et pour les pairs, il les avait trop maltraités pour croire que
cette seule fois il fût devenu tout à coup épris de leur dignité et de
l'honneur de leur séance. Je soupçonnai donc fortement que M. le duc
d'Orléans, battu de M. le Duc, au pied du mur pour un lit de justice de
grande exécution\,; cherchait quelque voie de le rompre. Le délai de
trois jours m'en avait donné l'inquiétude, et ceci si fort contraire à
son génie me l'augmenta beaucoup. Je craignis que, n'osant rompre à
découvert un projet de cette sorte, n'ayant plus par où le différer au
delà du vendredi, ni moins encore rien à alléguer pour changer une
résolution si concertée, il se jetait où il pouvait pour former un
délai, dans l'espérance de faire ébruiter, puis échouer la chose. Cela
me mit dans un grand malaise\,; je cherchai dans le reste de la
conversation à m'éclaircir de ce grand point, mais je compris bien que
mes soins seraient inutiles, et que, si le régent en avait la pensée, il
me la cacherait avec plus de précaution qu'à nul autre.

Delà, il passa à un récit bien considérable. «\,Vous ai-je dit, me
demanda-t-il, la conversation que j'ai eue mardi dernier avec le comte
de Toulouse\,?» Et sur ce que je lui répondis que non, il me conta
qu'après avoir travaillé avec le maréchal d'Estrées et lui, il resta
seul, et lui demanda s'il pouvait lui faire une question, et que cette
question fut s'il était content de lui et de sa conduite\,; que sur les
assurances de toute satisfaction suivies de réponses du comte de
Toulouse les plus convenables, même les plus nettes, il lui dit que,
puisqu'il en était ainsi, il en avait encore une autre à lui faire sur
son frère, qui était dans l'inquiétude d'un bruit répandu qu'il le
voulait faire arrêter et le maréchal de Villeroy. Son Altesse Royale
s'était mise à rire comme d'une chose qui ne méritait que cela\,; il fut
pressé\,; il répondit qu'il n'y avait songé. Le comte lui demanda s'il
en pouvait assurer son frère, et sur le oui, lui demanda s'il en était
mécontent, et d'où pouvait venir ce bruit. Le régent répondit que pour
le bruit il en ignorait la cause, mais que, pour content, il ne pouvait
l'être. Le comte voulut approfondir\,; sur quoi M. le duc d'Orléans lui
demanda ce qu'il penserait de remuer le parlement. Le comte lui répondit
avec franchise que cela lui paraîtrait très criminel, et s'informa s'il
y en avait quelque chose sur le compte de son frère. M. le duc d'Orléans
répondit qu'il n'en pouvait douter par des preuves très sûres, et tout
de suite lui demanda que lui semblerait d'un commerce en Espagne, et
avec le cardinal Albéroni. «\,Encore pis, répondit nettement le comte,
je ne regarderais pas cela différemment d'un crime d'État\,; » et sur ce
que M. le duc d'Orléans lui laissa entendre qu'il en savait le duc du
Maine coupable, le comte lui dit qu'il ne pouvait soupçonner son frère
jusqu'à ce point\,; qu'il le suppliait de bien prendre garde à la vérité
de ce qui en pouvait être\,; que pour lui, il lui avait donné sa parole,
parce qu'il considérait l'État et Son Altesse Royale comme une seule et
même chose\,; qu'ainsi il lui répondait de soi, mais qu'il ne lui
répondait pas de son frère.

Cette conversation me parut infiniment importante, et les réflexions que
j'y fis allongèrent fort la nôtre. Je dis à M. le duc d'Orléans que je
ne voyais rien de si net ni de plus estimable que le procédé du comte de
Toulouse, en, même temps rien de si fort contre le duc du Maine que ce
que son frère, si engagé à le soutenir, lui déclarait pourtant qu'il
n'en pouvait répondre. Le régent me parut y faire beaucoup d'attention.
Je lui dis qu'un tel propos la méritait tout entière, et lui faisait
sentir la grandeur de sa faute d'avoir laissé le duc du Maine entier\,;
que néanmoins il ne devait pas s'en frapper jusqu'à perdre de vue
l'espèce présente, je veux dire l'union du duc du Maine avec le
parlement, et le danger de les châtier ensemble\,; que ces conjonctures
demandaient toutes ses plus mûres réflexions. Après quelques séjours
là-dessus, moi ne voulant plus trop m'expliquer\,; et flottant entre le
danger nouveau, démontré par l'aveu du comte de Toulouse, et la crainte
extrême de moi-même sur ma vengeance et la restitution de notre rang, le
régent me conta que le maréchal de Villeroy lui avait parlé lui-même de
ce bruit de le faire arrêter avec M. du Maine, d'un ton fort humble et
fort alarmé\,; qu'il en avait été dire autant à l'abbé Dubois, et qu'il
était dans la dernière peine, quoi qu'on pût faire pour le rassurer. Je
dis à M. le duc d'Orléans que polir celui-là, quoi qu'il pût faire, il
fallait le laisser\,; qu'après les bruits anciens et nouveaux, il n'y
avait ni grâce ni sûreté à l'ôter d'auprès du roi, auquel s'il arrivait
malheur dans la suite, chacun renouvellerait d'horreurs contre Son
Altesse Royale.

Il en convint, et me témoigna d'ailleurs que l'âge et le peu de mérite,
du maréchal de Villeroy rendaient sa place très indifférente. J'ajoutai
que je regarderais sa mort, si elle arrivait devant la majorité, comme
un malheur pour Son Altesse Royale, parce qu'alors ce serait bien force
d'en nommer un autre\,; que je ne savais pas trop bien qui de mérite
propre à cette place en voudrait, et que ce serait en revenir presque au
même danger s'il arrivait malheur au roi.

Il en convint encore\,; puis nous revînmes à M. le Duc, moi bien aise de
prendre ma mission pour sentir où il en était sur le duc du Maine, et en
même temps sur notre rang. Il me parla faiblement sur l'un et sur
l'autre. Je le conjurai de nouveau de bien penser aux suites d'attaquer
le duc du Maine dans une partie aussi sensible que l'éducation, et de la
confier à un prince du sang de l'humeur arrêtée de M. le Duc, et, après
quelques raisonnements faits et abrégés là-dessus, je le suppliai de
sentir que, s'il faisait tant que d'ôter au duc du Maine l'éducation du
roi, il ne serait ni moins enragé ni moins irréconciliable\footnote{Le
  manuscrit porte \emph{réconciliable}. C'est une erreur évidente : il
  faut lire \emph{irréconciliable}.} d'y ajouter sa réduction à son rang
de pairie. Il me répondit qu'il l'avait déjà voulu une fois\,; que M. le
Duc s'y était opposé par l'idée de se séparer de nous par mettre entre
deux un rang intermédiaire\,; qu'il était bien aise de me le dire
nettement pour que je ne m'amusasse pas aux propos de M. le Duc, avec
lequel il faudrait bien voir, s'il se portait à lui donner l'éducation
du roi, mais sans lequel cela était impossible. Avec cela je m'en allai
avec un commencement d'espérance, dont voici le raisonnement, supposé
l'éducation changée de main.

Je comprenais de reste que ni M. le duc d'Orléans, ni M. le Duc ne se
souciaient de la restitution de notre rang. Je comptais bien même qu'ils
tâcheraient de l'éluder l'un par l'autre, le régent surtout, grand
maître en ces sortes de tours d'apparente souplesse qui se démêlent avec
exécration bientôt après\,; mais je sentis aussi qu'il ne résisterait
non plus à M. le Duc en ce point, si celui-ci se le mettait dans la
tète, que dans l'affaire de l'éducation, \emph{a fortiori}, et qu'il
n'était rien moins qu'impossible d'y déterminer M. le Duc qui croyait
avoir un besoin capital de moi, se conduisait avec moi de même, était
convaincu de son aveu fait à moi-même de la fausseté de son ancienne
idée de rang intermédiaire, et tacitement encore par ne le vouloir pas
dire par gloire, de la sottise qu'il avait faite de ne nous avoir pas
mis à leur suite contre les bâtards. Or il était à même de réparer l'une
et l'autre faute\,; lui-même y avait pensé, puisqu'il l'avait proposé
par l'un des trois projets d'édits. Il n'était donc plus question que de
lui parler ferme, et de me servir de sa passion démesurée de l'éducation
pour servir la mienne de la restitution de notre rang. C'est une des
choses que je roulai le plus dans ma tête le reste de la journée, mais
qui n'y roula qu'en second, tant j'eus peur de moi-même, et de ne pas
éloigner avec le désintéressement d'un coeur pur tout ce qui pouvait
nuire à l'État et y causer des troubles.

Plein de ces pensées, le duc de Chaulnes força ma porte au sortir de
dîner, que je tenais fermée en ces jours si occupés à tout ce qui
n'était point du secret. Fils et neveu des ducs de Chevreuse et de
Beauvilliers, notre union était intime. Je l'avais, comme on l'a vu,
fait duc et pair\,; il ne l'oublia jamais, et il était aussi sensible
que moi à ce qui était de cette dignité. Il venait, sur les bruits qui
couraient de la colère du régent contre le parlement, raisonner avec moi
si nous ne pourrions pas en tirer quelque parti. J'eus regret de ne
pouvoir lui rien dire\,; je battis la campagne sur les difficultés
générales, et je m'en défis le plus tôt que je pus.

J'étais attendu chez M. de La Force où Fagon et l'abbé Dubois devaient
se trouver. En les attendant, car je logeais fort près de lui et les
autres fort loin, je dissertai avec lui {[}sur{]} mes soupçons
renouvelés le matin par ce hoquet bizarre que M. le duc d'Orléans
m'avait fait des hauts sièges aux Tuileries. Il en fut effrayé comme
moi. Fagon vint qui ne le fut pas moins. Nous relûmes avec lui le
mémoire que je lui avais dicté chez moi, qui fut le fondement de toute
cette affaire. Il y avait ajouté diverses choses de pratique, mais
importantes, sur l'interdiction du parlement s'il refusait de venir aux
Tuileries, les scellés à mettre en différents lieux du palais et autres
choses de cette nature. L'abbé Dubois arriva après s'être fait attendre
assez longtemps avec d'excellentes notes d'ordres à donner pour
l'exécution mécanique de tous les ordres possibles, les signaux des
ordres pour les pouvoir donner en séance sans qu'il y parût, comme en
cas que le parlement voulût sortir du lit de justice, l'arrêter tout
entier ou quelques membres seulement, et quels, et mille choses de cette
nature qu'on ne peut trop soigneusement prévoir, et qui mettent en
désarroi quand elles arrivent sans qu'on y ait prévu d'avance.

Je n'eus pas le temps d'achever avec eux. Les sièges hauts me tenaient
en cervelle\,; je voulais ôter à M. le duc d'Orléans ce prétexte que je
redoutais. J'avais mandé à Fontanieu de m'attendre chez lui, et je
m'étais arrangé pour avoir, fait avec lui à temps de ne manquer pas mon
rendez-vous des Tuileries. Je trouvai moyen avec Fontanieu que les
sièges hauts eussent trois bonnes marches. Il se désolait du délai du
lit de justice, parce que dans l'intervalle, il craignait ses ouvriers
qui ne comprenaient point ce qu'il leur faisait faire, et qui mouraient
d'envie de le savoir et de s'en informer. Sortant de chez lui, je dis à
mes gens\,: «\,Au logis\,!» mais en passant devant ce pont tournant, du
bout du jardin des Tuileries, je tirai mon cordon, m'y fis descendre
comme séduit par le beau temps, et j'envoyai mon carrosse m'attendre au
bout du pont Royal.

Je ne tardai pas à trouver M. le Duc dans notre allée ordinaire, le long
du bas de la terrasse de la rivière. Comme c'était la seconde fois au
même lieu, je craignis les aventures imprévues et les remarques. Je lui
fis ôter son cordon bleu qu'il mit dans sa poche. Il avait vu M. le duc
d'Orléans le matin depuis moi, et je reconnus bientôt qu'il l'avait
trouvé beaucoup plus facile. Cela me fâcha, parce que j'en sentis la
conséquence et que je ne viendrais pas à bout d'un homme si arrêté dès
qu'il espérerait obtenir ce qu'il prétendait. Il me conta d'abord que le
régent lui avait fait la confidence des dix mille pistoles et la lui
avait faite entière en lui nommant le duc de Parme, dont je fus surpris,
parce que cela n'y ajoutait rien et découvrait ce qu'il ne fallait pas,
et me dit que Son Altesse Royale était demeurée persuadée sur ce qu'il
lui en avait dit que cette remise n'était pas pour M. le comte de
Charolais\,; je le pressai sur le retour de ce prince et sur
l'établissement. Lui se tint ferme à le différer jusqu'à un
établissement prêt, à en répondre dès qu'il le serait et à trouver qu'il
n'y en pouvait avoir que par le dépouillement du duc du Maine. Je le
suppliai de nouveau d'en sentir toutes les conséquences que je lui remis
devant les yeux. Nous les discutâmes encore, et ce ne fut de part et
d'autre que redites de nos précédentes conversations, parmi lesquelles
il me répéta à diverses reprises les manquements de parole qu'il avait
essuyés là-dessus et auxquelles il ne pouvait plus se fier, et sa
protestation encore plus durement que la veille d'attachement au régent
ou de ne faire pas un pas pour son service, selon que l'éducation lui
serait ou ne lui serait pas donnée dans le vendredi prochain.

Voyant que c'était perdre temps que d'espérer davantage de le ramener
là-dessus, il me vint dans l'esprit de lui faire une proposition qui me
parut devoir être goûtée\,: «\,Monsieur, lui dis-je, je vois bien ce qui
vous tient, vous ne voulez plus tâter des paroles et vous voulez user de
l'occasion présente\,; vous avez raison\,; mais vous convenez aussi que
si vous n'aviez pas été si souvent trompé, vous ne vous opiniâtreriez
pas à vouloir l'éducation dans la même séance qui doit si fort mortifier
le parlement, parce que vous en sentez toutes les dangereuses
conséquences. --- Cela est vrai, me répondit-il\,: je voudrais de bon
coeur pouvoir séparer l'un de l'autre\,; mais, après ce qui s'est passé
tant de fois, quelle sûreté aurais je et quelle folie à moi de m'y
laisser aller\,? --- Attendez, monsieur, répliquai-je. Il me vient
sur-le-champ une idée dans la tête que je ne vous réponds pas que M. le
duc d'Orléans adopte, mais que je vous réponds de lui proposer, si vous
la goûtez, et comme je la crois raisonnable de faire tout ce qui est en
moi pour qu'il l'exécute. Je voudrais que M. le duc d'Orléans vous
écrivît un billet signé de lui, par lequel il vous donnât sa parole de
vous donner l'éducation du roi à la rentrée du parlement. Par là elle
vous est immanquable\,; car, s'il vous tient parole, vous avez votre
but, s'il y voulait manquer, vous avez en main de quoi le rendre tout
aussi irréconciliable avec M. du Maine que s'il lui avait ôté
l'éducation, et par là vous le forcez à le faire, pour ne demeurer pas
tout à la fois brouillé avec vous et brouillé avec eux, si vous, hors de
toute mesure avec lui, montriez le billet de sa main. --- Monsieur, me
repartit M. le Duc d'un ton ferme, je ne me fie non plus aux écrits et
aux signatures de M. le duc d'Orléans qu'à ses paroles. Il m'a trompé
trop de fois, et ce serait être trop dupe.\,» Je contestai, mais ce fut
en vain, et il demeura ferme à vouloir l'éducation et rien autre.

Dépourvu de cette ressource qui s'était présentée à moi tout à coup
comme bonne, j'eus recours aux péroraisons. Je lui rebattis ce que je
crus de plus touchant sur le comte de Toulouse, et enfin sur les
mouvements qui pouvaient agiter l'État. Il me parut toujours le même,
c'est-à-dire inébranlable, et me dit qu'il devait écrire le lendemain
matin au régent pour le voir commodément l'après-dînée, et en venir
ensemble à une résolution\,; qu'il me priait de l'y préparer dans la
matinée, et de compter encore une fois que de l'éducation dépendrait son
attachement pour Son Altesse Royale, où le contraire avec un
ressentiment dans le coeur dont il ne serait pas le maître, et qui
durerait autant que lui\,: «\,Monsieur, lui répondis-je avec feu, vous
devez me connaître à présent sur les bâtards et sur mon rang. Je ne suis
point né prince du sang et habile à la couronne\,; cependant mon amour
pour ma patrie, que je crains de voir troubler bien dangereusement, me
fait combattre mon intérêt de rang le plus sensible et le plus précieux,
et ma vengeance la plus vive et la plus passionnément désirée. Vous donc
qui devez prendre d'autant plus de part que moi en cet État qui est
votre patrie comme la mienne, mais qui est de plus votre patrimoine
possible dont la couronne est dans votre maison depuis tant de siècles,
et ne peut tomber que sur vous et sur vos descendants à tour chacun
d'aînesse, je vous adjure par votre qualité de Français, par votre
qualité de prince du sang qui doit vous faire regarder la France avec
des yeux de tendresse et de propriété, je vous adjure de passer cette
nuit et demain toute la matinée à peser votre intérêt contre le duc du
Maine avec l'intérêt de l'État, d'être plus Français qu'intéressé dans
son abaissement, de vous représenter sans cesse les suites et les
conséquences de ce que vous, voulez faire\,; et quel serait votre juste
repentir, si par haine seulement ou par intérêt personnel vous nous
allez jeter dans des troubles et dans une guerre civile que vous
convenez vous-même qui perdrait l'État dans la situation où il se
trouve\,! Cela vaut bien la peine de prendre sur votre sommeil. Après
cela vous ferez ce que vous estimerez devoir faire, mais n'ayez pas à
vous reprocher aucune légèreté.\,»

Il me parut ému de ce discours si fort, et pour en profiter, je lui
parlai encore du comte de Toulouse, et lui demandai si cela ne touchait
point M\textsuperscript{me} la Duchesse, et s'il était d'accord avec M.
le prince de Conti. Il me répondit que pour M\textsuperscript{me} la
Duchesse, elle était là-dessus toute différente de M\textsuperscript{me}
la duchesse d'Orléans\,; que l'une était toute bâtarde, l'autre toute
princesse du sang\,; que, pour ce dont il s'agissait,
M\textsuperscript{me} la Duchesse n'en savait rien, parce qu'elle
l'avait prié de faire tout ce qu'il jugerait à propos contre ses frères,
pourvu qu'il ne lui en fit point de part, et qu'elle pût dire que
c'était à son insu, mais qu'il était assuré qu'elle en serait bien aise,
parce qu'elle sentait bien ce qu'elle était, et qu'avec elle ils
parlaient tout le jour de bâtards et de bâtardise\,; qu'il était vrai
qu'elle aimait le comte de Toulouse, quoique depuis leurs affaires il se
fût fort éloigné d'elle, mais que, pour le duc du Maine, elle le
connaissait trop pour l'aimer après ses procédés sur la succession de M.
le Prince et sur le rang\,; qu'à l'égard de M. le prince de Conti, il
m'en parlerait avec peine\,; que je voyais bien ce que c'était, qu'il ne
lui avait rien dit\,; et moins par des paroles que par des manières et
des tons il me fit bien comprendre, et qu'on n'y devait pas compter, et
qu'on, ne devait pas aussi s'en embarrasser. Tandis que nous en étions
sur ces espèces de parenthèses, il me vint dans l'esprit d'essayer à
déranger M. le Duc par la mécanique à la suite de l'émotion que je lui,
avais causée, par ce que je lui avais représenté de touchant.

Je lui dis donc que ce n'était pas le tout que vouloir et résoudre,
qu'il fallait descendre dans le détail, et voir comment arriver à ce
qu'il se proposait\,; que je sentais mieux que personne le néant du
conseil de régence et des personnes qui le composaient\,; que cependant
il ne fallait pas compter qu'on pût faire à l'éducation du roi un
changement de cette importance sans en parler à la régence, qu'il voyait
que les bâtards y prenaient pied comme ailleurs. Je lui contai là-dessus
ce que j'avais su de M. de La Force, et j'ajoutai qu'il devait regarder
les maréchaux de Tallard et d'Huxelles comme étant tout à fait à eux, le
premier par le maréchal de Villeroy, l'autre par lui-même, et par le
premier écuyer et le premier président, ses amis les plus intimes\,; que
d'Effiat, tout premier écuyer du régent {[}qu'{]} il était, il était si
lié, et de si longue main à M. du Maine qu'il le comptait beaucoup plus
à lui qu'à son maître\,; que Besons ne voyait et ne pensait que par
Effiat, et que le garde des sceaux était fort uni aux bâtards du temps
du feu roi\,; que, si quelqu'un d'eux venait à prendre la parole à la
régence, les autres du même parti le soutiendraient\,; que le maréchal
de Villeroy était capable de le prendre sur un ton pathétique par
rapport au feu roi, dont il couvrirait sa cabale\,; que, quel qu'il fût,
il était considéré, et imposait en présence à M. le duc d'Orléans qui
s'en dédommageait mal en s'en moquant en absence\,; que le maréchal de
Villars, ennemi d'abord du duc du Maine, par d'anciens faits, s'était
laissé regagner à lui, moins par ses souplesses que par la façon dont
lui, M. le Duc, l'avait traité.

Il m'interrompit pour m'en parler avec mépris, dire qu'il avait eu
raison, et que le maréchal était un misérable d'être demeuré à la tête
du conseil de guerre avec tous les dégoûts qu'il y avait reçus. «\,Tant
de mépris qu'il vous plaira, monsieur, lui répartis-je\,; personne ne
sait mieux que moi le peu qu'est né le maréchal de Villars, et n'a senti
plus vivement que moi la honte que nous avons reçue quand il a été fait
duc et pair. J'en ai été malade de honte et de dépit. Mais, après tout,
c'est le seul homme en France que vous ayez qui ait gagné des batailles,
qui n'en ait point perdu absolument parlant\,; et c'est encore lui qui,
par tant de bonheur qu'il vous plaira, a le nom d'avoir sauvé à Denain
la France prête à se voir la proie et le partage de ses ennemis, et qui,
par les traités de Rastadt et de Bade, a mis le dernier sceau à celui
d'Utrecht. C'est donc l'homme le plus glorieux qui soit en existence et
par des faits célèbres, et pardonnez-moi le terme, il est insensé à vous
de vous acharner après un tel homme, qui est tout ce que celui-ci est,
et vous voyez aussi ce qui vous en arrive. Il se prend à tout, à un fer
rouge\,; de rage il s'unit à M. du Maine, comme on n'en peut plus douter
après ce qu'a dit M. de La Force. Il tient des propos hardis en faveur
du chancelier et du parlement, et voilà un homme que votre fantaisie a
rendu votre ennemi et a écarté du régent par les niches que vous lui
avez fait faire. Or cet homme n'entend rien en affaires, cela est
vrai\,; mais il n'est pas moins vrai qu'il est éloquent, hardi, piqué,
outré\,; qu'il se déconcerte moins qu'homme du monde\,; que les paroles
lui viennent comme il lui plaît, et qu'un discours fort pour laisser les
choses comme elles sont, dans la bouche d'un homme aussi décoré
d'actions, d'emplois et des plus grands honneurs, ne ferait pas un
médiocre embarras. Le maréchal d'Huxelles parlera peu, mais avec poids.
Pensez-vous que ces gens-là n'entraînent personne, et pensez-vous encore
qu'entre ceux qu'ils n'ébranleront pas, il y en ait de pressés de
prendre la parole pour faire contre\,? Monsieur, ceci est bien
important, et vous ne connaissez pas la faiblesse de M. le duc
d'Orléans. --- En effet, me répondit M. le Duc, je n'avais pas songé à
cet embarras, et j'avoue qu'il est grand.\,» Et après un peu de silence
que je ne voulus pas troubler pour laisser fortifier l'impression qu'il
me semblait que je venais de faire. «\,Mais, reprit-il, monsieur, en
parlera-t-on à la régence\,? car ces bâtards y sont. --- Voilà,
monsieur, lui dis-je, où je vous attendais. Comment en parler devant eux
et comment l'éviter\,? Si c'est en face, se tairont-ils, et M. le duc
d'Orléans sera-t-il ferme\,? Ils parleront sans doute, et vous avez bien
vu M. du Maine parler à moins et en plus grande compagnie, en plein
parlement. Il y contesta au régent le commandement des troupes de la
maison du roi et celui de tous ses officiers, même de ceux qui sont sous
votre charge. Le comte de Toulouse le laissa faire. Mais ici, où il
s'agit de la totalité, non comme alors d'une partie seulement et
ajoutée, ne soutiendra-t-il point son frère\,? Ceux qui leur sont unis
de cabale et de parti oseront-ils les abandonner, ou plutôt joints à eux
comme ils sont, s'abandonneront-ils eux-mêmes\,? Sentez-vous le bruit
que cela fera dans le conseil\,? Comptez-vous sur quelqu'un pour tenir
tête\,? Vous flattez-vous que. M. le duc d'Orléans saura imposer\,? ---
Mais, me dit-il, le plus court est de n'en point parler à la régence\,;
car il est vrai que cet inconvénient est très grand, et que je n'y avais
pas fait réflexion. Il n'y a qu'à ne parler à la régence que de
l'affaire du parlement\,; l'autre ne sera que plus secrète. Je n'y vois
que cela, qu'en pensez-vous\,? --- Monsieur, lui répondis-je,
\emph{angustiae undique}. Si aucun membre du conseil de régence n'avait
de séance au lit de justice, ce serait un tour de passe-passe à tenter
effrontément. Le parlement croirait que le conseil y aurait passé, et le
conseil n'en saurait rien que tout enregistré et quand il n'y aurait
plus de remède. Mais songez-vous que la régence entière sied au lit de
justice, excepté trois ou quatre, et y opine\,? Que diront donc des gens
à la pluralité de l'avis desquels le régent s'est engagé en plein
parlement de déférer pour affaires, lorsqu'en plein parlement et au
sortir du conseil de régence\,; ils entendront une affaire de la qualité
de l'éducation dont ils n'auront su chose quelconque et dans le temps où
le parlement s'excuse de tout ce qu'il fait sur le peu de part qu'on
donne des affaires au conseil de régence, et ne feint pas de dire qu'il
est poussé par plusieurs de ce conseil\,? Qu'arrivera-t-il si un
maréchal de Villeroy, de dessus son tabouret de service de gouverneur du
roi, s'écrie que cela lui est tout nouveau, qu'un maréchal de Villars
harangue, que les autres maréchaux de France, qui tous tiennent aux
bâtards, clabaudent\,? Que sais-je, si des pairs même ne s'en mêleraient
pas de dépit contre vous sûr le rang intermédiaire que vous voulûtes
lors de votre procès, qui a valu celui de princes du sang aux bâtards,
et de dépit encore du bonnet contre M. le duc d'Orléans\,? N'est-ce pas
une voie toute simple aux uns de se venger, aux autres de faire une
plainte oblique, mais pourtant solennelle de l'anéantissement du conseil
de régence dans une compagnie aigrie, à ce moment si blessée\,? Et
puisqu'elle a enregistré les conseils et les engagements que le régent
s'est fait à cet égard, n'est-elle pas très intéressée à soutenir celui
de régence\,? Les amis et la cabale des bâtards n'aura-t-elle pas beau
jeu\,; et comment M. le duc d'Orléans soutiendra-t-il les clameurs du
conseil non consulté dans la forme, et de la délibération qu'on en
voudra prendre pour le fond\,? Et si, les bâtards y sont, monsieur, que
sera-ce à votre avis et quelle force de plus\,? --- Les bâtards n'y
seront point, me dit-il\,; car, depuis notre arrêt, ils ne vont point au
parlement pour qu'il ne soit pas dit qu'ils l'exécutent. --- Mais s'ils
en ont le vent, ils y iront pour parer ce coup de partie. De plus,
entrant et sortant avec le roi, rien dans l'exécution de votre arrêt qui
les empêche d'y aller, parce qu'alors point d'huissier devant vous tous,
et que tout l'accompagnement du roi traverse, quoique nouvellement et
fort mal à propos, le parquet, et ceux qui ont séance en haut y montent
et en descendent avec le roi par la même nouveauté\,: ainsi nul embarras
aux bâtards pour monter et sortir de séance. --- Ils n'auront le vent de
rien, me dit-il, et de plus, s'ils y viennent, je n'ai qu'à sortir et à
demander qu'ils sortent. ---A la bonne heure, répondis-je, c'est un
expédient\,; mais cela fera mouvement\,; et dans ce mouvement on aura le
temps de se parler, de se fortifier contre le premier étonnement. Ceux
qui seront pour vous n'auront plus votre présence, et, comme il s'agit
de nouveauté en votre faveur et de détruire l'effet de la volonté
domestique du feu roi enregistrée en lit de justice, il faut bien plus
pour l'emporter que pour l'empêcher. Monsieur, ceci est capital au
moins, et cette mécanique est bien à balancer\,; car entamer une telle
affaire et en recevoir l'affront, vous voyez où cela jette. Je n'ai pas
besoin de vous le commenter. Et si à tout ce bruit et à quelque sottise
que peut fort bien dire le maréchal de Villeroy, le roi se prend à
pleurer et à dire qu'il veut M. du Maine, où tout ceci aboutira-t-il\,?
Monsieur, je vous le répète, je vous adjure comme Français, comme
successeur possible à la couronne par le droit de votre naissance, comme
enfant de la maison, que votre haine pour M. du Maine n'y mette pas le
feu. Quand vous l'y aurez porté, votre douleur tardive ne l'éteindra
pas, et vous ne vous consolerez jamais d'avoir mis le comble aux maux
d'un État qui, à tant de titres, vous doit être si précieux et si
cher.\,» Je me tus pour lui laisser faire ses réflexions.

Après quelques moments de silence il me dit que ces difficultés lui
étaient nouvelles, et que M. le duc d'Orléans ne les lui avait point
faites\,; que pourtant il y fallait penser et trouver un remède avant de
nous séparer\,; qu'il me le répétait donc aussi que ce seraient troubles
pour troubles, parce que ces deux choses étaient également et très
exactement vraies\,; qu'il était perdu si l'éducation demeurait au duc
du Maine, et qu'il ne verrait pas quatre ans durant venir sa perte sans
mettre le tout pour le tout pour l'empêcher\,; que tout bien considéré
encore, il n'était pas moins vrai que plus le temps s'avancerait plus
les bâtards aussi se fortifieraient, et plus l'éducation deviendrait
dangereuse à leur ôter, plus les connaissances du roi qui croîtraient
avec l'âge deviendraient périlleuses, et pour se porter à vouloir garder
le duc du Maine, et pour prendre toutes les impressions qu'il lui
voudrait donner\,; qu'il y avait plus qu'il ne risquait rien à me le
dire, quoique, M. le duc d'Orléans le lui eût donné sous le secret, et
après m'avoir conté la conversation du régent avec M. le comte de
Toulouse, il ajouta que Son Altesse Royale avait conçu tout ce qu'il y
avait à juger du duc du Maine par l'aveu de son frère qui n'en répondait
point.

Comme je le visse fonder en raisonnements là-dessus, et compter de
m'ébranler par la nouveauté d'un fait si considérable, je lui avouai que
M. le duc d'Orléans me l'avait raconté aussi, mais que ce fait, tout
considérable qu'il était, ne levait aucune des difficultés que je venais
de lui montrer, et prouvait seulement l'ineptie consommée de n'avoir pas
traité les bâtards comme je le voulais à la mort du roi. «\,Oui,
monsieur, reprit vivement M. le Duc, et en homme qui a pris son parti,
vous aviez grande raison, sans doute\,; mais plus vous aviez raison
alors et moins vous l'avez aujourd'hui. Pardonnez-moi si je vous parle
si librement, car votre raisonnement ne va qu'à nous laisser égorger par
ces MM. les bâtards à leur bon point et aisément, et en attendant qu'ils
le puissent par la majorité, à leur en laisser tranquillement tous les
moyens et toutes les forces. Or, si M. le duc d'Orléans est de cette
humeur-là pour sa vade\footnote{Ce mot se trouve déjà dans Saint-Simon
  dans le sens de \emph{pour son compte}.}, je ne suis pas si paisible
pour la mienne. Il est si grand qu'il espère apparemment leur échapper
d'une façon ou d'une autre, par force ou par reconnaissance de ne les
avoir pas écrasés, en quoi je crois qu'il se trouverait pris pour dupé.
Moi qui n'ai ni les mêmes ressources ni la même grandeur, encore un coup
je n'en crois point de trouble, et je ne crois point, leur affaire assez
arrangée\,; mais troubles pour troubles ils seront pires en différant\,;
et, en un mot, comme que ce soit l'éducation vendredi, monsieur\,! Alors
je suis un à jamais avec M. le duc d'Orléans, et nous verrons, tous les
princes du sang unis, ce que pourront les bâtards\,; autrement mon
ressentiment sera plus fort que moi\,; il ne sortira jamais de mon
coeur, et je me sens dès à présent en ce cas incapable de marcher d'où
je suis jusqu'à vous, et si il n'y a pas loin, pour son service. Je sais
toute la différence qu'il y a de lui à moi, mais au bout c'est à lui à
savoir s'il me veut ou s'il ne se soucie pas de me perdre. Je n'en sais
pas davantage. Il est régent, il doit être le maître pour des choses
qui, tout à la fois, sont justes et raisonnables et de son intérêt
personnel. C'est donc à lui à les vouloir et à les savoir faire, sinon
ce n'est pas la peine d'être à lui.\,» C'était là trancher toutes
difficultés et non pas les lever.

J'allais répondre lorsque après un moment de silence\,: «\,Monsieur,
reprit-il d'un air doux, modéré et flatteur, je vous demande pardon de
vous parler si ferme et je sens très bien que je pourrais fort bien
passer dans votre esprit pour une tête de fer et bien opiniâtre. Je
serais bien fâché que vous eussiez si méchante opinion de moi, mais je
vous prie de vous mettre en ma place, de peser l'état où je me trouve,
tous les manquements de parole que j'ai essuyés là-dessus qui me jettent
où nous voici. Je compte sur votre amitié\,; me conseilleriez-vous de me
perdre, et voyez-vous ceci passé un bout et une fin à l'établissement de
M. du Maine auprès du roi\,? Voilà ce qui me rend si ferme\,; et si vous
voulez bien peser ce qui peut vous paraître opiniâtreté vous trouverez
que c'est nécessité.\,»

Ce propos m'embarrassa extrêmement, non par sa politesse que j'aurais
payée de respects, mais par une solidité trop effective et d'autant plus
fâcheuse, qu'elle nous mettait entre deux écueils. Son aliénation
capable de tout en France et en Espagne d'une part, et d'autre part la
difficulté de réussir et les troubles qui en pouvaient naître\,:
détestable fruit de cette débonnaireté insensible qui, contre le
souvenir des plus énormes offenses et des plus grands dangers, contre
tout intérêt, toute raison, toute justice, contre toute facilité, tout
cri public et universel, tout sens commun, avait à la mort du roi laissé
subsister les bâtards. Je me recueillis autant qu'une conversation si
importante et si vive me le put permettre, et je connus bien que cette
décision de M. le Duc, venue avec impétuosité au bout de mes difficultés
si fortes pour toute réponse à leur embarras avoué, et les raisons
apportées ensuite en excuses de cette impétuosité, démontraient qu'il
n'y avait plus rien à espérer de M. le Duc, d'autant plus raffermi par
les confidences que M. le duc d'Orléans lui avait faites, surtout celle
de sa conversation avec le comte de Toulouse dont il eût si bien pu se
passer, et encore plus de lui laisser sentir toute l'impression qu'elle
lui avait laissée. Dans cette conviction je cessai de tenter
l'impossible, et content en moi-même du témoignage de ma conscience, par
tous les efforts si sérieux que j'avais faits pour le déprendre ou pour
éluder son dessein contre le duc du Maine, je me crus permis de profiter
au moins pour nous de ce que je ne pouvais empêcher pour le bien de
l'État.

Je dis donc à M. le Duc qu'après lui avoir dit et représenté tout ce que
j'estimais du danger en soi, et des difficultés de cette grande affaire,
j'abuserais vainement de son temps à lui rebattre les mêmes choses,
n'ayant plus rien de nouveau à lui alléguer\,; que je voyais avec
douleur que, quoiqu'il sentît les embarras infinis et de la chose et de
sa mécanique, son parti était pris\,; que, cela étant, j'en souhaitais
passionnément le succès, puisqu'il n'y avait point de remède, mais
qu'avant de le quitter, je le suppliais de vouloir bien s'expliquer avec
moi sur la réduction des bâtards à leur rang de pairie.

Il me répondit qu'il consentait volontiers qu'ils n'en eussent point
d'autre, et que je savais bien que c'était un des trois projets d'édits
qu'il avait proposés et donnés à M. le duc d'Orléans. «\,J'entends bien,
lui répliquai-je\,; mais autre chose est de laisser faire, autre chose
de vouloir. Je vous supplie de ne pas perdre le souvenir que le rang
intermédiaire qu'on vous avait mis dans la tête lors de votre procès
avec les bâtards leur a valu celui de princes du sang qu'ils ont encore
comme à la mort du roi, et de demeurer en outre dans toute la grandeur
que vous redoutez aujourd'hui avec tant de sujet, et dans laquelle vous
les voulez, attaquer par la moelle, qui est l'éducation. Vous fûtes
trahi depuis le commencement de cette affaire jusqu'à la fin. Ne
retombez pas dans les pièges qui vous furent tendus par des gens payés
par M. et M\textsuperscript{me} du Maine, que vous vous croyiez avec
raison très attachés. --- Je vous nommerai bien qui\,? interrompit M. le
Duc\,; c'est Lassai qui nous trompa toujours. --- Puisque vous le
nommez, monsieur, lui dis-je, nommez-les, tous deux le père et le fils,
et tout le monde s'en aperçut bien hors vous. C'est encore quelque chose
que vous n'en soyez plus la dupe. Or, je vous le répète, la faute
radicale, et qui sauva les bâtards, ce fut de ne nous avoir voulu ni à
votre suite, ni protéger. En ce cas ils étaient réduits en leur rang de
pairie. Par là plus de place au conseil de régence, sans les en chasser,
plus de moyen d'imposer au monde le respect qu'ils avaient accoutumé,
plus d'éducation, car en quel honneur le maréchal de Villeroy eût-il pu
demeurer sous M. du Maine\,? Lorsque votre procès fut jugé, j'en parlai
fortement à M. de Villeroy et lui demandai comment il pouvait rester
sous un homme qui n'était plus prince du sang habile à la couronne. Il
en fut si embarrassé qu'il me parut ébranlé. Qu'eût-ce donc été s'ils
avaient fait le saut, et nous en honneur, et par là en force de faire
chanter le maréchal de Villeroy, quand bien même il n'eût pas voulu\,?
Alors quelle facilité à M. le duc d'Orléans de satisfaire son intérêt,
en ôtant M. du Maine d'auprès du roi\,! Quelle facilité encore de l'y
pousser, et quel embarras même au duc du Maine d'y rester sans les
honneurs et le service de prince du sang, et avec tous les affronts de
changement et de chute de rang, dont les occasions chez le roi lui
eussent été continuelles\,! --- Tout cela est vrai, me dit M. le Duc,
aussi voyez-vous que je consens et que je propose même la réduction que
vous voulez. --- Mais, monsieur, repris-je, cela ne suffit pas\,; me
permettez-vous de vous parler librement\,; comptez que par cette idée de
rang intermédiaire lors de votre procès, vous vous êtes aliéné tous les
ducs, je dis tous ceux qui ont du sang aux ongles. Je ne vous parle pas
de misérables comme un duc d'Estrées, un M. Mazarin, un M. d'Aumont,
mais de tout ce qui se sent et se tient, et parmi ceux-là les ducs qui
étaient le plus à l'hôtel de Condé par l'ancien chrême de père en fils
de guerres civiles. Nous ne paraissons pas, parce que nous sommes cent
fois pis que sous la tyrannie passée, mais nous ne nous en sentons pas
moins, et nous ne nous en tenons pas moins ensemble, comme vous l'avez
pu remarquer en toutes les occasions. Vous êtes bien grand, monsieur,
par votre naissance de prince du sang, et par la situation où vous vous
trouvez\,; mais croyez-moi, et ne pensez pas pour cela que nous voulions
vous rapprocher de trop près\,: quelque élevé que vous soyez, il ne vous
doit pas être indifférent que tout ce qu'il y a de ducs et pairs sensés
et sensibles soient à vous ou n'y soient pas, et voici une occasion de
vous les dévouer. Ne la manquez pas, et réparez par là le passé envers
eux, car je ne vous le déguiserai point, que M. le duc d'Orléans serré
de près ne leur a pas laissé ignorer, que, sans votre résistance, leur
requête eût été jugée avec la vôtre, et les bâtards réduits à leur rang
de pairie unique\,: et toute la haine en est tombée sur vous.\,»

M. le Duc fut un moment sans répondre, puis me dit qu'il avait bien
envie que je visse les trois projets d'édits qu'il avait donnés à M. le
duc d'Orléans\,; que celui, par qui il les avait fait dresser était fort
connu de moi, et désirait fort que je lui voulusse donner une heure chez
moi le plus tôt que je pourrais\,; que c'était Millain que j'avais fort
connu secrétaire du chancelier de Pontchartrain qui les avait dressés\,;
qu'il était très capable et très honnête homme\,; qu'il se fiait fort en
lui, et que je pourrais lui parler en toute confiance.

Je saisis cette ouverture avec une avidité intérieure que je couvris de
politesse et de complaisance. Millain était fort homme d'honneur, de
règle et de sens, et par son mérite fort au-dessus de son état. Les
distinctions que je lui avais témoignées chez M. le chancelier de
Pontchartrain, fondées sur l'estime qu'il en faisait et après sur ce que
j'en connus par moi-même, me l'avaient attaché. À la retraite du
chancelier, il avait voulu continuer à prendre soin de ses affaires et
ce n'avait été qu'à condition de ne pas cesser qu'il avait cédé à
l'empressement du chancelier Voysin de l'avoir auprès de lui, et ensuite
à passer chez M. le Duc. Il était toujours demeuré dans les mêmes termes
avec moi, quoique les occasions de nous voir fussent devenues fort rares
depuis la retraite de son premier maître que j'allais voir souvent, mais
chez qui je ne le rencontrais plus. Il me parut à souhait à mettre entre
M. le Duc et moi et à m'en servir auprès de lui. Nous convînmes donc
qu'il viendrait le lendemain matin chez moi avec ces trois projets, et
cette promptitude me parut faire plaisir à M. le Duc.

Après propos là-dessus, que je laissai aller pour laisser mâcher à M. le
duc ce que je lui venais de dire de fort, et pour mettre un intervalle à
ce que j'avais dessein d'ajouter, je crus lui devoir serrer la mesure.
Je lui dis donc que je le suppliais de ne pas regarder comme manque de
respect, mais bien comme une confiance que l'affaire exigeait, et, que
celle dont il m'honorait dans tout ceci me donnait droit de prendre en
lui avec un aveu naturel que je lui allais faire dont je le conjurais de
ne se point avantager d'une part et de ne le point trouver mauvais de
l'autre\,; que, voyant sa fermeté à vouloir l'éducation, j'avais déjà
soupçonné qu'on ne viendrait pas à bout de l'en déprendre, et que dans
cette crainte j'avais voulu à tout hasard ce matin même sonder le régent
à fond sur la réduction des bâtards à leur simple rang de pairie\,; que
le régent pressé m'avait laissé voir que cela dépendrait de ce que lui
M. le Duc voudrait\,; et que serré de plus près il m'avait dit qu'il
doutait de la volonté par l'expérience contraire qu'il en avait\,; que
poussé par degrés j'en avais tiré l'aveu que, s'il le demandait
formellement, Son Altesse Royale le trouvait juste et utile et n'y
ferait aucune difficulté. Puis, sans donner à M. le Duc le temps de
penser, je continuai tout de suite d'un ton de désir et de respect\,:
«\,Vous voyez donc, monsieur, que notre sort est entre vos mains\,; nous
abandonnerez-vous encore une fois, et les grands du royaume qui le
demeureront quoi qu'on fasse et dont beaucoup sont grandement établis,
ne vous paraîtront-ils pas dignes d'être recueillis par vous\,? Je vous
dirai plus, monsieur, leur intérêt est si grand ici que je croirai bien
principal si on leur fait une justice si désirée qu'ils la sussent en
entrant en séance. En ce moment plus de péril pour le secret quand ils
seraient capables d'en manquer contre eux-mêmes, puisqu'ils ne peuvent
se déplacer, et ce serait un véhicule certain pour tourner en votre
faveur tout ce que vous avez lieu de craindre en haine de ce qui s'est
passé et en vengeance du bonnet contre le régent même. Près d'obtenir ce
qui leur tient le plus vivement au coeur de l'équité de Son Altesse
Royale par votre seul secours, comptez pour vous tout le banc des pairs
s'il s'agit de parler, et croyez qu'en un lit de justice cette portion
est bien capitale à avoir et impose grandement au reste de ce qui s'y
trouve. »

Cela dit, je pris un autre ton, et je continuai tout de suite avec un
air de chaleur et de force\,: «\,Après cela, monsieur, je ne puis vous
tromper\,; tout ceci, vous le voyez, vous le sentez comme moi. Mais
mettez-vous en notre place, comment seriez-vous touché pour qui vous
tirerait d'opprobre ou qui vous y laisserait\,? Je ne vous le dissimule
point, je dois trop à mes confrères, je dois trop à moi-même pour ne les
pas instruire à fond de ce qui se sera passé, pour qu'ils ne sachent
point par moi que c'est de votre main qu'ils tiendront ou leur honneur
rendu ou leur ignominie. Et moi, monsieur, qui ai l'honneur de vous
parler, permettez-moi de me servir de vos propres paroles sur M. le duc
d'Orléans, quoiqu'il y ait bien plus loin de nous à vous que de vous à
lui. Si vous nous abandonnez, je sens en moi un ressentiment contre vous
dont je ne serai point maître, qui durera autant que moi et que ma
dignité, qui se perpétuera dans tous ceux qui en sont revêtus, qui nous
éloignera de vous pour jamais, et qui, se ployant au seul respect
extérieur qui ne vous peut être refusé, me détournera le premier, et
tous les autres avec moi, des plus petites choses de votre service. Que
si, au contraire, vous nous remettez en honneur et les bâtards en règle,
moi plus que tous, et tous avec moi, sommes à vous, monsieur, pour
jamais et sans mesure, parce que je vous crois très incapable de rien
vouloir faire contre l'État, le roi et le régent, et je vous mène dans
l'hôtel de Condé tous les pairs de France vous vouer leur service, et
des leurs, et toute leur puissance dans leurs charges et leurs
gouvernements. Pesez, monsieur, pesez l'un avec l'autre, pesez bien ce
qu'il vous en coûtera, comptez bien sur la solidité de tout ce que je
vous dis en l'un comme dans l'autre cas, et puis choisissez.\,» Je me
tus tout court après cette option si vivement offerte, bien fâché que
l'obscurité empêchât M. le Duc de bien distinguer le feu de mes yeux, et
moi-même de perdre par la même raison toute la finesse de la
connaissance que j'aurais pu tirer de son visage et de son maintien dans
sa réponse.

Il me dit tout aussitôt, et voici les propres paroles\,: «\,Monsieur,
j'ai toujours honoré votre dignité et la plupart de ceux qui en sont
revêtus. Je sens très bien quelle est pour moi la différence de les
avoir pour amis ou pour indifférents, encore pis pour ennemis. Je vous
l'ai déjà avoué, j'ai fait une faute à votre égard, messieurs, et j'ai
envie de la réparer\,; je sens encore qu'il est juste qu'il n'y ait rien
entre nous et vous. Mais M. le duc d'Orléans vous parle-t-il bien
sincèrement quand il vous promet la réduction des bâtards à leur rang de
pairie si je la lui demande\,? Car ne m'allez pas charger d'une iniquité
qui ne serait pas mienne. --- Monsieur, lui répondis-je, c'est mon
affaire\,; la vôtre est d'opter nettement. Voulez-vous de nous à ce
prix, ou vous paraît-il trop cher\,? --- Moi, monsieur, interrompit-il
avec vivacité, de tout mon coeur\,; mais en faisant de mon mieux, vous
aurai-je, ou dépendrai je du succès\,?» J'interrompis aussi avec
véhémence\,: «\,Point de cette distinction, s'il vous plaît. Le succès
est en vos mains\,; il ne s'agit que de demander la réduction du rang,
du ton et de la force dont vous demandez l'éducation\,; ne les séparez
point, insistez également\,; vous en sentez les raisons, en elles-mêmes
bonnes et vraies\,; vous en devez sentir autant les raisons
particulières à vous. En vous y prenant de la sorte, c'est moi qui vous
en réponds. M. le duc d'Orléans, vous accordant le plus difficile, ne
peut vous refuser le plus simple et le plus aisé, le jugement équitable,
avoué tel de lui et de vous, d'un procès pendant. --- Oh bien, monsieur,
reprit M. le Duc, je vous en donne ma parole\,; j'y ferai comme pour
l'éducation dans demain\,; mais promettez-moi aussi de faire de votre,
mieux. --- Doucement, monsieur, repris-je\,; avec cette parole vous avez
la mienne, et j'ose vous dire celle de tous les ducs, d'être à vous sans
mesure, le roi, l'État et le régent, exceptés, qui sont la même chose,
et contre qui vous ne voudrez jamais rien. Mais sur M. du Maine je ne
puis vous promettre que ce que j'ai déjà fait, de proposer à M. le duc
d'Orléans les raisons pour et contre\,; et, s'il se détermine à ce que
vous désirez, de m'y mettre jusqu'au, cou pour le succès.\,» Là-dessus,
protestations, embrassades et retour aux moyens sur les inconvénients
mécaniques.

Je lui dis que je croyais qu'il fallait séparer les deux frères, et pour
le bien de l'État qu'il nous en coûtât le rang du comte de Toulouse tel
qu'il l'avoir. M. le Duc me demanda avec surprise comment je
l'entendais. «\,Le voici, dis-je\,: je ne puis m'ôter de l'esprit que
celui-ci ne mette le tout pour le tout en cette occasion par toutes les
raisons que je vous en ai alléguées, ni que sa jonction et personnelle
et par ses charges ne donne un grand poids à leur parti. Écartons donc
cet écueil par notre propre sacrifice, qui n'en est pas un pour vous, et
au lieu de ce poids donné au duc, du Maine, accablons-l'en. Mettons le
monde de notre côté, et tâchons de jeter entre les deux frères une
division dont ils ne reviennent jamais. --- De tout mon coeur, s'écria
M. le Duc\,; vous voyez si j'aime le comte de Toulouse, et dès que vous
le voudrez bien, de tout mon coeur je contribuerai à le laisser comme il
est. Mais en serons-nous plus avancés\,? --- Oui, monsieur, lui
dis-je\,; écoutez-moi de suite, et puis vous verrez ce qu'il vous en
semblera. Je voudrais, par un seul et même acte, faire la réduction des
bâtards au rang de leurs pairies, et par un autre, tout au même instant,
rendre au comte de Toulouse seul, et pour sa seule personne, le rang
entier dont il jouit aujourd'hui\,; ne rien omettre dans le premier de
tout ce qui le peut rendre plus fort\,; insérer dans le second tout ce
que l'exception peut avoir de plus flatteur, et en même temps de plus
uniquement personnel et de plus confirmatif de la règle du premier. Par
là nul retour pour le rang en soi\,; les enfants exclus s'il vient à se
marier et à en avoir\,; par là un honneur sans exemple fait à la
personne du cadet, qui retombe à plomb en opprobre sur l'aîné, qui lui
devient un outrage à toujours à lui et à ses enfants à cause de lui, qui
met sa femme dans une fureur à n'en jamais revenir contre son
beau-frère, et qui constitue ce beau-frère dans une situation très
embarrassante dont nous n'avons qu'à profiter, quoi qu'il fasse\,; car,
monsieur, suivez-moi, je vous prie, ce comte de Toulouse, si droit, si
honnête homme, si sage, si considéré, que deviendra-t-il dans un cas si
inouï et auquel il n'aura pu se préparer\,? Il n'aura que deux partis à
prendre, et à prendre sur-le-champ\,: refuser ou accepter. Refuser, il y
pensera plus de quatre fois de sacrifier tout ce qu'il est et une
distinction aussi éclatante à un frère qu'il n'aima ni n'estima jamais,
qui, contre son avis, s'est exposé à tout ceci par un essor effréné
d'ambition, que celui-ci a blâmé en public et en particulier\,; de se
dévouer ainsi aux caprices, aux folies, aux fureurs d'une belle-sœur
qu'il abhorre comme une folle, une furieuse, une enragée, qui a poussé
son frère aux entreprises dont voici l'issue\,; au danger de passer de
la simple ingratitude à la révolte ouverte. Attaché au sort de son frère
conduit et mené par sa femme, à tout le moins mal avec eux s'il ne suit
leur fortune et toutes leurs entreprises, et plongé, pour le reste d'une
vie encore peu avancée, dans une retraite oisive et volontaire, point
différente d'un exil, dont la solitude lui deviendra tous les jours plus
pesante, qui ne le nourrira que des regrets les plus cuisants de ce
qu'il aura abandonné pour rien, croyez-vous que cette idée, branchue et
affreuse dans l'une et dans l'autre de ses deux branches, ne l'effrayera
point, et que cette indolence naturelle, cette probité, cet honneur, se
laisseront porter aisément à embrasser ce parti\,? S'il s'y précipite,
plus rien à craindre du public en sa faveur pour révoquer la déclaration
et le traiter sûr le rang comme son frère. Il l'aura mérité alors, parce
qu'il l'aura voulu, en méprisant une grâce sans exemple, et grâce
uniquement fondée sur l'estime que sa conduite alors démentira
publiquement\,; alors il ne sera pas plus à craindre que son frère, et
il ne lui ajoutera personnellement aucun poids. Le gouvernement sera
pleinement disculpé à cet égard, et les amis du comte de Toulouse seront
les premiers à le blâmer parce qu'il sera blâmable, et par leur chagrin
de se voir privés de son appui par la sottise de son choix. Le danger
prévenu n'en paraîtra qu'avec plus d'évidence, parce qu'on verra alors
la force et le nerf de la cabale se montrer supérieur à l'éclat inouï et
aux devoirs les plus grands et les plus nouveaux de la reconnaissance,
dont la seule estime avait été si puissante. Cette estime tombera, et
avec elle la distinction offerte éclatera par la modération et la
sagesse, et acquerra une pleine liberté de se tourner contre les effets
d'une passion si dangereuse dans des bâtards sans mesure agrandis et
ménagés sans mesure. Si le comte de Toulouse accepte, rien à craindre de
lui, tout au moins en ayant attention sur sa conduite. Il est dès lors,
par ce choix, hors de portée d'agir pour son frère contre le
gouvernement sans se déshonorer, ce qu'il ne fera jamais\,; tout son
poids non plus réuni à son frère, mais retombé à plomb sur lui. Ce frère
et encore plus M\textsuperscript{me} du Maine, accablés de la douleur et
de la rage de ce poids qui les écrasera, de cette séparation qui leur
ôtera tant de, force, de cette distinction si injurieuse pour eux et si
pesante à leurs enfants, tourneront une partie de leur fureur secrète
contre le comte de Toulouse, avec lequel désormais ils ne pourront
jamais plus avoir ni liaison ni confiance. Tout ce qui est
personnellement uni au comte de Toulouse, ravi de le voir si
glorieusement échappé, rira des éclats de la duchesse du Maine et des
désolations de son mari. Par cette voie, rien à craindre de la Bretagne
demi soulevée, ni de ce peu de marine, ni du public amoureux de la vertu
du comte de Toulouse, parce que cette vertu devient sans force s'il
refuse, et s'il accepte, se trouve récompensée outre mesure\,; et avec
cela plus de reproches à se faire, quelque parti qu'il prenne, de
l'avoir forcé à la révolte et précipité dans le malheur. Plus on ira en
avant, plus l'aigreur s'augmentera entre les frères et entre leurs
maisons\,; plus le comte de Toulouse achèvera de se dégoûter de M. et de
M\textsuperscript{me} du Maine, et s'applaudira intérieurement de la
différence de son état au leur, plus ses amis et ses principaux
domestiques la lui feront sentir et mettront peine à l'empêcher de
tomber dans les filets qui lui seront tendus de, cette part. Tout le
monde, qui aime et estime l'un, et qui méprise et déteste les autres,
applaudira, les uns par goût, les autres par équité, à la modération de
cette différence, qui, devenue la pomme de la discorde entre les deux
frères, rassurera contre eux. Voilà, monsieur, ce que j'imagine aux
dépens de mon rang pour le bien de l'État et pour sauver un homme dont
le mérite simple m'a captivé\,; qu'en pensez-vous\,? --- Rien de mieux,
me dit M. le Duc, mon amitié y trouve son compte\,; et en effet le comte
de Toulouse sera bien embarrassé. S'il refuse, il s'attire tout, et
n'aura que ce qu'il mérite, dont le public sera juge et témoin\,; s'il
accepte, et je le crois à cette heure que j'ai tout entendu, nous avons
notre but\,; mais j'avoue que d'abord j'ai cru qu'il n'accepterait pas.
--- Mais, monsieur, repris je, il serait fou de refuser, et il a des
gens auprès de lui qui, pour leur part, y perdraient trop et qui
n'oublieront rien pour qu'il accepte. Quoi qu'il fasse, son sort sera
entre ses mains. Cela nous doit satisfaire pour le coeur\,; mis pour
l'esprit, l'êtes-vous, et trouvez-vous quelque difficulté ou quelque
autre chose à y faire\,? --- Non, me dit-il, monsieur, et je suis charmé
de cette vue\,; je vais dire à Millain de travailler à un projet de
déclaration pour cela. --- Et moi, monsieur, j'en raisonnerai demain
matin avec lui\,; mais j'en veux dresser une aussi\,; et qu'il soit dit
que, pour le bien de l'État, des pairs l'aient faite eux-mêmes contre
eux-mêmes.\,»

Il loua ce désintéressement si peu commun, et les différentes raisons et
vues de ce projet de distinction du comte de Toulouse, après quoi il me
remit sur les difficultés mécaniques que moi-même j'avais formées. Je
lui dis qu'il y fallait bien penser, les proposer à M. le duc d'Orléans,
et sonder surtout ce qu'on pouvait attendre de sa fermeté, qui serait
perpétuellement et principalement en jeu dans toute cette grande
exécution\,; que maintenant qu'il me donnait sa parole pour ce qui
regardait notre rang, je ne craignais pas de lui engager celle de tous,
les pairs d'être pour lui au lit de justice\,; que parmi eux, le duc de
Villeroy, par ordre du maréchal son père, donné à lui de ma
connaissance, et le maréchal de Villars, tenants principaux du duc du
Maine, avaient signé la requête que nous avions présentée au roi et au
régent en corps contre les bâtards, qui était pour eux en cette occasion
une furieuse entrave\,; que les pairs pour lui entraîneraient presque
tous les autres au lit de justice\,; que je doutais que les autres
maréchaux de France, destitués de ceux-là, osassent y faire, du bruit\,;
mais que les deux grands embarras consistaient à dire ou à taire à la
régence les déclarations ou édits sur les bâtards, et à savoir que faire
tant au conseil qu'au lit de justice, si les bâtards s'y trouvaient.

Après avoir bien raisonné, nous crûmes pouvoir espérer assez de la
misère de messieurs de la régence pour préférer de n'y hasarder point ce
qui regarderait les bâtards, s'ils étaient au conseil, et ne le déclarer
qu'au lit de justice, et que là, si les bâtards y étaient, c'était au
régent à payer de fermeté.

En nous quittant, je pris encore la parole positive de M. le Duc qu'il
ferait auprès du régent sa propre affaire de la réduction des bâtards au
rang de leur pairie, comme de l'éducation même, et je l'adjurai encore
comme Français et comme prince du sang, de passer la nuit et la matinée
prochaines à méditer sur de si grandes choses, et à préférer le bien de
l'État à ce qui lui était personnel. Il me le promit, me dit encore
mille choses obligeantes, et me demanda l'heure pour Millain, que je lui
donnai pour le lendemain matin entre huit et neuf heures. Il me pria de
voir le régent dans la matinée, et quoique je lui répétasse que ce
serait sans plaider sa cause, mais en remontrant les dangers pour et
contre, il ne laissa pas que de me faire encore l'honneur de
m'embrasser. Il était fort tard, et, sans l'accompagner de peur de
rencontre, j'enfilai l'allée basse sous la terrasse de la rivière, et
revins chez moi dans une grande espérance pour notre rang, mais la tête
bien pleine du grand coup de dé que je voyais sur le point de se
hasarder.

\hypertarget{chapitre-xix.}{%
\chapter{CHAPITRE XIX.}\label{chapitre-xix.}}

1718

~

{\textsc{Millain chez moi, avec ses trois projets d'édits, me confirme
la parole de M. le Duc sur le rang\,; me promet de revenir le lendemain
matin.}} {\textsc{- Satisfaction réciproque.}} {\textsc{- Je rends
compte au régent de ma conversation avec M. le Duc.}} {\textsc{- Son
Altesse Royale déterminée à lui donner l'éducation.}} {\textsc{- Je
proteste avec force contre la résolution de toucher au duc du Maine\,;
mais, ce parti pris, je demande alors très vivement la réduction des
bâtards au rang de leur pairie.}} {\textsc{- Cavillations du régent.}}
{\textsc{- Je le force dans tous ses retranchements.}} {\textsc{- Je
propose au régent le rétablissement du comte de Toulouse, qu'il
approuve.}} {\textsc{- Reproches de ma part.}} {\textsc{- Je propose au
régent les inconvénients mécaniques, et les discute avec lui.}}
{\textsc{- Je l'exhorte à fermeté.}} {\textsc{- Avis d'un projet peu
apparent de finir la régence, que je mande au régent.}} {\textsc{- M. le
Duc vient chez moi me dire qu'il a demandé au régent la réduction des
bâtards au rang de leurs pairies, et s'éclaircir de sa part sur l'avis
que je lui avais donné.}} {\textsc{- J'apprends chez moi au duc de La
Force à quoi en sont les bâtards à notre égard, et le prie de dresser la
déclaration en faveur du comte de Toulouse.}} {\textsc{- Frayeur du
parlement.}} {\textsc{- Ses bassesses auprès de Law.}} {\textsc{-
Infamie effrontée du duc d'Aumont.}} {\textsc{- Frayeur et bassesses du
maréchal de Villeroy.}} {\textsc{- Conférence chez moi avec Fagon et
l'abbé Dubois sur tous les inconvénients et leurs remèdes.}} {\textsc{-
Fagon m'avise sagement de remettre au samedi d'arrêter les membres du
parlement, qui le devaient être le vendredi.}} {\textsc{- Le duc de La
Force et Millain chez moi avec la déclaration en faveur du comte de
Toulouse.}} {\textsc{- Millain m'avertit de la part de M. le Duc, chargé
par le régent, de me trouver le soir à huit heures chez le régent, pour
achever de tout résumer avec lui et M. le Duc en tiers, et d'y mener
Millain.}} {\textsc{- Je parle à Millain sur la réduction des bâtards à
leur rang de pairie avec la dernière force, et je le charge de le dire
mot pour mot à M. le Duc.}} {\textsc{- Contre-temps à la porte secrète
de M. le duc d'Orléans.}} {\textsc{- Je lui fais approuver le court
délai d'arrêter quelques membres du parlement.}} {\textsc{- Discussion
entre le régent et moi sur plusieurs inconvénients dans l'exécution du
lendemain.}} {\textsc{- M. le Duc survient en tiers.}} {\textsc{- Je les
prends tous deux à témoin de mon avis et de ma conduite en toute cette
affaire.}} {\textsc{- Je les exhorte à l'union et à la confiance
réciproque.}} {\textsc{- Je leur parle de la réduction des bâtards au
rang de leur pairie avec force et comme ne pouvant plus en douter, en
ayant leur parole à tous les deux.}} {\textsc{- Ils m'avertissent de ne
pas manquer à revenir le soir au rendez-vous avec eux deux.}} {\textsc{-
M. le Duc m'envoie par Millain la certitude de la réduction des bâtards
au rang de leur pairie, dont j'engage M. le Duc à s'assurer de plus en
plus.}} {\textsc{- Conférence chez moi avec le duc de La Force, Fagon et
l'abbé Dubois.}} {\textsc{- Tout prévu et remédié autant que possible.}}
{\textsc{- Conférence, le soir, entre M. le duc d'Orléans, M. le Duc et
moi, seuls, où Millain fut en partie seul avec nous, où tout se résume
pour le lendemain et les derniers partis sont pris.}} {\textsc{- Je suis
effrayé de trouver le régent au lit avec la fièvre.}} {\textsc{-
Solutions en cas de refus obstiné du parlement d'opiner.}} {\textsc{-
Pairs de France, de droit, et officiers de la couronne, de grâce et
d'usage, ont seuls voix délibérative au lit de justice et en matière
d'État, et les magistrats au plus consultative, le chancelier ou garde
des sceaux excepté.}} {\textsc{- Je confie, avec permission de Son
Altesse Royale, les événements si prochains au duc de Chaulnes.}}
{\textsc{- Contade fait très à propos souvenir du régiment des gardes
suisses.}} {\textsc{- Frayeur du duc du Maine d'être arrêté par lui.}}
{\textsc{- On avertit du lit de justice à six heures du matin ceux qui y
doivent assister.}} {\textsc{- Le parlement répond qu'il obéira.}}
{\textsc{- Discrétion de mon habit de parlement.}} {\textsc{- Je fais
avertir le comte de Toulouse d'être sage et qu'il ne perdra pas un
cheveu.}} {\textsc{- Valincourt\,; quel.}}

~

Le lendemain mercredi 24 août, Millain entra chez moi précisément à
l'heure donnée avec les trois projets qu'il avait dressés. Il me fit
mille compliments de la part de M. le Duc, et me dit la joie qu'il
sentait de le savoir maintenant convaincu du panneau du rang
intermédiaire, qu'il avait inutilement tâché de lui démontrer lors du
procès des princes du sang avec les bâtards. Après être entrés en
matière avec les propos de gens qui se connaissent de longue main, et
qui, à différents égards, sont bien aises de se retrouver ensemble en
affaires, il me conta que le matin même, M. le Duc l'avait envoyé
chercher, lui avait rendu le précis de nos conversations, et lui avait
avoué qu'il n'avait pas fermé l'oeil de toute la nuit dans l'angoisse en
laquelle il se trouvait\,; que néanmoins, son parti était pris par les
raisons qu'il m'avait dites\,; qu'il me tiendrait parole aussi sur notre
rang\,; et que lui Millain m'apportait les projets d'édits qu'il avait
toujours désiré pouvoir me communiquer. Nous les lûmes\,: premièrement,
celui pour le seul changement de la surintendance de l'éducation du
roi\,; après, celui du rang intermédiaire\,; enfin, celui de la
réduction des bâtards au rang de leurs pairies, révoquant tout ce qui
avait été fait au contraire en leur faveur. J'entendis le second avec
peine\,; et ne m'arrêtai qu'au premier et au dernier qui étaient
parfaitement bien dressés, le dernier surtout, selon mon sens, et tel
qu'il a paru depuis. Je dis à Millain qu'il fallait travailler à celui
du rétablissement du comte de Toulouse, sans préjudice de celui que je
voulus aussi dresser\,; et que, s'il voulait revenir le lendemain à
pareille heure, nous nous montrerions notre thème l'un à l'autre, pour
convenir de l'un des deux ou d'un troisième pris sur l'un et sur
l'autre. Je le chargeai de bien entretenir M. le Duc dans la fermeté
nécessaire sur ce qui nous regardait, en lui en inculquant les
conséquences, et, après une assez longue conférence, nous nous
séparâmes.

Aussitôt après j'allai au Palais-Royal, par la porte de derrière, où
j'étais attendu pour rendre compte au régent de ma conversation avec M.
le Duc. Il ferma la porte de son grand cabinet, et nous nous promenâmes
dans la grande galerie. Dès le premier demi-quart d'heure je m'aperçus
que son parti était pris sur l'éducation en faveur de M. le Duc, et que
je n'avais pas eu tort la veille, aux Tuileries, de l'avoir soupçonné de
s'être trop ouvert et trop laissé aller à ce prince, comme je m'en étais
bien aperçu avec lui dans ce jardin. Mes objections furent vaines.
L'éclaircissement sur M. le comte de Charolais et l'aveu du comte de
Toulouse sur son frère avaient fait des impressions, que le repentir
d'avoir différé et les raisons et les empressements de M. le Duc, dans
la conjoncture présente et si critique, avaient approfondies. Je ne
laissai pas de représenter à Son Altesse Royale le danger évident
d'attaquer le duc du Maine à demi, les embarras qu'il trouverait chez
lui-même à le dépouiller, celui de retirer M. le comte de Charolais des
pays étrangers par un grand gouvernement s'il ne le trouvait chez le duc
du Maine. Le régent convint de tout cela, et, dans le désir d'ôter
l'éducation à ce dernier, son dépouillement lui parut facile, parce
qu'il ne le considéra qu'en éloignement et ne voulut point ouïr parler
de tout faire ensemble, encore qu'il n'y eût point de comparaison, et
dans ce dépouillement il trouvait à tenir parole au comte de Charolais.

Je le vis si arrêté dans ces pensées que je crus inutile de disputer
davantage. Je me contentai de le supplier de se souvenir que ce qu'il
méditait contre le duc du Maine était contre mon sentiment, et de le
sommer de n'oublier pas que, contre mon intérêt le plus précieux et ma
vengeance la plus chère, j'avais lutté de toutes mes forces contre lui
et contre M. le Duc en faveur du duc du Maine, parce que je croyais
dangereux au repos de l'État de l'attaquer avec le parlement.

Ensuite, je lui proposai la réduction des bâtards au rang de leurs
pairies, et je me gardai bien de lui laisser entrevoir ce dont j'étais
convenu là-dessus avec M. le Duc. J'étais bien fort par les preuves que
je donnais sans cesse depuis cinq jours de mon désintéressement à cet
égard, et par la raison évidente que le duc du Maine, chassé d'auprès du
roi, et dans l'idée présente près d'être dépouillé de tous ses
établissements, n'était bon qu'à affaiblir d'autant. J'y ajoutai
l'ancienne et palpable raison que cette réduction de rang de plus ou de
moins ne rendrait le duc du Maine ni plus outré ni moins
irréconciliable, et la justice et la facilité de cette opération qui ne
consistait qu'à prononcer sur un procès pendant et instruit.

Le régent me passa tout, hors ce dernier point\,; il me voulut soutenir
que le procès existait bien à la vérité par la présentation de notre
requête en corps signée au roi et à lui lors du procès des princes du
sang et des bâtards\,; mais il me contesta les formes. La réponse fut
aisée\,: point de formes devant le roi, notre requête admise, puisque le
roi et lui l'avaient reçue, et que lui-même l'avait communiquée aux
bâtards\,; qu'il n'y en avait point eu d'autres au procès long et
célèbre que les pairs eurent et gagnèrent en 1664 devant le roi contre
les présidents à mortier au parlement de Paris et le premier président,
sur la préopinion aux lits de justice. Cela ferma la bouche à M. le duc
d'Orléans, mais il se rejeta à m'objecter que les bâtards n'avaient pas
répondu. Je répliquai qu'ils en avaient eu tout le temps, et que, si
cette raison était admise, il ne tiendrait qu'à celui qui aurait un
mauvais procès devant le roi de ne répondre jamais, puisqu'il n'y avait
point de formalités pour l'y forcer, moyennant quoi il n'en verrait
jamais la fin. Après quelques légères disputes, il se rendit et m'ouvrit
la carrière à lui représenter, pour ne pas dire reprocher, ses méfaits à
notre égard sur le bonnet et sur tant d'autres choses. Il m'allégua pour
dernier retranchement la noblesse qu'il ne voulait pas soulever. Je lui
remontrai, avec une indignation que je ne pus contraindre, que c'était
lui-même qui l'avait soulevée, et qui s'en était trouvé bien empêché
après\,; que la noblesse n'avait que voir ni aucun intérêt à ce que le
duc du Maine nous précédât ou que nous le précédassions\,; que toutes
les lois et les exemples étaient pour nous, et qu'il n'y avait que son
acharnement à lui régent contre nous, jusque contre son intérêt propre,
qui nous pût être contraire. Enfin je le réduisis à m'avouer que ce que
je lui demandais était plutôt bon que mauvais, que la noblesse n'avait
ni intérêt ni droit de s'en mêler, et qu'il était vrai encore que notre
demande était juste\,; mais il m'objecta M. le Duc, et c'était où je
l'attendais. Je le laissai dire là-dessus, et comme prendre haleine de
l'acculement où j'avais réduit son incomparable fausseté, et je le
contredis faiblement pour l'attirer à la confiance en cet obstacle, à
avouer que c'était le seul.

Quand je l'y tins de manière à ne pouvoir échapper, je lui dis que M. le
Duc sentait mieux que lui la conséquence de nous avoir tous pour amis,
et de réparer par là le mal qu'il nous avait fait\,; qu'il n'ignorait
pas que Son Altesse Royale avait eu la bonté, lors de son procès avec
les bâtards, de se décharger sur lui de toute notre haine\,; qu'il
désirait la faire cesser, d'autant plus qu'il sentait maintenant
l'illusion et la faute du rang intermédiaire\,; qu'il lui demanderait
expressément la réduction des bâtards au rang d'ancienneté de leurs
pairies, et que nous verrions alors jusqu'où Son Altesse Royale
pousserait sa mauvaise volonté à notre égard\,; que, pour moi, je lui
avouais que j'étais tous les jours étonné de moi-même de ce que je
pouvais le voir, lui parler, lui demeurer attaché, avec la rage que
j'aurais dans le coeur contre tout autre qui nous aurait traités comme
il avait fait\,; que c'était le fruit de trente années d'habitude et
d'amitié, dont je m'émerveillais tous les jours de ma vie\,; mais qu'il
ne fallait pas qu'il jugeât du coeur des autres par le mien à son égard,
qui n'étaient pas retenus par les mêmes prestiges, et qu'il avait grand
besoin de se rattacher.

Je me tus alors et m'attachai moins à écouter sa réponse qu'à examiner à
son visage l'effet d'un discours si sincère, et qui, pour en dire la
vérité, aurait pu l'être davantage. Je le vis rêveur et triste, la tête
basse, et comme un homme flottant entre ses remords et sa faiblesse, et
en qui même sa faiblesse combattait de part et d'autre. Je ne voulus pas
le presser pour lui donner lieu de sentir une sorte d'indignation qui
aurait usurpé un autre nom avec un autre homme, et que j'estimai qui
ferait une plus forte impression sur lui que plus de paroles et de
véhémence. Néanmoins, le voyant toujours pensif et taciturne un temps
assez long\,: «\,Eh bien\,! monsieur, lui dis-je, nous égorgerez-vous
encore et malgré M. le Duc.\,» Il se prit à sourire, et répondit d'un
air flatteur qu'il n'en avait pas du tout envie\,; qu'il verrait si M.
le Duc le voulait tout de bon, et que, cela étant, il le ferait\,: «\,Je
n'en suis point en peine, repris-je, si vous tenez parole\,; car vous
verrez ce que M. le Duc vous dira, mais le ferez-vous\,? --- Oui
assurément, repartit-il\,; je vous dis que j'en ai envie, et que je
l'eusse fait dès l'autre fois sans lui, et je le ferai celle-ci s'il le
veut.\,» Je craignis l'échappatoire, mais je ne voulus pas le pousser
plus loin. Je répondis que c'était ce qu'il pouvait faire de plus sage
et de plus de son intérêt, et je tournai sur le comte de Toulouse.

Je lui déduisis ma pensée, mon projet, mes raisons. Il les approuva
toutes et parce qu'elles étaient bonnes, et parce, encore plus, que cela
le déchargeait de la moitié de la besogne. Après je m'avantageai d'une
proposition qui nous ôtait la moitié de notre rétablissement, et lui fis
honte qu'il eût besoin de la demande de M. le Duc pour nous faire une
justice reconnue telle par lui-même et de son intérêt, tandis que je
m'étais si fortement opposé au mien le plus cher sur le duc du haine
pour l'amour de l'État, que je ne revendiquais que sur ce qu'il n'y
pouvait plus nuire dès que M. du Maine perdait l'éducation, et tandis
encore que je proposais moi-même de conserver le rang au comte de
Toulouse par la même considération du repos du royaume. Il ne put
désavouer des vérités si présentes, que je ne crus pas devoir presser
davantage, et je passai aux inconvénients mécaniques que j'avais
objectés à M. le Duc.

Le régent n'y avait pas fait la plus petite réflexion. Je les lui
présentai tous. Nous convînmes que, s'il pouvait compter sur les pairs
au lit de justice, il valait mieux risquer le paquet de ne point parler
des bâtards au conseil de régence. Cela me donna lieu de lui faire faire
légèrement attention au besoin qu'il avait des pairs, et sur l'utilité
que je leur pusse dire en entrant en séance la justice qui leur était
préparée. Il en convint. Après, nous traitâmes la grande question, qui
fut sa fermeté à y soutenir la présence des bâtards, et ce qui, par eux
et par leurs adhérents, pourrait être disputé en leur faveur. Je lui
proposai l'expédient de faire sortir M. le Duc, que ce prince m'avait
fourni, pour faire aussi sortir les bâtards. Le régent l'approuva fort
et promit merveilles de lui-même, espérant toujours que les deux frères
ne viendraient pas au lit de justice pour n'y pas exécuter le dernier
arrêt. Je lui fis sentir le frivole de cette espérance, par les mêmes
raisons dont j'en avais désabusé M. le Duc. Mais le régent, toujours
porté à l'espérance, voulut toujours se flatter là-dessus.

Je l'exhortai à se préparer à bien payer de sa personne\,; je lui
inculquai que du succès de ce lit de justice dépendait toute son
autorité au dedans et toute sa considération au dehors. Il le sentit
très bien et promit merveilles\,; mais ma défiance ne laissait pas de
demeurer extrême. Je le suppliai de se souvenir de toute la faiblesse
qu'il montra en la première séance de la déclaration de sa régence où
tout lui était si favorable, des propos bas et embarrassés qu'il y tint
pour le parlement, qui en tirait maintenant de si grands avantages,
jusqu'à en fonder de nouvelles prétentions et lui alléguer ces faits
devant le roi en pleines remontrances. Je lui rappelai de plus l'état où
dans cette première séance le réduisit l'insolente contestation du duc
du Maine sur le commandement des troupes de la maison du roi, dans
laquelle il eût succombé si je ne lui avais pas fait rompre la séance,
et remettre à l'après-dînée, et dans l'entre-deux si je ne lui avais pas
fait concerter tout ce qu'il y avait à dire et à faire. J'ajoutai que,
maintenant qu'il s'agissait du tout pour le duc du Maine, il devait
ranimer et ramasser toutes ses forces pour résister à un homme qui,
ayant su l'embarrasser dans un temps où tout était contre lui, mettrait
ici le tout pour le tout, appuyé d'un parlement aigri et pratiqué, et
sentant lui-même ses propres forces. Le régent entra bien dans toutes
ces réflexions, essaya de s'excuser sur la nouveauté pour lui de cette
première séance, et promit de soi plus je pense qu'il n'en espérait.

Nous descendîmes ensuite dans une autre sorte de mécanique à l'égard du
parlement, et nous convînmes qu'il prendrait ses mesures à tous égards
là-dessus dans la journée avec le garde des sceaux. Il me dit que l'abbé
Dubois était allé en conférer avec lui, et avait fait un mémoire de tout
ce qui pourrait arriver de difficultés de la part du parlement. Il
ajouta qu'il désirait que j'en conférasse avec ceux du secret, et
s'efforça de me montrer une résolution entière. Il n'oublia pas de me
demander avec grand soin si j'avais remédié à l'élévation des hauts
sièges. Il eut bien de la peine à se contenter des trois marches qu'ils
devaient avoir\,; c'est une grippe, pour user de ce mauvais mot, que je
n'ai jamais pu démêler en lui. En le quittant, je lui dis encore un mot
de la réduction des bâtards au rang de leur pairie. Il me la promit,
mais ma défiance me fit élever la voix et lui répondre\,: «\, Monsieur,
vous n'en ferez rien, et vous vous en repentirez toute votre vie, comme
vous vous repentez de n'avoir pas culbuté les bâtards à la mort du roi.
» Il était déjà à la porte de son grand cabinet pour l'ouvrir, et je
gagnai les petits pour m'en revenir chez moi dîner.

Au sortir de table j'eus avis d'une cabale du duc du Maine et de
plusieurs du parlement, prête à éclater, pour déclarer le roi majeur, et
former immédiatement sous Sa Majesté un conseil de leurs confidents et
de quelques membres du parlement, dont le duc du Maine serait chef. Cela
me parut insensé, parce que toutes les lois y résistaient, ainsi que
l'usage et le bon sens. Mais les menées de tous ces gens-là, l'aversion,
le mépris de la faiblesse du régent, dont on n'avait pris une idée que
trop juste\,; le manteau du bien public par rapport aux choses de
finance, la frayeur du duc du Maine, l'audace effrénée de son épouse, et
son extrême hardiesse, la terreur du maréchal de Villeroy, leurs
intrigues avec le prince de Cellamare, ambassadeur d'Espagne et le
cardinal Albéroni, lié de tout temps avec le duc du Maine par le feu duc
de Vendôme son maître, et toujours cultivé depuis\,; le grand mot du
comte de Toulouse à M. le duc d'Orléans sur son frère\,; tout cela me
parut pouvoir donner de la solidité à ce qui n'en pouvait avoir par
nature, et dans le cours ordinaire. Je le mandai par un billet au
régent, et demeurai tout le jour chez moi avec le duc d'Humières et
Louville, barricadé pour tout ce qui n'était point du secret.

Entre quatre et cinq de l'après-dînée, on m'avertit que M. le Duc
sortait de ma porte, où il avait fait beaucoup d'instances pour entrer,
et qu'il était allé chez le duc de La Force\,; fort près de chez moi.
J'avais demandé le matin au régent la permission de confier au duc de La
Force ce qui regardait les bâtards, dont jusqu'alors il n'avait pas su
un mot, parce que j'en avais besoin pour dresser la déclaration en
faveur du comte de Toulouse, et je compris que M. le Duc, ne m'ayant pu
voir, était allé raisonner avec lui sur le lit de justice. J'envoyai
aussitôt à l'hôtel de La Force dire à M. le Duc que je ne m'étais pas
attendu à l'honneur de sa visite, et s'il avait agréable de me faire
celui de revenir. Il arriva sur-le-champ. J'avais grande curiosité de ce
qui pouvait l'amener. Je lui fis mes excuses de la clôture de ma porte,
où l'affaire présente me tenait, et où ne devinant point qu'il pourrait
venir, je ne l'avais point excepté comme les autres du secret, et deux
ou trois autres mes intimes amis, pour qui elle n'était jamais fermée,
de peur de donner inutilement à penser à mes gens. Après cela je lui
demandai des nouvelles.

Il me dit, avec la politesse d'un particulier, qu'il venait me rendre
compte de ce qu'il avait fait avec Son Altesse Royale, à qui il avait
demandé la réduction des bâtards au rang de leurs pairies, comme
l'éducation, et qu'il l'espérait\,; mais qu'il venait aussi envoyé par
elle, sur le billet que je lui avais écrit l'après-midi, et savoir de
moi ce que j'avais appris.

Je lui répondis qu'il ne pouvait venir plus à propos, parce que {[}ce
que{]} j'en savais, je le tenais du duc d'Humières, que j'avais fait
passer avec Louville dans un autre cabinet. Je l'allai chercher, et il
dit à M. le Duc que M. de Boulainvilliers l'avait ouï dire à des gens du
parlement, et l'en avait averti aussitôt. J'ajoutai que M. le duc
d'Orléans pouvait envoyer chercher Boulainvilliers, et remonter à la
source. Avec cela M. le Duc retourna au Palais-Royal. Je fus bien aise
de la démarche qu'il y avait faite pour notre rang, mais je restai en
doute si ç'avait été avec suffisance.

M. de La Force vint après, à qui M. le Duc n'avait pas eu le temps de
rien dire, et que je n'avais pas vu depuis le Palais-Royal, où j'avais
eu la permission de lui confier ce qui regardait les bâtards. Je
{[}le{]} lui appris donc alors. Je ne sais ce qui l'emporta en lui, de
l'extrême surprise ou de la vive joie d'un événement si peu attendu et
si prochain. Je l'informai de tout ce à quoi j'en étais là-dessus, et je
le priai de travailler tout à l'heure à la déclaration en faveur du
comte de Toulouse\,; de prendre garde à y bien restreindre ce
rétablissement de rang à lui seul, à l'exclusion bien formelle des
enfants qu'il pourrait avoir et de tous autres quelconques, et de ne pas
manquer d'y insérer que c'était du consentement des princes du sang et à
la réquisition des pairs, pour bien mettre notre droit à couvert. Je le
renvoyai promptement la dresser, et je passai le reste de la journée
chez moi avec Law, Fagon et l'abbé Dubois ensemble et séparément.

Law était depuis quelques jours retourné chez lui, où, au lieu
d'attendre les huissiers, pour le mener pendre, le parlement, étonné du
grand silence qui avait succédé à la résolution prise au conseil de
régence de casser tous leurs arrêts, cette compagnie lui avait envoyé de
ses membres, pour entrer en conférence avec lui, et lui faire l'apologie
de Blamont, président d'une des chambres des enquêtes, et des intentions
du parlement\,; et, dans la matinée de ce jour mercredi, le duc d'Aumont
avait été le haranguer, pour s'entremettre avec lui dans cette affaire
et raccommoder le parlement avec le régent. Law nous en conta des
détails tout à fait ridicules, qui montrèrent combien promptement la
peur avait succédé à l'insolence, et combien aisément quelque peu de
fermeté eût prévenu ces orages et y pouvait aussi remédier.

Le duc d'Aumont, valet du duc du Maine et du premier président, chercha
à justifier ce dernier auprès de Law et à se fourrer dans l'intrigue. Il
lui dit qu'il en avait parlé au régent, qu'il lui avait demandé de l'en
entretenir à fond, lequel lui avait donné samedi ou dimanche pour
cela\,; qu'il espérait que tous les malentendus se raccommoderaient
aisément, et qu'il fallait aussi se servir de gens comme lui sans
intérêt, qui n'avait point voulu prendre de part à toutes ces sottises
du bonnet et cent verbiages de la sorte pour vanter sa bassesse, voiler
sa turpitude, son infamie, ses trahisons\,; se faire rechercher, s'il
eût pu, surtout tirer de l'argent, comme son premier président et lui
s'en étaient déjà fait donner quantité, l'un pour se faire acheter,
l'autre par l'importunité la plus effrontée. L'abbé Dubois me dit que le
maréchal de Villeroy mourait de peur d'être arrêté, au point que rien ne
le pouvait rassurer\,; qu'il avait été lui conter ses frayeurs, son
apologie, vanter son attachement pour feu Monsieur et cent mille
vieilles rapsodies. De toutes ces choses je conclus que ces gens-là
n'étaient pas encore en ordre de bataille, qu'on les prenait encore au
dépourvu, qu'il fallait frapper, tant sur le parlement que sur cet
exécrable bâtard, avec une fermeté qui assurât l'autorité et la
tranquillité du reste de la régence. L'abbé Dubois, Fagon et moi
concertâmes tout ce dont nous pûmes nous aviser sur toute espèce
d'inconvénient et de remède, à quoi le premier alla achever de méditer
chez lui, pour en corriger et augmenter son mémoire. Nous convînmes
cependant de plusieurs déclarations et arrêts du conseil signés et
scellés, qu'à tout événement le garde des, sceaux aurait dans son sac
avec les sceaux hors de leur cassette, pour qu'on ne s'en aperçût pas et
être en état de sceller sur-le-champ, s'il en était besoin, avec la
mécanique nécessaire, toute prête et portée dans une pièce voisine.
Demeurés et repassant toute notre affaire, il me fit faire réflexion que
le délai du mardi au vendredi et la résolution prise en la régence de
casser les arrêts du parlement pouvait rendre dangereuse, tout au moins
embarrassante, la capture des membres du parlement, qu'on avait résolu
de punir par une prison dure et éloignée, si on persistait à la faire le
matin même du lit de justice\,; que le parlement, qui en serait, ou
n'oserait s'assembler, ou refuserait de venir aux Tuileries, ou y ferait
des remontrances sur ce châtiment qui ne conviendraient pas au temps\,;
que tous ces partis étaient embarrassants, tellement qu'après avoir bien
raisonné et balancé, nous résolûmes à différer au samedi matin\,: ce qui
donnerait lieu de mieux connaître par la séance du lit de justice à qui
on avait affaire, et je me chargeai de le faire agréer ainsi à M. le duc
d'Orléans. Je lui mandai donc que j'avais à lui parler le lendemain
matin par la porte de derrière, pour qu'elle me fût ouverte, et je me
retirai si las de penser, d'espérer, de craindre par la nature de celui
qui devait donner consistance et mouvement à tout, que je n'en pouvais
plus.

Le lendemain, jeudi 25 août, le duc de La Force vint dès le matin chez
moi avec sa déclaration dressée en faveur du comte de Toulouse. Elle
était bien et tout à fait dans mon sens. Ce fut celle qui fut imprimée,
ainsi que l'instrument que Millain m'avait montré la veille pour la
réduction des bâtards au rang de leurs pairies. Il entra peu après M. de
La Force, et se retint dès qu'il le vit, mais je lui dis que M. de La
Force était maintenant de tout le secret\,: ainsi nous lûmes les deux
déclarations que chacun d'eux avait dressées en faveur du comte de
Toulouse. Nous raisonnâmes sur la totalité de la grande affaire du
lendemain. Millain me dit de la part de M. le Duc qu'il me priait de me
trouver le soir à huit heures, par la petite porte, chez M. le duc
d'Orléans, tandis que lui y entrevoit par la porte ordinaire, pour
prendre là tous trois ensemble nos dernières mesures sur le point de
l'exécution. Il ajouta que M. le duc d'Orléans avait chargé M. le Duc de
m'en avertir, et qu'il me priait, lui Millain, de trouver bon qu'il
m'accompagnât pour être introduit secrètement par moi en cas qu'on eût
besoin de lui pour les formes.

J'acceptai le tout avec joie et bon augure\,; mais non assez nettement
éclairci sur notre rang, j'en voulus avoir le coeur net. Je demandai
donc à Millain où en était son maître sur cela. Il ne me dit que les
mêmes choses que M. le Duc m'avait dites chez moi la veille. Je me mis à
répéter à Millain toutes les raisons dont j'avais battu et convaincu M.
le Duc là-dessus, dans lesquelles Millain entra très bien, en quoi je ne
fus que médiocrement aidé de M. de La Force. Ne croyant pas me devoir
abandonner à ce que M. le Duc avait fait la veille avec M. le duc
d'Orléans, qui ne me mettait pas suffisamment à mon aise, je fis sentir
à Millain le juste éloignement où nous étions tous de M. le Duc, par
l'excuse que M. le Duc d'Orléans nous avait faite de nous avoir laissés
dans la nasse lors du procès des princes du sang contre les bâtards\,;
l'ébranlement avoué de Son Altesse Royale pour réparer cette faute, si
M. le duc le désirait\,; l'état de rage ou, d'attachement où M. le Duc
avait le choix actuel de nous mettre à son égard\,; son intérêt de nous
avoir pour amis\,; l'engagement formel et net où il était entré
là-dessus avec moi. Quand je crus avoir suffisamment persuadé mon homme
par la tranquille solidité de mes raisons, je crus pouvoir le mener avec
plus de véhémence. «\,Vous m'avez donc bien entendu, lui dis-je, et par
moi tous les pairs de France, qui ne sont pas moins sensibles que moi.
Rendez-en compte de ma part à M. le Duc\,; vous ne lui pouvez trop
fortement déclarer que je sais précisément de M. le duc d'Orléans, et
que tous les pairs de France le sauront par moi, quoi qu'il arrive, que
notre sort est entre ses mains\,; que du succès de demain dépend notre
honneur ou notre ignominie\,; que l'un ou l'autre nous le devrons à M.
le Duc, avec les plus vifs sentiments et les plus durables, et les
partis les plus conformes à ce que nous lui devrons\,; qu'il n'en
regarde pas la déclaration réitérée par vous comme un discours frivole
(il sera suivi et comme substitué en maxime et en actions par nous et
par les nôtres) ni comme un manque de respect ni un air de menace, mais
qu'il le considère comme les mouvements véritables de l'honneur et d'une
sincérité qui ne veut point le laisser ni se tromper ni se séduire.
Monsieur, dites-le-lui bien. S'il nous abandonne, je me sens capable, et
avec moi tous les pairs, de nous jeter à M. du Maine contre lui\,; car,
au moins, dans tous les maux que nous a faits M. du Maine, il lui en est
résulté un bien et des avantages qu'il a jugés préférables à tout. Mais
M. le Duc, qui ne peut rien craindre de nous en matière de rang, avec
lequel non pas la préséance, mais l'égalité est impossible, son abandon
dans une telle crise serait nous vouloir le plus grand mal qui se
puisse, et nous le faire encore sans cause, sans intérêt, sans raison,
sans excuse, d'une manière purement gratuite, avec tout l'odieux du
\emph{malum quia malum appetere}\footnote{Il y a dans le manuscrit
  \emph{malum qua malum}. C'est une erreur évidente\,; puisque le sens
  de la phrase est \emph{rechercher le mal pour le mal}. Aussi
  avons-nous maintenu le changement de \emph{qua} en \emph{quia} fait
  dans les éditions précédentes.}**, qui est tel que les philosophes
prétendent que la méchanceté humaine ne peut aller jusque-là. Or, si
nous l'éprouvons, il n'y a fer rouge, désespoir, bâtardise, à quoi nous
ne nous prenions contre lui, et moi à la tête de tous\,; comme aussi,
s'il nous restitue en rang contre son ennemi, je n'ai point de paroles
pour vous témoigner notre abandon à lui et jusqu'à quel point il sera
maître de nos coeurs. Vous m'entendez. Ceci est clair. N'en oubliez pas
une parole, et revenez, s'il vous plaît, nous articuler sur quoi nous
devons compter.\,» J'eus peine à achever cette phrase si décisive et à
entendre les protestations de Millain, parce qu'un valet de chambre, que
j'avais envoyé au Palais-Royal, me vint dire que M. le duc d'Orléans
m'attendait, et que Millain lui-même était pressé d'aller retrouver M.
le Duc. M. de La Force me servit plutôt de témoin que d'appui en cette
forte conversation, dont il me parut effrayé. J'achevai promptement de
m'habiller et m'en allai au Palais-Royal par la petite porte.

Ibagnet, qui m'attendait, me conduisit à l'ordinaire\,; mais comme il
m'ouvrait la porte secrète des cabinets, La Serre, écuyer ordinaire de
M\textsuperscript{me} la duchesse d'Orléans, passa sur le degré et me
vit là avec un étonnement que je lus sur son visage. Cette rencontre me
fâcha fort d'abord\,; mais M\textsuperscript{me} la duchesse d'Orléans
était à Saint-Cloud heureusement, et je pris courage par la réflexion
qu'il n'y avait plus que vingt-quatre heures à ramer. Je trouvai le
régent qui travaillait avec La Vrillière, lequel se voulut retirer. Je
l'arrêtai et dis à Son Altesse Royale que je serais bien aise de lui
faire faire une réflexion devant lui. C'était celle de Fagon, qui fut
extrêmement goûtée. M. le duc d'Orléans me dit qu'il l'avait faite dans
la nuit qu'il avait passée avec un peu de fièvre, incommodité qui
m'alarma infiniment et qui me présenta tout le déconcertement du projet
qu'elle pouvait opérer. Il fut donc arrêté là que ceux qui devaient être
arrêtés le lendemain ne le seraient que le surlendemain matin, et il
était temps de s'en aviser, car La Vrillière allait donner les ordres
qu'il remit au lendemain au soir. Il s'en alla et je demeurai seul avec
M. le duc d'Orléans à nous promener dans sa grande galerie.

Il me parla d'abord du projet dont je lui avais écrit la veille, qu'il
m'assura être sans fondement\,; ensuite il vint à la grande journée du
lendemain. Il avait fait dire qu'il y aurait conseil de régence cette
même après-dînée, qui était celui qu'il avait annoncé extraordinaire le
lundi précédent, pour voir l'arrêt du conseil qui cassait ceux du
parlement.

Je le fis souvenir qu'il avait oublié de le contre-mander\,; il le fit
sur-le-champ en le mandant pour le lendemain après dîner. Tout cela
n'était que pour couvrir le projet en amusant même les parties
nécessaires, ce qui fut très à propos\,; mais les deux pénibles
difficultés restaient toujours, savoir le silence au conseil de régence
sur les bâtards, et leur présence très possible au lit de justice. Je
m'avisai d'une solution qui me vint dans l'esprit sur-le-champ. Je lui
proposai que le lit de justice se tînt à portes ouvertes, parce qu'alors
les affaires s'y traitent comme aux audiences et que le garde des sceaux
y prend les voix tout bas, allant le long des bancs, merveilleuse
commodité pour fermer la bouche à qui n'a pas la hardiesse de faire une
chose insolite en voulant parler tout haut et non moins sûre pour
rapporter les avis comme il plaît au maître\,; nous étions sûrs du garde
des sceaux\,; ainsi, nul risque pour les opinions du timide conseil de
régence ni même du parlement\,: car il eût fallu y trouver des gueules
bien fortes et bien ferrées pour vouloir opiner haut, contre les formes,
en face du roi et de son garde des sceaux et au milieu des gardes du
roi, dans les Tuileries.

Restait l'embarras des bâtards présents. Il n'était pas levé par la
sortie de M. le Duc qui eût demandé la leur, car ils pouvaient, avant de
le suivre, demander qu'il ne fût rien statué à leur égard sans les avoir
ouïs\,; mais cette sortie en levait la plus embarrassante partie pour la
faiblesse du régent, en ce qu'elle ôtait le face-à-face. Aller au delà,
c'était passer le but, et impossibilité entière. Restait à se vouer à la
fermeté du régent, en laquelle ma confiance était légère. Il promit
pourtant merveilles, et, dans la vérité, il tint même et bien au delà de
ce qu'il avait promis.

Parmi ces discussions M. le Duc arriva\,: nous les continuâmes tous
trois ensemble, et nous conclûmes la cadence des grands coups du
lendemain, qu'il est inutile de marquer ici parce que chaque chose sera
racontée en son ordre.

Après cela je pris la liberté de leur déclarer à tous les deux que je
les prenais tous les deux à témoin de mon avis et de ma conduite dans
cette affaire, et que je les y prenais l'un devant l'autre\,; qu'ils
savaient tous deux combien j'avais été contraire à rien ôter au duc du
Maine dans la crainte de l'unir trop au parlement, et de frapper un coup
dont le trop grand ébranlement remuât et troublât l'État\,; que je leur
répétais de nouveau que tel était encore mon sentiment, bien que je n'en
espérasse plus rien après tout ce que je leur avais représenté
là-dessus\,; que j'avais aussi été d'avis, et que j'y persistais, que
l'éducation ôtée au duc du Maine ne devait être donnée à personne en sa
place\,; mais que, puisqu'il en était résolu autrement, je les suppliais
de me permettre de les exhorter à une union intime, qui ne pourrait
subsister sans la confiance et une attention infinie à écarter les
soupçons et les fripons qui seraient appliqués à les brouiller\,; que
leur gloire, leur repos, le salut de l'État dépendaient de leur
intelligence, ainsi que la grandeur ou la perte de leurs communs
ennemis. Là-dessus, protestations de reconnaissance, d'attachement et de
toutes les sortes de M. le Duc, et politesses, avances même de M. le duc
d'Orléans. J'étendis ces propos à mesure que les compliments y donnèrent
lieu, après quoi je vins à mon fait du rang\,; non plus en homme qui
doute, mais en homme qui a pour soi le sacrifice qu'il a voulu faire à
l'État de son plus cher intérêt, qui le premier a proposé ensuite le
sacrifice d'une partie en conservant le comte de Toulouse entier, choses
dont je les pris encore tous les deux à témoin\,; en homme enfin qui a
pour soi justice, raison politique, paroles de tous les deux\,; et avec
cet air de confiance entière, je les quittai en souhaitant toute fermeté
à l'un, toute fidélité à l'autre, tout succès aux grands coups qui
s'allaient ruer.

Comme je m'éloignais déjà d'eux, ils me rappelèrent pour me dire de ne
manquer pas au rendez-vous du soir, à huit heures, par la petite porte,
et M. le Duc ajouta si je n'avais pas vu Millain, qui m'y suivrait.
C'était pour résumer tout, et prendre tous trois ensemble nos dernières
mesures sur tout ce qui pouvait arriver. Je leur rendis compte alors de
la déclaration en faveur du comte de Toulouse, que j'avais fait faire,
et que je l'avais laissée à Millain avec celle qu'il avait faite, duquel
je louai aussi l'ouvrage pour la réduction des bâtards à leur rang de
pairie\,; je l'avais oublié dans la conversation. Le nom de Millain,
quand M. le Duc me demanda si je l'avais vu, m'en fit souvenir.

Je m'en revins chez moi plus content et plus tranquille que je n'avais
encore été. Je croyais notre besogne aussi arrangée qu'il était
possible, les inconvénients prévus et prévenus le plus qu'il se trouvait
dans la nature des choses\,; la nôtre à nous tout à fait assurée, le
régent prenant force et courage, nul de nous ne se démentir, le secret
encore tout entier, la mécanique toute prête et les moments s'approcher.
Satisfait de moi-même d'avoir sincèrement fait tout ce qui avait été en
moi, de front, de biais, par adresse et de toutes parts, tant envers le
régent qu'auprès de M. le Duc, pour sauver le duc du Maine, dans la
seule vue du bien de l'État, malgré mes intérêts communs et personnels
les plus sensibles, je me crus permis de me réjouir enfin de ce qui
était résolu malgré moi, et plus encore de ce qui en allait être le
fruit. Toutefois, je n'osais encore m'abandonner à des pensées si douces
sans avoir une plus grande certitude de cette si désirée réduction des
bâtards au rang de leurs pairies, et je demeurai près de deux heures
dans ce resserrement de joie, à laquelle je ne pouvais me résoudre de
laisser prendre un plein essor. Libre alors des grandes affaires dont
l'arrangement était pris, j'étais tout occupé de procurer moi seul aux
pairs de France un rétablissement auquel nous n'avions pu, arriver par
nos efforts communs, et que je voyais sur le point d'éclater, à leur
insu et en leur présence.

Tandis que tout cela me roulait dans la tête, Millain arriva chez moi\,;
il me dit que M. le Duc le renvoyait m'assurer qu'il avait la parole du
régent pour la réduction des bâtards à leur rang d'ancienneté de leurs
pairies\,; qu'il en avait envoyé la déclaration avec celle en faveur du
comte de Toulouse à La Vrillière, telle que je les avais vues et au
garde des sceaux pour les expédier, et qu'il était en état de me
répondre qu'elles passeraient le lendemain. Jamais baiser donné à une
belle maîtresse ne fut plus doux que celui que j'appuyai sur le gros et
vieux visage de ce charmant messager. Une embrassade étroite et
redoublée fut ma première réponse, suivie après de l'effusion de mon
coeur pour M. le Duc et pour Millain même, qui nous avait dignement
servis dans ce grand coup de partie. Mais au milieu de ce transport je
ne perdis pas le jugement\,; je dis à Millain que La Vrillière, tout mon
ami qu'il était, et le garde des sceaux, se sentaient du vieux chrême du
feu roi\,; que le dernier était de tout temps lié avec les bâtards\,;
que l'un et l'autre avaient fait des difficultés sur notre affaire au
régent qui me l'avait dit la veille\,; qu'il fallait que M. le Duc
couronnât son oeuvre d'une nouvelle obligation sur nous\,; que
j'exigeais de son amitié qu'il prît la peine d'aller de ce pas lui-même
chez l'un et chez l'autre leur témoigner qu'il ne regardait pas la
réduction des bâtards au rang de leurs pairies différemment de
l'éducation, et que, par la manière dont ils en useraient pour faciliter
cette réduction telle qu'il la leur avait envoyée, il connaîtrait et
sentirait jusqu'où ils le voudraient obliger et comment il devrait aussi
se conduire dans la suite avec eux. Millain n'y fit point de difficulté,
et m'assura que M. le Duc n'y en ferait point non plus. Il ajouta même
qu'il l'y accompagnerait pour voir avec lui les deux déclarations et si
on n'y avait rien changé. Je redoublai mes remerciements, lui dis qu'il
fallait absolument que M. le Duc trouvât ces deux hommes chez eux, et me
hâtai de le renvoyer pour n'y pas perdre un instant.

Le reste du jour se passa chez moi avec l'abbé Dubois, Fagon et le duc
de La Force, l'un après l'autre, à remâcher encore toute notre besogne.
Tout était prévu, et les remèdes à chaque inconvénient tout dressés\,:
si le parlement refusait de venir aux Tuileries, l'interdiction prête,
avec attribution des causes y pendantes et des autres de son ressort au
grand conseil, les maîtres des requêtes choisis pour l'aller signifier
et mettre le scellé partout les lieux où il était nécessaire\,; les
officiers des gardes du corps choisis, et les détachements du régiment
des gardes destinés pour les y accompagner\,; si une partie du parlement
venait et une autre refusait, même punition pour les refusants\,; si le
parlement venu refusait d'entendre et voulait sortir, même punition\,;
si une partie restait, une autre s'en allait, de même pour les sortants,
c'est-à-dire si c'était des chambres entières, sinon interdiction
seulement des membres sortis\,; si refus d'opiner, passer outre, de même
pour peu qu'il restât de membres du parlement\,; au cas que tous fussent
sortis, tenir également le lit de justice, et huit jours après en tenir
un autre au grand conseil pour y enregistrer ce qui aurait été fait\,;
si les bâtards ou quelque autre seigneur branlaient, les arrêter dans la
séance, si l'éclat était grand, sinon à la sortie de séance\,; s'ils
sortaient de Paris les arrêter de même. Tout cela bien arrangé et les
destinations et les expéditions faites, l'abbé Dubois fit une petite
liste de signaux, comme croiser les jambes, secouer un mouchoir, et
autres gestes simples, pour la donner dans le premier matin aux
officiers des gardes du corps choisis pour les exécutions, qui, répondus
dans la salle du lit de justice, devaient continuellement regarder le
régent pour obéir au moindre signal, et entendre ce qu'ils auraient à
faire. Il fit plus, car, pour décharger M. le duc d'Orléans, il lui
dressa, pour ainsi dire, une horloge, c'est-à-dire des heures auxquelles
il devait mander ceux à qui il aurait nécessairement des ordres à donner
pour ne les pas mander un moment plus tôt que le précisément nécessaire,
et de ce qu'il aurait à leur dire pour n'aller pas au delà, n'en oublier
aucun et donner chaque ordre en son temps et en sa cadence, ce qui
contribua infiniment à conserver le secret jusqu'au dernier instant.

Vers huit heures du soir, Millain me vint trouver pour le rendez-vous du
Palais-Royal. Il me dit que M. le Duc avait été chez le garde des sceaux
et chez La Vrillière\,; qu'il avait pris leur parole sur notre affaire,
et vu chez eux les deux déclarations telles qu'il les leur avait
envoyées signées et scellées. Après les remerciements, j'envoyai Millain
m'attendre à la petite porte à cause de mes gens\,; et, un moment après,
je l'y suivis sans flambeaux. Ibagnet nous attendait, et nous
introduisit à tâtons de peur de rencontre\,: Je fus effrayé de trouver
M. le duc d'Orléans au lit, qui me dit qu'il avait la fièvre. J'avoue
que je ne sus si ce n'était point celle du lendemain. Je lui pris le
pouls assez brusquement, il l'avait en effet. Je lui dis que ce n'était
que fatigue de corps et d'esprit, dont il serait soulagé dans
vingt-quatre heures\,; lui, de sa part, protesta que, quoi que ce fût,
il tiendrait le lit de justice. M. le Duc, qui venait d'entrer, était au
chevet de son lit, et une seule bougie dans la chambre où il n'y avait
que nous quatre. Nous nous assîmes, M. le Duc et moi, et repassâmes les
ordres donnés et à donner, non sans une grande inquiétude à part moi de
cette fièvre venue si étrangement mal à propos à l'homme du monde le
plus sain, et qui ne l'avait jamais.

Là il fut résolu que le lit de justice serait institué à six heures du
matin au parlement pour, entre neuf et dix, aux Tuileries\,; le conseil
de régence, annoncé la surveille pour l'après-dînée, mandé pour sept
heures du matin pour être tenu à huit, et les chefs des conseils avertis
d'y porter toutes leurs affaires pressées, afin de le prolonger autant
qu'on le jugerait à propos\,; que Son Altesse Royale prendrait les avis
contre l'ordinaire par la tête, pour montrer son concert avec les
princes du sang, et pour intimider quiconque aurait envie de parler mal
à propos. Je proposai qu'au cas que le conseil manquât d'affaires avant
que la séance du lit de justice fût prête, Son Altesse Royale ordonnât
que chacun demeurât en place, et défendît surtout à qui que ce soit de
sortir sous quelque prétexte que ce fût.

Ensuite, M. le Duc voulut lire ce qu'il avait préparé pour demander
l'éducation. Il le venait de faire de sa main à peu près tel qu'il a
paru depuis. Son Altesse Royale y changea quelque chose et moi aussi, et
puis je m'avisai qu'il y fallait flatter la vanité du maréchal de
Villeroy, et je dictai à M. le Duc ce qui y est là-dessus, sur une niche
à chien que j'allai chercher faute de table portative.

Après, grande question sur les bâtards. Décidé\,: qu'à cause de leur
présence, on ne dirait rien au conseil de ce qui les regardait\,; que,
pour les éviter au lit de justice, ils n'en seraient point avertis, sous
prétexte que, depuis l'arrêt intervenu entre les princes du sang et eux,
ils ne voulaient plus aller au parlement. M. le duc d'Orléans, toujours
enclin à l'espérance, voulut se figurer que cette raison les en
empêcherait\,; que, de plus, pris au dépourvu, ils n'y pourraient venir
faute de rabat et de manteau. Je soutins que c'était s'abuser\,; que le
duc du Maine logeait sous l'appartement du roi\,; que le duc de Villeroy
était en quartier de capitaine des gardes, logé aussi aux Tuileries,
qu'on ne se pouvait passer de lui pour la mécanique de la séance, que
jusqu'à un certain temps\,; qu'averti, il avertirait son père couché
dans la chambre du roi, s'il lui était possible\,; qu'au même instant M.
du Maine le serait par le père ou par le fils, et aussitôt après le
comte de Toulouse par le duc du Maine\,; par conséquent qu'ils auraient
tout loisir depuis six heures du matin de prendre leur parti, et l'habit
convenable à ce qu'ils voudraient faire\,; que plus leur entreprise
serait grande, plus ils devaient être résolus à se trouver au lit de
justice pour s'y défendre courageusement, à quoi le remède ne pouvait se
trouver que dans la force de M. le duc d'Orléans en face, sans colère,
sans émotion, quoi qu'il pût arriver, mais aussi sans mollir sur quoi
que ce fût, en lieu et en état de faire justice, en droit de la rendre
et de faire valoir l'autorité royale déposée en ses mains.

Après cela, je me mis à chercher dans la forme de marcher en place les
moyens de les exclure par embarras\,; mais nous eûmes beau faire\,: la
raison que j'avais déjà trouvée et ce bel arrêt de plus rendu entre les
princes du sang et eux, qui leur laissait tous leurs honneurs, les
maintenait aussi dans celui de traverser le parquet, tellement que, de
façon ni d'autre, nous n'y pûmes trouver de remède.

Il fut convenu que j'avais eu raison de ne vouloir point de M. le duc de
Chartres en ce lit de justice, pour ne s'y point charger d'un enfant en
tout ce qu'il pouvait y arriver, ne point avertir M\textsuperscript{me}
la duchesse d'Orléans, avec laquelle il était à Saint-Cloud, de si bonne
heure que ses soupçons et ses inquiétudes ne lui fissent avertir ses
frères, surtout pour ne point séparer dans la séance M. le Duc de M. le
duc d'Orléans, qui pourraient avoir à se parler bas et à se concerter
sur-le-champ.

Ensuite, je remis sur le tapis l'affaire de la réduction des bâtards au
rang de leurs pairies. Le régent et M. le Duc me dirent nettement
qu'elle était ordonnée et les instruments signés et scellés tels que je
les avais vus\,; sur quoi, remerciements et louanges de ma part. Je
proposai qu'il me fût permis, entrant en séance, d'en dire un mot aux
pairs, qui alors lie le pouvaient communiquer à personne. Il fut jugé
qu'il était bon que je le fisse pour les bien disposer, et j'en répondis
hardiment. Mais pour m'assurer davantage de quelques douteux, soit de
cabale, soit de silence gardé à cet égard et à celui de l'éducation
jusqu'au lit de justice, je demandai à M. le duc d'Orléans et à M. le
Duc si à tout hasard, je ne ferais pas bien de mettre dans ma poche
notre requête contre les bâtards sur laquelle il serait fait droit, qui
entre autres, était signée du duc de Villeroy, par ordre de son père, et
par le maréchal de Villars, desquels nous avions tous soupçons\,: cela
fut fort approuvé, et dans la vérité je crus voir dans l'exécution que
la précaution n'avait pas été inutile.

Une autre question fut après traitée, savoir, ce qu'on ferait en cas de
refus du parlement d'opiner. J'y donnai deux solutions\,: au refus
silencieux et modeste, le prendre pour avoir opiné, le garde des sceaux
continuant également d'aller de banc en banc, et ne faisant aucun
semblant qu'on n'opinât point. Ce cas, et bien plus celui de s'opposer
aux enregistrements, avait été l'objet de la résolution prise, et que
j'avais pour cela suggéré de tenir un lit de justice, et à huis ouvert,
à la manière des audiences, pour y prendre bas les avis, allant le long
des bancs. Au cas de refus d'opiner, déclaré tout haut, soit de
quelques-uns du parlement, soit du premier président, et du banc des
présidents, en manière de protestation pour la compagnie, passer outre,
et déclarer que le roi n'est point tenu de prendre ni de se conformer
aux avis du parlement\,; qu'il les demandait par bonté et pour honorer
la compagnie\,; mais, qu'étant le maître et les sujets n'ayant qu'à
obéir à la volonté connue du souverain, il les avait mandés pour
l'entendre déclarer et l'enregistrer avec soumission\,; et tenir ferme.
M. le duc d'Orléans m'objecta qu'encore bien qu'il n'y eût que cela à
faire, il m'avait bien des fois ouï disputer le contraire, et qu'au lit
de justice il y avait voix non simplement consultative, mais
délibérative.

Je lui répondis que je le soutenais bien encore, mais qu'il fallait
distinguer les personnes et les cas\,; que, pour les personnes, il n'y
avait que les pairs assesseurs et conseillers nés de la couronne et des
rois, \emph{laterales regis}, qui eussent droit de délibérer sur les
affaires d'État, à parler étroitement, et pour s'élargir au plus qu'il
était possible, les officiers de la couronne avec eux, par la dignité,
encore plus par l'importance de leurs offices, par grâce toutefois, dont
la marque évidente ainsi que du droit des pairs, est que les officiers
de la couronne ne peuvent venir au lit de justice que mandés, et n'y
entrer qu'à la suite du roi, non pas même un seul instant devant lui, à
la différence des pairs qui ont et ont toujours eu séance par leur
dignité, sont mandés par nécessité, et qui, sans être mandés, ont droit
égal de s'y trouver, y entrent avant le roi, et sont en place quand il
arrive\,; mais qu'à l'égard des officiers du parlement, ils sont et ont
toujours été les assesseurs des pairs, de la présence desquels ils
tirent uniquement la liberté d'opiner en matière d'État, d'où est venue
la nécessité de la clause insérée toujours et jusqu'à aujourd'hui dans
ces sortes d'arrêts, \emph{la cour suffisamment garnie de pairs}. De là
vient encore l'essentielle différence de leur serment d'avec celui des
pairs, d'où résulte que la tolérance à ces officiers du parlement et
autres magistrats ou seigneurs d'opiner en matière d'État, ne leur y
donne que voix consultative, la délibérative y demeurant inhérente de
droit aux seuls pairs, et de grâce avec eux aux officiers de la
couronne, desquels il plaît au roi de se faire accompagner. Pour la
matière, qu'il ne s'en agissait ici que de deux sortes\,: la première,
si le roi serait obéi\,; ou, si le parlement l'emporterait sur lui. Si
c'était un procès, le parlement n'en pouvait être juge et partie\,;
sinon, il avait rempli tout devoir et pouvoir par ses remontrances. Il
n'avait pu décider, et sans aucuns pairs de France, d'affaires
concernant l'État, telles que sont les arrêts rendus par le parlement,
qu'il s'agit de casser. Il n'avait donc pas voix délibérative sur les
édits qu'il s'agit d'enregistrer, encore moins sur l'édit en forme de
règlement pour réprimer leurs désobéissances\,; que l'éducation était
encore une autre matière d'État à laquelle ils n'avaient que voir, et
qui même, absolument parlant, n'avait besoin d'aucune forme\,; que, pour
ce qui était du droit à faire à notre requête, le roi pouvait, à
meilleur titre, se passer d'eux pour, de son seul mouvement et de son
autorité, remettre les choses en règle\,; que le feu roi, par cette
seule voie, les en avait pu tirer\,; que formes, lois divines et
humaines, exemples, tout y était tellement en notre faveur, qu'il n'y
avait pas à craindre que le parlement y pût rien opposer\,; que, par
toutes ces raisons, je persistais à soutenir mon opinion ancienne et
continuelle sur le lit de justice, et à être en même temps persuadé que,
ne trouvant point de résistance dans les hauts sièges, omettant le garde
des sceaux qui parlait pour le roi en sa place, il n'y avait nulle voix
délibérative à reconnaître dans l'es bas sièges, et toute vérité de
droit à passer outre, quoi que les bas sièges pussent dire et faire. M.
le duc d'Orléans n'eut rien à répliquer, et convint de la force de ces
raisons, que j'eusse infiniment fortifiées s'il en eût été besoin et
loisir, et se résolut aussi à suivre cet avis\footnote{Voyez la note II
  en fin du volume.}.

Je lui demandai si les mesures étaient bien réglées à prendre dans la
nuit avec les gens du roi. Il me dit qu'ils seraient avertis d'être
sages en même temps que le parlement le serait du lit de justice, et en
particulier Blancmesnil, premier avocat général, frère de Lamoignon,
président à mortier, et que toute sa fortune répondrait à l'instant de
la moindre ambiguïté de ses conclusions sur tout ce qui serait proposé,
sans lui rien expliquer davantage.

De là M. le duc d'Orléans nous expliqua en gros l'horloge de sa nuit
jusqu'à huit heures du matin, qu'il se rendrait chez le roi en manteau.
Je l'exhortai à se reposer cependant le plus qu'il pourrait, et à
constituer le salut de sa régence dans les exécutions du lendemain, et
celui de ces exécutions dans sa résolution, sa fermeté, sa présence
d'esprit, son attention aux plus petites choses, surtout à se posséder
entièrement. Avec cela je lui souhaitai bonne nuit, et, me retirant vers
le pied du lit, je remerciai M. le Duc des visites qu'il avait faites,
avec des protestations qui partirent du coeur, qui furent suivies des
siennes et de deux embrassades les plus étroites. Millain avait assisté
debout, et très judicieusement parlé pendant une partie de cette
conférence. Avant de sortir je me rapprochai du lit et je demandai à M.
le duc d'Orléans permission de confier tout le mystère au duc de
Chaulnes, puisque aussi bien {[}il{]} le devait apprendre pour l'écorce
de Son Altesse Royale dans la nuit pour l'ordre aux chevau-légers dont
il était le capitaine, et il y consentit. Je lui pris le pouls, non sans
inquiétude. Je l'assurai toujours que ce ne serait rien, sans en être
trop sûr moi-même. Je pris congé enfin et me retirai à dix heures
précises, avec Millain, par où nous étions entrés, et M. le Duc par la
porte ordinaire. Quand je me vis seul avec Millain dans le cabinet par
où nous passions, je l'embrassai avec un plaisir extrême. Ces effusions
de coeur avec M. le Duc et lui furent suffoquées pour n'être pas
entendues, les unes du régent, au pied du lit duquel nous étions, les
autres par d'Ibagnet, qui nous attendait dans les cabinets voisins pour
nous éclairer et ouvrir sur le degré que nous descendîmes à tâtons,
comme nous l'avions monté\,; et après une embrassade en bas, dont je ne
pus me refuser le plaisir, nous nous séparâmes pour nous en revenir
chacun chez nous.

J'arrêtai tout près de chez moi devant l'hôtel de Luynes, où j'envoyai
prier le duc de Chaulnes de me venir parler à mon carrosse. Il y vint
sans chapeau, y monta, et aussitôt le cocher, qui avait l'ordre, marcha
et nous mena chez moi, sans que jusque dans mon cabinet je dise un mot
au duc de Chaulnes, fort surpris de se voir enlevé de la sorte. Il le
fut bien davantage lorsque, après avoir fermé mes portes, je lui appris
le grand spectacle préparé pour le lendemain matin. Nous nous livrâmes,
lui et moi, au ravissement d'un rétablissement si imprévu, si subit, si
prochain, si secret, dont la seule espérance, fondée comme que ce fût,
nous avait uniquement soutenus sous l'horrible marteau du feu roi. La
dissipation et la fonte de ces montagnes entassées l'une sur l'autre,
par degrés infinis, sur notre dignité par ces géants de bâtards, ces
Titans de la France\,; leur état prochain, la commune surprise, mais si
différente, si extrême en eux et dans les pairs\,; notre renaissance,
notre réexistence des anéantissements passés, cent vues à la fois, nous
dilatèrent le coeur d'une manière à ne le pouvoir rendre, la juste
rétribution des profondes noirceurs si pourpensées du duc du Maine sur
le bonnet et l'accomplissement d'une partie de la menace que je lui
avais faite chez lui à l'avortement de cette affaire, qu'on a vue ici en
son lieu. M. le Duc ne fut pas oublié, ni Millain même, dans ce
tête-à-tête. Nous nous séparâmes enfin dans cette grande attente.

J'avais retenu quelques jours auparavant Contade, major des gardes,
homme sûr et fort intelligent, que le hasard m'avait appris devoir aller
passer quelque temps chez lui en Anjou. Je le rencontrai au
Palais-Royal, comme je descendais de carrosse. Il me donna la main, je
lui dis à l'oreille que je lui conseillais et le priais de différer son
départ sans faire semblant de rien. Il me le promit, et le tint sans que
je lui en disse davantage, et me dit qu'il n'en parlerait point. Bien
nous prit de cette prévoyance. Depuis une heure après minuit, M. le duc
d'Orléans manda successivement les ducs de Guiche, de Villeroy et de
Chaulnes, colonel des gardes, capitaine des gardes du corps en quartier,
capitaine des chevau-légers de la garde\,; Artagnan et Canillac,
capitaines des deux compagnies des mousquetaires, et en l'absence de
Dreux, qui était à Courcelles, chez Chamillart son beau-père, des
Granges, maître des cérémonies, pour leur donner ses ordres, tandis que
La Vrillière les donnait à tout l'intérieur de la ville et aux
expéditions nécessaires.

On avait pensé à tout, excepté aux Suisses, car il échappe toujours
quelque chose, et souvent d'important. Contade, averti par le duc de
Guiche, s'en avisa sur ce que le duc de Guiche lui dit que le régent ne
lui en avait point parlé, et alla trouver Son Altesse Royale pour en
prendre ses ordres. Il lui fit entendre que, par l'affection fidèle du
régiment des gardes suisses, le commandement et la supériorité en nombre
du régiment des gardes françaises sur l'autre, il n'y avait rien à
craindre, et qu'on l'offenserait par une marque de défiance. Il reçut
donc ordre d'y pourvoir. Sur les quatre heures du matin, Contade alla
aux Tuileries, éveiller le duc du Maine, colonel général des Suisses. Il
n'y avait pas une heure qu'il était couché, revenant d'une fête que
M\textsuperscript{me} du Maine s'était donnée à l'Arsenal, où elle était
encore. Le duc du Maine fut sans doute étonné, mais il se contint, et
dans sa frayeur cachée, il demanda d'un air assez libre si Contade était
seul, qui l'entendit de la porte. Il se rassura sur ce qu'il apprit
qu'il était seul, et le fit entrer. Contade lui expliqua son ordre de la
part de M. le duc d'Orléans, et aussitôt le duc du Maine envoya avertir
les compagnies du régiment des gardes suisses. Je pense qu'il dormit mal
depuis, dans l'incertitude de ce qui allait arriver, mais je n'ai point
su ce qu'il fit depuis, non plus que la duchesse du Maine.

Vers cinq heures du matin on commença d'entendre des tambours par la
ville, et bientôt après d'y voir des soldats en mouvement. À six heures
des Granges fut au parlement rendre sa lettre de cachet. Messieurs, pour
parler leur langage, ne faisaient que de s'assembler. Ils mandèrent le
premier président, qui fit assembler les chambres. Tout cela dura une
demi-heure. Ils répondirent après qu'ils obéiraient\,: après ils
débattirent en quelle forme ils iraient aux Tuileries en carrosse ou à
pied. Le dernier prévalut, comme étant la forme la plus ordinaire, et
dans l'espoir d'émouvoir le peuple et d'arriver aux Tuileries avec une
foule hurlante. Le reste sera raconté mieux en sa place plus bas. En
même temps des gens à cheval allèrent chez tous les pairs et les
officiers de la couronne, et chez ceux des chevaliers de l'ordre, et des
gouverneurs ou lieutenants généraux des provinces dont on voulut
accompagner le roi, pour les avertir du lit de justice, des Granges,
dans ce subit embarras, n'ayant pas eu le temps d'aller lui-même. Le
comte de Toulouse était allé souper auprès de Saint-Denis, chez M. de
Nevers, et ne revint qu'assez avant dans la nuit. Les gardes françaises
et suisses furent sous les armes en divers quartiers, le guet des
chevau-légers, et les deux compagnies des mousquetaires tout prêts dans
leurs hôtels\,; rien des gens d'armes qui n'ont point de guet, et la
seule garde ordinaire des régiments des gardes françaises et suisses aux
Tuileries.

Si j'avais peu dormi depuis huit jours, je dormis encore moins cette
dernière nuit, si proche d'événements si considérables. Je me levai
avant six heures, et peu après je reçus mon billet d'avertissement pour
le lit de justice, au dos duquel il y avait de ne me point éveiller,
politesse de des Granges, à ce qu'il me dit depuis\,; dans la persuasion
que ce billet ne pouvait me rien apprendre. On avait marqué d'éveiller
tous les autres, dont la surprise fut telle qu'il se peut penser. Vers
sept heures, un huissier de M. le duc d'Orléans vint m'avertir du
conseil de régence pour huit heures, et d'y venir en manteau. Je
m'habillai de noir, parce que je n'avais que cette sorte d'habit en
manteau, et un autre d'étoffe d'or magnifique, que je ne voulus pas
prendre, pour ne pas donner lieu à dire, quoique fort mal à propos, que
j'insultais au parlement et au duc du Maine. Je pris avec moi deux
gentilshommes dans mon carrosse, et j'allai être témoin de tout ce qui
allait s'exécuter. J'étais en même temps plein de crainte, d'espérance,
de joie, de réflexions, de défiance de la faiblesse de M. le duc
d'Orléans, et de tout ce qui en pourrait résulter. J'étais aussi dans
une ferme résolution de servir de mon mieux sur tout ce qui pourrait se
présenter, mais sans paraître instruit de rien, et sans empressement, et
je me fondai en présence d'esprit, en attention, en circonspection, en
modestie et en grand air de modération.

Sortant de chez moi j'allai à la porte de Valincourt, qui logeait
vis-à-vis la porte de derrière de l'hôtel de Toulouse. C'était un homme
fort d'honneur, de beaucoup d'esprit, mêlé avec la meilleure compagnie,
secrétaire général de la marine, qui était au comte de Toulouse depuis
sa première jeunesse, et toujours depuis dans sa plus grande confiance.
Je ne voulus laisser aucune peur personnelle au comte de Toulouse ni
l'exposer à se laisser entraîner par son frère. J'envoyai donc prier
Valincourt, que je connaissais fort, de me venir parler. Il vint
effrayé, demi-habillé, de la rumeur des rues, et d'abordée me demanda ce
que c'était que tout cela. Je le pris par la tête, et je lui dis\,:
«\,Écoutez-moi bien, et ne perdez pas un mot. Allez de ce pas dire de ma
part à M. le comte de Toulouse qu'il se fie en ma parole, qu'il soit
sage, qu'il va arriver des choses qui pourront lui déplaire par rapport
à autrui\,; mais qu'il compte avec assurance qu'il n'y perdra pas un
cheveu\,; je ne veux pas qu'il puisse en avoir un instant d'inquiétude,
allez, et ne perdez pas un instant.\,» Valincourt me serra tant qu'il
put. «\,Ah\,! monsieur, me dit-il, nous avions bien prévu qu'à la fin il
y aurait un orage. On le mérite bien, mais non pas M. le comte, qui vous
doit être éternellement obligé.\,» Il l'alla avertir sur-le-champ, et le
comte de Toulouse, qui sut après que je l'avais sauvé de la chute de son
frère, ne l'a jamais oublié.

\hypertarget{chapitre-xx.}{%
\chapter{CHAPITRE XX.}\label{chapitre-xx.}}

1718

~

{\textsc{J'arrive aux Tuileries.}} {\textsc{- Le lit de justice posé
promptement et très secrètement.}} {\textsc{- J'entre, sans le savoir,
dans la chambre où se tenaient, seuls, le garde des sceaux et La
Vrillière.}} {\textsc{- Tranquillité du garde des sceaux.}} {\textsc{-
Le régent arrive aux Tuileries.}} {\textsc{- Duc du Maine en manteau.}}
{\textsc{- J'entre dans le cabinet du conseil.}} {\textsc{- Bon maintien
et bonne résolution du régent.}} {\textsc{- Maintien de ceux du
conseil.}} {\textsc{- Divers mouvements en attendant qu'il commence.}}
{\textsc{- Le comte de Toulouse arrive en manteau.}} {\textsc{- Le
régent a envie de lui parler.}} {\textsc{- Je tâche de l'en détourner.}}
{\textsc{- Colloque entre le duc du Maine et le comte de Toulouse, puis
du comte de Toulouse avec le régent, après du comte de Toulouse avec le
duc du Maine.}} {\textsc{- Le régent me rend son colloque avec le comte
de Toulouse\,; me déclare qu'il lui a comme tout dit.}} {\textsc{- Les
bâtards sortent et se retirent.}} {\textsc{- Le conseil se met en
place.}} {\textsc{- Séance et pièce du conseil dessinée pour mieux
éclaircir ce qui s'y passa le vendredi matin 26 août 1718.}} {\textsc{-
Remarques sur la séance.}} {\textsc{- Discours du régent.}} {\textsc{-
Lecture des lettres du garde des sceaux.}} {\textsc{- Tableau du
conseil.}} {\textsc{- Discours du régent et du garde des sceaux.}}
{\textsc{- Lecture de l'arrêt du conseil de régence en cassation de ceux
du parlement.}} {\textsc{- Opinions marquées.}} {\textsc{- Légers
mouvements au conseil sur l'obéissance du parlement.}} {\textsc{-
Discours du régent sur la réduction des bâtards au rang de leurs
pairies.}} {\textsc{- Effet du discours du régent.}} {\textsc{- Lecture
de la déclaration qui réduit les bâtards au rang de leur pairie.}}
{\textsc{- Effet de cette lecture dans le conseil.}} {\textsc{- Je mets
devant moi sur la table la requête des pairs contre les bâtards ouverte
à l'endroit des signatures.}} {\textsc{- Opinions.}} {\textsc{- Je fais
au régent le remerciement des pairs de sa justice, et je m'abstiens
d'opiner.}} {\textsc{- Le régent saute de moi au maréchal d'Estrées.}}
{\textsc{- Discours de M. le duc d'Orléans sur le rétablissement du
comte de Toulouse, purement personnel.}} {\textsc{- Impression de ce
discours sur ceux du conseil.}} {\textsc{- Lecture de la déclaration en
faveur du comte de Toulouse.}} {\textsc{- Opinions.}} {\textsc{- M. le
duc d'Orléans dit deux mots sur M. le Duc, qui demande aussitôt après
l'éducation du roi.}} {\textsc{- Mouvements dans le conseil.}}
{\textsc{- Opinions.}} {\textsc{- Le maréchal de Villeroy se plaint en
deux mots du renversement des dispositions du feu roi et du malheur du
duc du Maine, sur lequel le régent lance un coup de tonnerre qui
épouvante la compagnie.}} {\textsc{- Le garde des sceaux, et par lui le
régent, est averti que le premier président tâche d'empêcher le
parlement d'obéir.}} {\textsc{- Le régent le dit au conseil\,; montre
qu'il ne s'en embarrasse pas.}} {\textsc{- Mouvements et opinions
là-dessus.}} {\textsc{- Le parlement, en marche à pied, pour venir aux
Tuileries.}} {\textsc{- Attention du régent pour le comte de Toulouse et
pour les enregistrements.}} {\textsc{- Le maréchal de Villars, contre
son ordinaire, rapporte très bien une affaire du conseil de guerre.}}
{\textsc{- Le conseil finit.}} {\textsc{- Mouvements.}} {\textsc{-
Divers colloques.}} {\textsc{- D'Antin obtient du régent de n'assister
point au lit de justice.}} {\textsc{- Je parle à Tallard sur le maréchal
de Villeroy.}} {\textsc{- La Vrillière bien courtisan.}} {\textsc{- La
Maintenon désolée.}} {\textsc{- Mouvements dans la pièce du conseil.}}
{\textsc{- Je propose au régent d'écrire à M\textsuperscript{me} la
duchesse d'Orléans, etc.}}

~

J'arrivai sur les huit heures dans la grande cour des Tuileries, sans
avoir rien remarqué d'extraordinaire en chemin. Les carrosses du duc de
Noailles et des maréchaux de Villars et d'Huxelles et de quelques
autres, y étaient déjà. Je montai sans trouver beaucoup de monde, et je
me fis ouvrir les deux portes d'entrée et de sortie de la salle des
gardes, qui étaient fermées. Le lit de justice était préparé dans la
grande antichambre où le roi avait accoutumé de manger. Je m'y arrêtai
un peu, à bien considérer si tout y était dans l'ordre, et j'en
félicitai Fontanieu à l'oreille. Il me dit de même qu'il n'était arrivé
qu'à six heures du matin aux Tuileries, avec ses ouvriers et ses
matériaux\,; que tout s'était si heureusement construit et passé que le
roi n'en avait rien entendu du tout\,; que le premier valet de chambre
étant sorti pour quelque besoin de la chambre du roi, sur les sept
heures du matin, avait été bien étonné de voir cet appareil\,; que le
maréchal de Villeroy ne l'avait appris que par lui, et qu'il y avait eu
si peu de bruit à le dresser, que personne ne s'en était aperçu. Après
avoir bien tout examiné de l'oeil, j'avançai jusqu'au trône qu'on
achevait de préparer\,; voulant entrer dans la seconde antichambre, des
garçons bleus vinrent après me dire qu'on n'y passait point, et qu'elle
était fermée. Je demandai où on se tenait en attendant le conseil, et où
étaient ceux dont j'avais vu les carrosses dans la cour. Plusieurs
s'offrirent de me mener en haut où ils étaient. Le fils de Coste me mena
par un petit degré, au haut duquel il y avait beaucoup de gens de toutes
sortes et d'officiers de chancellerie. Il me fit aller à une porte qu'on
tenait, et qui me fut ouverte dès que je parus. J'y trouvai le garde des
sceaux et La Vrillière avec toutes leurs bucoliques. Nous fûmes bien
aises de nous trouver encore seuls ensemble pour nous bien recorder
avant les opérations. Ce n'était pourtant pas ce que je m'étais proposé.
Je n'avais remarqué dans la cour de carrosses que de gens suspects. Sous
prétexte de ne les avoir point pour tels, et d'ignorer tout moi-même,
sans affectation toutefois, je voulais aller où ils étaient, pour
déranger leur conférence, et y apprendre par leurs mouvements tout ce
qu'il se pourrait. Tombé par hasard en la chambre du garde des sceaux,
je crus qu'il y aurait de l'affectation de demander d'aller ailleurs\,;
ainsi j'abandonnai ma première vue.

Le garde des sceaux était debout, tenant une croûte de pain, aussi à
lui-même que s'il n'eût été question que d'un conseil ordinaire, sans
embarras de tout ce qui allait rouler sur lui ni d'avoir à parler en
public sur des matières aussi différentes, aussi importantes et aussi
susceptibles d'inconvénients. Il me parut seulement en peine de la
fermeté du régent et rempli avec raison de la pensée qu'il ne s'agissait
plus de mollir, beaucoup moins de reculer d'une ligne. Je le rassurai
là-dessus beaucoup plus que je ne l'étais moi-même. Je leur demandai si
leurs mesures étaient bien prises pour être avertis à tout instant de ce
qui se passerait au parlement. Ils m'en répondirent et furent en effet
très bien servis. Je voulus ensuite non pas lire, car cela était
inutile, mais voir tous les instruments à enregistrer\,; ils me les
montrèrent en leur ordre. Je voulus aussi voir de plus près que les
autres celui de la réduction des bâtards au rang d'ancienneté de leurs
pairies. «\,Tenez, me dit le garde des sceaux en me le montrant, voici
votre affaire.\,» Je le remarque exprès, parce que cela me fut redit
dans la suite comme une preuve que j'étais dû secret entendu apparemment
par quelque curieux collé derrière la porte\,; car nous étions tous
trois seuls à porte fermée. Je voulais parcourir les endroits
capitaux\,; ils m'assurèrent qu'il n'y avait été changé aucune chose, et
je le reconnus parfaitement lorsque j'en entendis après la lecture.
J'eus la même curiosité sur la déclaration en faveur de M. le comte de
Toulouse, avec même réponse et même succès. Puis je me fis montrer les
sceaux à nu dans le sac de velours et les instruments de précaution
signés et scellés, tout prêts en cas de besoin. Il y avait deux gros
sacs de velours, tout remplis, qu'il ne quitta point de vue et qui
furent toujours portés sous ses yeux et mis à ses pieds, tant au conseil
qu'au lit de justice, parce que les sceaux y étaient. Qui que ce soit ne
le sut que le régent, M. le Duc, le garde des sceaux La Vrillière et
moi. Son chauffe-cire et sa boutique étaient dans une chambre à part, et
tout proche, avec de l'eau et du feu tout allumé, tout prêt sans que
personne s'en fût aperçu. Comme nous achevions ainsi notre inventaire,
toujours raisonnant sur ce qui pouvait arriver, on le vint avertir de la
venue de M. le duc d'Orléans. Nous achevâmes en un moment ce que nous
avions encore à voir et à nous dire, et, tandis qu'il prit sa robe du
lit de justice pour n'avoir pas à en changer après le conseil, je
descendis pour ne paraître pas venir d'avec lui. Je voulus même que La
Vrillière demeurât, pour ne pas entrer ensemble dans le lieu du conseil.

Depuis les grandes chaleurs on l'avait tenu dans cette pièce, qui est la
dernière du reste de l'enfilade, parce que le roi, incommodé dans sa
très petite chambre, était venu coucher dans le cabinet du conseil\,;
mais, ce grand jour-ci, dès que le roi fut hors de son lit, on le mena
s'habiller dans sa petite chambre et de là dans ses cabinets. On tira
les housses de son lit et celui du maréchal de Villeroy, au pied
desquels on mit la table du conseil, et il y fut tenu. En entrant dans
la pièce de devant, j'y trouvai beaucoup de monde que le premier bruit
d'une chose si peu attendue avait sans doute amené, et parmi ce monde
quelques-uns du conseil. M. le duc d'Orléans était dans un gros de gens
au bas bout de cette pièce et, ce que je sus depuis, sortait de chez le
roi, où il avait vu le duc du Maine en manteau, qui l'avait suivi
jusqu'à la porte, comme il sortait, sans s'être dit un mot l'un à
l'autre.

Après un assez léger coup d'oeil sur cette demi-foule, j'entrai dans le
cabinet du conseil. J'y trouvai épars la plupart de ceux qui le
composaient avec un sérieux et un air de contention d'esprit qui
augmenta la mienne. Personne presque ne se parlait, et chacun, debout ou
assis, çà et là, se tenait assez en sa place. Je ne joignis personne
pour mieux examiner. Un moment après M. le duc d'Orléans entra d'un air
gai, libre, sans aucune émotion, qui regarda la compagnie d'un air
souriant\,: cela me fut d'un bon augure. Un moment après je lui demandai
de ses nouvelles. Il me répondit tout haut qu'il était assez bien\,;
puis, s'approchant de mon oreille, il ajouta que, hors les réveils qui
avaient été fréquents pour les ordres, il avait très bien dormi et qu'il
venait délibéré de ne point mollir. Cela me plut infiniment, car il me
sembla, à son maintien, qu'il me disait vrai et je l'y exhortai en deux
paroles.

Vint après M. le Duc, qui ne tarda pas à s'approcher de moi et à me
demander si j'augurais bien du régent et qu'il fût ferme. Celui-ci avait
un air de gaieté haute qui se faisait un peu sentir à qui était au fait.
Le prince de Conti, morosif, distrait, envieux de son beau-frère, ne
paraissait qu'occupé, mais de rien. Le duc de Noailles dévorait tout des
yeux et les avait étincelants de colère de se voir au parterre dans un
si grand jour, car il ne savait chose quelconque. Je l'avais ainsi
demandé à M. le Duc expressément, croyant leur liaison plus grande que
je ne la trouvai. Il en pensait avec défiance, sans estime, encore moins
d'amitié, indépendamment de ce qu'il y avait nouvellement à craindre de
lui avec M. du Maine.

Celui-ci parut à son tour en manteau, et entra par la petite porte du
roi. Jamais il ne fit tant et de si profondes révérences, quoiqu'il n'en
fût pas avare, et se tint seul perché sur son bâton, près de la table du
conseil, du côté des lits, considérant tout le monde. Ce fut là, où, de
vis-à-vis de lui, la table entre deux, je lui tirai la plus riante
révérence que je lui eusse faite de ma vie, avec la plus sensible
volupté. Il me la rendit pareille et continua d'observer chacun avec des
yeux tirant au fixe, un visage agité, partant tout seul presque
toujours.

Presque personne ne se demandait qu'est-ce que c'était que tout cela\,;
tous savaient la résolution prise de casser les arrêts du parlement pour
avoir assisté à cette délibération. Ce conseil était l'extraordinaire,
indiqué puis remis, pour y voir l'arrêt du conseil en cassation. Il fut
donc clair à tous que c'était ce qu'on allait voir pour le faire
enregistrer tout de suite, non peut-être sans peine d'un lit de justice
de surprise, surtout pour quelques-uns qui se croient privilégiés auprès
du régent. M. le Duc revint encore à moi assez de suite me témoigner sa
peine de voir là le duc du Maine en manteau et pour m'exhorter à
fortifier M. le duc d'Orléans, puis le garde des sceaux vint à moi pour
la même chose. Un moment après M. le duc d'Orléans m'en vint parler,
assez empêché de ce manteau, mais sans témoigner de faiblesse. Je lui
représentai que je lui avais toujours dit qu'il devait s'y attendre\,;
que mollir serait sa perte\,; que le Rubicon était passé. J'ajoutai ce
que je pus de plus fort et de plus concis pour le soutenir et pour ne
paraître pas aussi trop longtemps en conférence avec lui. Aussitôt que
je me fus séparé de lui, M. le Duc impatient et inquiet me vint demander
en quelle disposition d'esprit était le régent. Je lui dis bonne, en
monosyllabe, et l'envoyai l'y entretenir.

Je ne sais si ces mouvements, sur lesquels chacun commençait d'avoir les
yeux, effarouchèrent le duc du Maine\,; mais à peine M. le Duc eut-il,
en me quittant, joint le régent, que le duc du Maine alla parler au
maréchal de Villeroy et à d'Effiat, assis l'un près de l'autre au bas
bout vers la petite porte du roi, le dos à la muraille. Ils ne se
levèrent point pour le duc du Maine, qui demeura debout vis-à-vis et
tout près d'eux, où ils tinrent tous trois des propos bas assez longs,
comme gens qui délibèrent avec embarras et surprise, à ce qu'il me
paraissait au visage des deux assis que je voyais assez bien, et que je
tâchais à ne pas perdre de vue. Pendant ce temps-là M. le Duc d'Orléans
et M. le Duc se parlaient vers la fenêtre, près de la porte ordinaire
d'entrée, ayant le barde des sceaux assez près d'eux, qui les joignit.
M. le Duc, en ce moment, se tourna un peu, ce qui me donna moyen de lui
faire signe de l'autre conférence, qu'il avisa aussitôt. J'étais seul
vers la table du conseil, très attentif à tout, et les autres, épars,
commencèrent à le devenir davantage. Un peu après le duc du Maine vint
se remettre d'où il était parti, les deux étant restés assis où ils
étaient. M. du Maine alors se retrouva vis-à-vis de moi, la table entre
deux. J'observai qu'il avait l'air égaré, et qu'il parlait tout seul
plus que devant.

Le comte de Toulouse arriva en manteau, comme le régent venait de
quitter les deux avec qui il était. Le comte de Toulouse était en
manteau, et salua la compagnie d'un air grave et concentré, n'abordant
ni abordé de personne. M. le duc d'Orléans se trouva vis-à-vis de lui et
se tourna vers moi, quoiqu'à quelque distance, comme me le montrant et
m'en témoignant sa peine. Je baissai un peu la tête en le regardant
fixement, comme pour lui dire\,: «\,Eh bien, quoi\,? » M. le duc
d'Orléans s'avança au comte de Toulouse, et lui dit tout haut, devant
tout ce qui était là proche, qu'il était surpris de le voir en
manteau\,; qu'il n'avait pas voulu le faire avertir du lit de justice,
parce qu'il savait que, depuis leur dernier arrêt, il n'aimait pas aller
au parlement. Le comte de Toulouse répondit qu'il était vrai\,; mais
que, quand il s'agissait du bien de l'État, il mettait toute autre
considération à part. M. le duc d'Orléans se tourna sur-le-champ sans
rien répliquer, vint à moi, et me dit tout bas en me poussant plus
loin\,: «\,Voilà un homme qui me perce le coeur. Savez-vous bien ce
qu'il vient de me dire\,?» et me le répéta. Je louai le procédé de l'un,
le sentiment de l'autre\,; lui remontrai que le rétablissement du comte
de Toulouse étant résolu, et pour la même séance, son état ne devait pas
lui faire de peine, et je me mis doucement à le réconforter. Il
m'interrompit pour me dire l'envie qu'il avait de lui parler. Je lui
représentai que cela était bien délicat, et qu'au moins avant de s'y
résoudre, fallait-il attendre à toute extrémité. Je me tournai aussitôt
pour le ramener vers le gros du monde, pour abréger ce particulier que
je craignis qui ne fût trop remarqué. Le comte de Toulouse nous voyait
et était resté à la même place, et chacun nous voyait aussi, cantonné à
part soi.

Le duc du Maine était retourné au maréchal de Villeroy et à d'Effiat,
eux assis sans branler en la même place, et lui debout devant eux, comme
l'autre fois. Je vis ce petit conciliabule très ému. Il dura quelque
espace, pendant lequel M. le Duc me vint parler, puis le garde des
sceaux nous joignit, inquiets tous deux de ce qu'avait produit l'arrivée
du comte de Toulouse, sur laquelle M. le duc d'Orléans m'avait pris en
particulier. Je le leur dis, et me séparai d'eux le plus tôt que je pus.
Ce qui m'en hâta encore, fut que je venais de m'apercevoir que le duc de
Noailles n'ôtait pas les yeux de dessus moi, et me suivait de la vue,
quelque mouvement que je fisse, changeant même de place ou de posture
pour se trouver toujours en situation de me voir. Le duc de La Force me
voulut joindre alors\,; cela fut cause que je l'éconduisis
promptement\,; La Vrillière ensuite, à qui je dis quelque chose, et
l'envoyai au garde des sceaux pour qu'il fortifiât le régent. Cependant
M. du Maine quitta ses deux hommes et fit signe à son frère de le venir
trouver au pied du lit du maréchal de Villeroy où il venait de se
poster. Il lui parla avec agitation assez peu, l'autre répliqua de même,
comme n'étant pas trop d'accord. Le duc du Maine redoubla\,; puis le
comte de Toulouse alla entre les pieds des deux lits et la table gagner
la cheminée, où M. le duc d'Orléans était avec M. le Duc, et s'arrêta à
distance, en homme qui attend pour parler. M. le duc d'Orléans, qui s'en
aperçut, quitta M. le Duc quelques moments après, et alla au comte de
Toulouse. Ils se tournèrent le nez tout à fait à la muraille, et cela
dura assez longtemps sans qu'on en pût rien juger, parce qu'on ne voyait
que leur dos, et qu'il n'y parut ni émotion ni presque aucun geste.

Le duc du Maine était demeuré seul où il avait parlé à son frère. Il
présentait un visage demi-mort, regardait comme à la dérobée le colloque
qu'il avait envoyé faire, puis passait des yeux égarés sur la compagnie
avec un trouble de coupable et une agitation de condamné. Alors le
maréchal d'Huxelles m'appela. Il était vis-à-vis du duc du Maine, la
table entre deux, y avait le dos tourné, par conséquent au duc du Maine.
Le maréchal était là en groupe avec les maréchaux de Tallard et
d'Estrées et l'ancien évêque de Troyes, desquels le duc de Noailles
s'approcha en même temps que moi.

Huxelles me demanda ce que c'était donc que toutes ces allées et venues,
et sur ce que je lui en fis pour réponse la même question à lui-même, il
me demanda s'il y avait quelque difficulté au lit de justice pour ces
princes ou peut-être pour les enfants de M. du Maine. Je lui répondis
que, pour MM. du Maine et de Toulouse, il n'y en pouvait avoir, parce
que l'arrêt intervenu entre les princes du sang et eux les laissait dans
la jouissance de tous les honneurs qu'ils avaient\,; mais que, pour les
enfants du duc du Maine, nous ne les y souffririons pas.

Nous restâmes quelque peu ainsi en groupe, moi occupé à regarder M. du
Maine, et de me tourner quelquefois à regarder le colloque du régent et
du comte de Toulouse, qui persévérait. Il se sépara enfin, et j'eus le
temps de bien remarquer les deux frères, parce que le comte de Toulouse
revint vers nous, la table entre-deux, le long des pieds des lits,
trouver son frère, toujours resté seul debout sur son bâton, au pied du
lit du maréchal de Villeroy, à la même place d'où il n'avait bougé. Le
comte de Toulouse avait l'air fort peiné, même colère. Le duc du Maine,
le voyant venir à lui de la sorte, changea tout à fait de couleur.

Je demeurais là bien attentif, les considérant se joindre, sans que le
duc du Maine eût branlé de sa place, pour pénétrer leur conversation de
mes yeux, lorsque je m'entendis appeler. C'était M. le duc d'Orléans
qui, après avoir fait quelques pas seul le long de la cheminée, me
voulait parler. Je le joignis et le trouvai en trouble de coeur. «\,Je
lui viens de tout dire, me déclara-t-il à l'instant, je n'ai pu y
tenir\,; c'est le plus honnête homme du monde et qui me perce le plus le
coeur. --- Comment, monsieur, repris-je, et que lui avez-vous dit\,? ---
Il m'est venu trouver, me répondit-il, de la part de son frère, qui
venait de lui parler, pour me dire l'embarras où il se trouvait\,; qu'il
voyait bien qu'il y avait quelque chose de préparé\,; qu'il voyait bien
aussi qu'il n'était pas bien avec moi\,; qu'il l'avait prié de me venir
demander franchement si je voulais qu'il demeurât, ou s'il ne ferait pas
aussi bien de ne pas rester. Je vous avoue que j'ai cru bien faire de
lui dire qu'il ferait aussi bien de s'en aller, puisqu'il me le
demandait. Là-dessus, le comte de Toulouse a voulu entrer en
explication\,; j'ai coupé court, et lui ai dit que, pour lui, il pouvait
rester en sûreté, parce qu'il demeurerait tel qu'il est sans nulle
altération\,; mais qu'il pourrait se passer des choses désagréables à M.
du Maine, dont il ferait aussi bien de n'être pas témoin. Le comte de
Toulouse a insisté comment il pouvait rester comme il est dès qu'on
attaquait son frère, et qu'ils n'étaient qu'un parce qu'ils étaient
frères, et par honneur. J'ai répondu que j'en étais bien fâché\,; que
tout ce que je pouvais était de distinguer le mérite et la vertu, et de
la séparer, et puis quelques propos et des amitiés qu'il a reçues assez
froidement, et de là l'est allé dire à son frère. Trouvez-vous que j'aie
mal fait\,? --- Non, lui dis-je, car il n'était plus question d'en
délibérer, ni moins encore d'embarrasser un homme qu'il ne s'agissait
que de fortifier\,; j'en suis bien aise, ajoutai-je, c'est parler net en
homme qui a ses mesures bien prises et qui ne craint rien. Aussi faut-il
montrer toute fermeté encore plus avec cet engagement pris.\,» Il m'y
parut très résolu\,; mais en même temps très désireux que les bâtards
s'en allassent, qui fut, à ce que je crus voir, le vrai motif de ce
qu'il venait de faire.

M. le Duc vint à nous, je demeurai avec eux le moins que je pus, et je
leur conseillai de se séparer aussi, d'autant que toute la compagnie
partageait ses regards entre nous et les deux frères.

Le duc du Maine, pâle et comme mort, me parut près de se trouver mal\,;
il s'ébranla à peine pour gagner le bas bout de la table, dont il était
assez près, pendant quoi le comte de Toulouse vint dire un mot très
court au régent, et se mit en marche le long du cabinet. Tous ces
mouvements se firent en un clin d'oeil. Le régent, qui était auprès du
fauteuil du roi, dit haut\,: «\, Allons messieurs, prenons nos
places.\,» Chacun s'approcha de la sienne, et comme je regardais de
derrière la mienne, je vis les deux frères auprès de la porte ordinaire
d'entrée comme des gens qui allaient sortir. Je sautai, pour ainsi dire,
entre le fauteuil du roi et M. le duc d'Orléans pour n'être pas entendu
du prince de Conti, et je dis à l'oreille avec émotion au régent, qui
était déjà en place\,: «\,Monsieur, les voilà qui sortent. --- Je le
sais bien, me répondit-il tranquillement. --- Oui, répliquai-je avec
vivacité, mais savez-vous ce qu'ils feront quand ils seront dehors\,?
--- Rien du tout, me dit-il\,; le comte de Toulouse m'est venu demander
permission de sortir avec son frère\,; il m'a assuré qu'ils seront
sages. --- Et s'ils ne le sont pas\,? répliquai-je. --- Mais ils le
seront, et s'ils ne le sont pas, il y a de bons ordres de les bien
observer. --- Mais s'ils font sottise ou qu'ils sortent de Paris. --- On
les arrêtera, il y a de bons ordres, je vous en réponds.\,» Là-dessus,
plus tranquille, je me mis en place\,; à peine y fus-je qu'il me
rappela, et me dit que, puisqu'ils sortaient, il changeait d'avis, et
avait envie de dire ce qui les regardait au conseil. Je lui répondis que
le seul inconvénient qui l'en empêchait étant levé par cette sortie, je
croirais que ce serait très mal fait de ne le pas dire à la régence. Il
le communiqua à M. le Duc, tout bas à travers la table et le fauteuil du
roi, puis appela le garde des sceaux, qui tous deux l'approuvèrent, et
alors nous nous mîmes tout à fait en place.

Tous ces mouvements avaient augmenté le trouble et la curiosité de
chacun. Les yeux de tous occupés sur le régent, avaient fait tourner le
dos à la porte ordinaire d'entrée, et on ne s'aperçut point pour la
plupart que les bâtards n'y étaient plus. À mesure que chacun ne les vit
point en se plaçant, il les cherchait des yeux, et restait debout en
attendant. Je me mis au siège du comte de Toulouse. Le duc de Guiche,
qui était à mon autre côté, laissa un siège entre nous deux, le nez
haut, attendant toujours les bâtards. Il me dit de m'approcher de lui,
et que je me méprenais de siège. Je ne répondais mot, en considérant la
compagnie qui était un vrai spectacle. À la seconde ou troisième
semonce, je lui répondis qu'au contraire il s'approchât de moi. «\,Et M.
le comte de Toulouse, répliqua-t-il. --- Approchez-vous,\,» repris-je,
et le voyant immobile d'étonnement, regardant vis-à-vis où était le duc
du Maine, dont le garde des sceaux avait pris la place, je le tirai par
son habit, moi tout assis, en lui disant\,: «\,Venez çà et
asseyez-vous.\,» Je le tirai si fort qu'il s'assit près de moi sans
comprendre. «\,Mais qu'est-ce que ceci, me dit-il dès qu'il fut assis,
où sont donc ces messieurs\,? --- Je n'en sais rien, repris-je
d'impatience, mais ils n'y sont pas.\,» En même temps le duc de
Noailles, qui joignait le duc de Guiche, et qui, enragé de n'être de
rien dans une aussi grande préparation de journée, avait apparemment
compris à force de regarder et d'examiner que j'étais dans la bouteille,
et vaincu par sa curiosité, s'allongea sur la table par-devant le duc de
Guiche, et me dit\,: «\, Au nom de Dieu, monsieur le duc, faites-moi la
grâce de me dire ce que c'est donc que tout ceci.\,» Je n'étais en nulle
mesure avec lui, comme on l'a vu souvent, mais bien en usage de le
traiter très mal. Je me tournai à lui d'un air froid et dédaigneux, et,
après l'avoir ouï et regardé, je retournai la tête. Ce fut là toute ma
réponse. Le duc de Guiche me pressa de lui dire quelque chose, jusqu'à
me dire que je savais tout. Je le niai toujours, et cependant chacun se
plaçait lentement, parce qu'on ne songeait qu'à regarder et à deviner ce
que tout cela pouvait être, et qu'on fut longtemps à comprendre qu'il
fallait se placer sans les bâtards, bien qu'aucun n'en ouvrît la bouche.

Avant d'entrer dans ce qui se passa au conseil, il en faut donner la
séance de ce jour-là, et la disposition de la pièce\footnote{Sur
  l'exemplaire des Mémoires de Saint-Simon (édit Sautelet, t. XVII,
  p.~103) conservé à la Bibl. imp. de Louvre, le roi Louis-Philippe a
  écrit de sa main une note marginale conçue en ces termes\,: «\,C'est
  aux Tuileries la même salle qui a été celle des conseils sous Louis
  XVI, Napoléon, Louis XVIII et Charles X. J'y ai vu trois de ces
  souverains en conseil.\,»} où il se tint, pour mieux faire entendre ce
qui vient d'être raconté, et donner plus de jour à ce qui va l'être.

Il faut remarquer, sur la séance, que le maréchal d'Huxelles se mettait
toujours à droite, pour mieux lire les dépêches à contre-jour, et M. de
Troyes toujours auprès de lui, pour le soulager dans cette lecture. Ils
s'y mirent ce jour-là par habitude, quoiqu'ils n'eussent rien à lire, et
intervertirent ainsi le bas bout de la séance, ce qui n'empêcha pas
néanmoins que les avis ne fussent pris au rang où ils devaient l'être.
Il faut remarquer encore que la table du conseil n'étant pas assez
longue pour que chacune des deux rangées y fût commodément, d'Effiat et
Torcy étaient au bout, de manière qu'Effiat était presque au milieu du
bout, pour laisser plus de terrain à La Vrillière pour écrire
commodément. M. le duc d'Orléans, à l'autre bout, s'y tourna aussi un
peu vers le fauteuil vide du roi, pour voir mieux des deux côtés, ce
qu'il ne faisait jamais. Mais, outre que ce jour-là il voulait voir son
côté, il ne fut pas fâché de l'affecter, et de le laisser voir. Le garde
des sceaux avait à ses pieds, à terre, le sac de velours noir où étaient
les sceaux à nu, avec les instruments de précaution, signés et scellés,
et l'autre sac devant lui sur la table où il avait rangé tout ce qu'il
devait lire au conseil, dans l'ordre où chaque chose devait l'être, et
ce qui devait {[}être{]} enregistré, toutes choses et pièces qui furent
aussi lues au lit de justice. Le roi cependant était dans ses cabinets
et ne parut point du tout dans le lieu où se tint ce conseil ni dans les
pièces qui y tenaient.

Lorsqu'on fut tout à fait assis en place, et que M. le duc d'Orléans eut
un moment considéré toute l'assistance dont tous les yeux étaient fichés
sur lui, il dit qu'il avait assemblé ce conseil de régence pour y
entendre la lecture de ce qui avait été résolu au dernier\,; qu'il avait
cru qu'il n'y avait d'expédient pour faire enregistrer l'arrêt du
conseil dont on allait entendre la lecture que de tenir un lit de
justice, et que les chaleurs ne permettant pas de commettre la santé du
roi à la foule du palais, il avait estimé devoir suivre l'exemple du feu
roi, qui avait fait quelquefois venir son parlement aux Tuileries\,;
que, puisqu'il fallait tenir un lit de justice, il avait jugé devoir
profiter de cette occasion pour y faire enregistrer les lettres de
provision de garde des sceaux, et commencer par là cette séance, et il
ordonna au garde des sceaux de les lire.

Pendant cette lecture, qui n'avait d'autre importance que de saisir une
occasion de forcer le parlement de reconnaître le garde des sceaux dont
la compagnie haïssait la personne et la commission, je m'occupai
cependant à considérer les mines. Je vis en M. le duc d'Orléans un air
d'autorité et d'attention, qui me fut si nouveau, que j'en demeurai
frappé. M. le Duc, gai et brillant, paraissait ne douter de rien. Le
prince de Conti, étonné, distrait, concentré, ne semblait rien voir ni
prendre part à rien. Le garde des sceaux, grave et pensif, paraissait
avoir trop de choses, dans la tête\,; aussi en avait-il beaucoup à faire
et pour un coup d'essai. Néanmoins, il se déploya avec son sac en homme
bien net, bien décidé, bien ferme. Le duc de La Force, les yeux en
dessous, examinait les visages. Les maréchaux de Villeroy et de Villars
se parlaient des instants\,: ils avaient tous deux l'oeil irrité et le
visage abattu. Nul ne se composa mieux que le maréchal de Tallard\,;
mais il ne put étouffer une agitation intérieure qui étincela souvent au
dehors. Le maréchal d'Estrées avait l'air stupéfait et de ne voir qu'un
étang. Le maréchal de Besons, enveloppé plus que d'ordinaire dans sa
grosse perruque, paraissait tout concentré, et l'oeil bas et colère.
Pelletier, très dégagé, simple, curieux, regardait tout. Torcy, plus
empesé trois fois que de coutume, semblait considérer tout à la dérobée.
Effiat, vif, piqué, outré, prêt à bondir, le sourcil froncé à tout le
monde, l'oeil hagard, qu'il passait avec précipitation et par élans de
tous côtés. Ceux de mon côté, je ne pouvais les bien examiner je ne les
voyais que des moments par des changements de postures des uns et des
autres, et si la curiosité me faisait m'avancer sur la table et me
tourner vers eux pour en regarder l'enfilade, ce n'était que bien
rarement et bien courtement. J'ai déjà parlé de l'étonnement du duc de
Guiche, du dépit et de la curiosité du duc de Noailles. D'Antin,
toujours si libre dans sa taille, me parut tout emprunté et tout
effarouché. Le maréchal d'Huxelles cherchait à faire bonne mine, et ne
pouvait couvrir le désespoir qui le perçait. Le vieux Troyes, tout
ébahi, ne montrait que de la surprise, de l'embarras, et de ne savoir
proprement où il en était.

Dès l'instant de cette première lecture chacun vit bien, au départ des
bâtards, après tout ce qui s'était passé dans ce cabinet du conseil
avant la séance, qu'il s'agirait de quelque chose contre eux. La nature
et le plus ou le moins de ce quelque chose tenaient tous les esprits en
suspens, et cela joint à un lit de justice aussitôt éclaté et prêt
qu'annoncé, marquait une grande résolution prise contre le parlement,
annonçait aussi tant de fermeté et de mesures dans un prince si reconnu
pour en être entièrement incapable que tous en perdaient terre. Chacun,
suivant ce qu'il était affecté de bâtardise ou de parlement, semblait
attendre avec frayeur ce qui allait éclore. Beaucoup d'autres
paraissaient vivement blessés de n'avoir eu part à rien, de se trouver
dans la surprise commune, et que le régent leur eût échappé. Jamais
visages si universellement allongés, ni d'embarras plus général ni plus
marqué. Dans ce premier trouble, je crois que peu de gens prêtèrent
l'oreille aux lettres dont le garde des sceaux faisait la lecture. Quand
elle fut achevée, M. le duc d'Orléans dit qu'il ne croyait pas que ce
fût la peine de prendre les voix un à un, ni sur leur contenu ni sur
leur enregistrement, et qu'il pensait que tous seraient d'avis de
commencer la séance du lit de justice par là.

Après une petite pause, mais marquée, le régent exposa en peu de mots
les raisons qui avaient fait résoudre au dernier conseil de régence de
casser les arrêts du parlement qu'on y avait lus, et de le faire par un
arrêt du conseil de régence. Il ajouta qu'à la conduite présente du
parlement, c'eût été commettre de nouveau l'autorité du roi d'envoyer
cet arrêt au parlement, qui eût donné au public une désobéissance
formelle en refusant sûrement de l'enregistrer\,; que n'y ayant que la
voie du lit de justice pour y parvenir, il avait estimé le devoir faire
tenir fort secret pour ne pas donner lieu aux cabales et aux
malintentionnés d'y essayer à continuer la désobéissance, en leur
donnant le temps de s'y préparer\,; qu'il avait cru, avec M. le garde
des sceaux, que la fréquence et la manière des remontrances du parlement
méritait que cette compagnie fût remise dans les bornes du devoir, que
depuis quelque temps elle avait perdu de vue\,; que M. le garde des
sceaux allait lire au conseil un arrêt qui contenait la cassation
délibérée et les règles qu'elle devait observer à l'avenir. Puis,
regardant le garde des sceaux\,: «\,Monsieur, lui dit-il, vous
l'expliquerez mieux que moi à ces messieurs\,: prenez la peine de le
faire avant que de lire l'arrêt.\,»

Le garde des sceaux prit la parole, et paraphrasa ce que Son Altesse
Royale avait dit plus courtement\,; il expliqua ce que c'était que
l'usage des remontrances, d'où il venait, ses utilités, ses
inconvénients, ses bornes, la grâce de les avoir rendues, l'abus qui en
était fait, la distinction de la puissance royale d'avec l'autorité du
parlement émanée du roi, l'incompétence des tribunaux en matière d'État
et de finances, et la nécessité de la réprimer par une manière de code
(ce fut le terme dont il se servit), qui fût à l'avenir la règle
invariable du fond et de la forme de leurs remontrances. Cela expliqué
sans longueur, avec justesse et grâce, il se mit à lire l'arrêt tel
qu'il est imprimé, et entre les mains de tout le monde, à quelques
bagatelles près, mais si légères, que leur ténuité me les a fait
échapper.

La lecture achevée, le régent, contre sa coutume, montra son avis par
les louanges qu'il donna à cette pièce\,; puis, prenant un air et un ton
de régent que personne ne lui avait encore vu, qui acheva d'étonner la
compagnie, il ajouta\,: «\,Pour aujourd'hui, messieurs, je m'écarterai
de la règle ordinaire pour prendre les voix, et je pense qu'il sera bon
que j'en use ainsi pour tout ce conseil.\,» Puis, après un léger coup
d'oeil passé sur les deux côtés de la table, pendant lequel on eût
entendu un ciron marcher, il se tourna vers M. le Duc, et lui demanda
son avis. M. le Duc opina pour l'arrêt, alléguant plusieurs raisons
courtes, mais fortes. Le prince de Conti parla aussi en même sens. Moi
ensuite, car le garde des sceaux avait opiné tout de suite après sa
lecture. Je fus du même avis, mais plus généralement, quoique aussi
fortement, pour ne pas tomber inutilement sur le parlement, et pour ne
m'arroger pas d'appuyer Son Altesse Royale à la manière des princes du
sang. Le duc de La Force s'étendit davantage. Tous parlèrent, mais la
plupart très peu\,; et quelques-uns, tels que les maréchaux de Villeroy,
Villars, Estrées, Besons, M. de Troyes et d'Effiat laissèrent voir leur
douleur de n'oser résister au parti pris, dont il était clair qu'il n'y
avait pas à espérer d'en rien rabattre. L'abattement se peignit sur
leurs visages, et vit qui voulut que celui du parlement n'était ni ce
qu'ils désiraient ni ce qu'ils avaient cru qui pouvait arriver. Tallard
fut le seul d'eux qui en cela ne parut pas\,; mais le monosyllabe
suffoqué du maréchal d'Huxelles fit tomber ce qu'il lui restait de
masque. Le duc de Noailles se contint avec tant de peine qu'il parla
plus qu'il ne voulait, et avec une angoisse digne de Fresnes\footnote{Allusion
  au chancelier d'Aguesseau, alors exilé dans sa terre de Fresnes.}. M.
le duc d'Orléans opina le dernier, mais avec une force très insolite\,;
puis fit encore une pause, repassant tout le conseil sous ses yeux.

En ce moment le maréchal de Villeroy, plein de sa pensée, se demanda
entre ses dents\,: «\,Mais viendront-ils\,?» Cela fut doucement relevé.
M. le duc d'Orléans dit qu'ils en avaient assuré des Granges, et ajouta
qu'il n'en doutait pas, et tout de suite qu'il faudrait faire avertir
quand on les saurait en marche. Le garde des sceaux répondit qu'il le
serait. M. le duc d'Orléans reprit qu'il le faudrait toujours faire dire
à la porte\,; et, tout aussitôt voilà M. de Troyes debout. La peur me
prit si brusque qu'il n'allât jaser à la porte, que j'y courus plus tôt
que lui. Comme je revenais, d'Antin, qui s'était tourné pour me guetter
au passage, me pria en grâce de lui dire ce que c'était que ceci. Je
coulai, disant que je n'en savais rien\,: «\,Bon, reprit-il, à
d'autres\,!» Remis en place, M. le duc d'Orléans dit encore je ne sais
plus quoi\,; et M. de Troyes encore en l'air, moi aussi comme l'autre
fois. En passant je dis à La Vrillière de se saisir de toutes les
commissions pour aller à la porte, de peur du babil de M. de Troyes ou
de quelque autre, parce que de l'éloignement d'où j'étais assis, cela
marquait trop. En effet, cela était essentiel, et La Vrillière le fit
depuis. Retournant en ma place, encore d'Antin en embuscade,
m'interpellant, au nom de Dieu et les mains jointes, je tins bon, et lui
dis\,: «\,Vous allez voir.\,» Le duc de Guiche à mon retour en place me
pressa aussi inutilement, jusqu'à me dire qu'on voyait bien que j'étais
dans la bouteille\,: je demeurai sourd.

Ces petits mouvements passés, M. le duc d'Orléans, redressé sur son
siège d'un demi-pied, dit à la compagnie, d'un ton encore plus ferme et
plus de maître qu'à la première affaire, qu'il y en avait une autre à
proposer bien plus importante que celle qu'on venait d'entendre. Ce
prélude renouvela l'étonnement des visages, et rendit les assistants
immobiles. Après un moment de silence, le régent dit qu'il avait jugé le
procès qui s'était élevé entre les princes du sang et les légitimés\,:
ce fut le terme dont il usa sans y ajouter celui de princes\,; qu'il
avait eu alors ses raisons pour n'en pas faire davantage\,; mais qu'il
n'était pas moins obligé de faire justice aux pairs de France, qui
l'avaient demandée en même temps au roi par une requête en corps, que Sa
Majesté avait reçue elle-même, et que lui-même régent avait communiquée
aux légitimés\,; que cette justice ne se pouvait plus différer à un
corps aussi illustre, composé de tous les grands du royaume, des
premiers seigneurs de l'État, des personnes les plus grandement
revêtues, et dont la plupart s'étaient distingués par les services
qu'ils avaient rendus\,; que, s'il avait estimé au temps de leur requête
n'y devoir pas répondre, il ne se sentait que plus pressé de ne plus
différer une justice qui ne pouvait plus demeurer suspendue, et que tous
les pairs désiraient de préférence à tout\,; que c'était avec douleur
qu'il voyait des gens (ce fut le mot dont il se servit) qui lui étaient
si proches, montés à un rang dont ils étaient les premiers exemples, et
qui avait continuellement augmenté contre toutes les lois\,; qu'il ne
pouvait se fermer les yeux à la vérité\,; que la faveur de quelques
princes, et encore bien nouvellement, à voit interverti le rang des
pairs\,; que ce préjudice fait à cette dignité n'avait duré qu'autant
que l'autorité qui avait forcé les lois\,; qu'ainsi les ducs de Joyeuse
et d'Épernon, ainsi MM. de Vendôme avaient été remis en règle et en leur
rang d'ancienneté parmi les pairs, aussitôt après la mort de Henri III
et de Henri IV\,; que M. de Beaufort n'avait point eu d'autre rang sous
les yeux du feu roi, ni M. de Verneuil, que le roi fit duc et pair, en
1663, avec treize autres, et qui fut reçu au parlement, le roi y tenant
son lit de justice, avec eux, et y prit place après tous les pairs ses
anciens y séants\,; et n'y en a jamais eu d'autre\,; que, l'équité, le
bon ordre, la cause de tant de personnes si considérables et la première
dignité de l'État ne lui permettaient pas un plus long déni de
justice\,; que les légitimés avaient eu tout le temps de répondre, mais
qu'ils ne pouvaient alléguer rien de valable contre la force des lois et
des exemples\,; qu'il ne s'agissait que de faire droit sur une requête
pour un procès existant et pendant, qu'on ne pouvait pas dire qui ne fût
pas instruit\,; que, pour y prononcer, il avait fait dresser la
déclaration dont M. le garde des sceaux allait faire la lecture, pour la
faire enregistrer après au lit de justice que le roi allait tenir.

Un silence profond succéda à un discours si peu attendu et qui commença
à développer l'énigme de la sortie des bâtards. Il se peignit un brun
sombre sur quantité de visages. La colère étincela sur celui des
maréchaux de Villars et de Besons, d'Effiat, même du maréchal d'Estrées.
Tallard devint stupide quelques moments, et le maréchal de Villeroy
perdit toute contenance. Je ne pus voir celle du maréchal d'Huxelles,
que je regrettai beaucoup, ni du duc de Noailles que de biais par-ci,
par-là. J'avais la mienne à composer, sur qui tous les yeux passaient
successivement. J'avais mis sur mon visage une couche de plus de gravité
et de modestie. Je gouvernais mes yeux avec lenteur, et ne regardais
qu'horizontalement pour le plus haut. Dès que le régent ouvrit la bouche
sur cette affaire, M. le Duc m'avait jeté un regard triomphant, qui
pensa démonter tout mon sérieux, qui m'avertit de le redoubler et de ne
m'exposer plus à trouver ses yeux sous les miens. Contenu de la sorte,
attentif à dévorer l'air de tous, présent à tout et à moi-même,
immobile, collé sur mon siège, compassé de tout mon corps, pénétré de
tout ce que la joie peut imprimer de plus sensible et de plus vif, du
trouble le plus charmant, d'une jouissance la plus démesurément et la
plus persévéramment souhaitée, je suais d'angoisse de la captivité de
mon transport, et cette angoisse même était d'une volupté que je n'ai
jamais ressentie ni devant ni depuis ce beau jour. Que les plaisirs des
sens sont inférieurs à ceux de l'esprit, et qu'il est véritable que la
proportion des maux est celle-là même des biens qui les finissent.

Un moment après que le régent eut cessé de parler il dit au garde des
sceaux de lire la déclaration. Il la lut tout de suite, sans discourir
auparavant, comme il avait fait dans l'affaire précédente. Pendant cette
lecture qu'aucune musique ne pouvait égaler à mes oreilles, mon
attention fut partagée à reconnaître si elle était entièrement la même
que Millain avait dressée et qu'il m'avait montrée, et j'eus la
satisfaction de la trouver la même parfaitement, et à examiner
l'impression qu'elle faisait sur les assistants\,; peu d'instants me
découvrirent, par la nouvelle altération de leurs visages, ce qui se
passait dans leur âme, et peu d'autres m'avertirent, à l'air de
désespoir qui saisit le maréchal de Villeroy, et de fureur qui surprit
Villars, qu'il fallait apporter un remède à ce que le désordre, dont ils
ne paraissaient plus les maîtres, pouvait leur arracher. Je l'avais dans
ma poche et je l'en tirai alors. C'était notre requête contre les
bâtards que je mis devant moi sur la table et que j'y laissai ouverte au
dernier feuillet, qui contenait toutes nos signatures imprimées en gros
caractères majuscules. Elles furent incontinent regardées par ces deux
maréchaux et reconnues sans doute, au farouche abattu de leurs yeux qui
succéda sur-le-champ et qui éteignit je ne sais quel air de menace,
surtout dans le maréchal de Villars. Mes deux voisins me demandèrent ce
que c'était que ce papier, je le leur dis en leur montrant les
signatures. Chacun regarda ce bizarre papier sans que personne
s'informât d'une chose si reconnaissable, et que la seule facilité du
voisinage me l'avait fait demander par le prince de Conti et le duc de
Guiche, deux hommes qui, chacun fort différemment l'un de l'autre, ne
voyaient guère ce qu'ils voyaient. J'avais balancé cette démonstration
entre la crainte de trop montrer par là que j'étais du secret et le
hasard du bruit que je voyais ces maréchaux si près de faire et du
succès que ce bruit pouvait avoir. Rien n'était plus propre à les
contenir que l'exhibition de leur propre signature. Mais {[}ne{]} la
faire qu'après qu'ils auraient eu parlé, cela n'eût servi qu'à leur
faire honte et point à arrêter ce qu'ils auraient excité. J'allai donc
au plus sûr, et j'eus lieu de juger que j'avais fait utilement. Toute
cette lecture fut écoutée avec la dernière attention jointe à la
dernière émotion. Quand elle fut achevée, M. le duc d'Orléans dit qu'il
était bien fâché de cette nécessité, qu'il s'agissait de ses
beaux-frères, mais qu'il ne devait pas moins justice aux pairs qu'aux
princes du sang\,; puis, se tournant au garde des sceaux, lui ordonna
d'opiner. Celui ci parla peu, dignement, en bons termes, mais comme un
chien qui court sur de la braise, et conclut à l'enregistrement. Après,
Son Altesse Royale, regardant tout le monde, dit qu'il continuerait de
prendre les avis par la tête, et fit opiner M. le Duc. Il fut court,
mais nerveux et poli pour les pairs\,; M. le prince de Conti de même
avis, mais plus brièvement\,; puis M. le duc d'Orléans me demanda mon
avis. Je fis, contre ma coutume, une inclination profonde, mais sans me
lever, et dis qu'ayant l'honneur de me trouver l'ancien des pairs du
conseil, je faisais à Son Altesse Royale mes très humbles remerciements,
les leurs et ceux de tous les pairs de France, de la justice si
ardemment désirée qu'elle prenait la résolution de nous rendre sur ce
qui importait le plus essentiellement à notre dignité et qui touchait le
plus sensiblement nos personnes\,; que je la suppliais de vouloir bien
être persuadée de toute notre reconnaissance et de compter sur tout
l'attachement possible à sa personne pour un acte d'équité si souhaité
et si complet\,; qu'en cette expression sincère de nos sentiments
consisterait toute une opinion, parce qu'étant parties il ne nous était
pas permis d'être juges\,; je terminai ce peu de mots par une
inclination profonde, sans me lever, que le duc de La Force imita seul
en même temps. Je portai aussitôt mon attention à voir à qui le régent
demanderait l'avis, pour interrompre, si c'était à un pair, afin d'ôter
les plus légers prétextes de formes aux bâtards pour en revenir, mais je
ne fus pas en cette peine. M. le duc d'Orléans m'avait bien entendu et
compris, il sauta au maréchal d'Estrées. Lui et tous les autres
opinèrent presque sans parler, en approuvant ce qui ne leur plaisait
guère pour la plupart. J'avais tâché de ménager mon ton de voix de
manière qu'il ne fût que suffisant pour être entendu de tout le monde,
préférant même de ne l'être pas des plus éloignés, à l'inconvénient de
parler trop haut, et je composai toute ma personne au plus de gravité,
de modestie et d'air simple de reconnaissance qu'il me fut possible. M.
le Duc me fit malicieusement signe, en souriant, que j'avais bien dit\,;
mais je gardai mon sérieux et me tournai à examiner tous les autres. On
ne peut rendre les mines ni les contenances des assistants. Ce que j'en
ai raconté, et les impressions qui les occupaient se fortifièrent de
plus en plus. On ne voyait que gens oppressés et dans une surprise qui
les accablait, concentrés, agités, quelques-uns irrités, quelque peu
bien aises, comme La Force, et Guiche qui me le dit aussitôt très
librement.

Les avis pris presque aussitôt que demandés, M. le duc d'Orléans dit\,:
«\, Messieurs, voilà donc qui a passé\,; la justice est faite, et les
droits de MM. les pairs en sûreté. J'ai à présent un acte de grâce à
vous proposer, et je le fais avec d'autant plus de confiance, que j'ai
eu soin de consulter les parties intéressées, qui y veulent bien donner
les mains, et que je l'ai fait dresser en sorte qu'il ne pût blesser
personne. Ce que je vais exposer regarde la seule personne de M. le
comte de Toulouse. Personne n'ignore combien il a désapprouvé tout ce
qui a été fait en leur faveur, et qu'il ne l'a soutenu depuis la régence
que par respect pour la volonté du feu roi. Tout le monde aussi connaît
sa vertu, son mérite, son application, sa probité, son désintéressement.
Cependant je n'ai pu éviter de le comprendre dans la déclaration que
vous venez d'entendre. La justice ne fournit point d'exception en sa
faveur, et il fallait assurer le droit des pairs. Maintenant qu'il ne
peut plus souffrir d'atteinte, j'ai cru pouvoir rendre par grâce au
mérite ce que j'ôte par équité à la naissance, et faire une exception
personnelle de M. le comte de Toulouse, qui, en confirmant la règle, le
laissera lui seul dans tous les honneurs dont il jouit, à l'exclusion de
tous autres, et sans que cela puisse passer à ses enfants s'il se marie
et qu'il en ait, ni être tiré à conséquence pour personne sans
exception. J'ai le plaisir que les princes du sang y consentent, et que
ceux des pairs à qui j'ai pu m'en ouvrir sont entrés dans mes sentiments
et ont bien voulu même m'en prier. Je ne doute point que l'estime qu'il
s'est acquise ici ne vous rende cette proposition agréable\,;» et se
tournant au garde des sceaux\,: «\,Monsieur, continua-t-il, voulez-vous
bien lire la déclaration\,?» lequel, sans rien ajouter, se mit
incontinent à la lire.

J'avais pendant le discours de Son Altesse Royale porté toute mon
attention à examiner l'impression qu'il faisait sur les esprits.
L'étonnement qu'il y causa fut général\,; il fut tel, qu'il semblait, à
voir ceux à qui il s'adressait, qu'ils ne le comprenaient pas, et ils ne
s'en remirent point de toute la lecture. Ceux surtout que la précédente
avait le plus affligés témoignèrent à celle-ci une consternation qui fit
le panégyrique de cette distinction des deux frères, en ce qu'en
affligeant davantage ceux de ce parti, ce premier mouvement involontaire
marquait le parti même, non l'affection des personnes, qui leur eût été
ici un motif de consolation, au lieu que ce leur fut une très vive
irritation de douleur, par l'approfondissement où cette distinction
plongeait le duc du Maine et le privait du secours de son frère, au
moins avec grâce de la part d'un cadet si hautement distingué. Je
triomphai en moi-même d'un succès si évidemment démontré, et je ne reçus
pas trop bien le duc de Guiche, qui me témoigna le désapprouver.
Villeroy confondu, Villars rageant, Effiat roulant les yeux, Estrées
hors de soi de surprise, furent les plus marqués. Tallard, la tête en
avant, suçait pour ainsi dire toutes les paroles du régent à mesure
qu'elles étaient proférées, et toutes celles de la déclaration à mesure
que le garde des sceaux la lisait. Noailles, éperdu en lui-même, ne le
cachait pas même au dehors. Huxelles, tout occupé à se rendre maître de
soi, ne sourcillait pas. Je partageai mon application entre le maintien
de l'assistance et la lecture de la déclaration, et j'eus la
satisfaction de l'entendre parfaitement conforme à celle que le duc de
La Force avait dressée, et avec les deux clauses expresses du
consentement des princes du sang et à la réquisition des pairs, que j'y
fis insérer sous prétexte d'assurer à toujours l'état personnel du comte
de Toulouse, et en effet pour mettre le droit des pairs en sûreté avec
honneur, clauses qui réveillèrent d'une dose de plus les affections de
ceux dont je viens de parler.

La déclaration lue, M. le duc d'Orléans la loua en deux mots, et dit
après au garde des sceaux d'opiner. Il le fit en deux mots, à la louange
du comte de Toulouse. M. le Duc, après quelques louanges du même,
témoigna sa satisfaction par estime et par amitié. M. le prince de Conti
ne dit que deux mots. Après lui, je témoignai à Son Altesse Royale ma
joie de lui voir concilier la justice et la sûreté du droit des pairs
avec la grâce inouïe qu'il faisait à la vertu de M. le comte de
Toulouse, qui la méritait par sa modération, sa vérité, son attachement
au bien de l'État\,; que plus il avait reconnu l'injustice du rang
auquel il avait été élevé, plus il s'en rendait digne, puis il était
avantageux aux pairs de céder le personnel au mérite, lorsque cette
exception était renfermée à sa seule personne, avec les précautions si
formelles et si législatives contenues dans la déclaration, et de
contribuer ainsi du nôtre volontairement à une élévation sans exemple,
d'autant plus flatteuse qu'elle n'avait de fondement que la vertu, pour
exciter cette même vertu de plus en plus au service et à l'utilité de
l'État\,; que j'opinais donc avec joie à l'enregistrement de la
déclaration, et que je ne craignais point d'y ajouter les très humbles
remerciements des pairs, puisque j'avais l'honneur de me trouver
l'ancien de ceux qui étaient présents. En fermant la bouche, je jetai
les yeux vis-à-vis de moi, et je remarquai aisément que mon
applaudissement n'y plaisait pas, et peut-être mon remerciement encore
moins. Ils y opinèrent en baissant la tête à un coup si sensible\,; fort
peu marmottèrent je ne sais quoi entre leurs dents, mais le coup de
foudre sur la cabale fut de plus en plus senti, et à mesure que la
réflexion succéda à la première surprise, à mesure aussi une douleur
aigre et amère se manifesta sur les visages d'une manière si marquée,
qu'il fut aisé de juger qu'il était temps de frapper.

Les opinions finies, M. le Duc me jeta une oeillade brillante, et voulut
parler\,; mais le garde des sceaux qui, à son côté, ne s'en aperçut pas,
voulant aussi dire quelque chose, M. le duc d'Orléans lui dit que M. le
Duc voulait parler, et tout de suite, sans lui en donner le temps, et se
redressant avec majesté sur son siège\,: «\,Messieurs, dit-il, M. le Duc
a une proposition à vous faire\,; je l'ai trouvée juste et
raisonnable\,; je ne doute pas que vous n'en jugiez comme moi.\,» Et se
tournant vers lui\,: «\,Monsieur, lui dit-il, voulez-vous bien
l'expliquer\,?» Le mouvement que ce peu de paroles jeta dans l'assemblée
est inexprimable. Je crus voir des gens poursuivis de toutes parts et
surpris d'un ennemi nouveau qui naît du milieu d'eux dans l'asile où ils
arrivent hors d'haleine\,: «\,Monsieur, dit M. le Duc, en s'adressant au
régent à l'ordinaire, puisque vous faites justice à MM. les ducs, je
crois être en droit de vous la demander pour moi-même le feu roi a donné
l'éducation de Sa Majesté à M. le duc du Maine. J'étais mineur, et dans
l'idée du feu roi M. du Maine était prince du sang, et habile à succéder
à la couronne. Présentement je suis majeur, et non seulement M. du Maine
n'est plus prince du sang, mais il est réduit à son rang de pairie. M.
le maréchal de Villeroy est aujourd'hui son ancien et le précède
partout\,: il ne peut donc plus demeurer gouverneur du roi, sous la
surintendance de M. du Maine. Je vous demande cette place que je ne
crois pas qui puisse être refusée à mon âge, à ma qualité, ni à mon
attachement pour la personne du roi et pour l'État. J'espère,
ajouta-t-il en se tournant vers sa gauche, que je profiterai des leçons
de M. le maréchal de Villeroy pour m'en bien acquitter, et mériter son
amitié.\,»

À ce discours, M. le maréchal de Villeroy fit presque le plongeon, dès
qu'il entendit prononcer le mot de surintendance de l'éducation\,; il
s'appuya le front sur son bâton, et demeura plusieurs moments en cette
posture. Il parut même qu'il n'entendit rien du reste du discours.
Villars, Besons, Effiat ployèrent les épaules comme gens qui ont reçu
les derniers coups\,; je ne pus voir personne de mon côté que le seul
duc de Guiche, qui approuva à travers son étonnement prodigieux. Estrées
revint à soi le premier, se secoua, s'ébroua, regarda la compagnie comme
un homme qui revient de l'autre monde.

Dès que M. le Duc eut fini, M. le duc d'Orléans passa des yeux toute la
compagnie en revue, puis dit que la demande de M. le Duc était juste\,;
qu'il ne croyait pas qu'elle pût être refusée\,; qu'on ne pouvait faire
le tort à M. le maréchal de Villeroy de le laisser sous M. du Maine,
puisqu'il le précédait à cette heure\,; que la surintendance de
l'éducation du roi ne pouvait être plus dignement remplie que de la
personne de M. le Duc, et qu'il était persuadé que cela irait tout d'une
voix, et tout de suite demanda l'avis à M. le prince de Conti, qui opina
en deux mots, après au garde des sceaux, qui ne fut pas plus long,
ensuite à moi. Je dis seulement, en regardant M. le Duc, que j'y opinais
de tout mon coeur. Tous les autres, excepté M. de La Force qui dit un
mot, opinèrent sans parler, en s'inclinant simplement, les maréchaux à
peine, d'Effiat aussi, ses yeux et ceux de Villars étincelant de fureur.

Les opinions prises, le régent, se tournant vers M. le Duc\,:
«\,Monsieur, lui dit-il, je crois que vous voulez lire ce que vous avez
dessein de dire au roi au lit de justice.\,» Là-dessus M. le Duc le lut
tel qu'il est imprimé. Quelques moments de silence morne et profond
succédèrent à cette lecture, pendant lesquels le maréchal de Villeroy,
pâle et agité, marmottait tout seul. Enfin, comme un homme qui prend son
parti, il se tourna vers le régent, la tête basse, les yeux mourants, la
voix faible\,: «\,Je ne dirai que ces deux mots-là, dit-il\,: voilà
toutes les dispositions du roi renversées, je ne le puis voir sans
douleur. M. du Maine est bien malheureux. --- Monsieur, répondit le
régent d'un ton vif et haut, M. du Maine est mon beau-frère, mais j'aime
mieux un ennemi découvert que caché.\,» À ce grand mot plusieurs
baissèrent la tête. Effiat secoua fort la sienne de côté et d'autre. Le
maréchal de Villeroy fut près de s'évanouir, les soupirs commencèrent
vis-à-vis de moi à se faire entendre par-ci, par-là, comme à la
dérobée\,; chacun sentit qu'à ce coup le fourreau était jeté et ne
savait plus s'il y aurait d'enrayure. Le garde des sceaux, pour faire
quelque diversion, proposa de lire le discours qu'il avait préparé pour
servir de préface à l'arrêt de cassation de ceux du parlement et qu'il
prononça au lit de justice avant de proposer l'arrêt. Comme il le
finissait on entra pour lui dire que quelqu'un le demandait à la porte.

Il sortit et revint fort peu après, non à sa place, mais à M. le duc
d'Orléans, qu'il tira dans une fenêtre, et cependant, grand
concentrement de presque tous. Le régent remis en place dit à la
compagnie qu'il recevait avis que toutes les chambres assemblées, le
premier président, nonobstant ce qu'il avait répondu à des Granges,
avait proposé de n'aller point aux Tuileries et demandé ce qu'ils
iraient faire en ce lieu où ils n'auraient point de liberté\,; qu'il
fallait mander au roi que son parlement entendrait sa volonté dans son
lieu de séance ordinaire, quand il lui plairait lui faire cet honneur
que d'y venir ou de la lui envoyer dire\,; que cela avait fait du bruit
et qu'on délibérait actuellement. Le conseil parut fort étourdi de cette
nouvelle, mais Son Altesse Royale dit, d'un air très libre, qu'il
doutait d'un refus et ordonna au garde des sceaux de proposer néanmoins
ce qu'il croyait qu'il y aurait à faire au cas que l'avis du premier
président prévalût.

Le garde des sceaux témoigna qu'il ne pouvait croire que le parlement se
portât à cette désobéissance\,; qu'en ce cas elle serait formelle et
contraire également au droit et à l'usage. Il s'étendit un peu à montrer
que rien n'était si pernicieux que de commettre l'autorité du roi pour
en avoir le démenti, et conclut à l'interdiction du parlement
sur-le-champ s'il tombait dans cette faute. M. le duc d'Orléans ajouta
qu'il n'y avait point à balancer, et prit l'avis de M. le Duc, qui y
opina fortement\,; M. le prince de Conti aussi, moi de même, MM. de La
Force et de Guiche encore plus. Le maréchal de Villeroy, d'une voix
cassée, cherchant de grands mots qui ne venaient pas à temps, déplora
cette extrémité et fit tout ce qu'il put pour éviter de donner une
opinion précise. Forcé enfin par le régent de s'expliquer, il n'osa
contredire, mais il ajouta que c'était à regret, et voulut en étaler les
suites fâcheuses. Mais le régent l'interrompit encore, dit qu'il ne s'en
embarrassait pas\,; qu'il avait prévu à tout\,; qu'il serait bien
fâcheux d'avoir le démenti, et demanda tout de suite l'avis au duc de
Noailles, qui répondit tout court, d'un ton contrit, que cela serait
bien triste, mais qu'il en était d'avis. Villars voulut paraphraser,
mais il se contint, et dit qu'il espérait que le parlement obéirait.
Pressé par le régent, il proposa d'attendre des nouvelles avant qu'on
opinât\,; mais, pressé de plus près, il fut pour l'interdiction avec un
air de chaleur et de dépit extrêmement marqué. Personne après n'osa
branler et la plupart n'opinèrent que de la tête.

L'avis passé, cette nouvelle donna lieu à M. le duc d'Orléans de traiter
la manière de l'interdiction, et les différentes manières de se conduire
selon les divers contre-temps, tel que je l'ai exposé plus haut, excepté
qu'il ne fut parlé de signaux ni d'arrêter personne. Seulement il fut
agité ce que l'on ferait sur une remontrance, si le parlement s'en
avisait. Le garde des sceaux proposa d'aller au roi, puis de prononcer
que le roi voulait être obéi sur-le-champ. Cela fut approuvé.

Peu après, des Granges entra et vint dire à M. le duc d'Orléans que le
parlement était en marche, à pied, et commençait à déboucher le palais.
Cette nouvelle rafraîchit fort le sang à la compagnie, plus encore à M.
le duc d'Orléans qu'à aucun autre.

Des Granges retiré, avec ordre d'avertir quand le parlement
approcherait, M. le duc d'Orléans dit au garde des sceaux que, lorsqu'il
proposerait au lit de justice l'affaire des légitimés, il eût soin de le
faire en sorte qu'on ne fût pas un moment en suspens sur l'état du comte
de Toulouse, parce qu'ayant dessein de le rétablir au même instant, il
ne convenait pas qu'il souffrît la moindre flétrissure. Ce soin si
marqué, et en de tels termes, frappa un nouveau coup sur l'aîné des deux
frères, et j'observai bien que ses partisans en parurent accablés de
nouveau. Le régent fit encore souvenir le garde des sceaux de ne pas
manquer de faire faire les enregistrements au lit de justice, la séance
tenant, et sous ses yeux\,; et l'importance de cette dernière
consommation, en présence du roi, fut très remarquée.

Ensuite le régent dit, d'un air libre, aux présidents des conseils de
rapporter leurs affaires, mais aucun n'ayant été averti d'en apporter,
quoique l'ordre en eût été donné, tous avaient jugé qu'il ne s'agissait
que de la cassation des arrêts du parlement, et pas un n'en avait. Le
maréchal de Villars dit qu'il pouvait en rapporter une, quoiqu'il n'en
eût pas les papiers, et en effet il en rendit un compte le plus juste et
le plus net que je lui eusse encore entendu rendre d'aucune autre, car
cette fonction n'était pas son fort. Je fus infiniment surpris qu'il
s'en acquittât de la sorte dans une agitation d'esprit aussi étrange que
celle où je le voyais, soit que cette agitation même y contribuât, en
réveillant fortement ses idées et sa facilité de parler, soit effort de
réflexion et de prudence, pour paraître plus à soi-même. Il ne fut pas
même trop court\,; mais quoique rapportant très bien, je crois que peu
l'entendirent. On était trop fortement occupé de choses peu
intéressantes, et chacun fut de son avis sans parler. Ce fut un bonheur
pour ceux qui avaient des affaires, de n'être pas rapportés ce
jour-là\,; peu de rapporteurs peut-être eussent su ce qu'ils auraient
dit, et moins encore d'auditeurs.

Le conseil fini de la sorte faute de matière, il se fit un mouvement
pour le lever à l'ordinaire. Je m'avançai par-devant M. le prince de
Conti sur la table à M. le duc d'Orléans qui m'entendit, et qui pria la
compagnie de demeurer en place. La Vrillière, par son ordre, sortit aux
nouvelles, mais rien ne paraissait encore. Il était un peu plus de dix
heures. On resta ainsi une bonne demi-heure en place avec assez de
silence, chacun avec ses voisins, se parlant peu entre soi. Après,
l'inquiétude commença à prendre quelques-uns qui se levèrent pour aller
vers les fenêtres. M. le duc d'Orléans les contint tant qu'il put\,;
mais des Granges étant venu dire que le premier président était déjà
arrivé en carrosse, et que le parlement s'avançait assez près, à peine
fut-il retiré, que le conseil se leva par parties, et qu'il n'y eut plus
moyen de le retenir. M. le duc d'Orléans se leva enfin lui-même, et tout
ce qu'il put fut de défendre tout haut que qui que ce soit sortît sous
quelque prétexte que ce pût être, ce qu'il répéta deux ou trois fois
ensuite en divers temps.

À peine fûmes-nous levés, que M. le Duc vint à moi, joyeux du succès, et
soulagé au dernier point de l'absence des bâtards, et de ce qu'elle
avait permis qu'il eût été parlé de leur affaire à la régence, ce qui
prévenait les inconvénients à craindre au lit de justice. Je lui dis en
peu de mots ce que j'avais remarqué des visages. Je ne voulus pas être
longtemps avec lui. Peu après l'avoir quitté, M. le duc d'Orléans me
vint prendre dans la plénitude des mêmes sentiments. Je lui expliquai
plus qu'à M. le Duc, ce qui m'avait paru dans la mine et la contenance
de chacun, et lui assenai bien celle de son d'Effiat, dont il ne fut
point surpris\,; il le parut davantage de Besons, dont il déplora la
faiblesse et l'abandon pour d'Effiat, qui, dès avant la mort du roi,
était devenu sa boussole. Je demandai au régent s'il ne craignait point
que les bâtards instrumentassent actuellement avec le parlement et leurs
amis, et ne vinssent même au lit de justice. Sa confiance accoutumée,
qui abrégeait soins, réflexions, inquiétudes, ne lui permit pas d'en
avoir le moindre soupçon\,; dans la vérité le duc du Maine m'avait paru
si mort, et ses amis du conseil si déconcertés, que je n'en craignis
rien moi-même\,; mais, de peur de surprise, j'y voulus préparer et
fortifier le régent.

Je le quittai après, et vis les maréchaux de Villeroy et de Villars
assis auprès d'Effiat, se parlant moins que réfléchissant ensemble en
gens pris au dépourvu, enragés, mais abattus. Besons et le maréchal
d'Estrées après s'y joignirent, puis ils se séparèrent, et se
rapprochèrent, en sorte que les deux, trois, ou les quatre ensemble, ne
furent presque point mêlés avec d'autres. Tallard les joignit, non
ensemble, mais quelques-uns d'eux par-ci, par-là, courtement et à la
dérobée\,; Huxelles aussi, et Le pelletier\,; le garde des sceaux, assez
seul, méditant son affaire, souvent avec M. le duc d'Orléans et M. le
Duc, quelquefois avec moi, souvent avec La Vrillière, quand il joignait
quelqu'un. Je me promenais cependant lentement et incessamment sans
m'attacher à personne, pour essayer que rien ne m'échappât, avec une
attention principale aux portes. Je me servis de ce long toupillage pour
parler aux uns et aux autres, passer continuellement auprès des
suspects, pour écumer et interrompre leurs conciliabules, d'Antin, fort
seul, souvent joint par le duc de Noailles. Celui-ci avait repris sa
façon du matin, de me suivre toujours des yeux. Il avait l'air
consterné, agité, et une contenance fort embarrassée, lui ordinairement
si libre et si maître du tripot. D'Antin me prit à part pour me
témoigner son embarras d'assister au lit de justice, par rapport aux
bâtards, et me consulter s'il hasarderait de demander au régent de l'en
dispenser. Sa situation à cet égard me fit juger que cela pouvait se
faire. Il me pria de m'en charger\,; je ne pus le faire sitôt, parce que
le colloque d'Effiat et des siens me parut se forlonger, et que je m'en
allai vers eux. Je m'y assis même un peu. D'Effiat, d'abordée, ne put
s'empêcher de me dire que nous venions d'entendre d'étranges
résolutions\,; qu'il ne savait qui les avait conseillées\,; qu'il priait
Dieu que M. le duc d'Orléans s'en trouvât bien. Je lui répondis que ces
résolutions-là étaient assurément fortes et bien grandes\,; que cela
même me faisait juger qu'il fallait que les raisons qui y avaient
déterminé le fussent également\,; que j'en étais dans la même surprise
et dans les mêmes souhaits. Le maréchal de Villeroy poussa des soupirs
profonds, et fit quelques exclamations vides et muettes, qu'il soutint
de secouements de perruque. Villars parla un peu plus, blâma aigrement,
mais courtement, laissa voir son désespoir sur le duc du Maine\,; mais
il débiaisa sur le parlement, pour moins montrer sa vraie douleur. Je
payai de mines et de gestes, je ne contredis rien, mais je ne dis rien
aussi, parce que je ne m'étais pas mis là pour parler ni persuader, mais
pour voir et entendre. De tout ce que j'ouïs d'eux, je recueillis que
c'était gens en désarroi, de cabale non préparée, qui n'espéraient rien
du parlement, aussi peu préparé qu'eux.

Je les quittai pour ne rien affecter, et fis la commission de d'Antin\,;
le régent me dit qu'il lui avait parlé\,; qu'il approuvait son embarras
et sa délicatesse\,; qu'il lui avait permis de ne venir point au lit de
justice, à condition qu'il ne le dirait à personne\,: qu'il demeurerait
dans le cabinet du conseil, comme devant y aller, et que, pendant le lit
de justice, il ne sortirait point du même cabinet qu'après que toute la
séance serait finie. J'allai après à d'Antin, qui me le redit, et qui
l'exécuta très bien. En effet, le fils légitime de M\textsuperscript{me}
de Montespan, mêlé de société au point où il l'était avec tous les
bâtards et bâtardes de sa mère, ne pouvait honnêtement se trouver à ce
lit de justice.

Après je pris Tallard sur l'inquiétude où je ne laissais pas d'être des
soupirs, des exclamations et du désespoir évident du maréchal de
Villeroy, {[}de{]} ce mot qu'il avait dit des dispositions du roi
renversées et du malheur de M. du Maine, en plein conseil et si hors de
temps. Je joignais à cela la peur terrible que nous lui savions d'être
arrêté. Tout cela me fit craindre qu'il n'en regardât comme
l'avant-coureur la chute du duc du Maine, et que son peu d'esprit et de
sens ne lui persuadât qu'il serait beau d'amplifier au lit de justice le
pathos qu'il avait suffoqué au conseil pour se faire un mérite au
parlement et auprès de leur cabale, et un de reconnaissance auprès du
public, qui le rendrait peut-être plus difficile à arrêter, au moins
plus considérable. Or, un pathos d'un homme dans ces places, au milieu
d'un parlement enragé, était meilleur à empêcher qu'à hasarder de le
laisser faire. Je dis donc à Tallard que, ne pouvant parler là longtemps
au maréchal de Villeroy, je le priais de le joindre quand il le
pourrait, et de lui dire de ma part que je ne pouvais m'empêcher de me
moquer beaucoup de lui de l'inquiétude qu'il avait témoignée d'être
arrêté, ce que je paraphrasai de tout ce qui pouvait flatter sa vanité
personnelle, sans rien dire qui la pût exciter à autre titre, ni
conséquemment lui donner du courage, mais seulement de la confiance en
l'estime et l'amitié du régent. J'ajoutai que Son Altesse Royale, en me
le racontant, m'avait parlé de lui d'une manière à lui devoir donner de
la honte de ses soupçons, et que, quand je pourrais l'entretenir, je ne
m'empêcherais pas de la lui faire tout entière. En effet, il n'y avait
ni sens ni raison à l'arrêter, et par n'en valoir pas la peine, et par
les tristes qu'en dira-t-on du monde d'ôter tous les deux hommes
distingués à la fois, mis auprès du roi par le roi, son bisaïeul
mourant. Je crus donc qu'il n'était que bon de rassurer celui-ci, et par
là de lui ôter l'envie de dire quelque sottise au lit de justice, par
lui faire sentir qu'il n'en avait pas besoin pour rendre sa capture plus
difficile, et que cette sottise le gâterait tout à fait, puisqu'il avait
à perdre dans l'estime et la confiance du régent. Tallard ne me nia
point les inquiétudes de son cousin, et glissa sur tout en homme de
beaucoup d'esprit, sans me montrer que lui-même crût les inquiétudes
fondées ou non. Il me remercia néanmoins beaucoup de mon attention
pleine d'amitié, qui lui faisait grand plaisir, et qui en ferait
beaucoup au maréchal de Villeroy dès qu'il pourrait la lui apprendre. Il
ne tarda pas à le faire, car dès la première fois que je le revis après,
il me dit, que le maréchal de Tallard lui avait parlé, et me remercia
diffusément\,; mais ce qu'il me conta lors n'est pas du sujet présent.

À peine eus-je fait avec Tallard, que La Vrillière, qui me guettait
depuis quelques moments, me prit à part. Il s'était aperçu sans doute de
ma liaison nouvelle avec M. le Duc, qui n'avait que trop paru avant et
depuis le conseil fini, outre la visite qu'il lui avait faite la veille,
sur la réduction des bâtards au rang de leurs pairies. La Vrillière donc
me pria de témoigner à M. le Duc sa satisfaction et sa joie, et de
l'assurer de son attachement, parce qu'il n'osait aller lui parler
devant le monde. Jamais compliment ne fut plus de courtisan. La
Vrillière était tout feu roi, conséquemment tout bâtard, lié avec eux
par la Maintenon, leur ébreneuse\footnote{Expression triviale pour
  désigner une nourrice.}, qui, pour le dire en passant, tomba bien
malade et pleura bien plus longtemps et plus amèrement cette déconfiture
de son bel ouvrage, qu'elle n'avait fait la mort du feu roi dont sa
santé ne fut pas même altérée. La Vrillière avait eu des prises avec M.
le Duc sur la Bourgogne, où il avait eu les ongles rognés, de manière
qu'il avait besoin de se raccommoder avec un prince à qui il voyait
prendre un commencement de grand vol.~Je m'en acquittai volontiers.

Cependant, on s'ennuyait fort de la lenteur du parlement, et on envoyait
souvent aux nouvelles. Plusieurs, tentés de sortir, peut-être de jaser,
se proposèrent\,; mais le régent ne voulut laisser sortir que La
Vrillière, et voyant que le désir de sortir croissait, il se mit
lui-même à la porte. J'eus avec lui plusieurs entretiens sur les
remarques des divers personnages, avec M. le Duc, avec le garde des
sceaux. Je fis réitérer plusieurs fois au régent la défense de sortir.
Dans un de ces courts entretiens à l'écart, je lui parlai de la douleur
qu'aurait M\textsuperscript{me} la duchesse d'Orléans\,; combien il y
devait compatir, et la laisser libre, et qu'il ne devait avoir rien de
plus pressé que de lui écrire une lettre pleine de tendresse. Je lui
proposai même de l'écrire sur la table du conseil, tandis qu'il n'avait
rien à faire, mais il me dit qu'il n'y avait pas moyen parmi tout ce
monde. Il fut assez aisément disposé à compatir à sa peine, mais il m'en
parut assez peu touché\,; néanmoins, il me promit de lui écrire dans la
journée, au premier moment de liberté qu'il aurait. J'étais inquiet de
ce que faisaient les bâtards, mais je n'osais trop le lui marquer. Il
parlait aux uns et aux autres d'un air libre, comme dans une journée
ordinaire, et il faut dire qu'il fut le seul de tous qui conserva cette
sérénité sans l'affecter.

\hypertarget{chapitre-xxi.}{%
\chapter{CHAPITRE XXI.}\label{chapitre-xxi.}}

1718

~

{\textsc{Le parlement arrive aux Tuileries.}} {\textsc{- Attention sur
les sorties du cabinet du conseil et sur ce qui s'y passe.}} {\textsc{-
On va prendre le roi.}} {\textsc{- Marche au lit de justice.}}
{\textsc{- Le roi sans manteau ni rabat.}} {\textsc{- Séance et pièce du
lit de justice aux Tuileries dessinée, pour mieux éclaircir ce qui s'y
passa le vendredi matin 26 août 1718.}} {\textsc{- J'entre au lit de
justice, et, allant en place, je confie l'affaire des bâtards à quelques
pairs.}} {\textsc{- Spectacle du lit de justice.}} {\textsc{- Maintien
de M. le duc d'Orléans, de M. le Duc et de M. le prince de Conti.}}
{\textsc{- Maintien du roi et du garde des sceaux.}} {\textsc{- Lettres
de garde des sceaux.}} {\textsc{- Discours du garde des sceaux au
parlement sur sa conduite et ses devoirs.}} {\textsc{- Cassation de ses
arrêts.}} {\textsc{- Présence d'esprit et capacité d'esprit de
Blancmesnil, premier avocat général.}} {\textsc{- Remontrance envenimée
du premier président, confondue.}} {\textsc{- Réduction des bâtards au
rang de leurs pairies.}} {\textsc{- Rétablissement uniquement personnel
du comte de Toulouse.}} {\textsc{- M. de Metz et quelques autres pairs
mécontents sur le rétablissement du comte de Toulouse.}} {\textsc{- Je
refuse d'une façon très marquée d'opiner, tant moi que tous les pairs,
comme étant parties, dans l'affaire des bâtards.}} {\textsc{- Discours
du régent et de M. le Duc pour demander l'éducation du roi.}} {\textsc{-
Lourde faute d'attention de ces deux princes en parlant.}} {\textsc{- M.
le Duc obtient sa demande.}} {\textsc{- Enregistrement en plein lit de
justice de tout.}} {\textsc{- Le roi très indifférent pour le duc du
Maine.}} {\textsc{- Levée du lit de justice.}}

~

Enfin le parlement arriva, et, comme des enfants, nous voilà tous deux
aux fenêtres. Il venait en robes rouges, deux à deux, par la grande
porte de la cour qu'il croisa pour aller gagner la salle des
Ambassadeurs, où le premier président, venu en carrosse avec le
président d'Aligre, les attendait. Il avait traversé de la petite cour
d'auprès, pour avoir moins de chemin à faire à pied. Tandis que nos deux
fenêtres s'entassaient de spectateurs, j'eus soin de ne pas perdre de
vue le dedans du cabinet, à cause des conférences et de peur des
sorties. Des Granges vint à diverses fois dire à quoi les choses en
étaient, sans qu'il y eût de difficultés, moi toujours me promenant et
considérant tout avec attention. Soit besoin, soit désir du défendu,
quelques-uns demandèrent l'un après l'autre à sortir pour des besoins.
Le régent le permit à condition du silence et du retour sur-le-champ. Il
proposa même à La Vrillière de s'aller précautionner en même temps que
le maréchal d'Huxelles et quelques autres suspects\,; mais en effet pour
ne les perdre pas de vue, et il l'entendit et l'exécuta très bien. J'en
usai de même avec les maréchaux de Villars et Tallard, et, ayant vu
Effiat ouvrant la petite porte du roi pour le maréchal de Villeroy, j'y
courus, sous prétexte de lui aider, mais au vrai pour empêcher qu'il ne
parlât à la porte et qu'il n'envoyât quelques messages aux bâtards. J'y
restai même avec Effiat jusqu'à ce que le maréchal de Villeroy fût
rentré, pour éviter le même inconvénient à cette autre ouverture de la
porte, que je refermai bien après\,; et il faut avouer que cette
occupation de tête et de corps, d'examen et d'attention continuelle à
interrompre, à prévenir, à être en garde sur toute une vaste pièce et un
nombre de gens qu'on veut contenir et déranger sans qu'il y paroisse, ne
fut pas un petit soin ni une petite fatigue. M. le duc d'Orléans, M. le
Duc et La Vrillière en portaient leur part, qui ne diminuait guère la
mienne.

Enfin le parlement en place, les pairs arrivés, et les présidents ayant
été en deux fois prendre leurs fourrures derrière des paravents disposés
dans la pièce voisine, des Granges vint avertir que tout était prêt. Il
avait été agité si le roi dînerait en attendant, et j'avais obtenu que
non, dans la crainte qu'entrant aussitôt après au lit de justice, et
ayant mangé avant son heure ordinaire, il ne se trouvât mal, qui eût été
un grand inconvénient. Dès que des Granges eut annoncé au régent qu'il
pouvait se mettre en marche, Son Altesse Royale lui dit de faire avertir
le parlement, pour la députation à recevoir le roi, au lieu du bout de
la pièce des Suisses, où elle avait été réglée, et dit tout haut à la
compagnie qu'il fallait aller prendre le roi.

À ces paroles, je sentis un trouble de joie du grand spectacle qui
s'allait passer en ma présence, qui m'avertit de redoubler mon attention
sur moi. J'avais averti Villars de marcher avec nous, et Tallard de se
joindre aux maréchaux de France, et de céder à ses anciens, parce qu'en
ces occasions les ducs vérifiés n'existent pas. Je tâchai de me munir de
la plus forte dose que je pus de sérieux, de gravité, de modestie. Je
suivis M. le duc d'Orléans, qui entra chez le roi par la petite porte,
et qui trouva le roi dans son cabinet. Chemin faisant, le duc d'Albret
et quelques autres me firent des compliments très marqués, avec grand
désir de découvrir quelque chose. Je payai de politesse, de plaintes de
la foule, de l'embarras de mon habit, et je gagnai le cabinet du roi.

Il était sans manteau ni rabat, vêtu à son ordinaire. Après que M. le
duc d'Orléans eut été quelques moments auprès de lui, il lui demanda
s'il lui plaisait d'aller\,: aussitôt on fit faire place. Le peu de
courtisans revenus là, faute d'avoir trouvé où se fourrer dans le lieu
de la séance, s'écarta, et je fis signe au maréchal de Villars, qui prit
lentement le chemin de la porte, le duc de La Force derrière lui, et moi
après, qui observai bien de marcher immédiatement avant M. le prince de
Conti. Monsieur le duc le suivait, et M. le duc d'Orléans après.
Derrière lui les huissiers de la chambre du roi avec leurs masses, puis
le roi environné des quatre capitaines des gardes du corps, du duc
d'Albret grand chambellan, et du maréchal de Villeroy son gouverneur.
Derrière, venait le garde des sceaux, parce qu'il n'était pas enregistré
au parlement, puis les maréchaux d'Estrées, Huxelles, Tallard et Besons,
qui ne pouvaient entrer en séance qu'à la suite, et non devant Sa
Majesté. Ils étaient suivis de ceux des chevaliers de l'ordre et des
gouverneurs et lieutenants généraux des provinces qu'on avait avertis
pour le cortège du roi, qui devaient seoir en bas, découverts et sans
voix, sur le banc des baillis. On prit en cet ordre le chemin de la
terrasse jusqu'à la salle des Suisses, au bas de laquelle se trouva la
députation du parlement, de quatre présidents à mortier et de quatre
conseillers à l'accoutumée.

Tandis qu'ils s'approchèrent du roi, je dis au duc de La Force et au
maréchal de Villars que nous ferions mieux d'aller toujours nous mettre
en place, pour éviter l'embarras de l'entrée avec le roi. Ils me
suivirent alors un à un en rang d'ancienneté, marchant en cérémonie. Il
n'y avait que nous trois à pouvoir marcher comme nous fîmes, parce que
d'Antin n'y venait pas\,; le duc de Guiche était démis, Tallard point
pair, et les quatre capitaines des gardes étaient autour du roi avec le
bâton en ces grandes cérémonies. Mais avant d'en dire davantage, je
crois à propos de donner le dessin figuré du lit de justice dont la
disposition éclaircira d'un coup d'oeil ce qui va être raconté.

EXPLICATION.

A. Le roi sur son trône.

B. Marches du trône avec son tapis et ses carreaux.

C. Le grand chambellan couché sur ces carreaux, sur les marches, couvert
et opinant.

D. Hauts sièges à droite et à gauche.

E. Petit degré du roi couvert de la queue de son tapis de pied sans
carreaux.

F. Le prévôt de Paris avec son bâton, couché sur ces degrés.

G. Les huissiers de la chambre du roi à genoux, leurs masses de vermeil
sur le col.

H. Le garde des sceaux dans sa chaire à bras sans dos.

J. Un petit bureau devant lui.

K. Marches pour monter aux hauts sièges.

L. Porte d'entrée ordinaire, mais condamnée ce jour-là, par laquelle MM.
de Troyes et de Fréjus et M. de Torcy virent la séance debout et
reculés. Devant eux, un peu à côté en dedans, M. d'Harcourt debout et
découvert, avec le bâton de capitaine des gardes sans opiner.

M. Fenêtres à gradins pour les spectateurs\,; les duchesses de Ventadour
et de La Ferté, les sous-gouverneurs du roi, le premier gentilhomme de
la chambre et le capitaine des gardes du régent étaient dans celle de
derrière lui.

N. Le maréchal de Villeroy sur un tabouret, comme gouverneur du roi,
couvert et opinant.

O. Le duc de Villeroy, capitaine des gardes, assis, en quartier, couvert
et opinant.

P. Beringhen, premier écuyer, tenant la place du grand écuyer, assis,
mais découvert, sans opiner.

Ces deux places à cause de l'âge du roi, ainsi que celle de son
gouverneur.

Q. Les hérauts d'armes en cotte, etc.

R. Le grand maître ou le maître des cérémonies, assis mais découvert,
sans opiner.

S. Entrée des hauts sièges à gauche pour les évêques-pairs et les
officiers de la couronne.

T. Parquet ou espace vide au milieu de la séance.

V. Passage de plain-pied aux sièges hauts qui les communique des deux
côtés.

Y. Banc redoublé dans les sièges en cas de besoin pour les pairs
laïques.

Z. Greffier en chef du parlement enregistrant les déclarations à la fin.

Je pense qu'il serait inutile d'entrer dans une explication plus
détaillée de la séance, et que celle-ci suffit, tant pour la faire
entendre que pour éclaircir par le local ce qui va être raconté. J'ai
seulement observé d'y nommer les pairs par le nom de leurs pairies,
comme il se pratique en prenant leurs voix, et non par celui qu'ils
portent d'ordinaire, et sous lequel ils sont connus dans le monde. M. de
Laon était Clermont-Chattes, et M. de Noyon Châteauneuf-Rochebonne, mort
depuis archevêque de Lyon avec brevet de conservation de rangs et
d'honneurs. Il n'y eut sur le banc redoublé des pairs laïques que les
ducs de La Feuillade et de Valentinois qui s'y mirent après que le roi
fut arrivé.

Comme le parlement était en place et que le roi allait arriver, j'entrai
par la même porte. Le passage se trouva assez libre, les officiers des
gardes du corps me firent faire place, et au duc de La Force, et au
maréchal de Villars, qui me suivaient un à un. Je m'arrêtai un moment en
ce passage, à l'entrée du parquet, saisi de joie de voir ce grand
spectacle, et les moments si précieux s'approcher. J'en eus besoin
aussi, afin de me remettre assez pour voir distinctement ce que je
considérais, et pour reprendre une nouvelle couche de sérieux et de
modestie. Je m'attendais bien que je serais attentivement examiné par
une compagnie dont on avait pris soin de ne me pas faire aimer, et par
le spectateur curieux, dans l'attente de ce qui allait éclore d'un
secret si profond, dans une si importante assemblée, mandée si fort à
l'instant. De plus, personne n'y pouvait ignorer que je n'en fusse
instruit, du moins par le conseil de régence dont je sortais.

Je ne me trompai pas\,: sitôt que je parus, tous les yeux s'arrêtèrent
sur moi. J'avançai lentement vers le greffier en chef, et reployant
entre les deux bancs, je traversai la largeur de la salle par-devant les
gens du roi qui me saluèrent d'un air riant, et je montai nos trois
marches des sièges hauts où tous les pairs, que je marque, étaient en
place, qui se levèrent, dès que j'approchai du degré\,; je les saluai
avec respect du haut de la troisième marche. En m'avançant lentement, je
pris La Feuillade par l'épaule, quoique sans liaison avec lui, et lui
dis à l'oreille de me bien écouter et de prendre garde à ne pas donner
signe de vie\,; qu'il allait entendre une déclaration à l'égard du
parlement, après laquelle il y en aurait deux autres\,; qu'enfin nous
touchions aux plus heureux moments et les plus inespérés\,; que les
bâtards étaient réduits au simple rang d'ancienneté de leurs pairies, le
comte de Toulouse seul rétabli sans conséquence, pas même pour ses
enfants. La Feuillade fut un instant sans comprendre, et saisi de joie à
ne pouvoir parler. Il se serra contre moi, et comme je le quittais, il
me dit\,: «\,Mais comment, le comte de Toulouse\,? --- Vous le
verrez,\,» lui répondis-je, et passai, mais en passant devant le duc
d'Aumont, je me souvins de ce beau rendez-vous qu'il avait pour
l'après-dînée ou le lendemain, avec M. le duc d'Orléans, pour le
raccommoder avec le parlement, et finir galamment tous ces malentendus,
et je ne pus m'empêcher, en le bien regardant, de lui lâcher un sourire
moqueur. Je m'arrêtai entre M. de Metz, duc de Coislin, et le duc de
Tresmes, à qui j'en dis autant. Le premier renifla, l'autre fut ravi et
me le fit répéter d'aise et de surprise. J'en dis autant au duc de
Louvigny, qui n'en fut pas si étonné que les autres, mais au moins aussi
transporté. Enfin, j'arrivai à ma place entre les ducs de Sully et de La
Rochefoucauld. Je les saluai, et nous nous assîmes tout de suite\,; je
donnai un coup d'oeil au spectacle, et tout aussitôt je fis approcher
les têtes de mes deux voisins de la mienne, à qui j'annonçai la même
chose. Sully y fut sensible au dernier point\,; l'autre me demanda
sèchement pourquoi l'exception du comte de Toulouse. J'avais plusieurs
raisons de réserve avec lui, et bien que depuis l'arrêt de préséance que
j'avais obtenu sur lui, il en eût parfaitement usé à cet égard, je
sentais bien que cette préséance lui faisait mal au coeur. Je me
contentai donc de lui répondre que je n'en savais rien, et sur le fait,
ce que je pus pour le lui faire goûter. Mais, s'il trouvait ma préséance
indigeste, il pardonnait beaucoup moins au comte de Toulouse d'avoir eu
sa charge de grand veneur. Son froid fut tel, que je ne pus m'empêcher
de lui en demander la cause, et de le faire souvenir de toute l'ardeur
qu'il avait témoignée sur cette même affaire dans nos premières
assemblées chez M. de Luxembourg, au temps qu'il avait la goutte, et
dans les autres dont notre requête contre les bâtards était sortie et
dont il allait, au delà de nos espérances, voir enregistrer les
conclusions. Il répondit ce qu'il put, toujours sec et morne\,; je ne
pris plus la peine de lui parler.

Assis en place dans un lieu élevé, personne devant moi aux hauts des
sièges, parce que le banc redoublé pour les pairs, qui n'auraient pas eu
place sur le nôtre, n'avançait pas jusqu'au duc de La Force, j'eus moyen
de bien considérer tous les assistants. Je le fis aussi de toute
l'étendue et de tout le perçant de mes yeux. Une seule chose me
contraignit, ce fut de n'oser me fixer à mon gré sur certains objets
particuliers\,; je craignais le feu et le brillant significatif de mes
regards si goûtés\,; et plus je m'apercevais que je rencontrais ceux de
presque tout le monde sous les miens, plus j'étais averti de sevrer leur
curiosité par ma retenue. J'assenai néanmoins une prunelle étincelante
sur le premier président et le grand banc, à l'égard duquel j'étais
placé à souhait. Je la promenai sur tout le parlement\,; j'y vis un
étonnement, un silence, une consternation auxquels je ne me serais pas
attendu, qui me fut de bon augure. Le premier président, insolemment
abattu, les présidents déconcertés, attentifs à tout considérer, me
fournissaient le spectacle le plus agréable. Les simples curieux, parmi
lesquels je range tout ce qui n'opine point, ne paraissaient pas moins
surpris, mais sans l'égarement des autres, et d'une surprise calme\,; en
un mot, tout sentait une grande attente, et cherchait à l'avancer en
devinant ceux qui sortaient du conseil.

Je n'eus guère de loisir en cet examen, incontinent le roi arriva. Le
brouhaha de cette entrée dans la séance, qui dura jusqu'à ce que Sa
Majesté, et tout ce qui l'accompagnait, fût en place, devint une autre
espèce de singularité. Chacun cherchait à pénétrer le régent, le garde
des sceaux et les principaux personnages. La sortie des bâtards du
cabinet du conseil avait redoublé l'attention, mais tous ne la savaient
pas, et tous alors s'aperçurent de leur absence. La consternation des
maréchaux, de leur doyen sur tous dans sa place de gouverneur du roi,
fut évidente. Elle augmenta l'abattement du premier président, qui, ne
voyant point là son maître, le duc du Maine, jeta un regard affreux sur
M. de Sully et sur moi, qui occupions les places des deux frères
précisément. En un instant tous les yeux de l'assemblée se posèrent tout
à la fois sur nous, et je remarquai que le concentrement et l'air
d'attente de quelque chose de grand redoubla sur tous les visages. Celui
du régent avait un air de majesté douce, mais résolue, qui lui fut tout
nouveau, des yeux attentifs, un maintien grave mais aisé\,; M. le Duc,
sage, mesuré, mais environné de je ne sais quel brillant qui ornait
toute sa personne et qu'on sentait retenu. M. le prince de Conti triste,
pensif, voyageant peut-être en des espaces éloignés. Je ne pus guère,
pendant la séance, les voir qu'à reprises et sous prétexte de regarder
le roi, qui était sérieux, majestueux, et en même temps le plus joli
qu'il fût possible, grave avec grâce dans tout son maintien, l'air
attentif et point du tout ennuyé, représentant très bien et sans aucun
embarras.

Quand tout fut posé et rassis, le garde des sceaux demeura quelques
minutes dans sa chaire, immobile, regardant en dessous, et ce feu
d'esprit qui lui sortait des yeux semblait percer toutes les poitrines.
Un silence extrême annonçait éloquemment la crainte, l'attention, le
trouble, la curiosité de toutes les diverses attentes. Ce parlement, qui
sous le feu roi même avait souvent mandé ce même d'Argenson et lui
avait, comme lieutenant de police, donné ses ordres debout et découvert
à la barre\,; ce parlement, qui depuis la régence avait déployé sa
mauvaise volonté contre lui, jusqu'à donner tout à penser, et qui
retenait encore des prisonniers et des papiers pour lui donner de
l'inquiétude\,; ce premier président, si supérieur à lui, si
orgueilleux, si fier de son duc du Maine, si fort en espérance des
sceaux\,; ce Lamoignon qui s'était vanté de le faire pendre à sa chambre
de justice, où lui-même s'était si complètement déshonoré, ils le virent
revêtu des ornements de la première place de la robe, les présider, les
effacer, et entrant en fonction, les remettre en leur devoir et leur en
faire leçon publique et forte, dès la première fois qu'il se trouvait à
leur tête. On voyait ces vains présidents détourner leurs regards de
dessus cet homme qui imposait si fort à leur morgue, et qui anéantissait
leur arrogance dans le lieu même d'où ils la tiraient, et rendus
stupides par les siens qu'ils ne pouvaient soutenir.

Après que le garde des sceaux se fut, à la manière des prédicateurs,
accoutumé à cet auguste auditoire, il se découvrit, se leva, monta au
roi, se mit à genoux sur les marches du trône, à côté du milieu des
mêmes marches où le grand chambellan était couché sur des oreillers, et
prit l'ordre du roi, descendit, se mit dans sa chaire et se couvrit. Il
faut dire une fois pour toutes qu'il fit la même cérémonie à chaque
commencement d'affaire, et pareillement avant de prendre les opinions
sur chacune et après\,; qu'au lit de justice lui ou le chancelier ne
parlent jamais au roi autrement, et qu'à chaque fois qu'il alla au roi
en celui-ci, le régent se leva et s'en approcha pour l'entendre et
suggérer les ordres. Remis en place après quelques moments de silence,
il ouvrit cette grande scène par un discours. Le procès-verbal de ce lit
de justice, fait par le parlement et imprimé\footnote{Le recueil des
  \emph{Anciennes lois françaises} (t. XXI, p.~159 et suiv.) contient
  les différents édits qui furent enregistrés dans ce lit de justice. On
  peut aussi comparer le \emph{Journal de l'avocat Barbier}, à la date
  du mois d'août 1718.}, qui est entre les mains de tout le monde, me
dispensera de rapporter ici les discours du garde des sceaux, celui du
premier président, ceux des gens du roi et les différentes pièces qui y
furent lues et enregistrées. Je me contenterai seulement de quelques
observations. Ce premier discours, la lecture des lettres de garde des
sceaux et le discours de l'avocat général Blancmesnil qui la suivit, les
opinions prises, le prononcé par le garde des sceaux, l'ordre donné,
quelquefois réitéré, d'ouvrir, puis de tenir ouvertes les deux doubles
portes, ne surprirent personne, ne servirent que comme de préface à tout
le reste, à en aiguiser la curiosité de plus en plus, à mesure que les
moments approchaient de la satisfaire.

Ce premier acte fini, le second fut annoncé par le discours du garde des
sceaux, dont la force pénétra tout le parlement. Une consternation
générale se répandit sur tous leurs visages. Presque aucun de tant de
membres n'osa parler à son voisin. Je remarquai seulement que l'abbé
Pucelle, qui, bien que conseiller-clerc, était dans les bancs vis-à-vis
de moi, fut toujours debout toutes les fois que le garde des sceaux
parla, pour mieux entendre. Une douleur amère et qu'on voyait pleine de
dépit, obscurcit le visage du premier président. La honte et la
confusion s'y peignit. Ce que le jargon du palais appelle le grand banc
pour encenser les mortiers qui l'occupent, baissa la tête à la fois
comme par un signal, et bien que le garde des sceaux ménageât le ton de
sa voix, pour ne la rendre qu'intelligible, il le fit pourtant en telle
sorte qu'on ne perdit dans toute l'assemblée aucune de ses paroles, dont
aussi n'y en eut-il aucune qui ne portât. Ce fut bien pis à la lecture
de la déclaration. Chaque période semblait redoubler à la fois
l'attention et la désolation de tous les officiers du parlement, et ces
magistrats si altiers, dont les remontrances superbes ne satisfaisaient
pas encore l'orgueil et l'ambition, frappés d'un châtiment si fort et si
public, se virent ramenés au vrai de leur état avec cette ignominie,
sans être plaints que de leur petite cabale. D'exprimer ce qu'un seul
coup d'oeil rendit dans ces moments si curieux, c'est ce qu'il est
impossible de faire, et, si j'eus la satisfaction que rien ne m'échappa,
j'ai la douleur de ne le pouvoir rendre. La présence d'esprit de
Blancmesnil me surprit au dernier point. Il parla sur chaque chose où
son ministère le requit, avec une contenance modeste et sagement
embarrassée, sans être moins maître de son discours, aussi délicatement
ménagé que s'il eût été préparé\footnote{Voyez note II.}.

Après les opinions, comme le garde des sceaux eut prononcé, je vis ce
prétendu grand banc s'émouvoir. C'était le premier président qui voulait
parler et faire la remontrance qui a paru pleine de la malice la plus
raffinée, d'impudence à l'égard du régent et d'insolence pour le roi. Le
scélérat tremblait toutefois en la prononçant. Sa voix entrecoupée, la
contrainte de ses yeux, le saisissement et le trouble visible de toute
sa personne, démentaient ce reste de venin dont il ne put refuser la
libation à lui-même et à sa compagnie. Ce fut là où je savourai avec
toutes les délices qu'on ne peut exprimer, le spectacle de ces fiers
légistes, qui osent nous refuser le salut, prosternés à genoux, et
rendre à nos pieds un hommage au trône, tandis qu'assis et couverts, sur
les hauts sièges aux côtés du même trône, ces situations et ces
postures, si grandement disproportionnées, plaident seules avec tout le
perçant de l'évidence la cause de ceux qui, véritablement et d'effet,
sont \emph{laterales regis} contre ce \emph{vas electum} du tiers état.
Mes yeux fichés, collés sur ces bourgeois superbes, parcouraient tout ce
grand banc à genoux ou debout, et les amples replis de ces fourrures
ondoyantes à chaque génuflexion longue et redoublée, qui ne finissait
que par le commandement du roi par la bouche du garde des sceaux, vil
petit gris qui voudrait contrefaire l'hermine en peinture, et ces têtes
découvertes et humiliées à la hauteur de nos pieds. La remontrance
finie, le garde des sceaux monta au roi, puis, sans prendre aucuns avis,
se remit en place, jeta les yeux sur le premier président, et
prononça\,: «\,Le roi veut être obéi, et obéi sur-le-champ.\,» Ce grand
mot fut un coup de foudre qui atterra présidents et conseillers de la
façon la plus marquée. Tous baissèrent la tête, et la plupart furent
longtemps sans la relever. Le reste des spectateurs, excepté les
maréchaux de France, parurent peu sensibles à cette désolation.

Mais ce ne fut rien que ce triomphe ordinaire en comparaison de celui
qui l'allait suivre immédiatement. Le garde des sceaux ayant, par ce
dernier prononcé, terminé ce second acte, il passa au troisième.
Lorsqu'il repassa devant moi, venant d'achever de prendre l'avis des
pairs sur l'arrêt concernant le parlement, je l'avais averti de ne
prendre point leur avis sur l'affaire qui allait suivre, et il m'avait
répondu qu'il ne le prendrait pas. C'était une précaution que j'avais
prise contre la distraction à cet égard. Après quelques moments
d'intervalle depuis la dernière prononciation sur le parlement, le garde
des sceaux remonta au roi, et, remis en place, y demeura encore quelques
instants en silence. Alors tout le monde vit bien que l'affaire du
parlement étant achevée, il y en allait avoir une autre. Chacun, en
suspens, tâchait à la prévenir par la pensée. On a su depuis, que tout
le parlement s'attendit à la décision du bonnet en notre faveur, et
j'expliquerai après pourquoi il n'en fut pas mention. D'autres, avertis
par leurs yeux de l'absence des bâtards, jugèrent plus juste qu'il
allait s'agir de quelque chose qui les regardait\,; mais personne ne
devina quoi, beaucoup moins toute l'étendue.

Enfin le garde des sceaux ouvrit la bouche, et dès la première période
il annonça la chute d'un des frères et la conservation de l'autre.
L'effet de cette période sur tous les visages est inexprimable. Quelque
occupé que je fusse à contenir le mien, je n'en perdis pourtant aucune
chose. L'étonnement prévalut aux autres passions. Beaucoup parurent
aises, soit équité, soit haine pour le duc du Maine, soit affection pour
le comte de Toulouse\,; plusieurs consternés. Le premier président
perdit toute contenance\,; son visage, si suffisant et si audacieux, fut
saisi d'un mouvement convulsif\,; l'excès seul de sa rage le préserva de
l'évanouissement. Ce fut bien pis à la lecture de la déclaration. Chaque
mot était législatif et portait une chute nouvelle. L'attention était
générale, tenait chacun immobile pour n'en pas perdre un mot, et les
yeux sur le greffier qui lisait. Vers le tiers de cette lecture, le
premier président, grinçant le peu de dents qui lui restaient, se laissa
tomber le front sur son bâton, qu'il tenait à deux mains, et, en cette
singulière posture et si marquée, acheva d'entendre cette lecture si
accablante pour lui, si résurrective pour nous.

Moi cependant je me mourais de joie. J'en étais à craindre la
défaillance\,; mon coeur, dilaté à l'excès, ne trouvait plus d'espace à
s'étendre. La violence que je me faisais pour ne rien laisser échapper
était infinie, et néanmoins ce tourment était délicieux. Je comparais
les années et les temps de servitude, les jours funestes où, traîné au
parlement en victime, j'y avais servi de triomphe aux bâtards à
plusieurs fois, les degrés divers par lesquels ils étaient montés à ce
comble sur nos têtes\,; je les comparais, dis-je, à ce jour de justice
et de règle, à cette chute épouvantable, qui du même coup nous relevait
par la force de ressort. Je repassais, avec le plus puissant charme, ce
que j'avais osé annoncer au duc du Haine le jour du scandale du bonnet,
sous le despotisme de son père. Mes yeux voyaient enfin l'effet et
l'accomplissement de cette menace. Je me devais, je me remerciais de ce
que c'était par moi qu'elle s'effectuait. J'en considérais la rayonnante
splendeur en présence du roi et d'une assemblée si auguste. Je
triomphais, je me vengeais, je nageais dans ma vengeance\,; je jouissais
du plein accomplissement des désirs les plus véhéments et les plus
continus de toute ma vie. J'étais tenté de ne me plus soucier de rien.
Toutefois je ne laissais pas d'entendre cette vivifiante lecture dont
tous les mots résonnaient sur mon coeur comme l'archet sur un
instrument, et d'examiner en même temps les impressions différentes
qu'elle faisait sur chacun.

Au premier mot que le garde des sceaux dit de cette affaire, les yeux
des deux évêques pairs rencontrèrent les miens. Jamais je n'ai vu
surprise pareille à la leur, ni un transport de joie si marqué. Je
n'avais pu les préparer à cause de l'éloignement de nos places, et ils
ne purent résister au mouvement qui les saisit subitement. J'avalai par
les yeux un délicieux trait de leur joie, et je détournai les miens des
leurs, de peur de succomber à ce surcroît, et je n'osai plus les
regarder.

Cette lecture achevée, l'autre déclaration en faveur du comte de
Toulouse fut commencée tout de suite par le greffier, suivant le
commandement que lui en avait fait le garde des sceaux en les lui
donnant toutes deux ensemble. Elle sembla achever de confondre le
premier président et les amis du duc du Maine, par le contraste des deux
frères. Celle-ci surprit plus que pas une, et à qui n'était pas au fait,
la différence était inintelligible\,: les amis du comte de Toulouse
ravis, les indifférents bien aises de son exception, mais la trouvant
sans fondement et sans justice. Je remarquai des mouvements très divers
et plus d'aisance à se parler les uns aux autres pendant cette lecture,
à laquelle néanmoins on fut très attentif.

Les importantes clauses du consentement des princes du sang et de la
réquisition des pairs de France réveillèrent l'application générale, et
firent lever le nez au premier président de dessus son bâton, qui s'y
était remis. Quelques pairs même, excités par M. de Metz, grommelèrent
entre leurs dents, chagrins, à ce qu'ils expliquèrent à leurs confrères
voisins, de n'avoir pas été consultés en assemblée générale sur un fait
de cette importance, sur lequel néanmoins on les faisait parler et
requérir. Mais quel moyen de hasarder un secret de cette nature dans une
assemblée de pairs de tous âges, pour n'en rien dire de plus, encore
moins d'y en discuter les raisons\,? Le très peu de ceux qui en furent
choqués alléguèrent que ceux de la régence avaient apparemment répondu
pour les autres sans mission, et cette petite jalousie les piquait
peut-être autant que la conservation au rang, etc., du comte de
Toulouse. Cela fut apaisé aussitôt que né\,: mais rien en ce monde sans
quelque contradiction.

Après que l'avocat général eut parlé, le garde des sceaux monta au roi,
prit l'avis des princes du sang, puis vint au duc de Sully et à moi.
Heureusement j'eus plus de mémoire qu'il n'en voulut avoir\,: aussi
était-ce mon affaire. Je lui présentai mon chapeau à bouquet de plumes
au-devant, d'une façon exprès très marquée, en lui disant assez haut\,:
«\,Non, monsieur, nous ne pouvons être juges, nous sommes parties, et
nous n'avons qu'à rendre grâces au roi de la justice qu'il veut bien
nous faire.\,» Il sourit et me fit excuse. Je le repoussai avant que le
duc de Sully eût eu loisir d'ouvrir la bouche\,; et regardant aussitôt
de part et d'autre, je vis avec plaisir que ce refus d'opiner avait été
remarqué de tout le monde. Le garde des sceaux retourna tout court sur
ses pas, et sans prendre l'avis des pairs en place de service, ni des
deux évêques pairs, fut aux maréchaux de France, puis descendit au
premier président et aux présidents à mortier, puis alla au reste des
bas sièges\,; après quoi, remonté au roi et redescendu en place, il
prononça l'arrêt d'enregistrement, et mit le dernier comble à ma joie.

Aussitôt après M. le Duc se leva, et, après avoir fait la révérence au
roi, il oublia de s'asseoir et de se couvrir pour parler, suivant le
droit et l'usage non interrompu des pairs de France\,: aussi nous ne
nous levâmes pas un. Il fit donc debout et découvert le discours, qui a
paru imprimé à la suite des discours précédents, et le lut peu
intelligiblement, parce que l'organe n'était pas favorable. Dès qu'il
eut fini, M. le duc d'Orléans se leva et commit la même faute. Il dit
donc, aussi debout et découvert, que la demande de M. le Duc lui
paraissait juste\,; et après quelques louanges ajouta que, présentement
que M. le duc du Maine se trouvait en son rang d'ancienneté de pairie,
M. le maréchal de Villeroy, son ancien, ne pouvait plus demeurer sous
lui, ce qui était une nouvelle et très forte raison, outre celles que M.
le Duc avait alléguées. Cette demande avait porté au dernier comble
l'étonnement de toute l'assemblée, au désespoir du premier président et
de ce peu de gens qui, à leur déconcertement, paraissaient s'intéresser
au duc du Maine. Le maréchal de Villeroy, sans sourciller, fit toujours
mauvaise mine, et les yeux du premier écuyer s'inondèrent souvent de
larmes. Je ne pus bien distinguer le maintien de son cousin et ami
intime le maréchal d'Huxelles, qui se mit à l'abri des vastes bords de
son chapeau enfoncé sur ses yeux, et qui d'ailleurs ne branla pas. Le
premier président, assommé de ce dernier coup de foudre, se démonta le
visage à vis, et je crus un moment son menton tombé sur ses genoux.

Cependant le garde des sceaux ayant dit aux gens du roi de parler, ils
répondirent qu'ils n'avaient pas ouï la proposition de M. le Duc, sur
quoi, de main en main, on leur envoya son papier, pendant quoi le garde
des sceaux répéta fort haut ce que le régent avait ajouté sur
l'ancienneté de pairie du maréchal de Villeroy au-dessus du duc du
Maine. Blancmesnil ne fit que jeter les yeux sur le papier de M. le Duc
et parla, après quoi le garde des sceaux fut aux voix. Je donnai la
mienne assez haut et dis\,: «\,Pour cette affaire-ci, monsieur, j'y
opine de bon coeur à donner la surintendance de l'éducation du roi à M.
le Duc.\,»

La prononciation faite, le garde des sceaux appela le greffier en chef,
lui ordonna d'apporter ses papiers et son petit bureau près du sien pour
faire tout présentement et tout de suite, et en présence du roi, tous
les enregistrements de tout ce qui venait d'être lu et ordonné, et les
signer. Cela se fit sans difficulté aucune, dans toutes les formes, sous
les yeux du garde des sceaux, qui ne les levait pas de dessus\,; mais,
comme il y avait cinq ou six pièces à enregistrer, cela fut long à
faire.

J'avais fort observé le roi lorsqu'il fut question de son éducation, je
ne remarquai en lui aucune sorte d'altération, de changement, pas même
de contrainte. Ç'avait été le dernier acte du spectacle, il en était
tout frais lorsque les enregistrements s'écrivirent. Cependant, comme il
n'y avait plus de discours qui occupassent, il se mit à rire avec ceux
qui se trouvèrent à portée de lui, à s'amuser de tout, jusqu'à remarquer
que le duc de Louvigny, quoique assez éloigné de son trône, avait un
habit de velours, à se moquer de la chaleur qu'il en avait, et tout cela
avec grâce. Cette indifférence pour M. du Maine frappa tout le monde et
démentit publiquement ce que ses partisans essayèrent de répandre que
les yeux lui avaient rougi, mais que, ni au lit de justice ni depuis, il
n'en avait osé rien témoigner. Or, dans la vérité, il eut toujours les
yeux secs et sereins et il ne prononça le nom du duc de Maine qu'une
seule fois depuis, qui fut l'après-dînée du même jour, qu'il demanda où
il allait d'un air très indifférent, sans en rien dire davantage, ni
depuis ni nommer ses enfants\,; aussi ceux-ci ne prenaient guère la
peine de le voir, et, quand ils y allaient, c'était pour avoir jusqu'en
sa présence leur petite cour à part et se divertir entre eux. Pour le
duc du Maine, soit politique, soit qu'il crût qu'il n'en était pas
encore temps, il ne le voyait que les matins, quelque temps à son lit,
et plus du tout de la journée, hors les fonctions d'apparat.

Pendant l'enregistrement je promenoir mes yeux doucement de toutes
parts, et, si je les contraignis avec constance, je ne pus résister à la
tentation de m'en dédommager sur le premier président, je l'accablai
donc à cent reprises, dans la séance, de mes regards assenés et
forlongés avec persévérance. L'insulte, le mépris, le dédain, le
triomphe lui furent lancés de mes yeux jusqu'en ses moelles\,; souvent
il baissait la vue quand il attrapait mes regards\,; une fois ou deux il
fixa le sien sur moi, et je me plus à l'outrager par des sourires
dérobés, mais noirs, qui achevèrent de le confondre. Je me baignais dans
sa rage et je me délectais à le lui faire sentir. Je me jouais de lui
quelquefois avec mes deux voisins, en le leur montrant d'un clin d'oeil,
quand il pouvait s'en apercevoir\,; en un mot, je m'espaçai sur lui sans
ménagement aucun autant qu'il ne fut possible.

Enfin, les enregistrements achevés, le roi descendit de son trône et
dans les bas sièges, par son petit degré, derrière la chaire du garde
des sceaux, suivi du régent et des deux princes du sang et des seigneurs
de sa suite nécessaire. En même temps les maréchaux de France
descendirent par le bout de leurs hauts sièges, et, tandis que le roi
traversait le parquet, accompagné de la députation qui avait été le
recevoir, ils passèrent entre les bancs des conseillers, vis-à-vis de
nous, pour se mettre à la suite du roi, à la porte de la séance par
laquelle Sa Majesté sortit comme elle y était entrée\,; en même temps
aussi les deux évêques pairs, passant devant le trône, vinrent se mettre
à notre tête et me serrèrent les mains et la tête, en passant devant
moi, avec une vive conjouissance. Nous les suivîmes, reployant deux à
deux le long de nos bancs, les anciens les premiers, et descendus des
hauts sièges par le degré du bout. Nous continuâmes tout droit, et
sortîmes par la porte vis-à-vis. Le parlement se mit après en marche, et
sorti par l'autre porte, qui était celle par où nous étions entrés
séparément et par où le roi était entré et sorti. On nous fit faire
place jusqu'au degré. La foule, le monde, le spectacle resserrèrent nos
discours et notre joie. J'en étais navré. Je gagnai aussitôt mon
carrosse, que je trouvai sous ma main, et qui me sortit très
heureusement de la cour, en sorte que je n'eus point d'embarras, et que
de la séance chez moi je ne mis pas un quart d'heure.

\hypertarget{note-i.-comparaison-entre-les-parlements-de-france-et-dangleterre.}{%
\chapter{NOTE I. COMPARAISON ENTRE LES PARLEMENTS DE FRANCE ET
D'ANGLETERRE.}\label{note-i.-comparaison-entre-les-parlements-de-france-et-dangleterre.}}

La comparaison entre les parlements de France et d'Angleterre, dont
parle Saint-Simon, a été tentée à plusieurs époques, quoique, dans la
réalité, il fût impossible d'assimiler des corps de magistrature, dont
les charges étaient vénales, avec des assemblées élues par la nation
pour représenter ses intérêts. Au XVIe siècle, Michel de Castelnau, qui
écrivit en Angleterre la plus grande partie de ses Mémoires, exalte la
puissance des parlements français\,; il en parle «\,comme de huit
colonnes fortes et puissantes, sur lesquelles est appuyée cette grande
monarchie de France\footnote{\emph{Mémoires de Castelnau}, liv. I,
  chap.~IV.}.\,» Il les compare positivement au parlement d'Angleterre,
et leur subordonne en quelque sorte la puissance royale, lorsqu'il
ajoute que «\,les édits ordinaires n'ont point force et ne sont
approuvés des autres magistrats, s'ils ne sont reçus et vérifiés des
parlements, qui est une règle d'État, par le moyen de laquelle le roi ne
pourrait, quand il voudrait faire des lois injustes, que bientôt après
elles ne fussent rejetées.\,»

Les prétentions des parlements trouvaient une sorte de sanction dans une
décision des états de Blois (1577), qui avaient déclaré que «\,tous les
édits devoient être vérifiés et comme contrôlés ès cours de parlement,
lesquelles, combien qu'elles ne soient qu'une forme des trois états,
raccourcie au petit pied, ont pouvoir de suspendre, modifier et refuser
les édits\footnote{\emph{Mémoires de Nevers}, t. I, p.~449.}.\,»A la
faveur des troubles des guerres de religion, le parlement accrut
considérablement son pouvoir. L'ambassadeur autrichien, Büsbeck, qui
résida à la cour de Henri III, écrivait, en 1584, «\,qu'en France les
parlements ne règnent pas moins que le roi lui-même\footnote{«\,Concilia,
  quae \emph{parlamenta} vocant, regnant in Gallia, non minus fere quam
  ipse rex.\,» Lettre du 4 octobre 1584.}.\,» Deux minorités
fortifièrent encore l'autorité du parlement de Paris. Il en vint,
pendant la Fronde, à se regarder comme supérieur aux états généraux. À
l'occasion d'une lettre du parlement de Rouen, qui demandait au
parlement de Paris, s'il devait envoyer une députation à l'assemblée
projetée des états généraux, M. de Mesmes dit\,: «\,que les parlements
n'y avaient jamais député, étant composés des trois états\,; qu'ils
tenaient un rang au-dessus des états généraux, étant juges de ce qui y
était arrêté, par la vérification\,; que les états généraux n'agissaient
que par prières et ne parlaient qu'à genoux, comme les peuples et
sujets\,; mais que les parlements tenaient un rang au-dessus d'eux,
étant comme médiateurs entre le peuple et le roi\footnote{\emph{Journal
  d'Oliv. d'Ormesson}, à la date du lundi 1er mars 1649.}.\,»Ces
prétentions hautaines des parlements furent réprimées par Louis XIV\,;
mais elles reparurent pendant la régence, et provoquèrent, des plaintes
dont Saint-Simon s'est fait l'interprète.

\hypertarget{note-ii.-querelle-entre-les-pruxe9sidents-du-parlement-et-les-ducs-pairs.}{%
\chapter{NOTE II. QUERELLE ENTRE LES PRÉSIDENTS DU PARLEMENT ET LES
DUCS-PAIRS.}\label{note-ii.-querelle-entre-les-pruxe9sidents-du-parlement-et-les-ducs-pairs.}}

La querelle des présidents du parlement et des ducs et pairs, sur
laquelle Saint-Simon revient si souvent, était déjà ancienne. Un
manuscrit des Archives de l'empire (sect. judiciaires, U. 96, f° 199 et
suiv.) fournit quelques renseignements sur ces discussions, et donne les
arrêts et requêtes dont parle Saint-Simon.

L'extrait que nous donnons de ce manuscrit, est le complément naturel
des Mémoires de Saint-Simon. L'auteur anonyme parle d'abord de l'arrêt
rendu en avril 1664\,:

«\,Les mémoires des ducs furent communiqués aux présidents, et, après
que la matière eut été amplement discutée par plusieurs arrêts imprimés
de part et d'autre, et alors remis entre les mains de M. le chancelier
le 26 avril 1664, le roi, par un arrêt de son conseil d'État, où il
était présent en personne, décida cette contestation, maintint et garda
les pairs au droit d'opiner et dire leurs avis avant les présidents au
parlement de Paris, lorsque Sa Majesté y tiendrait son lit de justice,
sans qu'ils y puissent être troublés pour quelque cause et occasion que
ce soit. Cet arrêt fut enregistré au parlement, le roi séant en son lit
de justice, le mardi 29 avril 1664, et exécuté le même jour par M. le
chancelier, qui prit l'avis de MM. les pairs avant que de le prendre de
MM. les présidents.\,»

~

«\,\emph{Arrêt du conseil d'État portant règlement entre les ducs et
pairs et les présidents du parlement de Paris sur leur droit d'opiner
lorsque le roi tient son lit de justice}. (26 août 1664. - Enregistré au
parlement le 29 dudit mois et an. - Extrait des registres du conseil
d'État.)

~

«\,Le roi s'étant fait représenter, en son conseil, les mémoires mis
entre les mains de M. le chancelier, tant par les officiers de la cour
du parlement de Paris que par les pairs de France, suivant le
commandement qu'ils en avaient reçu de Sa Majesté\,; et, ayant vu par
lesdits mémoires les raisons, par lesquelles le parlement prétend que
les présidents en icelui doivent opiner avant les pairs, lorsque Sa
Majesté y tient son lit de justice, comme aussi les moyens dont les
pairs se servent pour appuyer le droit par eux prétendu de dire leurs
avis en de pareilles séances avant les présidents\,; Sa Majesté voulant
terminer ce différend, et prévenir les difficultés qui pourraient naître
en de semblables occasions, étant en son conseil, a maintenu et gardé,
maintient et garde les pairs de France au droit d'opiner et dire leurs
avis avant les présidents au parlement de Paris, lorsque Sa Majesté y
tiendra son lit de justice, sans qu'ils puissent être troublés pour
quelque cause et occasion que ce soit\,; veut pour cette fin Sa Majesté
que le présent arrêt soit enregistré ès registres de ladite cour. Fait
au conseil du roi, Sa Majesté y étant, tenu le 26 avril 1664.
\emph{Signé\,:} deGuénégaud.

«\,Et attendu que, depuis l'arrêt du conseil du 26 avril 1664, il y a de
nouvelles contestations entre les ducs et les présidents, il est a
propos de rapporter en cet endroit les conclusions des requêtes, qui
sont à juger au conseil entre les ducs et les présidents, à l'occasion
de leurs séances et opinions aux lits de justice, où ces contestations
ont été formées.

«\,\emph{Extrait des conclusions des requêtes présentées par MM. les
ducs au roi Louis XV et à M. le régent, au sujet de nouvelles
contestations formées par MM\hspace{0pt}. les présidents à mortier
contre MM. les ducs, depuis l'arrêt du règlement du} 26 \emph{avril}
1664.

PREMIÈRE REQUÊTE.

«\,Les ducs demandent, par leur première requête qu'ils ont présentée au
roi et à M. le régent, et par leurs mémoires, imprimés chez Urbain
Coutelier, libraire, et par les conclusions de ladite requête, qu'il
soit ordonné\,: 1° que le premier président sera tenu, aux séances de
rapport, de prendre l'avis des pairs, en les saluant\,; 2° qu'à ces
mêmes séances de rapport et aux audiences des bas sièges, l'ordre de
séances des pairs ne pourra, sous quelque prétexte que ce soit, être
interrompu par des conseillers placés à l'extrémité des bancs remplis
par les pairs\,; 3° que les réceptions des pairs se feront dorénavant
aux lits de justice ou bien aux audiences des hauts sièges, suivant
l'usage constamment pratiqué avant l'année 1643\,; 4° que dans toutes
les affaires, où les pairs auront été invités, leur présence sera
exprimée dans le prononcé de l'arrêt par l'ancienne formule\,: \emph{La
cour suffisamment garnie de pairs}.

SECONDE REQUÊTE.

«\,Les pairs demandent, par les conclusions de leur requête et mémoires,
qu'en attendant son jugement sur les contestations avec les présidents,
Sa Majesté ordonne que l'arrêt du parlement de 1715, rendu avant
l'arrivée des pairs, pour leur ôter voix délibérative dans cette séance,
au cas qu'ils voulussent, en opinant, interrompre l'usurpation des
présidents, sera déclaré attentatoire à l'autorité de Votre Majesté,
contraire à toutes les lois, et, en conséquence, comme tel, il sera dès
à présent rayé et biffé des registres du parlement, et cancellé.

«\,L'arrêt du 2 septembre 1715 portait que, si les pairs persistaient à
demander le salut, c'est-à-dire que le premier président ôtât son bonnet
en leur demandant leur opinion, ou donnaient leurs avis le chapeau sur
la tête, leurs voix ne seraient pas comptées.\,»

\end{document}
